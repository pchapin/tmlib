\section{Examples}
\label{section-examples}

In this section we present some instances of $\RTR$ that illustrate
how the system is used, and how it is able to capture a variety of
risk management schemes.  

\subsection{Bound-of-Risks}  
\label{section-examples-bounds}

In \cite{Den:latt}, an information flow security model is presented
where all static data is assigned to a security class.  Security
classifications of variables are then assigned based on the
combination of security classes of data flowing into those variables,
as determined by an abstract program interpretation.  Security classes
are identified by elements in a complete lattice, where
``class-combination'' is defined as the lub of combined classes.

We propose that an adaptation of this model is useful in the context
of authorization risk assessment.  We do not propose an abstract
interpretation of authorization, incorporating some form of
``may-analysis'', but rather a purely dynamic authorization and risk
assessment model, so in this sense we differ from the model proposed
in \cite{Den:latt}.  Nevertheless, we may adopt the use of least upper
bounds as a ``class-combination'' mechanism-- in our terminology,
risk aggregation-- that assesses the risk of any authorization
decision as the least upper bound of risks associated with all
credentials used in the decision.  Thus, we define each of
$\riskplus$ and $\isectplus$ as the lub operator on
risks in the given partial ordering.

Consider a risk ordering where three classifications $\mathcal{K} =
\setdefn{\mathit{low,medium,high}}$ are defined, and the following
relations are imposed:
$$
\mathit{low} \po \mathit{medium} \po \mathit{high}
$$ 
Imagine also that an online vendor called $\mathit{Store}$
maintains a purchasing policy whereby representatives of the
$\mathit{Acme}$ corporation have $\mathit{buyer}$ power only if they
are both employees and official purchasers.  Since this policy is
maintained locally, it is associated with a $\mathit{low}$ risk of
usage, hence $\mathit{Store}$ could specify:
\begin{mathpar}
\mathitcred{Store.buyer}{Acme.purchaser \cap Acme.employee}{low}
\end{mathpar}
Imagine further that $\mathit{Ed}$ attempts to make a purchase from
$\mathit{Store}$, providing certificates claiming $\mathit{employee}$
and $\mathit{purchaser}$ status.  However, if we assume that these
certificates can possibly be faked, or that role membership within the
$\mathit{Acme}$ corporation has a volatile status, higher risk can be
assigned to these certificates:
\begin{mathpar}
\mathitcred{Acme.employee}{Ed}{medium}

\mathitcred{Acme.purchaser}{Ed}{high}
\end{mathpar}
We also assume that a less risky path of establishing $\mathit{Ed}$'s
membership in the $\mathit{Acme.purchaser}$ role is through a
$\mathit{manager}$ certificate obtained directly from the issuer
$\mathit{Personnel}$, and via $\mathit{Acme}$'s own policy
specifying $\mathit{purchaser}$ power for all $\mathit{manager}$s:
\begin{mathpar}
\mathitcred{Acme.purchaser}{Personnel.manager}{low}

\mathitcred{Personnel.manager}{Ed}{low}
\end{mathpar}
Although using $\mathit{Ed}$'s certificate asserting his membership in
the $\mathit{Acme.purchaser}$ role will incur a $\mathit{high}$ risk,
because of the less risky path to this relation, the risk assessment of
this set of credentials will find that establishing $\mathit{Ed}$'s
membership in the $\mathit{Store.buyer}$ role requires a lower bound of
$\mathit{medium}$ risk.  The solution for this set of credentials 
is as follows:
\begin{eqnarray*}
\mathit{Store.buyer}&:& \setdefn{(\mathit{Ed, medium})}\\
\mathit{Acme.employee}&:& \setdefn{(\mathit{Ed, medium})}\\
\mathit{Acme.purchaser}&:& \setdefn{(\mathit{Ed, low})}\\
\mathit{Personnel.manager}&:& \setdefn{(\mathit{Ed, low})}
\end{eqnarray*}
Of course, in certain cases it may be preferable to use the
certificate $\mathit{Ed}$ provides, instead of going through
$\mathit{Personnel}$-- if wait times for distributed communication
with that node are prohibitively long, for example.  In this case it
should be specified that a $\mathit{high}$ level of risk will be
tolerated in the credential chain.  This is accomplished by defining
an appropriate threshold.  Although the semantics do not explicitly
list a $\mathit{high}$ risk membership in $\mathit{Store.buyer}$, it
does exist, and may be used in practice as discussed in
\autoref{section-rtr-discovery}.

Returning to the example, for the purposes of illustration we imagine
that the risk ordering is extended with an element $\mathit{moderate}$,
that is incomparable with $\mathit{medium}$, inducing the lattice:
\begin{center}
\risklattice
\end{center}
We also imagine that $\mathit{Store}$ has cached an old certificate,
establishing $\mathit{Ed}$'s membership in the $\mathit{Acme.employee}$
role with $\mathit{moderate}$ risk:
\begin{mathpar}
\mathitcred{Acme.employee}{Ed}{moderate}
\end{mathpar}
In this case, since $\mathit{moderate}$ and $\mathit{medium}$ are 
incomparable, the risk assessment will reflect that $\mathit{Ed}$'s
membership in the $\mathit{Store.buyer}$ and $\mathit{Acme.employee}$
roles can be established via two paths with incomparable risk:
\begin{eqnarray*}
\mathit{Store.buyer}&:& \setdefn{(\mathit{Ed, medium}),(\mathit{Ed, moderate})}\\
\mathit{Acme.employee}&:& \setdefn{(\mathit{Ed, medium}),(\mathit{Ed, moderate})}
\end{eqnarray*}

\subsubsection{Agreement Decreases Risk}

In case a PGP-like scheme of allowing agreement to reduce risk is
desired, intersection aggregation can be modified appropriately
without having to change risk aggregation.  For example, returning to
the risk ordering comprising just
$\setdefn{\mathit{high},\mathit{medium},\mathit{low}}$, and specifying
that $\isectplus$ be commutative, we could define:
\begin{mathpar}
\mathit{low} \isectplus \mathit{low} = \mathit{low} 

\mathit{low} \isectplus \mathit{medium} = \mathit{low} 

\mathit{low} \isectplus \mathit{high} = \mathit{low} 

\mathit{medium} \isectplus \mathit{medium} = \mathit{low} 

\mathit{medium} \isectplus \mathit{high} = \mathit{medium} 

\mathit{high} \isectplus \mathit{high} = \mathit{medium} 
\end{mathpar}
Intersection aggregation thus defined is both monotonic 
and associative as can easily be checked.  Given these
definitions and the following credentials:
\begin{mathpar}
\mathitcred{Store.buyer}{Acme.purchaser \cap Acme.employee}{low}

\mathitcred{Acme.employee}{Ed}{medium}

\mathitcred{Acme.purchaser}{Ed}{high}
\end{mathpar}
role memberships will reflect the reduction in risk achieved
via intersection:
\begin{eqnarray*}
\mathit{Store.buyer}&:& \setdefn{(\mathit{Ed, medium})}\\
\mathit{Acme.employee}&:& \setdefn{(\mathit{Ed, medium})}\\
\mathit{Acme.purchaser}&:& \setdefn{(\mathit{Ed, high})}\\
\end{eqnarray*}

\subsection{Sum-of-Risks}
\label{sec:sum-of-risks}

An alternative to the bound-of-risks model is a sum-of-risks model,
where credentials are assigned numeric risk values and the total risk
for any authorization decision is the sum of all risks associated with
the credentials used in the decision.  Thus, we take the risk ordering
in this model to be the lattice of natural numbers up to $\omega$
induced by $\leq$, and we take $\riskplus$ and
$\isectplus$ to be addition.  This model is useful in case risk is
considered additive, or in case the number of credentials used in an
authorization decision is an element of risk, the more the riskier.

Imagining a similar situation as above, the following risks could 
be assigned, where 1 is considered ``not risky'' and 4 is considered
``risky'':
\begin{mathpar}
\mathitcred{Store.buyer}{Acme.purchaser \cap Acme.employee}{1}

\mathitcred{Acme.employee}{Ed}{3}

\mathitcred{Acme.purchaser}{Ed}{4}

\mathitcred{Acme.purchaser}{Personnel.manager}{2}

\mathitcred{Personnel.manager}{Ed}{3}
\end{mathpar}
Note that $\mathit{Ed}$'s certificate claiming membership in the role
$\mathit{Acme.purchaser}$ is still assigned higher risk than both
the certificate establishing his $\mathit{manager}$ status and the
certificate establishing $\mathit{purchaser}$ rights for
$\mathit{manager}$s.  However, the sum-of-risks model will still
ascertain that the use of $\mathit{Ed}$'s certificate will be the least
risky way to establish his membership in the $\mathit{Store.buyer}$
role.  The solution of the given credentials will comprise the following
risk assessments:
\begin{eqnarray*}
\mathit{Store.buyer}&:& \setdefn{(\mathit{Ed, 8})}\\
\mathit{Acme.employee}&:& \setdefn{(\mathit{Ed, 3})}\\
\mathit{Acme.purchaser}&:& \setdefn{(\mathit{Ed, 4})}\\
\mathit{Personnel.manager}&:& \setdefn{(\mathit{Ed, 3})}
\end{eqnarray*}
If a pure count of credentials used in authorization is the 
basis of risk assessment, this model can be formally obtained 
in the sum-of-risks model by associating risk 1 with every 
credential.

Just as in the bound-of-risks model, intersection aggregation can be
modified to interpret agreement as reducing risk.  For example,
$\isectplus$ can be defined as the average of its operands, or some
other fraction of their sum.

\subsection{Delegation Depth and Width}

In $\RT_0$, type 2 credentials allow delegation of authority across
domain boundaries.  For example, the credential
$\cred{A.r_0}{B.r_1}{}$ allows the entity $A$ to delegate authority to
define a role within its namespace to the entity $B$, which may denote
a different security domain.  Furthermore, $B$ is able to delegate
authority to define $A.r_0$ to another entity $C$ via the credential
$\cred{B.r_1}{C.r_2}{}$.  However, as observed by various authors,
trust is not necessarily transitive, so that $A$ may wish to prevent
$B$ from from further delegation of authority to define $A.r_0$ to $C$
or anyone else.  This sort of control might also be more fine-grained,
in that $A$ might wish to allow one level of delegation of authority
to define $A.r_0$, from $B$ to $C$ for example, but no further, so
that $C$ should not be allowed to delegate authority to define $A.r_0$
to another entity.  The idea clearly generalizes to delegations of
arbitrary depth.

The system $\RT$ as originally conceived \cite{Li:2003-04} does not
allow delegation depth to be restricted in this way.  An extension of
$\RT_0$ called $\RT_+$ \cite{hong-zhu-wang-aina05} was proposed to
allow expression of delegation depth policies.  Here, we show how to
specify similar delegation depth control policies in an instance of
$\RTR$.

Assume given $\ponat$ as defined in \autoref{example-ponat} as a risk
ordering.  The encoding is then based on a numeric representation of
depth for individual credentials.  We consider the specification of
risk values and aggregation operations by considering each credential
type in turn.  Type 1 credentials define a role membership directly
within a namespace, so the delegation depth associated with those
credentials is 0.  Hence, all credentials match the schema:
$$
\cred{A.r}{B}{0}
$$

Type 2 credentials do allow delegation of role definition authority,
so type 2 credentials have a delegation depth of 1 if the delegation
crosses namespaces.  Otherwise the depth of a type 2 credential is 
0.  Hence, all type 2 credentials match the schemas:
\begin{eqnarray*}
&\cred{A.r_1}{A.r_2}{0}\\
&\cred{A.r_1}{B.r_2}{1} \quad A \ne B
\end{eqnarray*}
Naturally, risk aggregation is defined as addition, so that depth is
added as credential edges are crossed:
$$
\risk_1 \riskplus \risk_2 \defeq \risk_1 + \risk_2
$$

Type 3 credentials allow indirect delegation.  Recall that an $\RT_0$
credential of the form $\cred{A.r_1}{B.r_2.r_3}{}$ allows us to assert
that $C \in A.r_1$ if $D \in B.r_2$ and $C \in D.r_3$.  In our view,
this means that $B$ is thereby capable of delegating to $D$ the
authority to define $A.r_1$, and therefore also $A$ delegates to $B$
the authority to define the role.  So firstly, this means that 
type 3 credentials should be assigned a delegation depth of 1 if 
$A$ and $B$ are distinct namespaces, and 0 otherwise:
\begin{eqnarray*}
&\cred{A.r_1}{A.r_2.r_3}{0}\\
&\cred{A.r_1}{B.r_2.r_3}{1} \quad A \ne B
\end{eqnarray*}
Secondly, given some credential $\cred{A.r_1}{B.r_2.r_3}{}$, linking
aggregation should sum the delegation depth associated with
determining $D \in B.r_2$ with the delegation depth associated
with determining $C \in D.r_3$ to determine $C \in A.r_1$.  Hence, 
linking aggregation is also defined as addition:
\begin{mathpar}
\risk_1 \linkplus \risk_2 \defeq \risk_1 + \risk_2 
\end{mathpar}
We note that our model differs from the $\RT_+$ model with respect to
risk aggregation and depth of type 3 credentials.  In that paper, type
3 credentials are always assigned depth 1, and the depth associated
with determining $D \in B.r_2$ is ignored in linking aggregation.  The
authors do not clarify the reasons for these choices, but we believe
they are flawed.  Regarding the depth of type 3 credentials, consider
the following $\RT_0$ example:
\begin{mathpar}
\cred{A.r_1}{A.r_2.r_3}{}  

\cred{A.r_2}{A.r_3}{}

\cred{A.r_3}{B}{}
\end{mathpar}
These credentials allow us to establish that $B \in A.r_1$ with no
delegation of authority, whereas the scheme in $\RT_+$ would assign
a delegation depth of 1.  Multiple links within the same namespace
would extend the spurious depth, allowing arbitrarily large
overestimates of delegation depth.

Underestimates of delegation depth are also possible given the scheme
in $\RT_+$, as follows.  Consider the following credentials, where
$C_1.s_1 \longleftarrow \cdots \longleftarrow C_n.s_n$ denotes $n-1$
type 2 credentials in the obvious manner.
\begin{mathpar}
\cred{A.r_1}{B.r_2}{}

\cred{B.r_2}{C_1.s_1}{}

C_1.s_1 \longleftarrow \cdots \longleftarrow C_n.s_n

\cred{C_n.s_n}{D}{}
\end{mathpar}
Assuming that $A$, $B$, $D$, and $C_1,\ldots,C_n$ all denote distinct
namespaces, establishing $D \in A.r_1$ involves a delegation depth of
$n+1$.  However, there is an easy attack that $B$ can use to 
reduce an $n+1$ delegation depth to a depth of 2:
$B$ could eliminate $\cred{B.r_2}{C_1.s_1}{}$, and 
add the following credentials:
\begin{mathpar}
\cred{B.r_2}{B.r_3.r_4}{}

\cred{B.r_3}{B}{}

\cred{B.r_4}{C_1.s_1}{}
\end{mathpar}
In contrast, our monotonicity requirements on aggregation prevents
such an attack.

In intersection roles, depth of components should not be summed,
instead each component should be considered an independent ``branch''.
Intersection aggregation is therefore defined as the max height
of its operands:
\begin{mathpar}
\risk_1 \isectplus \risk_2 \defeq \mathit{max}(\risk_1,\risk_2)
\end{mathpar}
Like the other credential forms, type 4 credentials are assigned a
depth risk on the basis of whether they cross domain boundaries.
Since role expressions of any form may be intersected, we need to
specify the \emph{subjects} of role expressions:
\begin{eqnarray*}
\subjects(A) &=& \varnothing \\
\subjects(A.r) &=& \setdefn{A} \\
\subjects(A.r_1.r_2) &=& \setdefn{A}\\
\subjects(f_1 \cap \cdots \cap f_n) &=& \subjects(f_1) \cup \cdots \cup \subjects(f_n)
\end{eqnarray*}
and we assign depth risks to type 4 credentials as follows:
\begin{mathpar}
\cred{A.r}{f_1 \cap \cdots \cap f_n}{0} \quad \text{ if } \subjects(f_1 \cap \cdots \cap f_n) \subseteq \setdefn{A}

\cred{A.r}{f_1 \cap \cdots \cap f_n}{1} \quad \text{ if } \subjects(f_1 \cap \cdots \cap f_n) \nsubseteq \setdefn{A}
\end{mathpar}

\subsubsection{Controlling Delegation Width} 

The designers of SPKI/SDSI provided a simple scheme of boolean control
for delegation, expressible in our model by specifying a threshold
$\thresh$ such that $\thresh(A.r) = \omega$ if it was desired that
authority over the role $A.r$ could be delegated, and $\thresh(A.r) =
0$ if not.  The complexity of full integer depth control as in our
general model was not adopted, in part because depth control does not
control delegation \emph{width}, hence does not address problems of
proliferation \cite{ellison-etal-rfc99}.  Any given principal may
delegate authority to an unlimited number of other principals:
$$
\cred{A.r}{B_1.s_1}{} \qquad \cdots \qquad \cred{A.r}{B_n.s_n}{}
\qquad A \ne B_1 \ne \cdots \ne B_n
$$ and notions of delegation depth provide no control on the size of
$n$.  Furthermore, this ``fanning out'' can continue more than one
level deep in the credential chain, in that authority over any role
$B_i.r_i$ may in turn be delegated to an arbitrary number of
principals, and so on.

However, forms of width control can be obtained by appropriate
instantiations of $\RTR$.  In particular, limits can be placed on the
sets of principals to which authority can be delegated for a given
role definition.  Letting $\mathcal{P}$ be the set of principals, we
take the set of risks $\mathcal{K}$ to be the powerset of
$\mathcal{P}$, risk ordering $\po$ to be set containment, and both
forms of aggregation $\setdefn{\riskplus, \isectplus}$ to be set
union.  Credential risks are then defined as the set of subjects in
the credential, so that all credentials adhere to the following
schema:
$$
B.s \xleftarrow{\subjects(f)} f
$$

Now, suppose that $A$ wished to specify that only principals in the
set $\setdefn{B,C,D}$ should be allowed any sort of authority over the
role $A.r$.  In this case a threshold $\thresh$ would be defined such
that $\thresh(A.r) = \setdefn{B,C,D}$, and suppose that $\thresh$ maps
all other roles to $\mathcal{P}$ for the purposes of the example.
Hence, given the following set of credentials:
\begin{mathpar}
\cred{A.r}{B.s}{\setdefn{B}}

\cred{B.s}{C.q}{\setdefn{C}}

\cred{B.s}{E.q}{\setdefn{E}}

\cred{C.q}{E}{\varnothing}

\cred{E.q}{D}{\varnothing}
\end{mathpar}
The $\thresh$-constrained solution is as follows:
\begin{eqnarray*}
E.q &:& \setdefn{(D,\varnothing)}\\
C.q &:& \setdefn{(E,\varnothing)}\\
B.s &:& \setdefn{(D,\setdefn{E}),(E,\setdefn{C})}\\
A.r &:& \setdefn{(E,\setdefn{B,C})}
\end{eqnarray*}
Note that $D$ cannot be established as a member of $A.r$, since it can
be so only under the authority of $E$ which is disallowed by the
width threshold for $A.r$.
