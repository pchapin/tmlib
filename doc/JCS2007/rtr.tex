\section{The System $\RTR$}
\label{section-rtr}

The system $\RTR$ is $\RT_0$ extended with a formal definition of risk
assessments.  In this section we define and characterize the syntax
and semantics of $\RTR$.  We define a set theoretic semantics for
$\RTR$, since this allows an easy correspondence with the graph
theoretic characterization of $\RTR$ for distributed chain discovery
given in the next section.  Unlike $\RT_0$, the semantics of $\RTR$ not
only takes into account role membership, but the risk associated with
role membership.  This semantic representation allows formal 
discussion of \emph{thresholds}, the specification of tolerable 
risk levels for role membership on a per-role basis.

\subsection{Syntax and Semantics}

The system $\RTR$ is defined as a framework, parameterized by a
\emph{risk ordering}, which is required to be a complete lattice
$(\mathcal{K},\po)$.  We let $\risk$ and $\risks$ range over elements
and subsets of $\mathcal{K}$ respectively, and let $\top$ and $\bot$
denote the top and bottom elements of the lattice.  Any instantiation
of $\RTR$ is expected to specify a set of lattice elements and a
decidable risk ordering $\po$.  Intuitively, $\po$ allows comparison
between greater and lesser risk, e.g.~if one credential is more
trusted than another, or less computationally expensive to retrieve.
Also, operations for \emph{aggregating} risks-- that is, combining
risks in proofs of authorization-- must be specified.  The system
provides two forms of aggregation, risk aggregation and intersection
aggregation.  Each of these are total functions from pairs of risks to
risks.  The latter is provided for combining risks in intersection
role definitions, while the former handles all other forms of
aggregation.  We believe this distinction should be made, because
intersection roles can represent agreement between principals, which
in practice may reduce risk.  Also, this distinction is necessary to
encode certain features such as delegation depth as observed in
\autoref{section-examples}.  To ensure termination and flexibility of
our chain discovery algorithm we require that both forms of
aggregation be monotonic and associative.
\begin{definition}
An \emph{instance of $\RTR$} is obtained by defining a complete
lattice $(\mathcal{K},\po)$, where $\mathcal{K}$ is a set of
\emph{risks} and $\po$ is a decidable \emph{risk order relation}.  We
let $\risk$ and $\risks$ range over elements and subsets of
$\mathcal{K}$ respectively, and let $\top$ and $\bot$ denote the top
and bottom elements of the lattice.  The instantiation also includes
two associative and monotonic \emph{aggregation}
operators: \emph{risk aggregation} $\riskplus$,
%\emph{linking aggregation} $\linkplus$, 
and \emph{intersection aggregation} $\isectplus$.
\end{definition}
As an example, we can instantiate $\RTR$ with the usual $\le$ 
ordering on natural numbers plus $\omega$ as an upper bound,
and specify that addition is the sole aggregation operation.
\begin{example}
\label{example-ponat}
Let $\ponat \defeq (\nat \cup \setdefn{\omega}, \le)$ where $\le$ is
the usual relation on natural numbers extended such that $n \le
\omega$ for all $n \in \nat$.  Let $\riskplus = \isectplus = +$,
with $+$ extended such that $n + \omega = \omega$ for all $n \in \nat \cup
\setdefn{\omega}$.  
These definitions together specify an instance of $\RTR$.
\end{example}

The basis of risk assessment is the association of risk with
individual credentials, since credentials are the fundamental
assertions used in authorization decisions.  Thus, credentials in
$\RTR$ are of the form $\cred{A.r}{f}{\risk}$ where $\risk$ is the
risk associated with the credential.  The manner in which risks are
assigned to credentials is left unspecified, but we discuss some
concrete possibilities in \autoref{section-risk-assignment}.  In
essence, the aggregation of risks associated with credentials used in
an authorization decision constitutes the risk of that decision.
\begin{definition}
Given an instance of $\RTR$, we let $\creds$ range over finite sets of
\emph{credentials}, and $c$ range over individual credentials of the
form $\cred{A.r}{f}{\risk}$.  We refer to credential forms type 1
through 4 as in \autoref{section-rt}.
\end{definition}

Formally, the semantics of $\RTR$ associates risk $\risk$ with the
membership of entities $B$ in roles $A.r$.  Thus, the meaning of roles
$A.r$ are finite sets of pairs of the form $(B,\risk)$, called
$\mathit{RiskAssessment}$s.  Note that any $\mathit{RiskAssessment}$
may associate more than one risk with any entity, i.e.~there may exist
$(A,\risk_1),(A,\risk_2) \in R$ such that $\risk_1 \ne \risk_2$.  This
reflects the possibility of more that one path to role membership,
each associated with incomparable risk.  Taking the glb of
incomparable risks in risk assessments as a semantic basis of $\RTR$
is unsound, since the glb will assess a lesser risk of membership than
is in fact possible to obtain through any path.

\begin{definition}
\label{def-risk-assessment}
Sets of type $\mathit{RiskAssessment}$, denoted $R$, are finite
collections of pairs of the form $(B,\risk)$. For any
$\mathcal{A}\subseteq \mathit{Entities}$, the type
$\mathit{RiskAssessment}(\mathcal{A})$ denotes the set of risk
assessments $R$ such that $(A,\risk) \in R$ implies $A \in
\mathcal{A}$.
\end{definition}

Of course, if a risk assessment associates two distinct but comparable
risks with a given role membership, the lesser of the two can be taken
as representative.  In other words, risk assessments can be taken as a
set of lower bound constraints on risk in authorization.  Not only is
this idea logically appealing, but it also ensures that the semantics
of $\RTR$ is constructive in the presence of cyclic credentials.
Otherwise, if every path with different risk were explicitly
represented, the monotonic aggregation of risk along a cycle could
generate an infinitely large set of increasingly expensive risk
assessments for a given role membership.  In
\autoref{section-rtr-discovery} we show that this semantic
specification does not disallow the use in practice of a
higher-than-minimal authorization path.  We define an equivalence
relation to capture the idea, inducing equivalence classes identified
by \emph{canonical} risk assessments containing no $(A,\risk_1)$ and
$(A,\risk_2)$ where $\risk_1 \po \risk_2$.  Furthermore, the canonical
representation of any assessment $R$, denoted $\hat{R}$, is decidable
since assessments are finite and $\po$ is decidable.
\begin{definition}
Equivalence classes of risk assessments are obtained by the 
following axiom schema:
$$
R \cup \setdefn{(A,\risk_1),(A,\risk_2)} = R \cup \setdefn{(A,\risk_1)} 
\qquad \text{where } \risk_1 \po \risk_2 
$$ 
We call \emph{canonical} those risk assessments $R$ where there
exist no $(A,\risk_1), (A,\risk_2) \in R$ such that $\risk_1 \po \risk_2$,
and observe that any equivalence class of risk assessments has 
a unique canonical form.  We extend the 
ordering $\po$ to risk assessments as follows:
\begin{eqnarray*}
R_1 \po R_2 &\iff& \forall (A,\risk_1) \in \hat{R_1} . \exists
\risk_2.  (A,\risk_2) \in \hat{R_2} \wedge \risk_1 \po \risk_2
\end{eqnarray*}
\end{definition}
Hereafter we restrict our consideration to canonical risk assessments
without loss of generality.  

The semantic definition is defined via several auxiliary operations.
Canonical risk assessments are combined by taking their canonical
union:
\begin{definition}
The \emph{canonical union} of risk assessments is denoted $\uplus$, 
i.e.~define $R_1 \uplus R_2 = \hat{R}\text{, where } R = R_1 \cup R_2$.
\end{definition}
Entire risk assessments may be aggregated together, or incremented by
some risk.  We therefore extend aggregation to risk assessments for
notational convenience as follows.  Note that these operations 
preserve canonical form of risk assessments.
\begin{definition}
\label{def-aggregation-risk-assessments}
Letting $\star$ range over $\setdefn{\riskplus, \isectplus}$,
aggregation is extended to risk assessments as follows:
\begin{eqnarray*}
R \star \risk &\ \defeq\ & \hat{R'}\text{ where } R' = \setdefn{(A,\risk' \star \risk) \ \mid\ (A,\risk') \in R}\\
R_1 \star R_2 &\ \defeq\ &\hat{R'}\text{ where } R' = 
\setdefn{(A,\risk_1 \star \risk_2) \ \mid\ 
(A,\risk_1), (A,\risk_2) \in R_1 \times R_2} 
\end{eqnarray*}
\end{definition}

Now we can specify the semantics of $\RTR$, via interpretations that
map roles to risk assessments.  
\begin{definition}
A \emph{role interpretation} is a total function of type:$$
\mathit{Role} \rightarrow \mathit{RiskAssessment}
$$ 
Letting $f$ and $g$ be role interpretations, define:
$$
f \po g \quad\iff\quad f(A.r) \po g(A.r) \text{ for all roles } A.r
$$
\end{definition}
An interpretation is taken to be a solution to a set of credentials if
it maps each role to the ``right'' risk assessment, given the intended
meaning of the given credentials.  This meaning is characterized by
the functionals $\bounds$ and $\expr$, defined in
\autoref{figure-rmem}, and the following definition.

\begin{definition}[Semantics of $\RTR$] 
\label{def-solution}
Given a set $\creds$ of $\RTR$ credentials, the \emph{solution
$\credsmean{\creds}$ of $\creds$} is the least role interpretation
$\rmem$ such that $\bounds[\rmem] \po \rmem$, where $\bounds$
and the auxiliary function $\expr$, mapping interpretations to 
interpretations, are defined in \autoref{figure-rmem}.
\end{definition}

We discuss several extended examples of the system in
\autoref{section-examples}.  Here is a brief example illustrating the
semantics.
\begin{example}
\label{example-solution}
Assume given the instance of $\RTR$ as defined in \autoref{example-ponat},
and let $\creds$ consist of the following:
\begin{mathpar}
\cred{A.r_0}{B.r_3}{2}

\cred{A.r_0}{C.r_1.r_2}{1}

\cred{C.r_1}{D}{3}

\cred{B.r_3}{E}{4}

\cred{D.r_2}{F}{0}
\end{mathpar}
Then $\credsmean{\creds}$ is the interpretation mapping every
role to $\varnothing$, except:
\begin{mathpar}
\credsmean{\creds}(C.r_1) = \setdefn{(D,3)}

\credsmean{\creds}(B.r_3) = \setdefn{(E,4)}

\credsmean{\creds}(D.r_2) = \setdefn{(F,0)}

\credsmean{\creds}(A.r_0) = \setdefn{(E,6), (F,4)}
\end{mathpar}
\end{example}

\rmemfig

\subsection{Existence of a Solution}
\label{section-rtr-solution}

The semantics of $\RTR$ just defined is meaningful only if any
credential set $\creds$ has a solution.  We establish this by an
inductive construction that converges to a fixpoint, obtained by
iterating $\bounds$ over an initial empty risk assessment.  
\begin{definition}
A \emph{partial role interpretation over
$\creds$} is a function of type: 
$$\mathit{Roles}(\creds) \rightarrow
\mathit{RiskAssessment}(\mathit{Entities}(\creds)))$$ 
The family of
partial role interpretations over $\creds$ denoted
$\setdefn{\rmem_i}_{i \in \mathbb{N}}$ is defined inductively by
taking $\rmem_0(A.r) = \varnothing$ for every role $A.r$, and letting:
$$
\rmem_{i+1}(A.r) = \bounds[\rmem_i](A.r)
$$ 
for every $A.r$.  
\end{definition}
The argument for existence of a solution proceeds by showing that this
construction converges to a fixpoint, which is clearly a solution of
$\creds$.  

To show that the construction converges to a fixpoint, we observe that
$\bounds$ is monotonic over a lattice of partial role
interpretations over given $\creds$.  The lattice is induced by a
relation that is similar to $\po$, but with an important difference.
In each iteration, the solution is built up by adding
new elements $(A,\risk)$ to the interpretations of roles.  However,
the use of canonical union ensures any such element is added only if
$\risk$ is lesser than or incomparable with existing elements of the
solution, rather than greater than as would be possible if $\bounds$
were monotonic in $\po$.  Hence, we define an appropriate ordering
denoted $\rmempo$:
\begin{definition}
Define $\rmempo$ as a relation on risk assessments:
\begin{eqnarray*}
R_1 \rmempo R_2 &\iff& \forall (A,\risk_1) \in \hat{R_1} . \exists
\risk_2.  (A,\risk_2) \in \hat{R_2} \wedge \risk_2 \po \risk_1
\end{eqnarray*}
The relation is extended pointwise to partial role interpretations,
i.e.~given that $f$ and $g$ are partial role interpretations, define $f
\rmempo g \iff \forall A.r \in \dom(f) . f(A.r) \rmempo g(A.r)$.
\end{definition}
We observe that $\rmempo$ is a partial order, since $\po$ is:
\begin{lemma}
The relation $\rmempo$ is a partial order on both risk assessments and
partial role interpretations.
\end{lemma} 
It is essential to show that $\bounds$ is monotonic over this ordering
of partial role interpretations.  The property is immediate, since 
both $\bounds$ and $\expr$ are defined via operations that preserve
canonical forms when combining risk assessments, i.e.~$\uplus$, 
and $\riskplus$ and $\isectplus$.
\begin{lemma}
The function $\bounds$ is monotonic in $\rmempo$ over
partial role interpretations.
\end{lemma}

Now we show that $\rmempo$ induces a complete lattice structure 
on risk assessments and partial role interpretations.  This allows
us to assert the existence of a solution, since this result
and monotonicity of $\bounds$ in $\rmempo$ implies that 
$\setdefn{\rmem_i}_{i \in \mathbb{N}}$ converges to a fixpoint.
\begin{lemma}
\label{lemma-assessmentlattice}
Given finite $\creds$, the posets
\begin{eqnarray*}
&(\mathit{RiskAssessment}(\mathit{Entities}(\creds)),\rmempo) \\
& \text{and} \\
&(\mathit{Roles}(\creds) \rightarrow
\mathit{RiskAssessment}(\mathit{Entities}(\creds)),\rmempo)
\end{eqnarray*}
are complete lattices. 
\end{lemma}
\begin{proof}
We begin by showing that the first poset is a complete lattice. 
Given $\mathcal{A} = \mathit{Entities}(\creds)$ and $\mathcal{R} \subseteq
\mathit{RiskAssessment}(\mathcal{A})$.  For each entity $A \in \mathcal{A}$
let $\risks_A = \setdefn{\risk\ |\ \exists R \in \mathcal{R}
. (A,\risk) \in R}$
%, let: 
%$$\risks_A =
%\setdefn{\risk\ |\ \exists R \in \mathcal{R} . (A,\risk) \in R}$$ 
and let $\risk_A$ be the glb of $\risks_A$, which must exist since 
we require risk orderings to be complete lattices.  Let
$R_{\mathcal{R}} = \setdefn{(A,\risk_A) \mid A \in \mathcal{A}}$.
Clearly $R_{\mathcal{R}}$  is an element of 
$\mathit{RiskAssessment}(\mathcal{A})$,
and is a lub of $\mathcal{R}$.  The existence of a glb for 
$\mathcal{R}$ follows dually.  The second poset is clearly also 
a complete lattice, since the ordering in that poset is just
the pointwise extension of $\rmempo$ for risk assessments.
\end{proof}

Of course, a remaining issue is whether $\setdefn{\rmem_i}_{i \in
\mathbb{N}}$ converges to a fixpoint in a \emph{finite} number of
iterations.  This is obviously true if we restrict our consideration
to finite risk domains $\mathcal{K}$, since any set of partial role
interpretations forms a finite lattice under $\rmempo$ by the
preceding result.  In case $\mathcal{K}$ is infinite, the situation is
complicated somewhat.  However, any finite set of credentials $\creds$
can only be combined in a finite number of ways to obtain role
membership, yielding a finite number of possible membership risks,
unless credentials are cyclic, but encountering cycles only obtains
increased risk of existing memberships due to monotonicity of risk
aggregation.  These increased risks will be ignored by $\bounds$ since
it preserves canonical forms of risk assessments.  Hence, the lattice
of partial role interpretations relevant to a sequence
$\setdefn{\rmem_i}_{i \in \mathbb{N}}$ is effectively finite, even if
$\mathcal{K}$ is infinite.  The details of the argument are mostly
tedious and are omitted here for brevity.  We assert:
\begin{lemma}
Given finite $\creds$, there exists finite $n$ such that $\rmem_n$ is 
a fixpoint of $\bounds$.
\end{lemma}
A solution is than effectively constructed from this fixpoint.
Specifically, given finite $\creds$, let $\rmem_n$ be the
least fixpoint of the sequence $\setdefn{\rmem_i}_{i \in \mathbb{N}}$, 
and define:
$$
\begin{array}{rclr}
\credsmean{\creds}(A.r) &=& \rmem_\omega(A.r) \qquad &  A.r \in \mathit{Roles}(\creds) \\
\credsmean{\creds}(A.r) &=& \varnothing & A.r \not\in \mathit{Roles}(\creds)
\end{array}
$$

\subsection{Thresholds and Constrained Solutions}

In an authorization setting that takes risk into account, acceptable
levels of risk should be specifiable as a component of policy.
Furthermore, these levels should be specifiable on a per-role basis,
to allow a fine-grained view of risk tolerance.  We believe this level
of flexibility is desirable, since it allows the disparate entities
maintaining particular roles to assign different levels of risk
tolerance, or allows a local authorizer to associate different levels
of risk with different entities, both of which are likely scenarios.
To formalize this in our model, we introduce thresholds $\thresh$,
which are mappings from roles to risks.    
\begin{definition}
A \emph{threshold} $\thresh$ is a total function of type:
$$
\mathit{Role} \rightarrow \mathcal{K}
$$ 
We write $\thresh_\top$ to denote the threshold such that 
$\thresh_\top(A.r) = \top$ for all $A.r$, and we write
$\shadowmap{\thresh}{A.r}{\risk}$ to denote $\thresh'$
such that $\thresh'(A.r) = \risk$, and 
$\thresh'(B.r') = \thresh(B.r')$ for all $B.r' \ne A.r$. 
\end{definition}

Incorporating thresholds into the model is not simply a matter of
eliminating those elements of role solutions that exceed the given
threshold.  The problem is that some role memberships that are not
explicitly constrained by a threshold may nevertheless be eliminated
by it, due to a dependence on other constrained role memberships; see
\autoref{example-threshold-solution} below.  Therefore, we make a slight
modification to $\bounds$, to take into account threshold constraints.
\begin{definition}
Given some $\thresh$, the \emph{$\thresh$-constrained solution
$\credsmean{\creds}^\thresh$ of $\creds$} is the least role
interpretation $\rmem$ such that: 
$$\bounds_\thresh[\rmem] \po \rmem$$
where for all $A.r$: 
\begin{eqnarray*}
& \bounds_\thresh[\rmem](A.r) \\
& = \\
& \setdefn{(B,\risk)\ \mid\ (B,\risk) \in \bounds[\rmem](A.r) \text{\ and\ } \risk \po \thresh(A.r)}
\end{eqnarray*}
\end{definition}
The existence of threshold constrained solutions is established by 
nearly the same argument as that given in \autoref{section-rtr-solution}.
\begin{lemma}
$\credsmean{\creds}^\thresh$ exists for arbitrary $\thresh$ and $\creds$.
\end{lemma}

Here is a brief example illustrating threshold constrained solutions,
highlighting non-local effects of constraints on role membership.
Note that even though all members of $A.r_0$ in the full solution are
below the given threshold, one of them will be eliminated in the
threshold constrained solution, due to its dependence on the role
$B.r_3$.  That is, since $(E,4)$ is eliminated from $B.r_3$'s solution
by the given threshold constraint, $(E,6)$ cannot be established
as part of $A.r_0$'s solution.  
\begin{example}
\label{example-threshold-solution}
Assume given the instance of $\RTR$ and credential set $\creds$
specified in \autoref{example-solution}.  Define:
$$
\thresh = \shadowmap{\shadowmap{\thresh_\top}{A.r_0}{10}}{B.r_3}{3}
$$ 
Then $\credsmean{\creds}^\thresh$ is the interpretation mapping
every role to $\varnothing$, except:
\begin{mathpar}
\credsmean{\creds}^\thresh(C.r_1) = \setdefn{(D,3)}

\credsmean{\creds}^\thresh(D.r_2) = \setdefn{(F,0)}

\credsmean{\creds}^\thresh(A.r_0) = \setdefn{(F,4)}
\end{mathpar}
\end{example}


