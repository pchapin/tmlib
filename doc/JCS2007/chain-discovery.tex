\section{$\RTR$ Credential Chain Discovery}
\label{section-chain-discovery}

In this section we discuss an algorithm for authorization with risk in a
distributed environment.  As in $\RT$ chain discovery,
\cite{Li:2003-02}, a central issue is that credentials need not
be stored locally, but may need to be retrieved from remote machines.
Furthermore, modulation of tolerable risk is intended to be a central
feature of $\RTR$; online authorization need not be risk-optimal, but
risk should be kept below a certain threshold.  This allows tolerable
levels of risk to be specified and enforced.  This flexibility will be
implemented by viewing authorization decisions as a reachability
analysis in a \emph{credential graph}, and by defining chain discovery
as a partial graph reconstruction algorithm parameterized by 
a \emph{maximum risk threshold}.

\subsection{Credential Graphs}

We begin by defining an interpretation of credential sets $\creds$ as
a credential graph.  More precisely, sets of credentials are
interpreted as a weighted multigraph, where nodes are role
expressions, edges are credentials, and weights are risks.
Authorization is implemented by determining reachability, via
\emph{risk weighted paths}, where the aggregation of edge risk along
the path is the risk of authorization.  Reachability is predicated on
simple paths, since traversing cycles can only increase risk, and any
path with a cycle would otherwise generate an infinite number of risk
weighted paths.  Allowing the latter would preclude a constructive 
definition of credential graphs, as will become clear below.
\begin{definition}[Risk weighted paths]
Let $\graph{} = (\nodes{}, \wtedges{})$ be a weighted multigraph with
nodes $f \in \nodes{}$ and edges $\wtedge{f_1}{f_2}{\risk} \in
\wtedges{}$ weighted by elements $\risk$ of a given risk ordering.
The pair:
$$((f_1,\ldots,f_n), \risk_1 \oplus \cdots \oplus \risk_{n-1})$$ is a
\emph{risk weighted path in $\graph{}$} iff for all $i \in [1..n-1]$,
there exists $\wtedge{f_i}{f_{i+1}}{\risk_i} \in \wtedges{}$.  A
weighted path $((f_1,\ldots,f_n), \risk)$ is \emph{simple} iff no node
is repeated in $(f_1,\ldots,f_n)$.  We write $\wtpath{f}{f'}{\risk}$,
pronounced ``$f'$ is reachable from $f$ with risk $\risk$'',
iff $((f,\ldots,f'), \risk)$ is a simple risk weighted path.  We
write $\wtpath{f}{f'}{\risk} \in \graph{}$ iff $\wtpath{f}{f'}{\risk}$
holds given $\graph{}$.
\end{definition}

Now, the definition of credential graphs is founded on the definition of
risk weighted paths, since edges derived from linked and intersection
credentials are supported by them.  
\begin{definition}[Credential graph]
\label{def-credentialgraph}
Given $\creds$, its \emph{credential graph} is a weighted multigraph
$\graph{\creds} = (\nodes{\creds}, \wtedges{\creds})$, where:
$$
\nodes{\creds} = \bigcup\limits_{\cred{A.r}{e}{\risk}} \setdefn{A.r,e}
$$
And $\wtedges{\creds}$ is the least set of risk-weighted edges satisfying 
the following closure properties:
\begin{enumerate}
\item If $\cred{A.r}{e}{\risk} \in \creds$ then 
$\wtedge{e}{A.r}{\risk} \in \wtedges{\creds}$.
\item If $B.r_2, A.r_1.r_2 \in \nodes{\creds}$ and
$\wtpath{B}{A.r_1}{\risk}$ is simple, then
$\wtedge{B.r_2}{A.r_1.r_2}{\risk} \in \wtedges{\creds}$.
\item If $D,f_1\cap \cdots \cap f_n \in \nodes{\creds}$ and for each
$i \in [1..n]$ there exists a simple path
$\wtpath{D}{f_i}{\risk_i}$, then $\wtedge{D}{f_1 \cap \cdots \cap
f_n}{\risk} \in \wtedges{\creds}$, where $\risk = \risk_1 \oplus
\cdots \oplus \risk_n$.
\end{enumerate}
\end{definition}

The definition of credential graphs can be made constructive by
iterating closure over an initial initial edge set
$\wtedges{\creds}^0$:
$$
\wtedges{\creds}^0 = \setdefn{\wtedge{A.r}{e}{\risk} \mid 
\cred{A.r}{e}{\risk} \in \creds}
$$ 
In rules (2) and (3), the paths predicating membership in
$\wtedges{\creds}$ are called \emph{support paths}, and the edges are
called \emph{derived}.  On each iteration, add a new weighted edge
according to closure rule (2) or (3).  Since $\creds$ is finite, and
support paths must be simple, the process will reach a fixpoint in
a finite number of iterations; this fixpoint is $\wtedges{\creds}$.

We observe that the characterization of credential sets $\creds$
is sound and complete with respect to the set theoretic semantics
given in the previous section.  These results will form a 
bridge with the semantics of $\RTR$ for establishing correctness
of credential chain discovery.  The statement of soundness reflects
the fact that while risk assessments of credential sets express
minimum risk bounds of role membership, the credential graph 
does not preclude reachability via paths of higher risk.
\begin{theorem}[Soundness]
For all $B,A.r$, if $\wtpath{B}{A.r}{\risk} \in \graph{\creds}$, then
$(B,\risk') \in \credsmean{\creds}(A.r)$ with $\risk' \po \risk$.
\end{theorem}
The statement of completeness reflects that any assessed risk is the
weight of some related path in the graph:
\begin{theorem}[Completeness]
For all $A.r$, if $(B,\risk) \in \credsmean{\creds}(A.r)$, then 
$\wtpath{B}{A.r}{\risk} \in \graph{\creds}$.
\end{theorem}

\subsection{Backward Chain Discovery Algorithm}

The credential chain discovery algorithm is a modification of the
backward search algorithm in \cite{Li:2003-02}. For a particular risk
lattice $\mathcal{K}$ and risk aggregration operation $\oplus$, the
algorithm takes a collection of credentials $\mathcal{C}$, the name of a
governing role, and a maximum risk $\risk_M \in \mathcal{K}$. The
algorithm returns a risk assessment for the governing role where each
entity in the assessment is associated with a risk $\risk \po \risk_M$.

The algorithm maintains a set of credential graph nodes $N$ and edges
$E$, adding nodes and edges to these sets as it works. The algorithm
also maintains a queue of nodes $Q$ that are waiting to be processed.
When this queue is exhausted the algorithm terminates. Associated with
each node in $N$ is a set of aggregated risk records (ARRs) of the form
$(n_d, n_s, \risk_A)$. The driving node $n_d$ contains the name of the
node that caused the ARR to be initially created. The support node $n_s$
is used when handling intersection roles. The aggregated risk $\risk_A$
reflects the risk accumulated from the point where the ARR is first
created. In what follows we use $N(e)$ to denote the set of ARRs in node
$e$.

We define a canonical representation for sets of ARRs $\mathcal{A}$
analogously to risk assessments. $\mathcal{A}$ is in canonical form if
there does not exist $(n_d, n_s, \risk_1), (n_d, n_s, \risk_2) \in
\mathcal{A}$ such that $\risk_1 \po \risk_2$. In addition we extend the
definition of $\oplus$ to include sets of ARRs
$$
\mathcal{A} \oplus \risk' = \{(n_d, n_s, \risk \oplus \risk')\,|\, (n_d,
  n_s, \risk) \in \mathcal{A}\}
$$

It is convenient to define an operation $\mathit{update}(\mathcal{A},
\risk, e_2)$ that describes how a set of ARRs $\mathcal{A}$ are copied
to a node $e_2$ over an edge in the credential graph with risk $\risk$.
The $\mathit{update}$ function first adds a node for $e_2$ to the graph
if one does not already exist. It then computes
$$
A' = \mathit{canonical}(\mathit{bound}(\mathcal{A} \oplus \risk) \cup N(e_2))
$$
where $\mathit{canonical}$ is a function that returns a set of ARRs that
is the canonical form of the set given as its argument, and where
$\mathit{bound}$ is a function that removes all ARRs $(n_d, n_s, \risk)$
from the given set where $\risk \po \risk_M$ is false. If $N(e_2) \ne
A'$ then $\mathit{update}$ sets $N(e_2) = A'$ and queues $e_2$ for
exploration.

The algorithm processes each type of node, except for entity nodes, as
follows
\begin{description}
\item[Role $A.r$] Look up all edges $\cred{A.r}{e}{\risk}$. For each
  such edge, $\mathit{update}(N(A.r), \risk, e)$.
\item[Linked role $A.s.t$] Look up all edges $\cred{A.s.t}{e}{\risk}$.
  For each such edge, $\mathit{update}(N(A.r.s), \risk\ e)$. In addition
  $\mathit{update}((A.s.t, A.s, \bot), \bot, A.s)$.
\item[Intersection role $f_1 \cap \cdots \cap f_n$] Look up all edges
  $\cred{f_1 \cap \cdots \cap f_n}{e}{\risk}$. For each such edge,
  $\mathit{update}(N(f_1 \cap \cdots \cap f_n), \risk, e)$. In addition,
  for each $f_i$ participating in the intersection $\mathit{update}((f_1
  \cap \cdots \cap f_n, f_i, \bot), \bot, f_i)$.
\end{description}

These rules describe how ARRs propagate through the credential graph.
The algorithm starts by creating and queuing a node for the governing
role $g$ and assigning to that node the ARR $(g, g, \bot)$. As this ARR
moves through the graph it accumulates risk information about the
various paths to $g$. If two paths converge, the node at the junction of
the paths will get re-explored if the second set of incoming ARRs cause
accumulated risks at that node to change.

Maintaining the canonical form of the ARR sets in each node keeps the
algorithm from diverging when there are cycles in the credential graph.
When the edge $\cred{e_1}{e_2}{\risk}$ that closes the cycle is
processed, the $\mathit{update}$ operation will not modify $N(e_2)$
since all ARRs in $N(e_1) \oplus \risk$ will have risk values greater
than or equal to those in $N(e_2)$. Thus there will be no change to the
canonical form of $N(e_2)$ and $e_2$ will not be re-explored.

If at any time during the search the total aggregated risk $\risk$ in an
ARR is such that $\risk \po \risk_M$ is false, that ARR no longer
propagates and the search in that direction is abandoned. Thus the
algorithm does not bother exploring portions of the credential graph
that have too much aggregated risk from the governing role.

Entity nodes must be processed in a special way so that the derived
edges they imply are properly handled. The operation
$\mathit{processEntity}$ loops over all ARRs in the entity node $E$ and
for each ARR $(n_d, n_s, \risk)$ considers one of three cases
\begin{description}
\item[$n_d$ is a plain role $A.r$] In this case, $A.r$ must be the
  governing role and $E$ can reach that node with an aggregated risk
  $\risk$. The algorithm records this discovery.
\item[$n_d$ is a linked role $A.s.t$] In this case $E$ can reach $A.s$
  with risk $\risk$. The algorithm adds the edge
  $\cred{A.s.t}{E.t}{\risk}$ to the edge set and queues $A.s.t$ for
  re-exploration.
\item[$n_d$ is an intersection role $f_1 \cap \cdots \cap f_n$] In this
  case $E$ can reach $n_s$ with risk $\risk$. The algorithm associates
  this information with the pair $(n_d, E)$. If all roles mentioned in
  $n_d$ have been associated with $(n_d, E)$ in this way, the algorithm
  adds the edge $\cred{f_1 \cap \cdots \cap f_n}{E}{\risk}$ to the edge
  set and queues $f_1 \cap \cdots \cap f_n$ for re-exploration.
\end{description}

To illustrate the action of the algorithm consider the sum-of-risks
example in Section \ref{sec:sum-of-risks}. In that example $\mathit{Ed}$
is in the \role{Store}{buyer} role both with risk 8 and through the
\role{Personnel}{manager} role with risk 9. If $\risk_M = 8$, the
algorithm returns a risk assessment for the \role{Store}{buyer} role of
$\{(\mathit{Ed}, 8)\}$. The path with risk 9 is not fully explored
because the total aggregated risk on that path exceeds $\risk_M$. The
algorithm processes this example as follows.
\begin{enumerate}

\item The node for \role{Store}{buyer} is created an initialized with an
  ARR of $(S.b, S.b, \bot)$.

\item When \role{Store}{buyer} is processed, a node is created for
  \role{Acme}{purchaser} $\cap$ \role{Acme}{employee} and the ARR in
  \role{Store}{buyer} is copied to the new node and updated to $(S.b,
  S.b, 1)$.

\item When \role{Acme}{purchaser} $\cap$ \role{Acme}{employee} is
  processed, there are no edges yet defining it. However, nodes for
  \role{Acme}{purchaser} and \role{Acme}{employee} are created and
  initialized with $(A.p \cap A.e, A.p, \bot)$ and $(A.p \cap A.e, A.e,
  \bot)$ respectively.

\item When \role{Acme}{purchaser} is processed, nodes for
  \role{Personnel}{manager} and $\mathit{Ed}$ are created. The ARRs in
  \role{Acme}{purchaser} are copied to these new nodes and updated
  resulting in $(A.p \cap A.e, A.p, 2) \in N(P.m)$ and $(A.p \cap A.e,
  A.p, 4) \in N(E)$.

\item When \role{Acme}{employee} is processed, the ARR $(A.p \cap A.e,
  A.e, 3)$ is added to $N(E)$

\item When \role{Personnel}{manager} is processed, the ARR $(A.p \cap
  A.e, A.p, 5)$ is computed from $N(P.m) \oplus 3$. However, this ARR is
  removed by the canonicalization process and thus is never added to
  $N(E)$.

\item When $\mathit{Ed}$ is processed $\mathit{processEntity}$ loops
  over the ARRs contained in $N(E)$. It finds that the intersection is
  satisfied and thus creates the edge $\cred{A.p \cap A.e}{E}{7}$. It
  then queues node $A.p \cap A.e$ for re-exploration.

\item When \role{Acme}{purchaser} $\cap$ \role{Acme}{employee} is
  processed again the algorithm copies the ARR $(S.b, S.b, \bot)$ to $E$
  and updates it to $(S.b, S.b, 8)$. Since this changes $N(E)$ that node
  is queued for re-exploration. However, \role{Acme}{purchaser} and
  \role{Acme}{employee} are not queued since their ARR sets are not
  modified.

\item When $\mathit{Ed}$ is processed again, the discovery is made that
  $E$ is in $S.b$ with risk 8. The other ARRs in $N(E)$ are also
  processed causing the node for $A.p \cap A.e$ to be re-explored.
  However, none of the canonical ARR sets are modified after that and
  thus the queue drains and the algorithm terminates.
\end{enumerate}

\begin{theorem}
If $\mbox{\texttt{checkmem}}(E, A.r, \risk)$ returns true, then there
exists $(E, \risk') \in \mathcal{S}_\mathcal{C}(A.r)$ such that $\risk'
\po \risk$.
\end{theorem}

\begin{theorem}
For all $(E, \risk) \in \mathcal{S}_\mathcal{C}(A.r)$,
$\mbox{\texttt{checkmem}}(E, A.r, \risk)$ returns true.
\end{theorem}

