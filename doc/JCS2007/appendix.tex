\appendix

\newcommand{\ocamlitem}[2]{\noindent\hspace{6pt}\ttt{#1}: #2\smallskip}

\section{Definition of $\checkmem$}
\label{appendix-checkmem}

In this appendix we describe our OCaml implementation of the
$\checkmem$ algorithm.  As described in the text, the implementation
is in a procedural style, but exploits higher order functions and
types available in OCaml.

\subsection{Type Definitions}
\label{appendix-checkmem-types}

To abstract the definition of risk orderings in the implementation, we
parameterize the implementation by a module \ttt{Risk}, that satisfies
the following type signature \ttt{RISK}.
\begin{verbatimtab}
  module type RISK =
    sig
      type risk
      val lt : risk -> risk -> bool
      val plus : risk -> risk -> risk
      val times : risk -> risk -> risk
      val bot : risk
      val top : risk
    end
\end{verbatimtab}

Any concrete instance \ttt{Risk} of \ttt{RISK} implements the risk
ordering provided in an instance of $RT^R$, insofar as each component
of the module implements a component of a given risk ordering, as
follows.
\medskip

\ocamlitem{Risk.risk}{the set of risk values $\mathcal{K}$.}

\ocamlitem{Risk.lt}{the risk ordering $\po$.}

\ocamlitem{Risk.plus}{risk aggregation $\riskplus$.}

\ocamlitem{Risk.times}{intersection aggregation $\isectplus$.}

\ocamlitem{Risk.bot}{the bottom element $\bot$.}

\ocamlitem{Risk.top}{the top element $\top$.}

\medskip

Roles, role expressions, credentials, risk assessments, and thresholds
are then specified as follows.  Let types \ttt{entity} and
\ttt{role\_name} be arbitrary identifiers with equality operations,
e.g.~\ttt{string}s.  We define:
\begin{verbatimtab}
type role = entity * role_name
type role_expr =     
    Ent of entity                           
  | Role of role                            
  | Link of entity * role_name * role_name  
  | Isect of role_expr list                 
type cred = Cred of role * Risk.risk * role_expr 
type risk_assessment = (entity * Risk.risk) list 
type threshold = role_expr -> Risk.risk 
\end{verbatimtab}


Now we specify types that are specific to the $\checkmem$ 
algorithm: search risks, solution monitors, and search risk
propagaters.  
\begin{verbatimtab}
type search_risk = role * risk
type monitor = (entity * risk) -> unit 
type propagater = (role * risk) -> unit
\end{verbatimtab}
Nodes are specified as records identified by constant role
expressions, with mutable fields for maintaining data structures
relevant to the algorithm, and a visited flag.
\begin{verbatimtab}
type node = { 
    id : role_expr;  
    mutable monitors : monitor list; 
    mutable propagaters : propagater list; 
    mutable search_risks : search_risk list;
    mutable solution : risk_assessment;
    mutable visited : bool
  }
\end{verbatimtab}

\begin{fpfig*}[t]{Search Risk Propagation Functions}{figure-propagation}
\footnotesize
\medskip
\begin{verbatimtab}
  let risk_notify nd (r,k) = 
    if not (subsumed (r,k) nd.search_risks) 
    then 
      begin
        add_search_risks [(r,k)] nd;
        apply_all nd.propagaters (r,k)
      end

  let make_propagater nd k' = (fun (r,k) -> risk_notify nd (r, Risk.plus k k')) 
\end{verbatimtab}
\smallskip
\end{fpfig*}

\subsection{Auxiliary Functions and Data Structures}
\label{appendix-checkmem-auxfns}

The $\checkmem$ algorithm uses two global mutable data structures, 
identified by global variables in our implementation:

\medskip

\ocamlitem{q}{global queue data structure for storing nodes to be 
searched.}

\ocamlitem{nodes}{global list of all nodes created 
during the course of authorization.}

\smallskip



The $\checkmem$ algorithm uses a library of auxiliary functions
whose semantics we describe below.  The definitions are not 
particularly relevant and are omitted for brevity, but we 
describe their argument lists and semantics as follows.   
Recall that in OCaml, argument lists can be written in a
``curried'' style, e.g.~\ttt{f x y} instead of \ttt{f(x,y)}.

\medskip

\ocamlitem{make\_target a r}{sets membership checking of entity \ttt{a} in 
role \ttt{r} as main target of the authorization decision.}

\ocamlitem{issuer\_retrieve r}{retrieves all certificates in 
$\creds$ for which \texttt{r} is the issuer.}

\ocamlitem{add\_monitor m nd}{adds a solution monitor \ttt{m} 
to \ttt{nd.monitors}.}

\ocamlitem{add\_propagater p nd}{adds a search risk propagater \ttt{p} 
to \ttt{nd.propagaters}.}

\ocamlitem{add\_search\_risks risks nd}{adds \ttt{risks} 
to \ttt{nd.search\_risks}.}

\ocamlitem{add\_new\_solution (b,k) nd}{given solution element 
\ttt{(b,k)} that is 
not subsumed by \ttt{nd.solutions}, returns canonical union of 
$\{\ttt{(b,k)}\}$ and \ttt{nd.solutions}.}

\ocamlitem{apply\_all fs x}{apply all functions in the list \ttt{fs}
to \ttt{x}.}

\ocamlitem{apply\_to\_all f l}{apply \ttt{f} to all elements of list \ttt{l}.}

\ocamlitem{aggregate\_all rs k}{transforms search risks or assessment \ttt{rs} 
so that every element \ttt{(x,k')} becomes \ttt{(x,Risk.plus k' k)}}

\ocamlitem{subsumed (x,k) risks}{a predicate that holds iff there exists
\ttt{(x,k')} in \ttt{risks} such that \ttt{Risk.lt k' k}.}

\ocamlitem{produce\_node f}{if \ttt{nodes} contains a node \ttt{nd} 
identified by \ttt{f}, returns \ttt{nd}, otherwise returns an initialized
node identified by \ttt{f}, and adds it to \ttt{nodes}.}

\ocamlitem{enqueue\_unvisited nd}{adds \ttt{nd} to \ttt{q} and
sets the flag \ttt{nd.visited} to true if \ttt{nd.visited} is false, otherwise
a noop.}

\ocamlitem{below\_thresh thresh risks}{a predicate that holds iff 
search risks \ttt{risks} are below threshold \ttt{thresh}
(\autoref{def-below-thresh}).}

\ocamlitem{searchable thresh q}{a predicate that holds iff queue \ttt{q}
contains at least one node whose search risks are below threshold
\ttt{thresh}.}

\ocamlitem{getnext thresh q}{gets the next searchable node from queue
\ttt{q}.}

\begin{fpfig*}[t]{Solution Monitor Functions}{figure-monitors}
\footnotesize
\medskip
\begin{verbatimtab}
  let soln_notify nd (b,k) thresh = 
    if (Risk.lt k (thresh nd.id)) && not (subsumed (b,k) nd.solution)
    then
      begin
        add_new_solution (b,k) nd;
        apply_all nd.monitors (b,k)
      end
        
  let make_nmonitor nd k thresh = 
    (fun (b,k') -> soln_notify nd (b, Risk.plus k k') thresh)
  
  let make_lmonitor nd thresh =
    match nd.id with
      Link(a,r1,r2) -> 
        (fun (b,k) -> 
  	  let m = make_nmonitor nd k thresh in
    	  let nd' = produce_node (Role(b,r2)) in
            begin
  	      apply_to_all m nd'.solution;
  	      add_monitor m nd';
  	      add_search_risks (aggregate_all nd.search_risks k) nd';
  	      add_propagater (make_propagater nd' k) nd;
  	      enqueue_unvisited nd';
  	   end
        )	   
    | _ -> raise BadNode
\end{verbatimtab}
\smallskip
\end{fpfig*}

\begin{fpfig*}{Node Processing ($\checkmem$) Function Definitions}{figure-checkmem}
\footnotesize
\medskip
\begin{verbatimtab}
  let process_cred nd thresh (Cred(_,k,f)) = 
    let m = make_nmonitor nd k thresh in 
    let nd' = produce_node f in 
    begin
      apply_to_all m nd'.solution;
      add_monitor m nd';
      apply_to_all (risk_notify nd') (aggregate_all nd.search_risks k);
      add_propagater (make_propagater nd' k) nd;
      enqueue_unvisited nd'
    end
  
  let process_isect nd m nd' = 
    begin
      add_monitor m nd';
      add_search_risks nd.search_risks nd';
      add_propagater (make_propagater nd' Risk.bot) nd;
      enqueue_unvisited nd'
    end

  let rec checkmem b r thresh = 
    begin
      Queue.clear q;
      nodes := [];
      make_target b r;
      enqueue_unvisited (produce_node (Role r));
      while (searchable thresh q) do
        let nd = getnext thresh q in 
        match nd.id with 
  	  Ent(a) -> soln_notify nd (a, Risk.bot) thresh
        | Role(r) -> 
  	    begin
  	      add_search_risks [(r,Risk.bot)] nd;
  	      apply_to_all (process_cred nd thresh) (issuer_retrieve r)
  	    end
        | Link(a,r1,r2) -> 
  	    let nd' = produce_node (Role(a,r1)) in 
  	    let m = make_lmonitor nd thresh in
  	    begin
  	      apply_to_all m nd'.solution;
  	      add_monitor m nd';
  	      add_search_risks nd.search_risks nd';
  	      add_propagater (make_propagater nd' Risk.bot) nd;
  	      enqueue_unvisited nd'
  	    end	      
        | Isect(fs) -> 
  	    let m = make_imonitor nd thresh in
  	    let nds = List.map produce_node fs in
  	    apply_to_all (process_isect nd m) nds
      done
    end
\end{verbatimtab}
\smallskip
\end{fpfig*}

\subsection{Propagaters and Role Monitors}
\label{appendix-checkmem-monitors}

To propagate search risks along search paths, we define functions
for notifying nodes of new risks, and for making propagaters for 
a given node and risk.  These functions are defined in 
\autoref{figure-propagation}.

\medskip

\ocamlitem{risk\_notify nd (r,k)}{notifies the node \ttt{nd} to 
add a new search risk \ttt{(r,k)}, and applies all of 
\ttt{nd.propagaters} to \ttt{(r,k)}.}

\ocamlitem{make\_propagater nd k'}{returns search risk propagater for 
node \ttt{nd} and risk \ttt{k'}.}

\smallskip

To propagate solutions along solution paths, we define functions
to notify nodes of solutions, and functions to generate solution
monitors.  Note that node, linking, and intersection monitors
are not defined explicitly, but are returned by higher order 
functions for making them.  All of these functions are defined 
in \autoref{figure-monitors} except for \ttt{make\_imonitor} which
is omitted for brevity.

\medskip

\ocamlitem{soln\_notify nd (b,k) thresh}{notifies node \ttt{nd} to
add a new solution element \ttt{(b,k)}, and applies all of 
\ttt{nd.monitors} to \ttt{(b,k)}.  Raises a \ttt{Solved}
exception if membership of \ttt{b} in 
\ttt{nd.id} is the target of authorization as set by \ttt{make\_target}.}

\ocamlitem{make\_nmonitor nd k thresh}{returns a node monitor for 
node \ttt{nd} and risk \ttt{k}.}

\ocamlitem{make\_lmonitor nd thresh}{returns a linking monitor for 
node \ttt{nd}.}

\ocamlitem{make\_imonitor nd thresh}{returns an intersection monitor for 
node \ttt{nd}.}



\subsection{Node Processing}
\label{appendix-checkmem-processing}

Putting all the pieces together, we can now define the \ttt{checkmem}
function that implements node processing.  To make the code more 
readable, we have factored out some blocks as auxiliary functions.
These functions are defined in \autoref{figure-checkmem}.  The invocation
of \ttt{make\_target} sets the goal of the authorization decision; 
if it is discovered before the queue is empty, an exception is raised
and computation is aborted.

\medskip

\ocamlitem{process\_cred nd thresh c}{does all the work for processing
a credential \ttt{c} that defines a role identifying \ttt{nd} in 
the role case of \ttt{checkmem}.}

\ocamlitem{process\_isect nd m nd'}{given \ttt{nd} identified by 
an intersection role, does most of the work of processing a node
\ttt{nd'} identified by a component of the intersection.}

\ocamlitem{checkmem b r thresh}{implementation of the $\checkmem$ 
algorithm.}




