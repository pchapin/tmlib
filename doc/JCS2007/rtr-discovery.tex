\section{$\RTR$ Distributed Credential Chain Discovery}
\label{section-chain-discovery}
\label{section-rtr-discovery}

In this section we discuss an algorithm for authorization with risk in
a distributed environment, where not all credentials are required to
be known a priori.  Rather, non-local certificates may be retrieved
automatically to establish new credentials if necessary.  Following
$\RT$ credential chain discovery \cite{Li:2003-02}, our technique is
to characterize credential sets graph-theoretically, except that our
credential graphs are risk-weighted multigraphs, to accommodate risk
assessments.  Credential graphs are shown to be a full abstraction of
solutions as in \autoref{def-solution}, and the $\RTR$ discovery
algorithm is shown to correctly reconstruct credential graphs.

In addition to theoretical correctness, our chain discovery algorithm
has two important practical features:
\begin{enumerate}
  \item The algorithm need not verify a role membership in a
  risk-optimal fashion, but rather is parameterized by 
  a threshold, specifying maximum tolerable risks for role 
  memberships.
 \item The discovery procedure is \emph{directed}, in the sense that
it is aborted along search paths whose risk overruns the maximum
threshold.
\end{enumerate}
The first feature allows end-users to modulate tolerable levels of
risk in authorization.  The second feature reaps any efficiency
benefits intended by associating risks with credentials, as high risk
may be associated with high expense, e.g.~if risks are wait times.


\subsection{Credential Graphs}
\label{section-credential-graphs}

We begin by defining an interpretation of credential sets $\creds$ as
a credential graph.  More precisely, a set of credentials is
interpreted as a weighted multigraph, where nodes are role
expressions, edges are credentials, and weights are risks.
Authorization is implemented by determining reachability, via risk
weighted paths, where the aggregation of edge risk along the path is
the risk of authorization.  Reachability is predicated on simple
paths, since traversing cycles can only increase risk due to
monotonicity of risk aggregation, and any path with a cycle would
otherwise generate an infinite number of risk weighted paths.
Allowing the latter would preclude a constructive definition of
credential graphs, since chains are distinguished by risk and cycle
traversal increases risk monotonically.
\begin{definition}[Risk weighted credential chains]
Letting $\graph{} = (\nodes{}, \wtedges{})$ be a multigraph with
nodes $f \in \nodes{}$ and edges $\wtedge{f_1}{f_2}{\risk} \in
\wtedges{}$ weighted by elements $\risk$ of a given risk ordering,
the pair:
$$((f_1,\ldots,f_n), \risk_1 \oplus \cdots \oplus \risk_{n-1})$$ is a
\emph{risk weighted path in $\graph{}$} iff for all $i \in [1..n-1]$,
there exists $\wtedge{f_i}{f_{i+1}}{\risk_i} \in \wtedges{}$.  A
weighted path $((f_1,\ldots,f_n), \risk)$ is \emph{simple} iff no node
is repeated in $(f_1,\ldots,f_n)$.  We write $\wtpath{f}{f'}{\risk}$,
pronounced ``there exists a credential chain from $f$ to $f'$
with risk $\risk$'', iff $((f,\ldots,f'), \risk)$ is a simple risk
weighted path.  We write $\wtpath{f}{f'}{\risk} \in \graph{}$ iff
$\wtpath{f}{f'}{\risk}$ holds given $\graph{}$.
\end{definition}
The ability to isolate different weighted paths between the same nodes
in a graph benefits our larger goals.  In particular, while
credential solutions in the sense defined in \autoref{section-rtr}
explicitly reflect only minimal risks associated with role membership,
the abstraction of paths allows a formal designation of role
memberships with comparable but unequal risk-- given some graph
$\graph{}$, it may be the case that $\wtpath{A}{B.r}{\risk} \in
\graph{}$ and $\wtpath{A}{B.r}{\risk'} \in \graph{}$ where $\risk \po
\risk'$ and $\risk \ne \risk'$.  The ability to establish role
membership with non-minimal risk is an important feature of our
distributed chain discovery algorithm defined below.

The definition of credential graphs
is founded on the definition of risk weighted chains, since edges
derived from linked and intersection credentials are supported by
them.
\begin{definition}[Credential graph]
\label{def-credentialgraph}
Given finite $\creds$, its \emph{credential graph} is a weighted multigraph
$\graph{\creds} = (\nodes{\creds}, \wtedges{\creds})$, where:
$$
\nodes{\creds} = \bigcup\limits_{\cred{A.r}{e}{\risk}} \setdefn{A.r,e}
$$
And $\wtedges{\creds}$ is the least set of risk-weighted edges satisfying 
the following closure properties:
\begin{enumerate}
\item If $\cred{A.r}{e}{\risk} \in \creds$ then 
$\wtedge{e}{A.r}{\risk} \in \wtedges{\creds}$.
\item If $B.r_2, A.r_1.r_2 \in \nodes{\creds}$ and
$\wtpath{B}{A.r_1}{\risk}$, then
$\wtedge{B.r_2}{A.r_1.r_2}{\risk} \in \wtedges{\creds}$.
\item If $D,f_1\cap \cdots \cap f_n \in \nodes{\creds}$ and for each
$i \in [1..n]$ there exists 
$\wtpath{D}{f_i}{\risk_i}$, then $\wtedge{D}{f_1 \cap \cdots \cap
f_n}{\risk} \in \wtedges{\creds}$, where $\risk = \risk_1 \isectplus
\cdots \isectplus \risk_n$.
\end{enumerate}
\end{definition}

The definition of credential graphs can be made constructive by
iterating closure over an initial edge set
$\wtedges{\creds}^0$:
$$
\wtedges{\creds}^0 = \setdefn{\wtedge{A.r}{e}{\risk} \mid 
\cred{A.r}{e}{\risk} \in \creds}
$$ 
In rules (2) and (3), the paths predicating membership in
$\wtedges{\creds}$ are called \emph{support paths}, and the edges are
called \emph{derived}.  On each iteration, add a new weighted edge
according to closure rule (2) or (3).  Since $\creds$ is finite, and
support paths must be simple, the process will reach a fixpoint in
a finite number of iterations; this fixpoint is $\wtedges{\creds}$.

We observe that the characterization of credential sets $\creds$
is sound and complete with respect to the set theoretic semantics
given in the previous section.  These results will form a 
bridge with the semantics of $\RTR$ for establishing correctness
of credential chain discovery.  The statement of soundness reflects
the fact that while risk assessments of credential sets express
minimum risk bounds of role membership, the credential graph 
does not preclude reachability via paths of higher risk.
\begin{theorem}[Soundness]
\label{theorem-graph-soundness}
For all $B,A.r$, if $\wtpath{B}{A.r}{\risk} \in \graph{\creds}$, then
$(B,\risk') \in \credsmean{\creds}(A.r)$ with $\risk' \po \risk$.
\end{theorem}
Following \cite{Li:2003-02}, the result follows by a double induction;
an outer induction on the number of closure iterations to obtain the
graph $\graph{\creds}$, and an inner induction on the length of
the path $\wtpath{B}{A.r}{\risk}$.

The statement of completeness reflects that any assessed risk is the
weight of some related path in the graph:
\begin{theorem}[Completeness]
\label{theorem-graph-completeness}
For all $A.r$, if $(B,\risk) \in \credsmean{\creds}(A.r)$, then 
$\wtpath{B}{A.r}{\risk} \in \graph{\creds}$.
\end{theorem}
Following \cite{Li:2003-02}, the result follows by induction on $n$,
where $\rmem_n$ is a fixpoint of $\bounds$ as constructed in
\autoref{section-rtr-solution}.  Proofs for both of these results
are uninteresting modifications of analogous results in 
\cite{Li:2003-02}, and we omit them here for brevity.

\subsubsection{Threshold-Constrained Credential Chains}

Some additional definitions are required to incorporate thresholds
into the graph model of credentials.  Given a particular threshold, we
need to filter out those paths that exceed a given threshold.  But
because a threshold has role-level granularity, the filtering of a
path must take into account the filtering of its role-terminated
subpaths.  In the case of intersection and linking nodes, we also need
to take into account the paths supporting the derived edges leading
into them.
\begin{definition}
Given $\graph{\creds}$.  A \emph{tail} of a chain
$\wtpath{f_1}{f_2}{\risk} \in \graph{\creds}$ is a chain
$\wtpath{f_1}{f'}{\risk'} \in \graph{\creds}$ such that
$\wtedge{f'}{f_2}{\risk''} \in \wtedges{\creds}$ and $\risk = \risk'
\riskplus \risk''$.
%, and we say that $\risk''$ is the \emph{head risk}
%of $\wtpath{f_1}{f_2}{\risk}$.  
The chains $\wtpath{B}{A.r_1}{\risk},\wtpath{f}{B.r_2}{\risk''}\in 
\graph{\creds}$ \emph{support} $\wtpath{f}{A.r_1.r_2}{\risk'}\in
\graph{\creds}$ iff $\wtpath{f}{B.r_2}{\risk''}\in \graph{\creds}$ is
a tail of $\wtpath{f}{A.r_1.r_2}{\risk}\in \graph{\creds}$ and $\risk =
\risk' \linkplus \risk''$.  The chains
$\wtpath{B}{f_1}{\risk_1},\ldots,\wtpath{B}{f_n}{\risk_n}\in \graph{\creds}$
are said to \emph{support} $\wtpath{B}{f_1 \cap \cdots \cap f_n}{\risk}\in \graph{\creds}$ iff
$\risk = \risk_1 \isectplus \cdots \isectplus \risk_n$.  
\end{definition}
%\begin{lemma}
%Given $\graph{\creds}$, there exist supports for any chains
%$\wtpath{D}{A.r_1.r_2}{\risk} \in \graph{\creds}$ and $\wtpath{D}{f_1
%\cap \cdots \cap f_n}{\risk} \in \graph{\creds}$.
%\end{lemma}
Now we can define the threshold constrained credential 
chains as a predicate on the paths in a credential graph.
\begin{definition}
The set of \emph{$\thresh$-constrained credential chains of
$\graph{\creds}$}, written $\graph{\creds}^\thresh$, is a subset of
$\graph{\creds}$ where membership is predicated on the following
inductively defined conditions:
\begin{eqnarray*}
\wtpath{D}{A.r}{\risk} \in \graph{\creds}^\thresh 
&\text{ if }& \risk \po \thresh(A.r) \text{ and } 
\wtedge{D}{A.r}{\risk} \in \wtedges{\creds} \\
\wtpath{D}{A.r}{\risk} \in \graph{\creds}^\thresh 
&\text{ if }& \risk \po \thresh(A.r) \text{ and } 
\exists c \in \graph{\creds}^\thresh .\ c \text{ is a tail of } \wtpath{D}{A.r}{\risk}\\
\wtpath{D}{A.r_1.r_2}{\risk} \in \graph{\creds}^\thresh 
&\text{ if }& \exists c_1,c_2 \in \graph{\creds}^\thresh .\ 
c_1,c_2 \text{ support } \wtpath{D}{A.r_1.r_2}{\risk}\\
\wtpath{D}{f_1 \cap \cdots \cap f_n}{\risk} \in \graph{\creds}^\thresh
&\text{ if }& \exists c_1,\ldots,c_n \in \graph{\creds}^\thresh  . \\  
& & \phantom{\exists} c_1,\ldots,c_n \text{ support }\wtpath{D}{f_1 \cap \cdots \cap f_n}{\risk}
\end{eqnarray*}
\end{definition}
Full abstraction of the constrained graph model with respect to the 
set theoretic model is established via the following theorems.
\begin{theorem}
For all $B,A.r$, if $\wtpath{B}{A.r}{\risk} \in \graph{\creds}^\thresh$, then
$(B,\risk') \in \credsmean{\creds}^\thresh(A.r)$ with $\risk' \po \risk$.
\end{theorem}

\begin{theorem}
For all $A.r$, if $(B,\risk) \in \credsmean{\creds}^\thresh(A.r)$, then 
$\wtpath{B}{A.r}{\risk} \in \graph{\creds}^\thresh$.
\end{theorem}
The results follow by a straightforward generalization of
\autoref{theorem-graph-soundness} and
\autoref{theorem-graph-completeness}.

\subsection{Backward Chain Discovery Algorithm $\checkmem$}
 
In centralized chain discovery, all credentials are maintained locally
by assumption.  In distributed chain discovery, some credentials may
be retrieved over the network, in the form of certificates.  This of
course presupposes that the location of certificates can be determined
in some manner.  \emph{Backwards} chain discovery assumes that the
certificates for credentials defining a role $A.r$ are obtained from
the entity $A$, so that chains need to be reconstructed ``backwards'',
beginning with the governing role of an authorization decision
\cite{Li:2003-02}.  We now define a backwards credential chain
discovery algorithm $\checkmem$ for $\RTR$, possessing features
described at the beginning of \autoref{section-rtr-discovery}.  We
abstract the details of credential retrieval and risk assignment,
other than its ``backwards'' nature, assuming that remote
risk-weighted credentials can always be retrieved on demand (and
cached, presumably).  Forwards and mixed discovery techniques for
$\RT$ are also discussed in previous work \cite{Li:2003-02}; analogous
techniques for $\RTR$ can be adapted in the same way as we have
adapted backward discovery here.

Much like the algorithm developed for $RT_0$, we define distributed
chain discovery for $\RTR$ as a credential graph reconstruction
algorithm.  The primary difference is that ours maintains a record of
the risk encountered along partially reconstructed paths.  If this
``search risk'' exceeds the maximum tolerable risk allowed by a given
threshold, then search along that path is aborted.  Monotonicity and
associativity of risk aggregation ensures that any fully reconstructed
paths in that direction would exceed the threshold, so aborting search
in this manner is a heuristic to improve efficiency of search for
threshold-constrained solutions.  

In the following text, we describe an algorithm $\checkmem$ in
English, for which we have developed a prototype implementation in
OCaml described in \autoref{appendix-checkmem}.  The English
description refers to the Appendix in key spots for clarification.
The algorithm $\checkmem$ itself is described in
\autoref{section-checkmem-processing}, after preceding 
sections that describe data structures and auxiliary functions
used by the algorithm.

\subsubsection{Data Structures and Strategy}

The overall strategy of the algorithm is based on graph search
techniques.  The central data structures of the algorithm are
\emph{nodes}, each of which are uniquely identified by a role
expression $f$.  Every node contains the following mutable component
structures:
\begin{description}
\item[Solution.] A risk assessment (\autoref{def-risk-assessment}) for
membership in the identifying role expression $f$.  That is, a set of
elements of the form $(B, \risk)$ denoting that $B$ has been
determined to be a member of the role expression $f$, with risk
$\risk$.  Note that this use of the term ``solution'' is slightly
abusive given \autoref{def-solution}, since here membership is
assessed for role expressions $f$ in general and not just roles.
\item[Search Risks.] The aggregate risk-so-far that it took to search
from particular roles to the identifying role expression $f$.  Search
risks differ from solutions, in that they reflect the cost of search,
not the cost of membership, and are in essence a running
under-approximation of how risky it would be to establish membership
in a role along the path traversed by search from that role.
\item[Search Risk Propagaters.] Functions for propagating search risks
along paths already searched.  Each risk propagation function living
at a node $f$ is defined with respect to a local node
parameter $f'$, such that a search edge from $f$ to $f'$ has been
explored.  Future search risks are propagated along this edge via
invocation of the function.
\item[Solution Monitors.] Functions for propagating newly-discovered
solutions along paths already searched.  Each solution monitor
function living at a node $f$ is defined with respect to a local node
parameter $f'$, such that a credential graph edge from $f$ to $f'$ has
been discovered.  Future solutions are propagated along this edge via
invocation of the function.
\end{description}  
Search risks, propagaters, and solution monitors are discussed in more
detail below.  In essence, during a run of the algorithm, new nodes
are created for role expressions discovered along credential paths.
The components of an \emph{initialized} node are all empty.  After
initialization, node solutions and search risks are updated to reflect
flow through the graph structure known at that point in time.  The
same nodes are later mutated to reflect newly discovered graph
structure, via propagater and monitor functions, that invoke each
other in chains reflecting discovered graph edge structure. Nodes are
specified by the type \ttt{node} in \autoref{appendix-checkmem-types}.

\subsubsection{Search Risks}

Thresholds allow per-role specification of tolerable membership risks.
Since backwards search proceeds backwards along credential graph
edges, search reconstructs partial risk-weighted credential paths.
Furthermore, the aggregate search risk along these paths is always a
conservative approximation of role membership along these paths.
Hence, as search proceeds away from a role node $A.r$, search can be
aborted when the aggregation of risks encountered exceeds
$\thresh(A.r)$ for given threshold $\thresh$, since any role membership along
that path is sure to exceed threshold.

Nodes $f$ therefore maintain a set of search risks of the form 
$(A.r, \risk)$ representing the aggregation of risks along
a search path from $A.r$ to $f$.  A node is searchable only 
if its search risk are below-threshold, in the following sense.
\begin{definition}
\label{def-below-thresh}
A \emph{search risk} is a tuple of the form $(A.r,\risk)$.  We let $S$
range over sets of search risks.  We say that $A.r$ is \emph{included}
in a set of search risks $S$ iff there exists $\risk$ such that $(A.r,
\risk) \in S$.  Given some $\thresh$, a set of search risks $S$ is
\emph{below threshold} iff for all $A.r$ included in $S$, there exists
$\risk$ such that $(A.r, \risk) \in S$ and $\risk \po \thresh(A.r)$.
%Define:
%$$
%S \riskplus \risk \ \defeq\ \setdefn{(A.r, \risk' \riskplus \risk) | (A.r,\risk') \in S}
%$$
\end{definition}
The type of search risks is specified as \ttt{search\_risk} in
\autoref{appendix-checkmem-types}.

\subsubsection{Search Risk Propagaters and Solution Monitors}

Any given node in the credential graph can be discovered more than
once during search, but to ensure termination we require that any
given node can only be searched once, as is usually the case in graph
search algorithms.  But it is important to maintain discovered graph
structure, so that newly discovered information about the graph can be
propagated to already-searched nodes.  We use functions called
\emph{search risk propagaters} and \emph{solution monitors}, to
maintain graph structure and propagate information, specified as types
\ttt{propagater} and \ttt{monitor} in
\autoref{appendix-checkmem-types}.  These functions propagate search
risks and solutions along edges.  Each function lives at a node $f$,
which is the source of the edge, and is always defined with respect to
a node $f'$ that is the sink of the edge.  

To propagate search risks backward along search paths, we define
side-effecting functions called search risk propagaters.  This
functionality is necessary since less risky search paths may be
discovered to already-visited nodes.  Whenever a node $f$ is notified
to add a new risk $(A.r,\risk)$ to its search risks (as defined by
function \ttt{risk\_notify} in \autoref{appendix-checkmem-monitors}),
if there does not exist $(A.r,\risk')$ already in $f$'s search risks
such that $\risk' \po \risk$, then $(A.r,\risk)$ is added, and all of
$f$'s search risk propagaters are invoked on $(A.r, \risk)$.  Each
propagater is defined with respect to a node $f'$ denoting the node to
which the search risk should be propagated.
\begin{enumerate}[\hspace{11pt}]
\item A \emph{search risk propagater} for a node $f'$ and a risk
$\risk'$ is a function abstracted on search risks $(A.r,\risk)$ that
notifies $f'$ to add $(A.r, \risk' \riskplus \risk)$ to its search
risks.  The function \ttt{make\_propagater} in
\autoref{appendix-checkmem-monitors} generates search risk
propagaters.
\end{enumerate}

Solution monitors propagate solution elements $(A,\risk)$ forward
along discovered edges, aggregating edge risks as they go; their
control flow structure mimics the discovered graph structure. At the
same time, they also enforce the threshold $\thresh$ supplied as a
parameter to $\checkmem$.  Whenever a node $f$ is notified to add a
solution element $(A,\risk)$ (as defined by function
\ttt{soln\_notify} in \autoref{appendix-checkmem-monitors}), the
element is added to $f$'s solution and all of $f$'s solution monitors
are applied to it, on two conditions: (1) there does not exist $\risk'
\po \risk$ such that $(A,\risk')$ is already in $f$'s solution (in
which case we say it is \emph{canonically new}), and (2) if $f$ is a
role $A.r$ then $\risk \po \thresh(A.r)$.  There are three classes of
solution monitors, generated by functions \ttt{make\_nmonitor},
\ttt{make\_lmonitor}, and \ttt{make\_imonitor} as specified in
\autoref{appendix-checkmem-monitors}.  Each monitor form is
defined with respect to a given role expression, denoting the node to
which the given solution should be propagated:
\begin{enumerate}
  \item A \emph{node monitor} for a given node $f$ and 
edge risk $\risk$ is a function abstracted on solution 
elements $(B,\risk')$, that notifies $f$ to add
$(B,\risk' \oplus \risk)$ to its solutions.  
  \item A \emph{linking monitor} for a given linked role $A.r_1.r_2$
is a function abstracted on solution elements $(B,\risk)$, that
creates a node monitor for $A.r_1.r_2$ and $\risk$, applies it to each
known element of $B.r_2$'s solution, and adds it to $B.r_2$'s solution
monitors to propagate solutions yet to be discovered.  Also, given all
search risks $(C.s, \risk')$ of $A.r_1.r_2$, $B.r_2$ is notified to
add $(C.s, \risk \linkplus \risk')$ to its search risks, a search risk
propagater for $B.r_2$ and $\risk$ is added to $A.r_1.r_2$'s search risk
propagaters, and $B.r_2$ is added to the queue if it hasn't already
been.
  \item An \emph{intersection monitor} for a given intersection role
$f_1 \cap \cdots \cap f_n$ is a function abstracted on solution
elements $(B,\risk)$, that applies a node monitor for $f_1 \cap \cdots
\cap f_n$ and $\bot$ to each element 
$(B,\risk')$ in the the assessment $R_1 \isectplus
\cdots \isectplus R_n$, where each $R_i$ is the assessment of $f_i$ in the
current solution.
\end{enumerate}

\subsubsection{Node Processing}
\label{section-checkmem-processing}

Given entity $A$, role $B.r$ and threshold $\thresh$, the algorithm
$\checkmem$ reconstructs a proof graph, to check membership of $A$ in
role $B.r$ within a given threshold $\thresh$.  The algorithm
maintains two mutable global data structures: a list of nodes created
during execution of the algorithm, and a queue of nodes to be
searched.  These are specified as \ttt{nodes} and \ttt{q} in
\autoref{appendix-checkmem-auxfns}.  Whenever a role expression $f$ is
first encountered during search, an initialized node identified by $f$
is added to \ttt{nodes}.

Upon invocation, the algorithm \checkmem, defined in
\autoref{appendix-checkmem-processing}, clears \ttt{q} and
\ttt{nodes}.  The node $B.r$ is initialized and added to \ttt{q} and
\ttt{nodes}.  While the queue contains at least one element whose
search risks are below threshold $\thresh$, such below threshold nodes
are taken from the queue individually for searching.  Above threshold
nodes are not explored, since any solution paths that encounter them
are sure to overrun the risk threshold specified for the node.  But
neither are they eliminated, since future search may find new below
threshold paths to them.  The algorithm runs until there are no below
threshold nodes left in the queue, or until an element $(A,\risk)$ is
added to $B.r$'s solutions with $\risk \po \thresh(B.r)$, signalling
that a $\thresh$-constrained path $\wtpath{A}{B.r}{\risk}$ has been
discovered.
  
Whenever nodes are taken from the queue, they are processed depending
on their form:
\begin{enumerate}
  \item To process an entity $A$, the node $A$ is notified to 
add $(A,\bot)$ as a solution to itself.
  \item To process a role $A.r$, the credentials defining $A.r$ are
retrieved.  For each such credential $\cred{A.r}{f}{\risk}$, a node
monitor for $A.r$ and $\risk$ is created, is applied to all of $f$'s
known solutions, and is added to $f$'s solution monitors for
propagating solutions still to be discovered.  Also, $A.r$ is notified
to add $(A.r,\bot)$ to its search risks, and given all search risks
$(B.s,\risk')$ of $A.r$, $f$ is notified to add $(B.s, \risk'
\riskplus \risk)$ to its search risks.  A search risk propagater for
$f$ and $\risk$ is added to $A.r$'s search risk propagaters, and $f$
is added to the queue if it hasn't already been.  The auxiliary function
\ttt{process\_creds} defined in \autoref{appendix-checkmem-processing}
implements much of this.
  \item To process a linked role $A.r_1.r_2$, a linking monitor for
$A.r_1.r_2$ is created, is applied to all of $A.r_1$'s known
solutions, and is added to $A.r_1$'s solution monitors.  The node
$A.r_1$ is notified to add the search risks of $A.r_1.r_2$ to its
own search risks, $A.r_1.r_2$ acquires a search risk propagater for
$A.r_1$ and $\bot$, and $A.r_1$ is added to the queue if it hasn't
already been.
  \item To process an intersection role $f_1 \cap \cdots \cap f_n$, an
intersection monitor for $f_1 \cap \cdots \cap f_n$ is created, and
added to each $f_i$.  For each $f_i$ and every search risk
$(A.r,\risk)$ of $f_1 \cap \cdots \cap f_n$, the node $f_i$ is
notified to add the search risk $(A.r, \risk)$ and the node $f_1 \cap
\cdots \cap f_n$ acquires a search risk propagater for $f_i$ and
$\bot$, and each $f_i$ is added to the queue if it hasn't already
been.  The auxiliary function \ttt{process\_isect} defined in
\autoref{appendix-checkmem-processing} implements much of this.
\end{enumerate}
The algorithm \ttt{checkmem} defined in \autoref{appendix-checkmem}
raises an exception \ttt{Solved} if the node $A.r$ is notified to add
a solution $(B,\risk)$ such that $\risk \po \thresh(A.r)$.  At the
top-level, we write $\checkmem(A,B.r,\thresh)$ to denote invocation of
a function that calls $\ttt{checkmem}$ on our OCaml representation of
arguments, and that returns true if this call raises a \ttt{Solved}
exception and false if it terminates unexceptionally.  Hence, given a
set of distributed credentials $\creds$, when an invocation
$\checkmem(A,B.r,\thresh)$ terminates, the algorithm returns true iff
there exists $\risk$ such that $(A,\risk) \in \credsmean{C}^\thresh$.

\subsubsection{Properties}

Assuming that defining credentials can always be obtained for any
role, we assert that $\checkmem$ satisfies the following properties,
demonstrating that it correctly reconstructs credential graphs.  Since
credential graphs are full abstractions of the $\RTR$ semantics as
discussed in \autoref{section-credential-graphs}, these results
demonstrate that $\checkmem$ is a correct implementation of $\RTR$.
The proofs, omitted here for brevity, are straightforward extensions
of results in \cite{Li:2003-02}, since our $\checkmem$ algorithm is an
extension of the \ttt{backward} algorithm described in that paper.
Note that these results presuppose that credentials are stored in a
manner that allows backwards chain reconstruction: we say that a set
of credentials $\creds$ is \emph{distributed} if they are stored in a
manner that allows lookup given the credential issuer.
\begin{theorem}[Soundness]
Given a set of distributed credentials $\creds$, if an invocation
$\checkmem(A,B.r,\thresh)$ holds then there exists $\risk$ such that
$(A,\risk) \in \credsmean{\creds}^\thresh$.
\end{theorem}

\begin{theorem}[Completeness]
Given a set of distributed credentials $\creds$, if there exists
$\risk$ such that $(A,\risk) \in \credsmean{\creds}^\thresh$ then
$\checkmem(A,B.r,\thresh)$ holds.
\end{theorem}
For our algorithm, an issue highly relevant to completeness is how
search risks are computed.  Recalling that discovery will not proceed
along paths where search risks are over-threshold, it is essential to
observe that search risks are indeed under-approximations of role
membership risk.  Othermise, search could be aborted along paths that
would otherwise discover solutions within the given threshold
constraint, resulting in incompleteness.  This property is expressed
via the following lemma.
\begin{lemma}
Given a set of distributed credentials $\creds$, at any point during
execution of $\checkmem(C,B.s,\thresh)$ for arbitrary $C$, $B.s$, and
$\thresh$, if a node identified by a role expression $f$ contains
$(A.r, \risk)$ in its search risks, then $\wtpath{f}{A.r}{\risk'} \in
\graph{\creds}$ such that $\risk' \po \risk$.
\end{lemma}

We also observe that the algorithm terminates given a finite set of
credentials, regardless of the given threshold.  This is because nodes
are never visited more than once, and solution monitors will not
traverse any graph cycle, and hence are guaranteed to terminate.
Solution monitors only propagate canonically new members, but
traversal of a cycle necessarily causes a monotonic increase in a
solution's risk assessment, hence canonical containment in an existing
solution.  Search risk propagaters are similarly guaranteed to
terminate.
\begin{theorem}[Termination]
Given a finite set of distributed credentials $\creds$, for all $B$, $A.r$,
and $\thresh$, $\checkmem(B,A.r,\thresh)$ terminates.
\end{theorem}

\subsubsection{Discussion: Example} 
\label{section-discovery-example}

\discoveryexamplefig

We now provide a graphical example that illustrates how the algorithm
works.  While the algorithm possesses a number of technical details,
its basic operation is fairly straightforward as the graphic shows.
Assume given the natural number risk ordering defined in
\autoref{example-ponat}, where aggregation is addition.  Modifying the
Example in \autoref{section-rt-example}, assume that the following
credentials are stored with their issuers, numbered here $(i)$ through
$(v)$ for later reference:
\begin{mathpar}
\cred{H.\mathit{discount}}{H.\mathit{preferred}}{15} \quad (i)

\cred{H.\mathit{discount}}{H.\mathit{orgs}.\mathit{members}}{5}   \quad(ii)

\cred{H.\mathit{orgs}}{\mathit{AAA}}{10} \quad (iii)

\cred{H.\mathit{preferred}}{\mathit{AAA}.\mathit{members}}{7}  \quad (iv)

\cred{\mathit{AAA}.\mathit{members}}{\mathit{Mary}}{4}   \quad (v)
\end{mathpar}
Finally, define $\thresh$ as the threshold that maps
$H.\mathit{discount}$ to 20 and every other role to $\omega$.  In
\autoref{figure-discovery-example}, we show how a credential graph
evolves during a run of: 
$$\checkmem(H.\mathit{discount}, \mathit{Mary}, \thresh)$$ in
subfigures (a) through (f).  Nodes in the graph are represented by
circles, and edges along which solution monitors will propagate
solutions by arrows, labeled by the risk aggregated by traversing that
edge.  Each node is labeled with three possibly blank lines: its role
expression identifier on the first line, its search risk from role
$H.\mathit{discount}$ on the second, and its solution on the third.
Only search risks from $H.\mathit{discount}$ are denoted since all
others are unconstrained by $\thresh$ and therefore irrelevant to the
algorithm.  We now describe each subfigure (a) through (f) in order.
\begin{enumerate}[\hspace{5pt}(a)]
\item Initially, the node $H.\mathit{discount}$ is added to the 
graph, with a search risk of 0 from itself.
\item After retrieval of credentials $(i)$ and $(ii)$ defining
role $H.\mathit{discount}$, the nodes $H.\mathit{preferred}$ and
$H.\mathit{orgs}.\mathit{members}$ are added to the graph with the
appropriate search risks from $H.\mathit{discount}$.
\item Assuming that node $H.\mathit{preferred}$ is added before
$H.\mathit{preferred}$, the former is the next to be searched,
resulting in retrieval of credential $(iv)$ and the addition of node
$\mathit{AAA}.\mathit{members}$ to the graph with its search risk from
$H.\mathit{discount}$ being the aggregate of preceding risks.
\item Since the current search risk of $\mathit{AAA}.\mathit{members}$
from $H.\mathit{discount}$ is above the threshold specified by 
$\thresh$, the search will proceed by exploring the next below-threshold
node in the queue, which is $H.\mathit{orgs}.\mathit{members}$.  
This results in the addition of the node  $H.\mathit{orgs}$ to 
the graph. 
\item Since $H.\mathit{orgs}$ is now the next unsearched
below-threshold node in the queue, credential (iii) will be retrieved,
so that node $\mathit{AAA}$ is added to the graph.  Also, for the
first time, solutions are added to the graph: $(\mathit{AAA},0)$ is 
added to $\mathit{AAA}$'s solution, as is $(\mathit{AAA},10)$ to
$H.\mathit{orgs}$ 's, the latter by a solution monitor.
\item The previous step reduced the search risk of
$\mathit{AAA}.\mathit{members}$ from $H.\mathit{discount}$ to 15, so
the former node is now below-threshold and searchable.  This results
in retrieval of credential $(v)$, and the addition of node
$\ttt{Mary}$ to the graph.  Since $\mathit{Mary}$ is an entity,
$(\mathit{Mary},0)$ is added to $\mathit{Marys}$'s solution, and
propagated backwards with aggregated risk by solution monitors.  This
results in the addition of $(\mathit{Mary},19)$ to node
$H.\mathit{discount}$, so the algorithm terminates with success.
\end{enumerate}


% Assume given the bound-of-risks ordering defined in
% \autoref{section-examples-bounds} with elements
% $\setdefn{\mathit{high},\mathit{medium},\mathit{low}}$, and let both
% risk and intersection aggregation be the lub operation.  Assume given
% the first five credentials in \autoref{section-examples-bounds} with
% one small change for purposes of the example, numbered here for later
% reference:
% %\begin{mathpar}
% %\mathitcred{Store.buyer}{Acme.purchaser \cap Acme.employee}{low}\ (1)
% %
% %\mathitcred{Acme.employee}{Ed}{medium}\ (2)
% %
% %\mathitcred{Acme.purchaser}{Ed}{high}\ (3) 
% %
% %\mathitcred{Acme.purchaser}{Personnel.manager}{low}\ (4)
% %
% %\mathitcred{Personnel.manager}{Ed}{low}\ (5)
% %\end{mathpar}
% \begin{eqnarray}
% \mathit{Store.buyer} &\cedge{\mathit{low}}& \mathit{Acme.purchaser \cap Acme.employee}\\
% \mathit{Acme.employee} &\cedge{\mathit{medium}}& \mathit{Ed}\\
% \mathit{Acme.purchaser} &\cedge{\mathit{high}}& \mathit{Ed}\\
% \mathit{Acme.purchaser} &\cedge{\mathit{medium}}& \mathit{Personnel.manager}\\
% \mathit{Personnel.manager} &\cedge{\mathit{low}}& \mathit{Ed}
% \end{eqnarray}
% We posit that these credentials are stored as certificates with 
% their issuers in keeping with the backwards discovery scheme.
% 
% Let threshold $\thresh_\mathit{high}$ be defined such that $\forall
% A.r\ .\ \thresh_\mathit{high} = \mathit{high}$.  Then the query
% $\checkmem(\mathit{Ed}, \mathit{Store.buyer},\thresh_\mathit{high})$
% will succeed, as the algorithm will retrieve credentials in the order
% (1),(3),(4),(2).  Note that this credential chain constitutes the path
% $\wtpath{\mathit{Ed}}{\mathit{Store.buyer}}{\mathit{high}}$.  Now, let
% $\thresh = \thresh_\mathit{high}[\mathit{Store.buyer} :
% \mathit{medium}]$.  Then the query $\checkmem(\mathit{Ed},
% \mathit{Store.buyer},\thresh)$ will also succeed.  However, while the
% algorithm will again retrieve credentials in the order (1),(3),(4),(2)
% in this case, $\mathit{Ed}$ will not have been added to
% $\mathit{Store.buyer}$'s solutions, so $\checkmem$ will additionally
% retrieve (5) and then succeed, as it has constituted the
% $\thresh$-constrained path
% $\wtpath{\mathit{Ed}}{\mathit{Store.buyer}}{\mathit{med}}$.  Letting
% $\thresh' = \thresh_\mathit{high}[\mathit{Store.buyer} :
% \mathit{low}]$, observe that $\checkmem(\mathit{Ed},
% \mathit{Store.buyer},\thresh)$ will fail.  Not only that, but it will
% only retrieve credentials (1),(3),(4),(2), not (5), since
% $\mathit{Personnel.manager}$'s search risks after retrieval of (2)
% will be:
% $$
% \setdefn{(\mathit{Store.buyer}, \mathit{medium}), (\mathit{Acme.purchaser}, \mathit{medium})}
% $$
% which is not below threshold for $\thresh'$, hence its 
% definition will not be retrieved.

\subsubsection{Discussion: Refinements} There are two particular 
instances where the definition of $\checkmem$ could be enhanced, for
more eager short-circuiting of chain discovery in case risk thresholds
are exceeded along discovery paths. First, observe that credentials
are retrieved before being checked to see if their risks will force
the discovery threshold to be exceeded.  However, risks such as
expected wait time suggest that it is more practical for credentials
to be retrieved after ensuring they won't overrun the threshold. A
number of minor variations on $\checkmem$ can be imagined that will
address this.

A more interesting enhancement is relevant to the propagation of
search risks along discovery paths leading from intersection nodes.
Observe that from any intersection role $f_1 \cap \cdots \cap f_n$,
the search risks of $f_1 \cap \cdots \cap f_n$ are propagated to each
$f_i$.  However, this could be a under-approximation of search risks
for any given $f_i$.  For example, suppose that $A$ is being checked
for authorization and $(A,\risk)$ is known to be the only possible 
assessment of $A$ in $f_1$'s solution.  When checking $f_n$, the
search risks of $f_n$ inherited from $f_1 \cap \cdots \cap f_n$
could be aggregated with $\risk$, since $\risk$ is certain to 
be a component risk of any authorization supported by
discovery from $f_n$.  

