\section{Overview of RT}
\label{section-rt}

Rather than defining a new trust management logic for a formalization
of risk, we take advantage of the existing $\RT$ system
\cite{Li:2003-04}.  This system combines the strengths of role-based
access control with an expressive trust management logic, and enjoys a
variety of existing implementation techniques \cite{Li:2003-02}.  We
believe these features make $\RT$ one of the most advanced trust
management systems, and an appealing setting for the development of
formal risk assessment.  

The RT role-based trust management system is actually a collection of
trust management logics, all of which are variations on a basic logic
called $\RT_0$ \cite{Li:2003-04}.  Variations include separation of
duties and delegation.  In this same spirit, we propose a variation on
$\RT_0$ to incorporate a formalization of risk assessment, so we
briefly review $\RT_0$ here to provide necessary background.

In $\RT_0$, individual actors, or principals, are called
$\mathit{Entities}$ and are defined by public keys.  We let $A,B,C,D,E$
range over entities.  Each entity $A$ can create an arbitrary number
of $\mathit{Roles}$ in a namespace local to the entity,
denoted $A.r$.  The $\mathit{RoleExpressions}$ of $\RTR$, denoted $f$, are either
entities or roles or constructed from other role expressions by
\emph{linking} and \emph{intersection}.  Formally, role 
expressions are generated by the following grammar:
$$
f ::= A \mid A.r \mid A.r.r \mid f \cap \cdots \cap f
$$ 
Role expressions are used to define roles via credentials. To define
a role an entity issues credentials that specify the role's
membership. Some of these credentials may be a part of private policy;
others may be signed by the issuer and made publically available. The
overall membership of a role is taken as the memberships specified by
all the defining credentials.

\newpage
$\RT_0$ provides four credential forms: 
\begin{enumerate}

\item $\cred{A.r}{E}{}$ 

 This form asserts that entity $E$ is a member of role $A.r$.

\item $\cred{A.r}{B.s}{}$ 

  This form asserts that all members of role $B.s$
  are members of role $A.r$. Credentials of this form can be used to
  delegate control over the membership of a role to another entity.

\item $\cred{A.r}{B.s.t}{}$ 

  This form asserts that for each member $E$ of
  $B.s$, all members of role $E.t$ are members of role
  $A.r$. Credentials of this form can be used to delegate control over
  the membership of a role to all entities that have the attribute
  represented by $B.s$.  The expression $B.s.t$ is called a 
  \emph{linked role}.

\item $\cred{A.r}{f_1 \cap \cdots \cap f_n}{}$
  
  This form asserts that each entity that is a member of all role
  expression forms $f_1,\ldots, f_n$ is also a member of role
  $A.r$. The expression $f_1 \cap \cdots \cap f_n$ is called an
  \emph{intersection role}.

\end{enumerate}
Authorization is then cast as a role membership decision: an access
target is represented as some role expression $f$, and authorization
for that target for some entity $A$ is equivalent to determining
whether $A$ is a member of $f$.  In such a decision, we call $f$ the
\emph{governing role}.  Authorization always assumes some given finite
set of credentials, denoted $\creds$.  We use
$\mathit{Entities(\mathcal{C})}$ to represent the entities used in a
particular set of credentials $\mathcal{C}$, and similarly
$\mathit{RoleNames(\mathcal{C})}$, $\mathit{Roles(\mathcal{C})}$, etc.

\subsection{Example}
\label{section-rt-example}

Suppose a hotel $H$ offers a room discount to certain preferred
customers, who are members of $H.\mathit{preferred}$. The policy of
$H$ is to grant a discount to all of its preferred customers in
$H.\mathit{preferred}$ as well as to members of certain organizations.
$H$ defines a role $H.\mathit{orgs}$ that contains the public keys of
these organizations. Into that role $H$ places, for example, the key
of the AAA, the American Auto Association. These
credentials are summarized as follows:
\begin{mathpar}
\cred{H.\mathit{discount}}{H.\mathit{preferred}}{}

\cred{H.\mathit{discount}}{H.\mathit{orgs}.\mathit{members}}{}

\cred{H.\mathit{orgs}}{\mathit{AAA}}{}
\end{mathpar}

Now imagine that at a later time a special marketing plan is created to
encourage travelers to stay at $H$. A decision is made that all
members of the AAA are automatically preferred customers and thus the
credential
$\cred{H.\mathit{preferred}}{\mathit{AAA}.\mathit{members}}{}$ is added
to the policy.

Finally suppose that Mary is a member of the AAA. She has a credential
issued by the AAA,
$\cred{\mathit{AAA}.\mathit{members}}{\mathit{M}}{}$, attesting to
that fact. By presenting this credential to $H$'s web service Mary can
prove in two distinct ways that she is authorized to receive the
discount. On one hand she is a member of an organization in
$H.\mathit{orgs}$. On the other hand she is, indirectly, a preferred
customer of $H$.  Certain practical considerations may motivate $H$'s
decision about which ``proof'' to use.  As we'll see in
\autoref{section-rtr-discovery}, specified risk thresholds in $\RTR$
can steer authorization in the right direction.
