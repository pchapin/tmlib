\title{Risk Management for Distributed Authorization}

\markboth{Christian Skalka, X.~Sean Wang and Peter Chapin}{Risk Management for Distributed Authorization}

\author{
Christian Skalka   \\ University of Vermont
\and 
X.~Sean Wang  \\ University of Vermont
\and 
Peter Chapin \\ University of Vermont
}

\begin{abstract} 
Distributed authorization takes into account several elements,
including certificates that may be provided by non-local actors.
While most trust management systems treat all assertions as equally
valid up to certificate authentication, realistic considerations may
associate risk with some of these elements, for example some actors
may be less trusted than others.  Furthermore, practical online
authorization may require certain levels of risk to be tolerated. In
this paper, we introduce a trust management logic that incorporates
formal risk assessment.  This formalization allows risk levels to be
associated with authorization elements, and promotes development of a
distributed authorization algorithm allowing tolerable levels of risk
to be precisely specified and rigorously enforced.
\end{abstract}

\category{C.2.0}{Computer-Communication Networks}{\\General}[Security and protection]

%\terms{Security, Languages, Theory}

%\keywords{Distributed Authorization, Trust Management Logic}

\maketitle