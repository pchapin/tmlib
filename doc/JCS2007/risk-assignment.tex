\section{Practical Motivations}
\label{section-risk-assignment}

Credentials in RT are all created equal, in that each represents a
true statement in the knowledge base of an authorization decision.
There is no facility for denoting that one credential may be more or
less ``believable'' than another.  In this RT is similar to other
trust management systems, such as SPKI/SDSI \cite{ellison-etal-rfc99}.
However, recent practice has shown that such a manichean view is not
consistent with reality.  In this section we discuss practical issues
that suggest the need for a more fine-grained view of the different
risks associated with credentials.

\subsection{Risk as an Authentication Metric}

The system RT treats PKI transparently.  That is, the semantics of RT
does not concern itself with the details of associating keys with
users, nor authenticating this association.  Nevertheless, credentials
for authorization decisions are established by certificates, that must
be authenticated.  Furthermore, authentication is not necessarily a
simple process in open distributed systems, rather modern PKI allows
for construction of global public-key namespaces on the basis of local
certification authorities.  In this setting authentication can involve
traversing a chain of intermediary authorities in distinct
administrative domains \cite{birrell-etal-oakland86}.  Distinct
domains may be trusted to varying degrees, so depending on what domain
boundaries are crossed, one public key certification may be more
trustworthy than another, or there may exist multiple paths of varying
degrees of trust to establish the same certification.
\emph{Authentication chains} of this sort resemble certificate chains
as we study them in this paper, though the latter is at a level of
abstraction above the former.

To address issues of trust in authentication chains,
\emph{authentication metrics} have been developed to assign
measurements of trust to particular certifications
\cite{reiter-stubblebine-tissec99}.  The measure of assurance of a
certification is based on the set of chains that establish it, which
are in turn based on a combination of trust measures assigned to nodes
in the chain, i.e.~particular administrative domains.  A number of
schemes have been proposed, including a \emph{legitimacy} measure for
PGP \cite{Abdul-Rahman-EDI97}, illustrating the practical relevance of
the idea.

Thus, in standard authorization frameworks, the treatment of
credentials as inerrable facts ignores any degrees of assurance that
are established for particular certifications.  But such degrees are
clearly of potential interest to the authorizer, for example if
multiple credentials are involved in a particular decision and each is
of low assurance, then these may compound to yield unacceptably low
assurance for the decision, whereas it may be tolerable if only one of
them has low assurance.  By allowing an assignment of risks to RT
credentials, and providing a means of combining them in the
authorization semantics, our proposed extension to RT allows a formal
accounting of these considerations.

\subsection{Balancing Security and Efficiency}

In addition to issues of trust, efficiency issues may affect the risk
associated with credentials.  For example, if cryptographic
certification of a particular credential via the PKI would be too
time-consuming, the authorizer may prefer to avoid using that
credential.  The algorithm we define for online authorization is given
a threshold of tolerable risk for authorization, and avoids
authorization chains that exceed this threshold.  Thus, our directed
search technique provides a heuristic for efficiency in case risk is
measured in computational cost.

A more interesting example is the use of cached credentials.  Caching
credentials is a useful technique to avoid re-retrieving and
re-authenticating certificates.  However, certificates are commonly
assigned expiration dates, as in x509 \cite{X509}, and most trust
management systems, including RT and SPKI/SDSI, do not consider
expiration in the formal authorization semantics but only in the
initial credential certification \cite{ellison-etal-rfc99}.
Therefore, re-use of a cached credential runs the risk of re-using
expired rights.  Similarly, systems with certificate revocation must
take care not to re-use cached credentials that have been revoked.
However, if the system uses certificate revocation lists as does for
example SPKI/SDSI \cite{ellison-etal-rfc99}, maintaining a current
view of revocation may be problematic due to the likely need to update
lists with non-local information.  This reality is reflected in the
\emph{verify-only} mode of QCM \cite{Gunter:PDCR}. To implement
certificate revocation, QCM relies on a database of revoked
certificate identifiers, but in verify-only mode its external
communications are blocked for the sake of efficiency.  In such
situations, our formalism allows trust management systems to represent
and manage the risks associated with credentials that have expired or
that may have been revoked.  A sufficiently abstract representation of
risk even allows to balance the cost of trusting questionable
credentials against the efficiency benefits gained by their usage, as
discussed in \autoref{section-applications-cost-benefit}.
    
