%%%%%%%%%%%%%%%%%%%%%%%%%%%%%%%%%%%%%%%%%%%%%%%%%%%%%%%%%%%%%%%%%%%%%%%%%%%%
% FILE         : modal.tex
% AUTHOR       : Peter C. Chapin
% LAST REVISED : 2007-10-23
% SUBJECT      : A few thoughts on using modal logic in authorization systems.
%
% Send comments or bug reports to:
%
%       Peter C. Chapin
%       Department of Computer Science
%       University of Vermont
%       Burlington, VT
%       pchapin@cems.uvm.edu
%%%%%%%%%%%%%%%%%%%%%%%%%%%%%%%%%%%%%%%%%%%%%%%%%%%%%%%%%%%%%%%%%%%%%%%%%%%%

%+++++++++++++++++++++++++++++++++
% Preamble and global declarations
%+++++++++++++++++++++++++++++++++
\documentclass{article}
%\documentclass[12pt]{article}

% Common packages. Uncomment as needed.
%\usepackage{amsmath}
\usepackage{amssymb}
%\usepackage{amstext}
%\usepackage{amsthm}
%\usepackage{doublespace}
%\usepackage{fullpage}
%\usepackage[dvips]{graphics}
%\usepackage{listings}
%\usepackage{mathpartir}
%\usepackage{url}
%\usepackage{hyperref}

% The following are settings for the listings package.
%\lstset{language=Java,
%        basicstyle=\small,
%        stringstyle=\ttfamily,
%        commentstyle=\ttfamily,
%        xleftmargin=0.25in,
%        showstringspaces=false}

\setlength{\parindent}{0em}
\setlength{\parskip}{1.75ex plus0.5ex minus0.5ex}

% Some handy macros.
\newcommand{\todo}[1]{\textit{TODO: #1}}
\newcommand{\po}{\preccurlyeq}

%++++++++++++++++++++
% The document itself
%++++++++++++++++++++
\begin{document}

%-----------------------
% Title page information
%-----------------------
\title{Modal Logic in Trust Management}
\author{Peter C. Chapin}
\date{October 23, 2007}
\maketitle

\section{Introduction}

Trust management systems that use contextual information as part of the authorization decision
need to contend with the volatility of that information. In a traditional environment, there is
an implicit assumption that a requester's credentials remain valid for the duration of a
particular session. Thus the authorizer need not be concerned about rechecking a previously made
authorization decision throughout a session. However, the validity of contextual information may
change on a time scale of minutes or even seconds in some cases. For example, a requester might
move to a different location inside a building, or the temperature on particular sensor might
change abruptly. In such an environment the authorizer must face the very practical problem of
monitoring the validity of each authorization decision and preemptively aborting sessions that
become invalid.

To support practical implementations, it is desirable to have a unified theory of context aware
authorization that takes into account the time changing nature of the contextual inputs. In a
traditional trust management system, invalid (expired) credentials are simply discarded before
the authorization decision is made. While such an approach might be theoretically unsatisfying,
it is feasible and practical. However, when contextual inputs are used the authorizer will need
to maintain a complex data structure that reflects the current status of all active sessions.
For reasons of efficiency this structure will need to be updated incrementally as the contextual
inputs change. Proving that the algorithms used for this purpose are correct will require a
reasonable theory together with a well defined notion of correctness.

Authorizers might also be interested in asking meta-questions about their policy. For example,
given a particular policy and set of contextual inputs, is it possible for a particular
principal to gain access to a particular resource? A related question might be, will a
particular principal necessarily have access to a particular resource, regardless of how the
contextual inputs might change? To frame and answer such questions formally suggests that the
theory should be based on a modal logic. The purpose of this document is to explore how a modal
logic might be used to construct a unified theory of context aware authorization.

This document is a collection of loosely related ideas. It is a scratch pad on which these ideas
can be stated and developed. It is hoped that as these ideas are refined the text of this
document will evolve to a more formal and comprehensive presentation of the use of modal logic
in trust management systems.

\section{Problem Scope}

In order to keep the problem focused we first consider a theory in a more limited context. Our
theory will be applicable to only role based access control systems, specifically those based on
the $RT$ framework. In such a system role membership is what defines resource access, and the
authorization decision is thus about computing which principals are in each role.

The second simplification is that we only consider temporal context. That is, the validity of
credentials depends on the time. Other kinds of contextual inputs will initially be ignored.
There are several reasons for this.

\begin{enumerate}
\item Temporal context potentially has many applications. For example employees might only have
  certain rights during work hours (doctors, emergency services personnel, etc).
\item Certificate expiration and perhaps revocation could be handled by an appropriate temporal
  theory. This allows us to create a good theoretical foundation for these important issues that
  are now largely handled in an ad-hoc manner.
\item A large body of existing work on handling temporal modalities already exists. We hope to
  be able to tap into that work.
\end{enumerate}

We believe that both of these restrictions can probably be lifted once the basic ideas are
worked out. For example, it would be valuable if the theory had application to other kinds of
access control models such as ACLs or other access control matrix approaches. In addition, it
would be valuable if the theory could be used with a collection of contextual inputs, perhaps
with properties that are quite different than time. Dealing with this more general approach
would probably require the use of more general modal logics. However, the theory of such logics
is reasonably well developed.

\section{Instantiation of $RT^R$}

An extension to the $RT$ framework is needed to write policies that make use of contextual
inputs, and issue credentials that reference them. Here we specify an instantiation of $RT^R$
that can serve as a base for later modal extensions. This $RT^R$ instantiation treats the
contextual inputs as time invariant propositions. We assume the authorizer can assign truth
values to each proposition based on data inputs arriving at the authorizer through some
unspecified means. We further assume that those inputs are trustworthy; we do not consider at
this time a mechanism for protecting the contextual inputs.

Let $\mathcal{P}$ be a countable set of proposition symbols $\{ p_0, p_1, \ldots, \}$. The
interpretation of these symbols is not specified but as an example $p_0$ might be interpreted as
``the reactor temperature is critical,'' $p_1$ might be interpreted as ``two or more people are
in the launch control room,'' etc. We assume that all participants in the distributed system
agree on the meaning of the proposition symbols so that credentials created by a principal will
be interpreted correctly by other principals\footnote{$RT$ also assumes principals agree on the
  significance of role names.}. \todo{Can we make this first order instead? That would probably
  be more useful.}

Let $\mathcal{F}$ be the set of propositional formula defined inductively as follows.
\begin{enumerate}
\item The symbols $\top$ (constant true) and $\bot$ (constant false) are formula.
\item $p_i$ is a formula.
\item $\neg p_i$ is a formula.
\item $f_i \wedge f_j$ is a formula where $f_i, f_j \in \mathcal{F}$.
\item $\mathcal{F}$ is the smallest set that contains all formula generated by the above rules.
\end{enumerate}

Note that $\mathcal{F}$ is a subset of all possible propositional formula. For example,
disjunction is not allowed since $\neg (\neg p_i \wedge \neg p_j) \notin \mathcal{F}$.

Following Definition 4.1 in \cite{Skalka:RMDA} we define $\mathcal{K} = \mathcal{F}$. We define
$\po$ as follows: For a particular formula $f_i$, let $P_i$ be the site of primitive
propositions used in $f_i$ where a ``primitive proposition'' is either $p_n$ or $\neg p_n$ for
some $p_n \in P$. Then $f_i \po f_j$ iff $P_i \subseteq P_j$. Furthermore let the bottom element
of the lattice be $\bot$. Let the top element of the lattice be the conjunction $\top \wedge p_0
\wedge p_1 \wedge \ldots$ \todo{Prove that $(K, \po)$ as defined here is a complete lattice as
  required by $RT^R$}.

$RT^R$ also requires two associative and monotonic aggregation operators $\oplus$ and $\otimes$.
In this instantiation we use $\wedge$ for both of these operators. It is clear that $\wedge$ is
monotonic in the sense of the ordering relation $\po$ defined above.

\subsection{Example}

Suppose Alice is a doctor at hospital $H$. She is allowed access to emergency room resources
only when she is on call. Let $p_c$ be true when Alice is on call. The hospital issues the
following $RT^R$ credential:

$H.e \leftarrow A, p_c$

Suppose further that the hospitals in the local area have formed a federation (with key $F$)
allowing them to share resources. The federation's membership is defined by the role $F.m$ and
the federation issues a credential such as

$F.m \leftarrow H, \top$

for each participating hospital. Assume the local police department (with key $P$) wants to
temporally grant emergency workers access to police reports in the case when an accident has
occurred. Specifically if $p_a$ is true when an accident is active, and $p_n$ is true when there
are unstablized injured persons, the police department might issue the credential.

$P.r \leftarrow F.m.e, p_a \wedge p_n$

Using these credentials, $RT^R$ finds the membership of $P.r$ to be the risk assessment $\{ (A,
\top \wedge p_c \wedge p_a \wedge p_n) \}$. Alice can thus be considered a member of the role
iff the formula $\top \wedge p_c \wedge p_q \wedge p_n$ evaluates to true.

\section{Role of Modal Logic}

Modal logic operates over a universe of states\footnote{In this document, we might also use the
  terms \textit{point}, \textit{world}, \textit{situation}, or \textit{instant}.}. Each formula
is evaluated in the context of a current state. In the case of temporal logic, this state
represents the current time. Modal formula then make statements about the nature of the states
(times) that can be reached from the current one. This approach makes sense in our case because
the authorizer will know the current context and is interested in proving statements or writing
policy about the states that can be reached from the current context. A situation is only
relevant if it can arise from the situation that exists \emph{now}. It is not necessary or even
desirable to consider every possible combination of contextual inputs, if some of those
combinations can never appear.

Modal logic provides a formal language for talking about the ways in which the universe might
evolve. The structure of the contextual information---the way in which states might change from
one to the next---is captured in an accessibility relation that underlies the semantics of the
modal logic used. The most appropriate proof system to use depends on this structure, and hence
on the nature and behavior of the contextual inputs.

There is much literature on the structure of time and on the corresponding modal axioms
necessary to capture that structure. Our initial goal, then, is to combine the ideas of temporal
logic with the $RT^R$ instantiation above. In this way we will build a system that allows
temporal reasoning about contextual inputs and role memberships.

\todo{Comment on the use of time intervals to model sessions.}

\section{Examples}

\todo{Describe one (or two?) compelling examples that motivate the theory and demonstrate its
  utility.}

\section{Implementation Issues}

\todo{Describe how the theory can be used to guide a verifiable implementation that dynamically
  and incrementally updates session information.}

\todo{Comment on the use of modal logic programming languages.}

\bibliographystyle{plain}
\bibliography{references}

\end{document}
