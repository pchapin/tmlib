%%%%%%%%%%%%%%%%%%%%%%%%%%%%%%%%%%%%%%%%%%%%%%%%%%%%%%%%%%%%%%%%%%%%%%%%%%%%
% FILE         : summsary.tex
% AUTHOR       : Peter C. Chapin, Christian Skalka, and Sean Wang
% LAST REVISED : 2005-03-16
% SUBJECT      : Summary of RT_0^D rules.
%
% This file contains a summary of the Li and TbV Datalog rules.
%
% Send comments or bug reports to:
%
%       Peter C. Chapin
%       University of Vermont
%       Department of Computer Science
%       Burlington, VT 05405
%       pchapin@cs.uvm.edu
%%%%%%%%%%%%%%%%%%%%%%%%%%%%%%%%%%%%%%%%%%%%%%%%%%%%%%%%%%%%%%%%%%%%%%%%%%%%

%+++++++++++++++++++++++++++++++++
% Preamble and global declarations
%+++++++++++++++++++++++++++++++++
\documentclass{article}

\setlength{\parindent}{0em}
\setlength{\parskip}{1.75ex plus0.5ex minus0.5ex}

% In case I want to change these later.
\newcommand{\TbV}{Trust-but-Verify}
\newcommand{\predicate}[1]{\texttt{#1}}

% Use whenever introducing a new term.
\newcommand{\newterm}[1]{\textit{#1}}

% Macros for credential formatting.
\newcommand{\memcert}[2]{\texttt{#1} $\leftarrow$ \texttt{#2}}
\newcommand{\delcert}[3]{\texttt{#1} $-$\texttt{(#2)}$\rightarrow$ \texttt{#3}}
\newcommand{\activate}[2]{\texttt{#1 as #2}}

%++++++++++++++++++++
% The document itself
%++++++++++++++++++++
\begin{document}

%-----------------------
% Title page information
%-----------------------
\title{Summary of $RT_0^D$ Datalog Rules}
% \author{Peter C. Chapin, Christian Skalka, and Sean Wang}
\date{March 16, 2005}
\maketitle

\section{Li/Mitchell/Winsborough Rules}

In these rules, read the predicate \predicate{forRole(C, B, A, r)} as
``\texttt{C} uses (can use) activation \activate{B}{A.r}.''

\begin{itemize}

\item \memcert{A.r}{D}

\texttt{forRole(d, d, a, r).}

\item \memcert{A.r}{B.r1}

\texttt{forRole(X, Y, a, r) :- forRole(X, Y, b, r1).}

\item \memcert{A.r}{A.r1.r2}
\begin{verbatim}
forRole(X, Y, a, r) :-
        forRole(Z, Z, a, r1),
        forRole(X, Y, Z, r2).
\end{verbatim}

\item \memcert{A.r}{B1.r1 $\cap \ldots \cap$ Bk.rk}
\begin{verbatim}
forRole(X, Y, a, r) :-
        forRole(X, Y, b1, r1),
        ...,
        forRole(X, Y, bk, rk).
\end{verbatim}

\item \delcert{B1}{D as A.r}{B2}

\texttt{forRole(b2, d, a, r) :- forRole(b1, d, a, r).}

\item \delcert{B1}{D as all}{B2}

\texttt{forRole(b2, d, X, Y) :- forRole(b1, d, X, Y).}

This credential delegates all role activations involving $D$ from $B_1$
to $B_2$.

\item \delcert{B1}{all}{B2}

\texttt{forRole(b2, X, Y, Z) :- forRole(b1, X, Y, Z).}

This credential delegates all of $B_1$'s role activations to $B_2$.

\end{itemize}

Requests in $RT_0^D$ are delegated role activations. For the purposes of
delegating activations to a request, a dummy entity is constructed to
represent the request. This entity is the target of one or more
delegation certificates. Access is then granted if
\texttt{forRole(reqID, X, a, r)} is true, where \texttt{reqID} is the
dummy entity representing the request and $A.r$ is the governing role
for the request. The value unified for $X$ is the entity making the
original request.

\section{TbV Rules}

Read the predicate \predicate{mem(b, a, r)} as asserting that $B$ is a
member of role $A.r$.

\begin{itemize}

\item \memcert{A.r}{B}

\texttt{mem(b, a, r).}

\item \memcert{A.r}{B.r1}

\texttt{mem(X, a, r) :- mem(X, b, r1).}

\item \memcert{A.r}{A.r1.r2}

\texttt{mem(X, a, r) :- mem(Z, a, r1), mem(X, Z, r2).}

\item \memcert{A.r}{B1.r1 $\cap \ldots \cap$ Bk.rk}
\begin{verbatim}
mem(X, a, r) :-
          mem(X, b1, r1),
          ...,
          mem(X, bk, rk).
\end{verbatim}

\end{itemize}

We introduce the predicate \predicate{validAct(a, b, r)} to represent
that \activate{A}{B.r} is a valid role activation. We introduce the
predicate \predicate{auth(a, b, r)} to represent that $A$ is authorized
to use the powers of role $B.r$.

\texttt{auth(X, Y, Z) :- validAct(X, Y, Z).}

\begin{itemize}

\item \memcert{A.r1}{B.r2}

\texttt{auth(X, a, r1) :- auth(X, b, r2).}

\item \memcert{A.r}{A.r1.r2}

\texttt{auth(X, a, r) :- mem(Z, a, r1), auth(X, Z, r2).}

\item \memcert{A.r}{B1.r1 $\cap \ldots \cap$ Bk.rk}
\begin{verbatim}
auth(X, a, r) :-
          auth(X, b1, r1),
          ...,
          auth(X, bk, rk).
\end{verbatim}
\end{itemize}

The predicate \predicate{has\_activation(A, B, C, r)} is true when
entity $A$ can use the activation \activate{B}{C.r}.

\texttt{has\_activation(A, A, X, Y) :- mem(A, X, Y).}

\begin{itemize}

\item \delcert{B1}{D as A.r}{B2}

\texttt{has\_activation(b2, d, a, r) :- has\_activation(b1, d, a, r).}

\item \delcert{B1}{D as all}{B2}

\texttt{has\_activation(b2, d, X, Y) :- has\_activation(b1, d, X, Y).}

\item \delcert{B1}{all}{B2}

\texttt{has\_activation(b2, X, Y, Z) :- has\_activation(b1, X, Y, Z).}

\end{itemize}

We introduce the predicate \predicate{act(a, b, r)} to represent that
the role activation \activate{A}{B.r} is part of the request.

An activation is valid if it is provided with the request, the immediate
requester $I$ has the activate (either by direct membership or via
delegation), and the entity mentioned in the activation is actually a
member of the role mentioned in the activation.

\begin{verbatim}
validAct(X, Y, Z) :-
            act(X, Y, Z),
            has_activation(i, X, Y, Z),
            mem(X, Y, Z).
\end{verbatim}

A request is authorized if

\texttt{auth(X, a, r).}

is true where $A.r$ is the governing role for the request. The value
unified for $X$ is the entity making the original request.

\end{document}
