%%%%%%%%%%%%%%%%%%%%%%%%%%%%%%%%%%%%%%%%%%%%%%%%%%%%%%%%%%%%%%%%%%%%%%%%%%%%
% FILE         : RT.tex
% AUTHOR       : (C) Copyright 2004 by Peter C. Chapin
% LAST REVISED : 2004-12-01
% SUBJECT      : Presentation on the RT Framework
%
%
% Send comments or bug reports to:
%
%       Peter C. Chapin
%       University of Vermont
%       Burlington, VT 05061
%       pchapin@cs.uvm.edu
%%%%%%%%%%%%%%%%%%%%%%%%%%%%%%%%%%%%%%%%%%%%%%%%%%%%%%%%%%%%%%%%%%%%%%%%%%%%

%+++++++++++++++++++++++++++++++++
% Preamble and global declarations
%+++++++++++++++++++++++++++++++++
\documentclass[landscape]{slides}

%\usepackage{amsmath}
%\usepackage{amssymb}
%\usepackage{amstext}
%\usepackage{mathpartir}

%
% For landscape mode slides:
%
%  1) use \documentclass[landscape]{slides}
%  2) dvips -tlandscape slides.dvi -o 
%
% This still gets the orientation wrong in the PS file.
% Only way I found to get a good pdf file was on the mac
%
\usepackage{graphics}
\usepackage{color}
\usepackage{epsfig}

%\setlength{\oddsidemargin}{-30pt}
%\setlength{\evensidemargin}{-30pt}
\setlength{\topmargin}{-0.75in}
\setlength{\textheight}{7in}

% colour defs
\input colordvi
\definecolor{headcolor}     {rgb}{0.55,0,0}
%\definecolor{itemcolor}    {rgb}{0.1,0.1,0.55}
%\definecolor{itemcolor}    {named}{SkyBlue}
%\definecolor{tickmarkcolor}{rgb}{0.5,0.1,0.5}
%\definecolor{tickmarkcolor}{named}{SkyBlue}
%\definecolor{titlecolor}   {rgb}{0.65,0.1,0.1}
%\definecolor{titlecolor}   {rgb}{0.7,0.5,0.9}
%\definecolor{tickmarkcolor}{rgb}{0.7,0.5,0.9}
%\definecolor{emphcolor}    {rgb}{0.7,0.5,0.9}
%\definecolor{titlecolor}   {rgb}{0.8,0.7,0.2}
%\definecolor{tickmarkcolor}{rgb}{0.85,0.84,0.1}
\definecolor{titlecolor}    {cmyk}{0.98,.3,0,0.43}
\definecolor{tickmarkcolor} {cmyk}{0.98,.3,0,0.43}
%\definecolor{emphcolor}    {rgb}{1.0,0.6,0.3}
\definecolor{emphcolor}     {named}{BrickRed}
\definecolor{refcolor}      {rgb}{0.,.4,.4}
%\def\Red#1{\Color{0 0.70 0.70 0.2}{#1}}
%\def\Green#1{\Color{0.70 0 0.5 0.2}{#1}}
%\def\Blue#1{\Color{0.70 0 0 0.2}{#1}}
\definecolor{Beige}         {rgb}{0.96,0.96,0.86}
%\definecolor{Background}    {rgb}{.97,.94,0.82}
\definecolor{Background}    {rgb}{1.,1.,1.}
\definecolor{MyYellow}      {rgb}{1.,0.84,0.8}
\definecolor{Gold}          {rgb}{1.0,0.84,0.}
\definecolor{White}         {named}{White}
%\definecolor{DarkGold}      {rgb}{.5,0.42,0.}
\definecolor{DarkGold}      {rgb}{.3,0.28,0.}
\definecolor{Blue}          {rgb}{0.,0.,1.}
\definecolor{Pink}          {rgb}{1.,0.75,0.8}
\definecolor{light-blue}    {rgb}{0.8,0.85,1}
\definecolor{mygrey}        {gray}{0.75}
\definecolor{darkgrey}      {gray}{0.40}
\definecolor{greenyellow}   {named}{GreenYellow}
\definecolor{myblack}       {named}{Black}

% Existing color names:
%
% GreenYellow, Yellow, Goldenrod, Dandelion, Apricot, Peach, Melon,
% YellowOrange, Orange, BurntOrange, Bittersweet, RedOrange, Mahogany,
% Maroon, BrickRed, Red, OrangeRed, RubineRed, WildStrawberry, Salmon,
% CarnationPink, Magenta, VioletRed, Rhodamine, Mulberry, RedViolet,
% Fuchsia, Lavender, Thistle, Orchid, DarkOrchid, Purple, Plum, Violet,
% RoyalPurple, BlueViolet, Periwinkle, CadetBlue, CornflowerBlue,
% MidnightBlue, NavyBlue, RoyalBlue, Blue, Cerulean, Cyan, ProcessBlue,
% SkyBlue, Turquoise, TealBlue, Aquamarine, BlueGreen, Emerald,
% JungleGreen, SeaGreen, Green, ForestGreen, PineGreen, LimeGreen,
% YellowGreen, SpringGreen, OliveGreen, RawSienna, Sepia, Brown, Tan,
% Gray, Black, White.

% new tick marks for itemize
\renewcommand{\labelitemi}{ {\color{tickmarkcolor}$\bullet$} }

% new tick marks for itemize
\renewcommand{\labelenumi}{
        {\color{tickmarkcolor} \arabic{enumi}}
}

% color versions of itemize/enumerate with less spacing
\newenvironment{citemize}{
        \vspace{-0.2in}
        \begin{itemize}
        \itemsep 0in
        %\color{itemcolor}
}{
        \end{itemize}
}

\newenvironment{cenumerate}{
        \vspace{-0.2in}
        \begin{enumerate}
        \itemsep 0in
        %\color{itemcolor}
}{
        \end{enumerate}
}

% headed slide takes 1 parameter: the heading of the slide, which will
% appear at the topright of the page
\newenvironment{hslide}[1]{
        \begin{slide}
        \begin{flushright}
                {\color{headcolor} \tiny #1}
        \end{flushright}
}{
        \end{slide}
}

% titled slide takes 1 parameter: the heading of the slide, which will
% appear at the topright of the page
\newenvironment{tslide}[1]{
        \begin{slide}
        \begin{flushleft}
                {\bf \color{headcolor} #1}
        \end{flushleft}
}{
        \end{slide}
}

% title - displayed as emphasised
\newcommand{\stitle}[1]{
        { \bf \color{titlecolor} #1 }
}

% comments
\newcommand{\mknote}[1]{
        \begin{note}
                \begin{flushright}
                        {\it NOTES}
                \end{flushright}
                {\scriptsize #1
                
                end}
        \end{note}
}

% end web borrowed macros

\def\cemph#1{{\color{emphcolor}\emph{#1}}}
\def\cref#1{{\color{refcolor}\small\emph{#1}}}
%\usepackage{times}
%\usepackage{mathptmx}
%%% beat slides class fonts into submission
%%% first, use most of package times
%%\renewcommand{\sfdefault}{phv}
%%\renewcommand{\rmdefault}{ptm}
%%\renewcommand{\ttdefault}{pcr}
%%% then, use most of package mathptm
%%\DeclareSymbolFont{operators}   {OT1}{ptmcm}{m}{n}
%%\DeclareSymbolFont{letters}     {OML}{ptmcm}{m}{it}
%%\DeclareSymbolFont{symbols}     {OMS}{pzccm}{m}{n}
%%\DeclareSymbolFont{largesymbols}{OMX}{psycm}{m}{n}
%%\DeclareSymbolFont{bold}        {OT1}{ptm}{bx}{n}
%%\DeclareSymbolFont{italic}      {OT1}{ptm}{m}{it}
%\setlength{\oddsidemargin}{-30pt}
%\setlength{\evensidemargin}{-30pt}
%\setlength{\textwidth}{7.4in}

%\long\def\inlong#1{{\def\titlefont{\it}{\it #1}}}
%\long\def\inlong#1{}
% slide macro abbreviations
\def\centerheader#1{\begin{center}
 {{\large\titlefont\color{titlecolor}#1}\par\vskip 1.5ex}
 \end{center}}
\def\titlefont{\bf}

\def\startslide#1{
\begin{slide}
\if !#1
 \else \centerheader{#1}
\fi
}
\def\stopslide{\end{slide}}

\newcounter{section} % make macro files load without complaining


%++++++++++++++++++++
% The document itself
%++++++++++++++++++++
\begin{document}

%-----------------------
% Title page information
%-----------------------
\title{The RT Trust Management Framework}
\author{Presented by Peter C. Chapin\\University of Vermont}
\date{December 1, 2004}
\maketitle

%%%%%

\startslide{References}

\begin{itemize}

\item N. Li, J. Mitchell, and W. Winsborough. \textit{Design of a Role-based Trust-management Framework}. In Proceedings of 2002 IEEE Symposium on Security and Privacy, pages 114--130. IEEE Computer Society Press, May 2002

\item M. Blaze, J. Feigenbaum, J. Lacy. \textit{Decentralized Trust Management}. In Proceedings of 1996 IEEE Symposium on Security and Privacy, pages 164--173, May 1996

\item D. Ferraiolo and R. Kuhn. \textit{Role-based access controls}. In Proceedings of 15th NIST-NCSC National Computer Security Conference, pages 554--563, Baltimore, MD, October 13-16 1992

\end{itemize}
\stopslide

%%%%%

\startslide{Outline}

\begin{itemize}

\item Trust Management

\item Role Based Access Control

\item RT: \textbf{R}ole Based \textbf{T}rust Management

\end{itemize}

\stopslide

%%%%%

\startslide{Trust Management}

\begin{itemize}

\item Traditional access control uses ``closed world'' assumption: all users are known to the system.

  \begin{itemize}
  \item Example: Unix or Windows
  \end{itemize}

\item In a decentralized, distributed system the closed world assumption is unreasonable.

	\begin{itemize}
	\item Infeasible to maintain a list of all potential users.
  \item Access must be granted based on certified characteristics of users (group membership, age, eye color, etc).
  \end{itemize}

\end{itemize}
\stopslide

%%%%%

\startslide{Who's There?}

Identity not particularly useful.

Example: We don�t know Jill Jones so knowing that she is requesting access isn�t helpful.
We might rather know that the client

\begin{itemize}
\item Is UVM CS faculty \textbf{OR}
\item Is a UVM CS student \textbf{AND} has a GPA $\ge$ 3.5
\end{itemize}

X.509 and PGP provide identity certificates. Useful in a closed world.
\stopslide

%%%%%

\startslide{Characteristics of TM Systems I}

Trust management systems provide a way of describing\ldots

\begin{enumerate}

\item Principals: entities to which access can be granted or trust extended. Public keys represent individuals (human and non-human), groups, roles, etc.

\item Actions or permissions: security sensitive operations. Often implicit in a request. Possibly high level. Application specific.

\item Policy: what actions can be done by which principals. Example: UVM students can read, UVM CS students can update.

\end{enumerate}

\stopslide

\startslide{Characteristics of TM Systems II}

\begin{enumerate}

\item Credentials: signed certificates. Proves the specified principal possesses a particular characteristic.

\item Trust relationships: who is trusted to certify what. For example, the UVM registrar is trusted to certify UVM student status.

\end{enumerate}
\stopslide

%%%%%

\startslide{Delegation}

There are two kinds of delegation.

\begin{itemize}

\item Delegation of trust: Who is trusted to certify characteristics? Example: Doctors can certify height. The AMA can certify doctors.

\item Delegation of authority: Who can act on a principal's behalf in a transaction?

\end{itemize}

First form is an essential part of trust management. Second form is optional, but often useful.

\stopslide

%%%%%

\startslide{Anatomy of a Request}

\input request
\centerline{\box\graph}

\stopslide

%%%%%

\startslide{Role Based Access Control (RBAC)}

Introduced to simplify administration of access control.

\begin{itemize}

\item Roles define both users and permissions.

\item Easy to look up both role members and role permissions; simply consult the role definition.

\item In contrast groups only define users. Permissions are defined separately (on ACLs, for example) and are thus harder to manage.

\item RBAC not discretionary. Only creator of role can modify role permissions. Resource owners not necessarily involved.

\end{itemize}

\stopslide

%%%%%

\startslide{More RBAC}

Roles can be used in several ways depending on the system's security needs.

\begin{itemize}

\item Organize users into groups. Example: UVM CS students.

\item Represent permissions. Example: Print on network printer. Role members have the indicated permission.

\item Represent attributes. Example: People more than 6 feet tall. Role members have the indicated attribute (ABAC).

\end{itemize}

Some systems allow roles to contain other roles forming role hierarchies.

\stopslide

%%%%%

\startslide{RT Framework}

RT combines ideas from trust management systems and RBAC.

\begin{itemize}

\item Trust management systems provide notion of trust delegation.

\item RBAC provides notion of flexible roles.

\item RT offers several variations: $RT_0$, $RT_1$, $RT_2$, $RT^T$, $RT^D$, $RT^C$, etc.  Variations trade off expressiveness and complexity.

\end{itemize}

\stopslide

%%%%%

\startslide{$RT_0$ Roles}

\begin{itemize}
\item Role names local to principals (like SDSI/SPKI).
\item Roles defined by specifying role members.
\item Some roles defined as part of local policy, some as signed certificates.
\item Role membership is the union of all policy and certificate definitions.
\end{itemize}

\stopslide

%%%%%

\startslide{Membership Assertions I}

$A.R \leftarrow B$

Principal $B$ is in $A$'s role $R$. Example: \texttt{WS.student} $\leftarrow$ \texttt{Alice}

$A.R \leftarrow B.R_1$

Members of $B$'s role $R_1$ are in $A$'s role $R$. This is trust delegation. Example: \texttt{WS.student} $\leftarrow$ \texttt{UVM.student}

\stopslide

%%%%%

\startslide{Membership Assertions II}

$A.R \leftarrow A.R_1.R_2$

$A$'s role $R$ contains all members of $B$'s $R_2$ role if $B$ is a member of $A$'s $R_1$ role. Example: \texttt{WS.student} $\leftarrow$ \texttt{WS.university.student} with an appropriate definition of \texttt{WS.university}.

$A.R \leftarrow B_1.R_1 \cap B_2.R_2 \cap \ldots \cap B_n.R_n$

$A$'s role $R$ contains the entities that are members of all the roles $B_1.R_1$, $B_2.R_2$, etc. Example: \texttt{WS.student} $\leftarrow$ \texttt{UVM.student} $\cap$ \texttt{UVM.gradcollege}.

\stopslide

%%%%%

\startslide{Example}

WS's policy:

\texttt{WS.readsite} $\leftarrow$ \texttt{WS.student}\\
\texttt{WS.student} $\leftarrow$ \texttt{WS.university.student}\\
\texttt{WS.university} $\leftarrow$ \texttt{ABU.accredited}

Submitted with Alice's request to read the site:

\texttt{UVMregistrar.student} $\leftarrow$ \texttt{Alice}\\
\texttt{UVM.student} $\leftarrow$ \texttt{UVMregistrar.student}\\
\texttt{ABU.accredited} $\leftarrow$ \texttt{UVM}

Access is granted.

\stopslide

\startslide{Datalog}

\begin{itemize}

\item RT semantics defined in terms of Datalog.

\item Datalog is a subset of Prolog that lacks function symbols and negation. Introduced as a database query language supporting recursion.

\item RT assertions can be translated into Datalog rules and facts.

\item Access is granted if the necessary fact can be proven.

\end{itemize}

\stopslide

%%%%%

\startslide{Example (Continued)}

\begin{verbatim}
% Policy
member(ws, readsite, X) :- member(ws, student, X).
member(ws, student, X) :- member(ws, university, Y),
                          member(Y, student, X).
member(ws, university, X) :- member(abu, accredited, X).

% Credentials submitted with request
member(uvmregistrar, student, alice).
member(uvm, student, X) :- member(uvmregistrar, student, X).
member(abu, accredited, uvm).
\end{verbatim}

Computing \texttt{member(ws, readsite, alice)} returns ``yes''.

\stopslide

%%%%%

\startslide{$RT_1$}

$RT_1$ adds parameterized roles.

\begin{itemize}

\item \texttt{WidgetsInc.managerOf(Alice)} $\leftarrow$ \texttt{Bob}
\item \texttt{WidgetsInc.evaluatorOf(?X)} $\leftarrow$ \texttt{WidgetsInc.managerOf(?X)}
\item \texttt{UVM.recentAlumni} $\leftarrow$ \texttt{UVM.graduated(?Year:[2001..2004])}
\item \texttt{UVM.graduated(2001)} $\leftarrow$ \texttt{Alice}
\end{itemize}

Multiple parameters allowed on a role. Parameters are typed and can take on discrete integer, float, or enumeration values.

$RT_1^C$ enriches role parameters by allowing more expressive constraint domains.

\stopslide

%%%%%

\startslide{Other RTs}

\begin{itemize}
\item $RT_2$. Supports parameterized object sets.
\item $RT^T$. Supports threshold and separation of duty policies. 
\item $RT^D$. Supports role activations and delegation of those activations.
\item $RT^C$. Supports structured resources (semantics requires Datalog with constraints).
\end{itemize}

The various options can be mixed: $RT_1^D$, $RT_2^{DT}$, etc.

\stopslide

%%%%%

\startslide{Questions}

\begin{center}\Large ?\end{center}

\stopslide

\end{document}
