%%%%%%%%%%%%%%%%%%%%%%%%%%%%%%%%%%%%%%%%%%%%%%%%%%%%%%%%%%%%%%%%%%%%%%%%%%%%
% FILE         : PolicyMaker-POC.tex
% AUTHOR       : (C) Copyright 2005 by Peter C. Chapin
% LAST REVISED : 2005-05-02
% SUBJECT      : Presentation on "Compliance Checking in the PolicyMaker..."'
%
% Send comments or bug reports to:
%
%       Peter C. Chapin
%       University of Vermont
%       Burlington, VT 05061
%       pchapin@cem.uvm.edu
%%%%%%%%%%%%%%%%%%%%%%%%%%%%%%%%%%%%%%%%%%%%%%%%%%%%%%%%%%%%%%%%%%%%%%%%%%%%

%+++++++++++++++++++++++++++++++++
% Preamble and global declarations
%+++++++++++++++++++++++++++++++++
\documentclass[landscape]{slides}

\usepackage{amsmath}
\usepackage{amssymb}
\usepackage{amstext}
%\usepackage{stmaryrd}
%\usepackage{mathpartir}

\input{slide-macros}

%++++++++++++++++++++
% The document itself
%++++++++++++++++++++
\begin{document}

%-----------------------
% Title page information
%-----------------------
\title{Compliance Checking in PolicyMaker}
\author{Presented by Peter C. Chapin\\University of Vermont}
\date{May 2, 2005}
\maketitle

%%%%%
\startslide{Outline}

\begin{itemize}
\item Overview of PolicyMaker
\item Proof of Compliance Problem
\item Locally Bounded Proof of Compliance Problem
\item Polynomial Time Variant
\end{itemize}

\stopslide

%%%%%
\startslide{References}

\begin{itemize}
\item ``Compliance Checking in the PolicyMaker Trust Management System.'' Matt Blaze, Joan Feigenbaum, and Martin Strauss. \textit{Proceedings of the Second International Conference on Financial Cryptography}, pp 254-274. Published by Springer-Verlag, 1998.
\item ``Decentralized Trust Management.'' Matt Blaze, Joan Feigenbaum, and Jack Lacy. \textit{Proceedings of the 1996 IEEE Symposium on Security and Privacy}, pp 164-173. Published by IEEE Computer Society Press, May 1996.
\end{itemize}

\stopslide

%%%%%
\startslide{Trust-Management/PolicyMaker}
A method for handling distributed authorization.
\begin{itemize}
\item Authorizer writes a policy to describe who can have access to a resource.
\item Requester sends signed credentials intended to prove compliance with policy.
\item Authorizer executes a ``proof of compliance'' algorithm that returns YES if requester complies with policy; NO otherwise.
\end{itemize}
In PolicyMaker the policy and credential assertions are arbitrary functions.
\stopslide

%%%%%
\startslide{Assertions}

Assertions are represented with the pair $(f_i, s_i)$ where
\begin{itemize}
  \item $f_i$ is the function
  \item $s_i$ is an identifier that represents the source of the assertion.
\end{itemize}

The policy is represented as an assertion with source POLICY. We'll use $f_0$ for the policy function.

\stopslide

%%%%%
\startslide{Acceptance Records}

Assertions produce \textit{acceptance records} of the form $(i, s_i, R_{ij})$ where

\begin{itemize}
\item $i$ is the assertion number that generated the record.
\item $s_i$ is the source identifier of the assertion that generated the record.
\item $R_{ij}$ is an \textit{action string} that specifies the $j^{th}$ possible action assertion $i$ is authorizing.
\end{itemize}

Action strings are strings over some alphabet $\Sigma$. The request itself is encoded as an action string.

\stopslide

%%%%%
\startslide{Acceptance Sets}

\begin{itemize}
\item The compliance checker maintains a workspace consisting of a set of acceptance records.
\item The assertions take this workspace as an argument and return an updated workspace.
\item Workspace initialized to $\{(\Lambda, \Lambda, R)\}$. Where $R$ is the encoding of the request.
\end{itemize}

\stopslide

%%%%%
\startslide{Proof of Compliance (POC)}

\textbf{Input:} A request $r$ and a set $\{(f_0, \mbox{POLICY}), (f_1, s_1), \ldots, (f_{n-1}, s_{n-1})\}$ of assertions.

\textbf{Question:} Is there a finite sequence $i_1, i_2, \ldots, i_t$ of indicies such that each $i_j$ is in $\{0, 1, \ldots, n-1\}$, but the $i_j$'s are not necessarily distinct and not necessarily exhaustive of $\{0, 1, \ldots, n-1\}$ and such that
\begin{displaymath}
  (0, \mbox{POLICY}, R) \in (f_{i_t}, s_{i_t}) \circ \cdots \circ (f_{i_1}, s_{i_1})(\{\Lambda, \Lambda, R)\})
\end{displaymath}

where $R$ is the encoding of the request as an action string.
\stopslide

%%%%%
\startslide{Promise Problems}
\begin{itemize}
\item Two predicates $(Q, R)$. 
\item Machine $M$ solves the problem if
\begin{displaymath}
\forall w (Q(w) \rightarrow (M_h(w) \wedge (M_a(w) \leftrightarrow R(w))))
\end{displaymath}
\item class NPP contains all promise problems with at least one solution in NP.
\item Promise problem is NP-hard if all solutions are NP-hard.
\end{itemize}
To prove NP-hardness, reduce an NP-hard language to an instance of the promise problem where $Q$ is always satisfied.
\stopslide

%%%%%
\startslide{Locally Bounded POC (LBPOC)}
\textbf{Input:} A request $r$, a set $\{(f_0, \mbox{POLICY}), (f_1, s_1), \ldots, (f_{n-1}, s_{n-1})\}$ of assertions, and integers $c$, $l$, $m$, and $s$.

\textbf{Promise:} Each $f_i$ runs in time $O(N^c)$. On any input set that contains $(\Lambda, \Lambda, R)$, for each $f_i$ there is a set $O_i$ of at most $m$ action strings such that any acceptance record produced by $f_i$ has an action string $R_{ij} \in O_i$. Furthermore, $s$ is the maximum size of an acceptance record produced by $f_i$.

\textbf{Question:} Is there a sequence $i_1, \ldots, i_t$, as before, such that $t \le l$ and
\begin{displaymath}
  (0, \mbox{POLICY}, R) \in (f_{i_t}, s_{i_t}) \circ \cdots \circ (f_{i_1}, s_{i_1})(\{\Lambda, \Lambda, R)\})
\end{displaymath}

\stopslide

%%%%%
\startslide{LBPOC $\in$ NPP}

\begin{itemize}
\item Non-deterministically guess $i_1, \ldots, i_t$.
\item Simulate assertions in guessed sequence.
\item Check if $(0, \mbox{POLICY}, R)$ is in the resulting acceptance set.
\end{itemize}

This is clearly polynomial: There are at most $l$ assertions. Each assertion runs in time $O(N^c)$. Thus overall running time of verification is $O(l(nms)^c)$

However, LBPOC is NP-hard.

\stopslide

%%%%%
\startslide{Bounded Post Correspondence (BPCP)}
\textbf{Input:} A finite alphabet $\Sigma$, two sequences $a = (a_1, a_2, \ldots, a_q)$ and $b = (b_1, b_2, \ldots, b_q)$ of strings from $\Sigma^{*}$, and a positive integer $K \le q$.

\textbf{Question:} Is there a sequence $i_1, i_2, \ldots, i_k$ of $k \le K$ (not necessarily distinct) positive integers between 1 and $q$, such that the two strings $a_{i_1} a_{i_2} \cdots a_{i_q}$ and $b_{i_1} b_{i_2} \cdots b_{i_q}$ are identical?

This problem is NP-complete (and thus NP hard). Note that the unbounded case is undecidable.

\stopslide

%%%%%
\startslide{LBPOC is NP-Hard}

Proof by reducing, in polynomial time, an arbitrary instance of BPCP to a corresponding instance of LBPOC where the promise is satisfied.

\begin{itemize}
\item Request and action string irrelevant in reduction. Use $\sigma$ as placeholder in LBPOC instance.
\item Sources of assertions irrelevant in reduction, except for the POLICY assertion. Use $\sigma$ as placeholder in LBPOC instance.
\item LBPOC instance has $n = q + 1$ assertions.
\item For $1 \le i \le q$, assertion $(f_i, \sigma)$ outputs $S \cup \{(i, \sigma, (a_i, b_i, |S|))\}$.
\item Let $c = 2$, $l = K + 1$, $m = K$.
\end{itemize}

\stopslide

%%%%%
\startslide{Proof (continued)}
Introduce assertion $(f_0, \mbox{POLICY})$ on input $S$ as follows

\begin{itemize}
\item Sort $S$ into increasing order with respect to the size field in the action string. Let $i_1, \ldots, i_t$ be the resulting sorted sequence of first coordinates of acceptance records.
\item If $a_{i_1} \cdots a_{i_t} = b_{i_1} \cdots b_{i_t}$, output $\{(0, \mbox{POLICY}, R)\}$, otherwise output the empty set.
\end{itemize}

The reduction is polynomial time. The resulting LBPOC instance satisfies the promise and accepts iff the original BPCP instance accepts.

\stopslide

%%%%%
\startslide{Monotonic Assertions}

Assertion $f_i$ is monotonic if, for all acceptance sets $A$ and $B$
\begin{displaymath}
A \subseteq B \rightarrow (f_i)(A) \subseteq (f_i)(B)
\end{displaymath}

In other words, if $f_i$ approves an action given evidence $A$, $f_i$ will continue to approve that action given addition evidence. There is no ``negative'' evidence.

LBPOC with monotonic assertions is still NP-hard\ldots even after removing restriction on using only $l$ assertions.

\stopslide

%%%%%
\startslide{Agglomeration}

Instead of composing the results of multiple assertions ($f_i \circ f_j$), we can agglomerate them ($f_i \star f_j$).

\begin{itemize}
\item $S_j = f_{i_j}(S_{j-1}) \cup S_{j-1}$
\end{itemize}

Thus assertions can't ``undo'' the effects of other assertions by removing their acceptance records from the work space. This is different than monotonicity.

Authors claim that agglomerative versions of LBPOC, including LBPOC with monotonic assertions are still NP-hard.

\stopslide

%%%%%
\startslide{Authenticity}

\begin{itemize}
\item An assertion $f_i$ is authentic if the new acceptance records that it outputs are only in the form $(i, s_i, R_{ij})$.
\item The assertion does not attempt to impersonate another assertion by outputting a record of the form $(i', s_{i'}, R_{i'j})$.
\item So far assertions have not needed to be authentic. They could have output an acceptance set containing new acceptance records ``produced'' by other assertions.
\end{itemize}

LBPOC using authentic, monotonic assertions and agglomeration is in P.

\stopslide

%%%%%
\startslide{Compliance Checking Algorithm}
\begin{verbatim}
CCA(r, {n assertions}, c, m, s)
{
  S <- {(_, _, R)}
  I <- { }
  for j <- 1 to nm {
    for i <- n-1 to 0 {
      if f_i notin I then S' <- f_i(S)
      if IllFormed(f_i) then I <- I union {f_i}
        else S <- S union S'
    }
  }
  if (0, POLICY, R) in S then return accept
  return reject
}
\end{verbatim}
\stopslide

%%%%%
\startslide{Proof of CCA}

Running time: $O(mn^2(nms)^c) = O(n^{(2+c)})$

Does it work?
\begin{itemize}
\item Let acceptance set be $A$ when the inner loop starts.
\item Each assertion is executed on a superset of $A$.
\item Each assertion authorizes every action $A$ implies to that assertion.
\item Multiple passes ensure that all possible authorizations are generated.
\end{itemize}

\stopslide

%%%%%
\startslide{Proof of CCA (continued)}
\begin{itemize}
\item Monotonicity: Assertions generate the same authorized action strings despite ``noise'' of authorizations generated by other assertions.
\item Agglomeration: Assertions can't ``fight'' each other by authorizing and unauthorizing each other's action strings.
\item Authenticity: \textit{I do not understand why this is necessary.}
\end{itemize}

\stopslide

%%%%%
\startslide{Questions}
\begin{center}\Large ?\end{center}
\stopslide

\end{document}
