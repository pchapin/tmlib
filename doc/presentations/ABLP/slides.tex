
\newcommand{\as}{\textit{ as }}
\newcommand{\controls}{\textit{ controls }}
\newcommand{\for}{\textit{ for }}
\renewcommand{\implies}{\supset}
\newcommand{\says}{\textit{ says }}
\newcommand{\serves}{\textit{ serves }}
\newcommand{\speaksfor}{\Rightarrow}

\startslide{Reference}

``Calculus for Access Control in Distributed Systems'' by Abadi,
Burrows, Lampson, and Plotkin. 1993. \textit{ACM Transactions on
  Programming Languages and Systems}. 15(4), pp 706-734.

Some figures and examples taken from ``Trust But Verify: Authorization
for Web Services'' by Skalka and Wang. 2004. \textit{ACM Workshop on
  Secure Web Services}. Submitted.

\stopslide

%%%

\startslide{ABLP Logic Overview}

ABLP Logic is a formal system for expressing resource requests and for
making access control decisions about those requests.

\begin{citemize}
\item Allows access requests and access control lists to be clearly defined.
\item Allows one to reason about the authorization system is a precise way.
\end{citemize}

\stopslide

%%%

\startslide{Principals}

\begin{citemize}
  \item Users
  \item Machines
  \item I/O channels
  \item Encryption keys
\end{citemize}

Simple requests have form ``$P \says s$'' where $s$ is a proposition
being asserted by principal $P$. For example ``$\text{Carol} \says
\mathrm{\bf read}$'' means Carol requests read access.

Other statements besides requests are also possible.

\stopslide

%%%

\startslide{Syntax: Principal Expressions}

$$\begin{array}{rcl@{\hspace{10mm}}r}
A,B,C,\ldots &\in& \mathit{Atom} & \gdesc{atomic principals} \\
P &::=& A \mid  P \wedge P \mid P|P & \gdesc{principals}\\
\end{array}$$

Principals can be compound.

\begin{citemize}
  \item For $P_1 \wedge P_2$ read ``$P_1$ and $P_2$''
  \item For $P_1|P_2$ read ``$P_1$ quoting $P_2$''
\end{citemize}

\stopslide

%%%

\startslide{Inference Rules: Part 1}

These rules formally define the meaning of the compound principals.
\begin{mathpar}
\inferrule[Agreement]{}{\vdash (A \wedge B) \says s \equiv (A \says s)
  \wedge (B \says s)}

\inferrule[Quote]{}{\vdash (B|A) \says s \equiv B \says (A \says s)}
\end{mathpar}

For example, the above imply that $(A \wedge B)|C \equiv (A|C) \wedge
(B|C)$.
\begin{eqnarray*}
  ((A \wedge B)|C) \says s & \equiv & (A \wedge B) \says (C \says s) \\
                           & \equiv & (A \says (C \says s)) \wedge
                                      (B \says (C \says s)) \\
                           & \equiv & ((A|C) \wedge (B|C)) \says s
\end{eqnarray*}

\stopslide
%%%

\startslide{Syntax: Primitive Formulae}

$$\begin{array}{rcl@{\hspace{10mm}}r}
p &\in& \mathit{Prop}   & \gdesc{primitive propositions} \\
s &::=&  p \mid \neg s \mid s \wedge s \mid P \speaksfor P \mid P \says s & \gdesc{formulae}
\end{array}$$

\begin{citemize}
  \item Other propositional logic connectives defined in terms of $\neg$
    and $\wedge$ in the usual way.
  \item For $P_1 \speaksfor P_2$ read ``$P_1$ speaks for $P_2$''
\end{citemize}

\stopslide

%%%

\startslide{Inference Rules: Part 2}

\begin{mathpar}
\inferrule[Taut]{\text{$s$ is a tautology of propositional logic}}{\vdash s}

\inferrule[Modus Ponens]{\vdash s' \\ \vdash s' \implies s}{\vdash s}

\inferrule[Subtext]{}{\vdash A \says (s \implies s') \implies (A \says s
  \implies A \says s')}

\inferrule[Parrot]{\vdash s}{\vdash A \says s}

\inferrule[Speaksfor]{}{\vdash (A \speaksfor B) \implies ((A \says s)
  \implies (B \says s))}
\end{mathpar}

Group membership can be modeled with ``speaks for''. Thus ``$P
\speaksfor G$'' can be used to indicate that $P$ is a member of group
$G$.

\stopslide

%%%

\startslide{Example Inference}

\begin{displaymath}
  K_J \speaksfor J
\end{displaymath}

A request for $s$ signed by $K_J$ is
\begin{displaymath}
  K_J \says s
\end{displaymath}

Thus
$$\begin{array}{c}
  (K_J \speaksfor J) \implies (K_J \says s \implies J \says s) \\
  K_J \says s \implies J \says s \\
  J \says s
\end{array}$$

Thus a message signed by $K_J$ can be taken as the word of $J$.
Consequently $K_J \speaksfor J$ is a statement of the idea that private
keys are private.
\stopslide

%%%

\startslide{Access Control Lists}

Let ``$P \controls s$'' be shorthand for ``$(P \says s) \implies s$.''

An access control list is a list of statements of the form $P \controls
s$ for various principal expressions $P$.
\begin{eqnarray*}
  \text{Carol} & \controls & \mathrm{\bf read} \\
  (\text{Dave} \wedge \text{Host}) & \controls & \mathrm{\bf enumerate} \\
  \text{Eric} & \controls & (\text{Carol} \says \mathrm{\bf read})
\end{eqnarray*}

The last ACE says
$$\begin{array}{c}
  (E \says (C \says r)) \implies (C \says r) \\
  ((E|C) \says r) \implies (C \says r)
\end{array}$$

\stopslide

%%%

\startslide{Access Control Decisions}

Form conjunction of request and ACL. Use inference rules to derive
necessary priviledge.

\begin{citemize}
  \item $\text{Carol} \says \mathrm{\bf read}$: Request
  \item $\text{Carol} \controls \mathrm{\bf read}$: ACL
\end{citemize}

$$\begin{array}{c}
  (C \says r) \wedge ((C \says r) \implies r)
\end{array}$$

By \textsc{Modus Ponens} $r$ can be derived immediately.

\stopslide

%%%

\startslide{Delegation}

Let there be a distinguished principal $D$ that acts as a delegation
server. If $A$ wishes to delegate to $B$ then $A$ says
$$(B \says (A \says s)) \implies (D \says (A \says s))$$

In other words
$$A \says (B|A \speaksfor D|A)$$

Since $B|A \says s$ and $D|A \says s$ then $((B \wedge D)|A) \says s$.
Let ``$B \for A$'' be shorthand for $(B \wedge D)|A$. Also let ``$B
\serves A$'' be shorthand for $(B|A \speaksfor D|A)$.

\stopslide

%%%

\startslide{Delegation Example}

A request comes from $B$ who has been previously authenticated:
\begin{citemize}
  \item $K_A \speaksfor A$: Private keys are private.
  \item $K_A \says (B \serves A)$: Delegation certificate.
  \item $B|A \says r$: Raw request.
\end{citemize}

ACL contains:
\begin{citemize}
  \item $A \controls (B \serves A)$: $A$ is allowed to delegate to $B$.
  \item $(B \for A) \controls r$: $B$ is allowed to read on behalf of $A$
\end{citemize}

\stopslide

%%%

\startslide{Delegation Derivation}

\begin{eqnarray*}
  (K_A \speaksfor A) \wedge (K_A \says (B \serves A)) & \implies & (A \says
    (B \serves A)) \\
  (A \says (B \serves A)) \wedge (A \controls (B \serves A)) & \implies
    & (B \serves A) \\
  (B \serves A) \wedge (B|A \says r) & \implies & (D|A \says r) \\
  (B|A \says r) \wedge (D|A \says r) & \implies & (B \for A) \says r \\
  (B \for A \says r) \wedge (B \for A \controls r) & \implies & r
\end{eqnarray*}

Note that $D$ can be fictitious. All that is required is for $A$ and the
server to agree on a $D$.

\stopslide
