%%%%%%%%%%%%%%%%%%%%%%%%%%%%%%%%%%%%%%%%%%%%%%%%%%%%%%%%%%%%%%%%%%%%%%%%%%%%
% FILE         : DistAuth.tex
% AUTHOR       : (C) Copyright 2005 by Peter C. Chapin
% LAST REVISED : 2005-04-15
% SUBJECT      : 
%
% Send comments or bug reports to:
%
%       Peter C. Chapin
%       University of Vermont
%       Burlington, VT 05061
%       pchapin@cs.uvm.edu
%%%%%%%%%%%%%%%%%%%%%%%%%%%%%%%%%%%%%%%%%%%%%%%%%%%%%%%%%%%%%%%%%%%%%%%%%%%%

%+++++++++++++++++++++++++++++++++
% Preamble and global declarations
%+++++++++++++++++++++++++++++++++
\documentclass[landscape]{slides}

\usepackage{amsmath}
\usepackage{amssymb}
\usepackage{amstext}
\usepackage{stmaryrd}
%\usepackage{mathpartir}

%
% For landscape mode slides:
%
%  1) use \documentclass[landscape]{slides}
%  2) dvips -tlandscape slides.dvi -o 
%
% This still gets the orientation wrong in the PS file.
% Only way I found to get a good pdf file was on the mac
%
\usepackage{graphics}
\usepackage{color}
\usepackage{epsfig}

%\setlength{\oddsidemargin}{-30pt}
%\setlength{\evensidemargin}{-30pt}
\setlength{\topmargin}{-0.75in}
\setlength{\textheight}{7in}

% colour defs
\input colordvi
\definecolor{headcolor}     {rgb}{0.55,0,0}
%\definecolor{itemcolor}    {rgb}{0.1,0.1,0.55}
%\definecolor{itemcolor}    {named}{SkyBlue}
%\definecolor{tickmarkcolor}{rgb}{0.5,0.1,0.5}
%\definecolor{tickmarkcolor}{named}{SkyBlue}
%\definecolor{titlecolor}   {rgb}{0.65,0.1,0.1}
%\definecolor{titlecolor}   {rgb}{0.7,0.5,0.9}
%\definecolor{tickmarkcolor}{rgb}{0.7,0.5,0.9}
%\definecolor{emphcolor}    {rgb}{0.7,0.5,0.9}
%\definecolor{titlecolor}   {rgb}{0.8,0.7,0.2}
%\definecolor{tickmarkcolor}{rgb}{0.85,0.84,0.1}
\definecolor{titlecolor}    {cmyk}{0.98,.3,0,0.43}
\definecolor{tickmarkcolor} {cmyk}{0.98,.3,0,0.43}
%\definecolor{emphcolor}    {rgb}{1.0,0.6,0.3}
\definecolor{emphcolor}     {named}{BrickRed}
\definecolor{refcolor}      {rgb}{0.,.4,.4}
%\def\Red#1{\Color{0 0.70 0.70 0.2}{#1}}
%\def\Green#1{\Color{0.70 0 0.5 0.2}{#1}}
%\def\Blue#1{\Color{0.70 0 0 0.2}{#1}}
\definecolor{Beige}         {rgb}{0.96,0.96,0.86}
%\definecolor{Background}    {rgb}{.97,.94,0.82}
\definecolor{Background}    {rgb}{1.,1.,1.}
\definecolor{MyYellow}      {rgb}{1.,0.84,0.8}
\definecolor{Gold}          {rgb}{1.0,0.84,0.}
\definecolor{White}         {named}{White}
%\definecolor{DarkGold}      {rgb}{.5,0.42,0.}
\definecolor{DarkGold}      {rgb}{.3,0.28,0.}
\definecolor{Blue}          {rgb}{0.,0.,1.}
\definecolor{Pink}          {rgb}{1.,0.75,0.8}
\definecolor{light-blue}    {rgb}{0.8,0.85,1}
\definecolor{mygrey}        {gray}{0.75}
\definecolor{darkgrey}      {gray}{0.40}
\definecolor{greenyellow}   {named}{GreenYellow}
\definecolor{myblack}       {named}{Black}

% Existing color names:
%
% GreenYellow, Yellow, Goldenrod, Dandelion, Apricot, Peach, Melon,
% YellowOrange, Orange, BurntOrange, Bittersweet, RedOrange, Mahogany,
% Maroon, BrickRed, Red, OrangeRed, RubineRed, WildStrawberry, Salmon,
% CarnationPink, Magenta, VioletRed, Rhodamine, Mulberry, RedViolet,
% Fuchsia, Lavender, Thistle, Orchid, DarkOrchid, Purple, Plum, Violet,
% RoyalPurple, BlueViolet, Periwinkle, CadetBlue, CornflowerBlue,
% MidnightBlue, NavyBlue, RoyalBlue, Blue, Cerulean, Cyan, ProcessBlue,
% SkyBlue, Turquoise, TealBlue, Aquamarine, BlueGreen, Emerald,
% JungleGreen, SeaGreen, Green, ForestGreen, PineGreen, LimeGreen,
% YellowGreen, SpringGreen, OliveGreen, RawSienna, Sepia, Brown, Tan,
% Gray, Black, White.

% new tick marks for itemize
\renewcommand{\labelitemi}{ {\color{tickmarkcolor}$\bullet$} }

% new tick marks for itemize
\renewcommand{\labelenumi}{
        {\color{tickmarkcolor} \arabic{enumi}}
}

% color versions of itemize/enumerate with less spacing
\newenvironment{citemize}{
        \vspace{-0.2in}
        \begin{itemize}
        \itemsep 0in
        %\color{itemcolor}
}{
        \end{itemize}
}

\newenvironment{cenumerate}{
        \vspace{-0.2in}
        \begin{enumerate}
        \itemsep 0in
        %\color{itemcolor}
}{
        \end{enumerate}
}

% headed slide takes 1 parameter: the heading of the slide, which will
% appear at the topright of the page
\newenvironment{hslide}[1]{
        \begin{slide}
        \begin{flushright}
                {\color{headcolor} \tiny #1}
        \end{flushright}
}{
        \end{slide}
}

% titled slide takes 1 parameter: the heading of the slide, which will
% appear at the topright of the page
\newenvironment{tslide}[1]{
        \begin{slide}
        \begin{flushleft}
                {\bf \color{headcolor} #1}
        \end{flushleft}
}{
        \end{slide}
}

% title - displayed as emphasised
\newcommand{\stitle}[1]{
        { \bf \color{titlecolor} #1 }
}

% comments
\newcommand{\mknote}[1]{
        \begin{note}
                \begin{flushright}
                        {\it NOTES}
                \end{flushright}
                {\scriptsize #1
                
                end}
        \end{note}
}

% end web borrowed macros

\def\cemph#1{{\color{emphcolor}\emph{#1}}}
\def\cref#1{{\color{refcolor}\small\emph{#1}}}
%\usepackage{times}
%\usepackage{mathptmx}
%%% beat slides class fonts into submission
%%% first, use most of package times
%%\renewcommand{\sfdefault}{phv}
%%\renewcommand{\rmdefault}{ptm}
%%\renewcommand{\ttdefault}{pcr}
%%% then, use most of package mathptm
%%\DeclareSymbolFont{operators}   {OT1}{ptmcm}{m}{n}
%%\DeclareSymbolFont{letters}     {OML}{ptmcm}{m}{it}
%%\DeclareSymbolFont{symbols}     {OMS}{pzccm}{m}{n}
%%\DeclareSymbolFont{largesymbols}{OMX}{psycm}{m}{n}
%%\DeclareSymbolFont{bold}        {OT1}{ptm}{bx}{n}
%%\DeclareSymbolFont{italic}      {OT1}{ptm}{m}{it}
%\setlength{\oddsidemargin}{-30pt}
%\setlength{\evensidemargin}{-30pt}
%\setlength{\textwidth}{7.4in}

%\long\def\inlong#1{{\def\titlefont{\it}{\it #1}}}
%\long\def\inlong#1{}
% slide macro abbreviations
\def\centerheader#1{\begin{center}
 {{\large\titlefont\color{titlecolor}#1}\par\vskip 1.5ex}
 \end{center}}
\def\titlefont{\bf}

\def\startslide#1{
\begin{slide}
\if !#1
 \else \centerheader{#1}
\fi
}
\def\stopslide{\end{slide}}

\newcounter{section} % make macro files load without complaining

\newtheorem{condition}{Condition}
\newcommand{\priv}{\mathbf{priv}}
\newcommand{\audit}{\text{audit}}
\newcommand{\cod}[1]{\llbracket#1\rrbracket}

%++++++++++++++++++++
% The document itself
%++++++++++++++++++++
\begin{document}

%-----------------------
% Title page information
%-----------------------
\title{Distributed Authorization}
\author{Presented by Peter C. Chapin\\University of Vermont}
\date{April 15, 2005}
\maketitle

%%%%%

\startslide{Outline}

\begin{itemize}

\item Overview of Trust Management

\item Introduction to the RT Framework

\item Trust-but-Verify

\item Trust-but-Verify in an RT Framework

\end{itemize}

\stopslide

%%%%%
\startslide{Distributed System Model}
Web service or remote procedure call model.
\begin{itemize}
\item Application distributed over many network nodes.
\item Network nodes call each other in arbitrary ways.
\item Network nodes in different administrative domains.
\end{itemize}
Access control potentially applied at each node. Must be efficient, flexible, and support delegation.
\stopslide

%%%%%
\startslide{Trust Management}
In a distributed system authorizers\ldots
\begin{itemize}
\item Often don't know (or want to know) the identities of requesters.
\item Might instead rely on third parties to certify characteristics of requesters.
\item Write policies that describe access in terms of these characteristics.
\item Require requesters prove compliance with policy.
\end{itemize}
\stopslide

%%%%%
\startslide{$RT$ Framework}
Introduced by Li, Mitchell, and Winsborough\footnote{\textit{Design of a Role-based Trust-management Framework}, IEEE Symposium on Security and Privacy, May 2002}. $RT_0$ specifies four assertion forms.
\begin{itemize}
\item $A.r \leftarrow B$
\item $A.r \leftarrow B.s$
\item $A.r \leftarrow A.s.t$
\item $A.r \leftarrow B.s \cap C.t$
\end{itemize}
\stopslide

%%%%%
\startslide{Example: VT Residents Only}
Authorizer wishes to only grant access to Vermont residents.

$A.g \leftarrow V.residents$

Vermont delegates authority to determine residency to the towns.

$V.resident \leftarrow V.towns.resident$\\
$V.towns \leftarrow Tunbridge$

Tunbridge delegates to neighborhood authorites.

$Tunbridge.resident  \leftarrow Tunbridge.locals.resident$\\
$Tunbridge.locals \leftarrow JohnSmith$

Requester has certificate signed by John Smith

$JohnSmith.resident \leftarrow Peter$
\stopslide

%%%%%
\startslide{Authorization as Logic Program}
\begin{verbatim}
mem(X, a, g) :- mem(X, v, resident).
mem(X, v, resident) :- mem(Y, v, towns), mem(X, Y, resident).
mem(tunbridge, v, towns).
mem(X, tunbridge, resident): mem(Y, tunbridge, locals),
                             mem(X, Y, resident).
mem(johnsmith, tunbridge, locals).
mem(peter, johnsmith, resident).
\end{verbatim}
Query: \texttt{mem(peter, a, g)}\\
Answer: ``yes''
\stopslide

%%%%%
\startslide{Delegation: $RT_0^D$}
An entity is allowed to activate only a subset of his/her potential roles and delegate those activations to an intermediary.
\begin{itemize}
\item $P \rightarrow (P\, \mbox{\texttt{as}}\, A.g)\rightarrow I_1$
\item $I_1 \rightarrow (P\, \mbox{\texttt{as}}\, A.g)\rightarrow I_2$
\end{itemize}
$I_2$ passes $P$ as $A.g$ to the request. Access is granted because
\begin{itemize}
\item $P$ as $A.g$ is valid.
\item There is a path from $A.g$ to the governing role.
\item $A$ gets signed delegation certificates that originate with $P$.
\end{itemize}
\stopslide

%%%%%
\startslide{Authorization is Expensive}
Complex authorization decisions might be time consuming.
\begin{itemize}
\item Fetch remote credentials
\item Cryptographic operations (signature checking)
\item Proof of compliance check
\end{itemize}
Authorizer obtains and checks certificates, builds Datalog program (or equivalent) and computes answer to relevant query.
\stopslide

%%%%%
\startslide{Distributed Credential Chain Discovery}
Li, Winsborough, and Mitchell describe a credential discovery algorithm for $RT_0$\footnote{\textit{Distributed Chain Discovery}, Journal of Computer Security, 11(1), pp 35-86, February 2003}.

\begin{center}$A \rightarrow B.r \rightarrow C.s \rightarrow D.t \rightarrow E.g$\end{center}

Requestor provides $B.r \leftarrow A$. Authorizer ($E$) has policy $E.g \leftarrow D.t$. How are these assertions related?

$B$ has $C.s \leftarrow B.r$\\
$D$ has $D.t \leftarrow C.s$

Finding these assertions takes time.
\stopslide

%%%%%
\startslide{Trust but Verify (TbV)}
Reliable, practical distributed authorization.
\begin{itemize}
\item Ideal security may take many details and relations into account.
\item Practical security may require abstractions or simplifications to achieve efficiency.
\end{itemize}
Simplification during authorization introduces elements of \cemph{trust} into online security decisions.

Essence of TbV: Online trust should be verifiable offline, to ensure that it is warranted.
\stopslide

%%%%%
\startslide{TbV Examples}
\begin{itemize}
\item \textit{Simple real-life credit card transactions}. Credit card companies trust vendors without proof of delegation from cardholder.
\item \textit{Business Transaction Protocol} (enroll/prepare/commit). Authorizes could potentially enroll quickly using trust.
\end{itemize}
TbV approach allows \cemph{auditing} of trusted transactions, in case of disputes or as a matter of course, to verify trust content.

Ideally we desire a continuum of trust. Authorizers should be able to trust a little or a lot, verify promptly or not at all, depending on administrative needs.
\stopslide

%%%%%
\startslide{A Formal Framework}
TbV is a framework for augmenting trust management systems.
\begin{citemize}	
\item \textit{Logically well-founded}. Provides precise meaning of ``trust'' and ``verification'', with provable guarantees.
\item \textit{General}. Framework intended to be applicable to variety of systems.
\end{citemize}
A TbV system is characterized by  high-level formal conditions.

TbV implementations must adhere to these conditions.
\stopslide

%%%%%
%\startslide{TbV Framework Conditions}
%\begin{condition}
%Let $s$ be an authorization context; then $s \vdash \priv$ is decidable.
%\end{condition}
%\begin{condition}
%Let $s$ be a trusted context; then if $\audit(s)$ succeeds, it is the case that $\audit(s) \vdash s'$ such that $\cod{s'} = s$.
%\end{condition}
%\begin{condition}
%et $s$ be a trusted context and $\priv$ be a privilege. If $s \vdash \priv$ and $\audit(s)$ succeeds, then $\audit(s) \vdash \priv$.
%\end{condition}
%\stopslide

%%%%%
\startslide{Trusted Assertions}
One possible trust transformation for $RT$: Add assertions to the policy compliance checking process. Logical weakening.
\begin{itemize}
\item Trusting context uses additional assertions.
\item Auditing performed by either
  \begin{itemize}
  \item Locating ``real'' assertions (credential discovery) that prove validity of added assertions.
  \item Proving authorization without the help of the added assertions.
  \end{itemize}
\end{itemize}
Additional assertions deduced as part of trust-transformation semantics.
\stopslide

%%%%%
\startslide{Trusted Assertions Example}

\begin{center}$A \rightarrow B.r \rightarrow C.s \rightarrow D.t \rightarrow E.g$\end{center}

\begin{center}$A \rightarrow B.r\phantom{ \rightarrow C.s \rightarrow }D.t \rightarrow E.g$\end{center}

Authorizer refuses to fetch remote credentials. Trusts they are available. Adds

\begin{center}$B.r \rightarrow D.t$\end{center}

Later auditing finds

\begin{center}$B.r \rightarrow C.s \rightarrow D.t$\end{center}
\stopslide

%%%%%
\startslide{Trust Cost}
Additional idea: Assign a specific ``trust cost'' to each assertion and accept authorization only if total trust cost is below a specified threshold.

\begin{tabular}{|l|r|} \hline
Assertion & Cost \\ \hline
$B.r \leftarrow A$   & 10\\
$C.s \leftarrow B.r$ & 20\\
$D.t \leftarrow B.r$ &  5\\
$E.g \leftarrow C.s$ &  0\\
$E.g \leftarrow D.t$ &  0\\ \hline
\end{tabular}

Proof $A \rightarrow B.r \rightarrow C.s \rightarrow E.g$ incurs cost 30.\\
Proof $A \rightarrow B.r \rightarrow D.t \rightarrow E.g$ incurs cost 15.

Operates outside the formal logic (for now).
\stopslide

%%%%%
\startslide{Bounded Trust Search}
Must backtrack during proof search if bound on trust cost exceeded. Example: let bound be 20.

$E.g \leftarrow C.s\, (0)$\\
$E.g \leftarrow C.s \leftarrow B.r\, (0 + 20 = 20)$\\
$E.g \leftarrow C.s \leftarrow B.r \leftarrow A\, (20 + 10 = 30)$\, (Abort)\\
$E.g \leftarrow C.s \leftarrow B.r$\, (No additional options)\\
$E.g \leftarrow C.s$\, (No additional options)\\
$E.g \leftarrow D.t\, (0)$\\
$E.g \leftarrow D.t \leftarrow B.r\, (0 + 5 = 5)$\\
$E.g \leftarrow D.t \leftarrow B.r \leftarrow A\, (5 + 10 = 15)$

Access is granted. Search is a modification of the normal proof searching methods used to evaluate logic programs.
\stopslide

%%%%%
\startslide{Sources of Trust Cost}
Trust costs might come from various sources.
\begin{itemize}
\item Signature checks skipped, but assertion gets high cost.
\item Certificates issued by questionable entities get high cost.
\item Certificates near or past expiration time get high cost.
\end{itemize}

Time dependent cost function?

\begin{displaymath}
C = a \left( 2^{\frac{t - T_i}{T_e - T_i}} - 1 \right)
\end{displaymath}
\stopslide

%%%%%
\startslide{Verification}
In addition to formal auditing, authorizer could verify trust costs by reducing cost threshold (to zero?) and recomputing authorization.
\begin{itemize}
\item Check signatures.
\item Delete questionable assertions.
\item Get fresh certificates.
\end{itemize}
\stopslide

%%%%%
\startslide{Practical Matters}
\begin{itemize}
\item Role memberships (probably) change slowly. Assertions effectively cachable (like caching DNS lookups).
\item Delegations more volatile. Certificates expire quickly.
\end{itemize}
Trust-but-Verify may be more useful in connection with delegation checking than with role membership checking.

Delegations can be handled similarly to role memberships: semantics defined by logic program; certificate graph can be constructed; trust transformations and proof searching works the same way.
\stopslide

%%%%%
\startslide{Questions}
\begin{center}\Large ?\end{center}
\stopslide

\end{document}
