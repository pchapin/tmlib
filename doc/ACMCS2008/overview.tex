\section{Overview}
\label{section-overview}

In this section we provide background in trust management systems for
the general reader.  We also clarify which trust management system
components are relevant to the authorization decision-- there turn out
to be some important subtleties in this regard.  In light of the
structure of authorization decisions so described, we outline our
approach to comparing trust management systems.  We also provide a
longer running example, which serves to illustrate the concepts
introduced and later serves as an explicit point of comparison
for the systems we survey.

\subsection{Components of Full Implementations}
\label{section-overview-components}

Trust Management Systems (TMSs) in practice comprise a number of
functions and subsystems, which we divide into three major components:
\emph{the authorization decision}, \emph{certificate storage and
retrieval}, and \emph{trust negotiation}.  Authorization decisions are
relevant to the elements and semantics of the access control decision
itself.  Certificate storage and retrieval is relevant to the physical
location of certificates that are the low-level representation of
access control elements such as credentials and policies.  For
example, systems have been proposed for storing SPKI certificates
using DNS \cite{nikander98storing} and for storing SDSI certificates using a
peer-to-peer file server \cite{ajmani02conchord}.  Trust negotiation
\cite{Winsborough:ATN,Yu:PECSATNI,Seamons:LDACPATN,Yu:ISATN,Winsborough:TPATN,Winsborough:SATN}
is necessary for access control decisions where some elements of access
policies or the credentials used to prove authorization with those polices
should not be arbitrarily disclosed. For example, in \cite{Winsborough:ATN}
a scheme is proposed whereby access rights held by requesters are protected
by their own policies, and both authorizers and requesters must show
compliance with policies (i.e.~negotiate) during authorization.  
We provide a brief summary and overview of trust negotation in 
\autoref{section-trust-negotiation}, to provide a more complete 
view of trust management functionality and challenges in modern practice.

The importance of these other components notwithstanding, in this
survey our focus will be on authorization decisions.  This is because
the authorization decision is the basis of any trust management
system.  Furthermore, not all the systems proposed in the literature
have been developed sufficiently to include certificate storage
implementations, nor trust negotiation strategies in the presence of
confidentiality.  Focusing on authorization decisions allows us to
sufficiently narrow our scope, and thoroughly review components that
endow systems with their characteristic features.  When we say that we
consider only those TMSs with a formal foundation in this survey as in
\autoref{section-introduction}, we mean that the authorization
decision is based on a mathematically well-founded semantics of some
sort, e.g.~propositional logic or relational algebra.

\subsection{Elements of Authorization: Glossary}

To clarify the remaining presentation and identify fundamental
elements of trust management authorization decisions, we now provide a
glossary of relevant terms.  More in depth discussion of these terms
occurs throughout the rest of the paper, this section is intended as a
succinct reference.

\medskip\emph{Entity}: an individual actor in a distributed system, also
frequently called a principal.

\medskip\emph{Resource}: anything that a local system might regard as
worthy of access control-- file access, database lookup, web browser
display area, etc.

\medskip\emph{Policy}: a specification of rules for accessing a particular
resource.  Policy is usually defined locally at least in part, but
TMSs sometimes allow policy to be defined
non-locally as well.

\medskip\emph{Authorizer}: the local authority that protects a
resource, by automatically allowing access only after an
appropriate proof of authorization has been shown.  Authorizers 
also specify policy.

\medskip\emph{Requester}: an entity (usually non-local) seeking to access a
resource.

\medskip\emph{Attribute}: a property of interest in some security 
domain, for example a role membership.

\medskip\emph{Credential}: endows entities with certain attributes.
Local policy usually specifies that requesters must be endowed with
certain attributes before resource access is allowed, so credentials
are essential to establish access rights to resources.

\medskip\emph{Issuer}: the authority that issues a particular credential.

\medskip\emph{Certificate}: a certified wire format representation of
a credential.
%
%\medskip\emph{Delegation}: the transmission of authority or rights from
%one entity to another.

\medskip\emph{Certificate revocation}: the removal of a requester's 
credential, typically by the issuer.

\medskip\emph{Credential negation}: Policy languages sometimes 
allow policy makers to specify that a credential \emph{not} be held.
Logically, this is expressed as credential negation.

\medskip\emph{Delegation of authority}: the (usually temporary) logical 
transfer of authority over policy from one entity to another.

\medskip\emph{Delegation of rights}: the (usually temporary) logical
transfer of an access right from one entity to another.

\medskip\emph{Authorization decision}: the determination of whether a
given requester possesses the necessary attributes to access a
particular resource as mediated by local policy, based on a preferably
well-defined semantics of policies and credentials.
%
%\medskip\emph{Proof of compliance}: synonymous with authorization
%decision.

\medskip\emph{Authorization mechanism}: the automated means by which
an authorization decision is reached. Depending on context this refers
to an algorithm or a module of software executed by the authorizer.

\medskip\emph{Core authorization semantics}: the mathematically
well-founded theory that constitutes the meaning of authorization
decisions.  

\medskip\emph{Role}: an attribute that requesters can activate when
requesting authorization.  Authorization is often based on the role
a requester is able to assume.

\medskip\emph{Role membership}: an entity is said to be a member of a role
if that entity is among the group of entities that can activate the role.

\medskip\emph{Threshold policy}: threshold policies require a 
minimum specified number of entities to agree on some fact. 
Threshold policies usually support separation of duty authorization
schemes \cite{Li:DRBTMF}. 

\medskip\emph{Domain}: the security locality administered by a given
authority.

\medskip\emph{Name space}: the names defined in a particular domain.


\subsection{Structure of an Authorization Decision}
\label{section-overview-structure}

The subsystem of a trust management system that constitutes its
authorization decision includes more than just a core authorization
semantics.  By \emph{system} we mean the set of components that
provide an implementation, not just an abstract specification of the
authorization semantics.  This distinguishes our presentation from a
survey of authorization logics \cite{Abadi:LAC}.  In this section we
identify the components of a generic authorization decision and
characterize its structure.  This provides a better understanding of
authorization decisions in general, and also a means to better
categorize features of particular systems later in the paper.

In \autoref{figure-tmstruct} we illustrate the components of a generic
authorization decision.  This graphic is meant as a rough sketch, not
a formal specification, and not all TMSs contain all the components we
describe.  Nevertheless, the illustration is a useful tool for
categorizing systems.  The graphic is read top to bottom, and shows
the flow of information through a particular authorization process,
with output computed in response to an authorization request.  The
diagram is intentionally vague about the nature of the output: in the
simplest case, the output is a simple ``yes'' or ``no'' decision as to
whether or not to grant resource access, but in systems that support
\emph{trust negotiation}, the output could be a partial answer that
provides direction for additional input.  This issue is better
discussed in \autoref{section-trust-negotiation}.  Within the scope of
this survey, we mainly consider the case where the output is a boolean
value, hence our terminology authorization \emph{decision}.  The core
authorization semantics $L$ implement the authorization decision, and
may be a specialized inference system, or a proof search in a generic
programming logic such as Prolog, for example.  The authorization
semantics takes as input parameters from $C$, $P$, and $Q$, which we
now describe in detail.

Local policy $P$ is defined in some specification language, that is
transformed into terms understood by the core semantics by the
transformation function $T_P$.  This translation may just consist of
parsing from concrete to abstract syntax, or $T_P$ may compile
statements in a high-level policy language into lower level terms for
the core semantics.  For example, TPL \cite{Herzberg:ACMPKI} provides
an XML-based ``trust policy language'' that is compiled into Prolog.

Credentials for a particular requester may be defined as part of local
policy. But an earmark of TMSs is their ability to extend local policies
with credentials conferred by non-local authorities. This is realized as
set of available certificates $C$ that are transformed by a function $T_C$
into credentials defined in terms understood by the core semantics. The
transformation $T_C$ provides a level of indirection allowing systems to
choose between various certificate wire formats and PKIs, though X.509
\cite{X509} or WS-Security \cite{OASIS:WSSTC} are obvious choices for
Internet and Web Services settings.


\tmstructfig

The transformation $T_C$ also has special significance for the
semantics of TMSs, since it is often not a straight parsing or
compilation procedure.  Rather, certificates may be rejected, or their
credential representations enhanced, by certificate validity
information.  Validity information is external to the authorization
semantics in some systems, but internal to it in others, so we
represent the certificate validation component of the authorization
decision $V$ as dashed box.  For example, any given certificate $c \in
C$ almost always defines a finite lifetime for the certification, also
called a validity interval \cite{winslett-adl97}. Some TMSs such as
PCA \cite{Bauer:GFACSW} support lifetime information in the
authorization semantics, and in such a case $T_C$ can map the lifetime
information in $c$ to its credential representation. However, other
systems do not represent lifetimes in the authorization semantics per
se (that is, in $L$), and in such cases the onus is on $T_C$ to filter
out expired certificates.  For example, SPKI provides a mechanism for
certificates to be checked on-line to see if they have been revoked
\cite{RFC-2693}, but this mechanism is not part of SPKI's formal
structure. This means on the one hand SPKI's revocation policy cannot
be expressed in the SPKI policy language itself, nor enforced by its
authorization semantics. On the other hand it allows a SPKI
implementation to apply a different revocation policy without changing
their underlying logical structure, and in general the difficulties
associated with formalizing certificate revocation
\cite{Stubblebine:RSAERDS,Stubblebine:ALSSRR,Rivest:CWECRL} can be
avoided, while a means for certificate revocation in the system is
still available.

In addition to policy $P$ and certificates $C$, the authorization decision
takes as input a question or goal $Q$ that is specialized for a particular
access request. As an example, some trust management systems, such as SDSI and
$\RT_0$ \cite{Li:DRBTMF,Li:RRBTMF}, define roles. These systems allow one to
prove that a particular principal is in a particular role. Resources are
associated with roles, and the authorization decision is based on whether the
requester is a member of the relevant role. The transformation $T_Q$
translates the goal into terms understood by the core semantics. Finally, the
core semantics combines policies and credentials established by input
certificates to determine whether the authorization goal is satisfied, and
outputs ``yes'' or ``no'' based on this determination.

However, as denoted by the dotted line, some systems also provide a
``feedback'' mechanism $D$ between the semantics of authorization and
certificate collection.  Rather than merely answering ``no'' outright
in case an authorization goal cannot be reached, the system might
identify credentials that are missing and attempt to collect them.
This functionality is sometimes called \emph{distributed certificate
chain discovery} \cite{Li:DCDTM} or \emph{policy directed certificate
retrieval} \cite{Gunter:PDCR}.
%, whereas
%\cite{koshutanski-massacci-xmls03} refers to it as \emph{abduction},
%s.~\emph{deduction} as implemented by the authorization semantics.
Whatever the specifics, it is clear that this functionality makes for
a more flexible system in terms of certificate distribution and
storage, but presents a significant challenge to system designers.

\subsection{Comparing Trust Management Systems}

A basis for comparing the features and functionality supported by
trust management systems is fundamental to our survey.  Since the
systems we consider have a formal foundation, some sort of formal
comparison seems appealing.  Indeed, in \cite{Weeks:UTMS} a framework
is presented for describing a variety of authorization semantics,
including KeyNote and SPKI/SDSI.  This uniform specification of
various semantics allows them to be compared on a completely formal
basis, so for example it can be shown how credentials in one system
can be faithfully encoded in another.

However, as we observe above, features of TMS authorization decisions
are not entirely realized in the authorization semantics $L$, but may
be realized in other components, as for example certificate revocation
is sometimes implemented as part of the translation $T_C$ from
certificates to credentials.  Since the definition of these components
is not included in the formal specification of the authorization
semantics, these system features can only be compared on an informal
basis.  We will therefore compare systems in light of the features
they possess.  In addition, we will observe whether the features are
realized formally as part of the authorization semantics, or whether
they are implemented by some other system component; this will clarify
in what sense particular TMSs ``possess'' a certain feature.  As a
concrete point of comparison, we will also show how various systems
encode the running example introduced next.  
 
\subsection{A Running Example}
\label{section-overview-example}

Suppose Alice is a cancer patient at a hospital being treated by Bob, a
doctor. Alice grants Bob access to her medical records and also allows Bob
to delegate such access to others as he sees fit.

Bob defines his team as a particular collection of individuals
together with the people supporting them. A person supporting one of
Bob's team members becomes a team member herself so Bob's definition
is open ended and can potentially refer to a large number of people he
does not know directly.  Here we assume that Bob's team includes both
medical and non-medical personnel (for example other doctors as well
as receptionists). Bob then delegates his access to Alice's medical
records to only the medical staff on his team-- that is, people on his
team that are also on the medical staff, as opposed to
e.g.~administrative staff.

Suppose further that Bob consults with another doctor, Carol, on Alice's
condition. Bob modifies his policy to add Carol temporarily to his team.
Carol orders some blood tests that are then analyzed by Dave, a lab
technician and one of Carol's support people. The policy described allows
Dave to access Alice's medical records, for example to input the test
results.

Dave signs the test results when he uploads them to the hospital database. He
also includes appropriate credentials so that the database will authorize his
access. The precise credentials needed depend on the trust management system
in use and on the way credentials and policy statements are distributed and
located by that system, however we imagine that these credentials should 
be able to express the following relations in some form:
\begin{itemize}
\item Bob has delegated his access to Alice's medical records to
people on his team who are members of the medical staff.

\item Carol is on Bob's team.

\item If someone is on Bob's team, than any person on their support
staff is also on Bob's team. 

\item Dave is one of Carol's support people.

\item Dave is a member of the hospital's medical staff.
\end{itemize}
On the basis of these relations, one may deduce that Dave has access to
Alice's medical records.  Realistically, Dave may not know to submit
all of this information, or have any knowledge of Bob's policy. If the
trust management system used by the hospital supports distributed
credential chain discovery, the hospital database would locate Bob's
policy automatically in order to complete the authorization decision.

Complex access control scenarios such as this are difficult to express
using traditional methods. Neither Alice nor Bob realize that Dave
needs to be granted access to Alice's medical records. Although Dave's
role as one of Carol's support people might be enough to grant him
access to the records of Carol's patients, Dave's relationship to Bob,
and hence to Alice, is indirect; it is Bob's act of adding Carol to
his team that causes Dave to gain access to Alice's records.  Observe
also that Bob's team policy is recursive.  A primary purpose of trust
management systems is to provide language features and authorization
semantics that support such complex policies.

%Once Alice is discharged from the hospital she may wish to continue
%allowing Bob to access her records so that he can provide appropriate
%follow-up care. However, Bob may wish to remove Carol as a member of
%his team if he no longer needs to consult with her.  To implement this
%change, a number of trust management systems also offer mechanisms for
%certificate expiration or revocation.  In some situations Alice may
%wish to explicitly forbid Dave from accessing her records. For
%example, perhaps Dave is Alice's brother and she wants to keep the
%details of her condition private from him. Systems that support
%credential negation would allow Alice to create such a policy.

