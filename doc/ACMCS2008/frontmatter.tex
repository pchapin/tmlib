%\acmVolume{V}
%\acmNumber{N}
%\acmYear{YY}
%\acmMonth{M}

\markboth{Peter Chapin et al.}{Authorization in Trust Management: Features and Foundations}

\title{Authorization in Trust Management: Features and Foundations}

\author{
PETER C.~CHAPIN, CHRISTIAN SKALKA, and X.~SEAN WANG \\ University of Vermont
}


\begin{abstract} 
Trust management systems are frameworks for authorization in modern
distributed systems, allowing remotely accessible resources to be
protected by providers.  By allowing providers to specify policy, and
access requesters to possess certain access rights, trust management
automates the process of determining whether access should be allowed
on the basis of policy, rights, and an authorization semantics.  In
this paper we survey modern state-of-the-art in trust management
authorization, focusing on features of policy and rights languages
that provide the necessary expressiveness for modern practice.  We
characterize systems in light of a generic structure that takes into
account components of practical implementations.  We emphasize systems
that have a formal foundation, since security properties of them can
be rigorously guaranteed.  Underlying formalisms are reviewed to
provide necessary background.
\end{abstract}

\category{C.2.0}{Computer-Communication Networks}{\\General}[Security and protection]

\terms{Security, Design, Languages}

\keywords{Distributed Authorization, Trust Management Systems}

\begin{bottomstuff}
Authors' addresses: Peter Chapin, University of Vermont, Department of
Computer Science, Burlington, VT 05405, \texttt{pchapin@cs.uvm.edu}.
Christian Skalka, University of Vermont, Department of Computer
Science, Burlington, VT 05405, \texttt{skalka@cs.uvm.edu}.
X.~Sean Wang, University of Vermont, Department of Computer
Science, Burlington, VT 05405, \texttt{xywang@cs.uvm.edu}.
\end{bottomstuff}

\maketitle
