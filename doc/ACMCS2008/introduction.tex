\section{Introduction}
\label{section-introduction}

Distributed applications that span administrative domains have become
commonplace in today's computing environment. Electronic commerce,
high performance scientific computing, groupware, and multimedia
applications all require collaborations between distinct social
entities.  In such systems each administrative domain, also called a
security domain, controls access to its own resources and operates
independently of other administrative domains. The problem of how to
best specify and implement access control in such an environment has
been a topic of considerable research. To address this problem the
idea of \newterm{trust management} was introduced
\cite{Blaze:DTM} and subsequently developed by many authors,
providing frameworks in which entities can specify independent access
control policies that are enforced upon access request.

At the heart of trust management systems is the \emph{authorization
procedure}, which determines whether resource access should be granted
or not based on a number of conditions.  The semantics of
authorization provide meaning to the features supported by trust
management systems, for both the policy maker and the resource
requester.  While a number of techniques have been proposed to
characterize authorization in trust management systems, we argue that
the most promising are those based on rigorous formal foundations.
This argument is not new, in fact it has motivated trust management
research since its inception \cite{woo93authorizations}.  In a
security setting, entities should be able to specify policies
precisely, to have an absolutely clear idea of the meaning of their
policies, and to have confidence that they are correctly enforced by
authorization mechanisms. Formally well-founded trust management
systems achieve this, providing a setting in which reliability can be
rigorously established by mathematical proof. In particular, various
logics have served as the foundation for trust management
\cite{Abadi:LAC,Bertino:LFRAACM}. In this paper we survey
state-of-the-art in trust management authorization, with an emphasis
on formally well-founded systems.  These systems are compared to each
other with respect to desirable high-level features of trust
management.

Our focus is the foundations and features of trust management systems, not
their application, though we note that trust management systems have been
shown to enforce security in many real applications. For example, the
KeyNote system has been shown capable of enforcing the IPsec network
protocol \cite{Blaze:TMIPS,Blaze:EKTMS}. SPKI/SDSI has been used to provide
security in component based programming language design \cite{Liu:CSI}.
Cassandra has been examined in the context of the United Kingdom's proposed
nationwide electronic health records system \cite{Becker:CFTMAEHR}. In
addition, the Extensible Access Control Markup Language (XACML)
\cite{OASIS:XACMLTC} and the Security Assertion Markup Language (SAML)
\cite{OASIS:SSTC}, both OASIS standards, define XML policy and assertion
languages that makes use of many trust management concepts.

\subsection{Authorization Frameworks}


%In this review we are concerned with authorization: how the decision is
%made to grant or deny access to a protected resource for a particular
%request. This review does not directly consider issues of authentication.
%Authentication is primarily concerned with how a requester proves his or
%her identity to an authorizer. Trust management systems of the sort we
%discuss here do not use the identities of requesters as the primary
%attribute considered in the access control decision and thus authentication
%does not play as important a role in trust management systems as it does
%traditionally.
%
%Also in this review we are only concerned with distributed systems that
%span administrative domains. We do not consider authorization as
%implemented under the control of a single entity, such as an operating
%system. Applications that are distributed over multiple hosts in a single
%administrative domain are also outside the scope of this review. Such
%systems can generally use the traditional approach to access control: a
%requester first has his or her identity authenticated and then all access
%requests are evaluated based on that identity. This is feasible because
%such systems, despite being spread over many hosts, are relatively closed.
%The list of possible requesters can be managed by the local authority. In
%contrast we are concerned with environments where the list of requesters is
%open ended and were an authorizer can not feasibly be expected to know all
%possible requesters.
%
%Our primary focus is on the formal logical structure of trust management
%systems. That is, we review the logical foundations on which distributed
%trust management rests. In systems designed for security properties such
%as soundness, completeness, and tractability should be precisely defined
%and rigorously proved. No matter how well implemented or carefully
%deployed a system might be, a loophole in its logical structure might
%introduce a vulnerability that will eventually be found and exploited.
%No amount of software engineering or vigilance by system administrators
%can correct a problem caused by a fundamental error in design.


The trust management systems we survey are primarily concerned with
\emph{authorization}, as opposed to \emph{authentication}.  The latter
addresses how to determine or verify the identity of actors or message
signers in a distributed transaction with a high degree of confidence.
Authorization, on the other hand, is based on calculi of principals
whose identities are taken for granted.  Although any real
implementation of an authorization system will rely on authentication
to establish these identities, and key-to-identity bindings may even
have an abstract representation in the system, authorization generally
treats authentication and public key infrastructure as orthogonal
issues.  Authorization is more properly concerned with non-trivial
access control policies-- how to specify them, what they mean, and how
to endow trusted principals with the credentials necessary to satisfy
them.

Authorization in trust management systems is more expressive than in
traditional access control systems such as role based access control
(RBAC) \cite{Sandhu:RBACM}.  In such simpler models, access is based
directly on identities of principals.  But in a large distributed
environment such as the Internet, creating a single local database of
all potential requesters is untenable.  Where there are multiple
domains of administrative control, no single authorizer can be
expected to have direct knowledge of all users of the system.
Furthermore, the Internet is a highly dynamic and volatile
environment, and no single entity can be expected to keep pace with
changes in an authoritative manner.  Finally, basing authorization
purely on identity is not a sufficiently expressive or flexible
approach, since security in modern distributed systems utilizes more
sophisticated features (e.g.~delegation) and policies (e.g.~separation
of duty \cite{Simon:SODRBE}).  These problems are addressed by the use of trust
management systems.  We now return to some of the applications
mentioned above, to illustrate how authorization in trust management
systems is suited to enforcing security in practical computing
scenarios.
\begin{description}
\item[IPsec] Blaze and Ioannidis \cite{Blaze:TMIPS} describe an extension
  to the IPsec architecture that uses KeyNote to check if packet filters
  proposed by a remote host comply with a local policy for the creation of
  such filters. This allows a system administrator to prevent an attacker
  from negotiating a secure connection and then using that connection to
  attack vulnerable services. This application is an instance of the more
  general idea of using a trust management system for firewall management.
\item[Web Page Content Ratings] Several authors describe the use of trust
  management systems to implement web page content rating schemes
  \cite{Gunter:ADDUQCM,Chu:RTMWA}. This is of significant practical
  interest; the World Wide Web Consortium has considered using trust
  management concepts in its Platform for Internet Content Selection
  \cite{Resnick:PIACWC}. In a rating scheme a client delegates the
  authority to rate web pages to a suitable ratings server. The server
  issues certificates that bind a web page (via its hash value) to a
  rating. When a page is fetched, the web server delivers this certificate
  to the browser where the browser's policy is consulted to determine if
  the page should be displayed.
\item[Medical Records] Several trust management systems have been
  applied to maintaining integrity and privacy in electronic health
  records \cite{Bacon:MORBACSAS,Becker:CFTMAEHR}, a topic of
  considerable importance in modern health care \cite{Ota:PPCMI}.
  Security in this setting involves policies spanning many loosely
  coupled domains such as clinics, hospitals, laboratories, and
  emergency services.  
\end{description}

\subsection{Goals and Outline of the Paper}

A summary and comparison of the features and formal underpinnings of
authorization procedures in trust management systems is a primary goal
of this paper, grounded in a review of their foundations in
authorization logics such as ABLP \cite{Abadi:CACDS}.  This summary
provides a useful explanation and overview of modern state-of-the-art
in trust management authorization technology.  Another contribution of
this survey is the characterization of authorization frameworks as
\emph{systems} that include other components in addition to the core
authorization semantics. This distinguishes our presentation from a
previous survey of authorization logics \cite{Abadi:LAC}. It is
important to consider these components, since some features of trust
management systems may be reflected in them rather than in the
authorization semantics, for example certificate expiration dates may
be checked when parsing wire format certificates but ignored by the
authorization semantics. This also sheds light on how much formal
support is provided for these features in various systems. We
summarize the components of trust management systems, and compare them
in light of which features are supported by which components. 

Because trust management is a broad and active field, it is important
to restrict the scope of our survey to provide sufficient depth as
well as breadth.  As the title suggests, we are mainly concerned with
the semantics and implementation of \emph{authorization} in trust
managements systems, versus other components such as certificate
storage and retrieval.  We delineate our scope more precisely
below in \autoref{section-overview-components}.

The remainder of this survey is organized as follows. In
\autoref{section-overview} we introduce important concepts and terminology,
summarize the method we use to compare and contrast various systems, and
introduce a running example. In \autoref{section-features} we highlight
several features offered by trust management systems.
\autoref{section-foundations} reviews in more detail the logical basis of
trust management. \autoref{section-review} reviews several trust management
systems with a focus on those that are logically well founded.
\autoref{section-trust-negotiation} gives an overview of trust negotiation,
an important component of some trust management applications. Finally we
conclude in \autoref{section-conclusion}.
