\subsection{RT}

The $\RT$ trust management framework is not a single trust management
system but rather a collection of trust management systems with varying
expressiveness and complexity \cite{Li:DRBTMF,Li:DCDTM,Li:RRBTMF}. The base
system, $\RT_0$, is similar to SDSI except that it limits extended names to
one level of indirection and provides intersection roles. The limitation of
linked roles to one level of indirection does not reduce the expressiveness
of the language since additional indirections are possible by introducing
intermediate roles.

$\RT_1$ is an extension of $\RT_0$ providing parameterized roles. $RT_1^C$
further extends $RT_1$ to allow for the description of structured resources
\cite{Li:DCFTML,Li:RRBTMF}. The system $\RT^D$ provides a mechanism to
describe the delegation of rights and role activations, and $\RT^T$ provides
support for threshold and separation of duty policies. $\RT^T$ and $\RT^D$ can
be used in combination with $\RT_0$, $\RT_1$, or $\RT_1^C$ to create trust
management systems such as $\RT_0^T$, $\RT_1^{TD}$, and so forth. A rich
complexity analysis has also been developed for the $\RT$ framework for
problems beyond simple authorization, e.g.\ role inclusion and role membership
bounds \cite{Li:BPOCSATM}.

\subsubsection{Features}

Like SPKI/SDSI, the $\RT$ framework represents principals as public keys
and does not attempt to formalize the connection between a key and an
individual. The $\RT$ literature usually refers to these principals as
\newterm{entities}. Also like SPKI/SDSI, the $\RT$ framework allows each
entity to define roles in a name space that is local to that entity. An
authorizer associates permissions with a particular role; to access a
resource a requester must prove membership in the role. In this way the
$\RT$ framework provides role based access control.

To define a role, an entity issues credentials that specify the role's
membership. Some of these credentials may be a part of private policy,
others may be signed by the issuer and made publicly available as
certificates. The overall membership of a role is taken as the union
of the memberships specified by all the defining credentials.

Let $A, B, C, \ldots$ range over entities and let $r, s, t, \ldots$ range
over role names. A role $r$ local to an entity $A$ is denoted by $A.r$.
$\RT_0$ credentials are of the form $\cred{A.r}{f}{}$, 
where $f$ can take on one of four forms to obtain one of 
four credential types:
\begin{enumerate}

\item $\cred{A.r}{E}{}$ 

 This form asserts that entity $E$ is a member of role $A.r$.

\item $\cred{A.r}{B.s}{}$ 

  This form asserts that all members of role $B.s$ are members of role
  $A.r$. Credentials of this form can be used to delegate authority over
  the membership of a role to another entity.

\item $\cred{A.r}{B.s.t}{}$ 

  This form asserts that for each member $E$ of $B.s$, all members of
  role $E.t$ are members of role $A.r$. Credentials of this form can be
  used to delegate authority over the membership of a role to all
  entities that have the attribute represented by $B.s$. The expression
  $B.s.t$ is called a \emph{linked role}.

\item $\cred{A.r}{f_1 \cap \cdots \cap f_n}{}$
  
  This form asserts that each entity that is a member of all roles
  $f_1,\ldots, f_n$ is also a member of role $A.r$. The expression $f_1
  \cap \cdots \cap f_n$ is called an \emph{intersection role}.

\end{enumerate}
For all credential forms $\cred{A.r}{f}{}$, the principal $A$ is called the
\emph{issuer} of the credential.

$\RT_1$ enhances $\RT_0$ by allowing roles to be parameterized. For
example, the second credential form above is extended to
$\cred{A.r(h_1, h_2, \ldots, h_n)}{B.s(k_1, k_2, \ldots, k_m)}{}$
where the $h_i$ and $k_j$ are parameters. Role parameters are typed
and can be integers, floating point values, dates and times,
enumerations, or finite sets or ranges of these datatypes. An $\RT_1$
credential is \newterm{well formed} if the parameters given to the
roles have the right type and if each variable in the credential
appears in the body of that credential.

As an example of an $\RT_1$ credential \cite{Li:DRBTMF}, suppose company $A$
has a policy that the manager of an entity also evaluates that entity. This
can be expressed in $\RT_1$ using a policy statement such as
$$\cred{A.\mathit{evaluatorOf}(?Y)}{A.\mathit{managerOf(?Y)}}{}$$
This policy can't be feasibly expressed in $RT_0$ because the role parameters
might take on an arbitrarily large number of values. In $RT_0$ individual
credentials would be needed for each possible value of the role parameter.

$\RT_1^C$ further enhances the expressive power of $RT_1$ by allowing
structured constraints to be applied to role parameters. In addition the
restriction on variables only appearing in the body of a rule is lifted
\cite{Li:DCFTML,Li:RRBTMF}. For example, suppose a host $H$ wishes to grant
access to a particular range of TCP ports to those entities that are employed
by the information technology department. The host might have as its local
policy:
$$\cred{\mathit{Host}.p(\mathit{port} \in
[1024..2048])}{\mathit{IT}.\mathit{employee}}{}$$ 
This example assumes
that an entity is granted access to a particular TCP port if that
entity is a member of the $\mathit{Host}.p$ role with the port
specified as a parameter.

To accommodate threshold structures, representing agreement between
a group of principals, the system $\RT^T$ interprets roles as sets of
sets of entities, called \newterm{principal sets}.  These principle
sets can be combined with role product operators $\odot$ and
$\otimes$.  
%New credential forms are as follows:
%\begin{enumerate}
%\item $\cred{A.r}{B_1.r_1 \odot B_2.r_2 \odot \cdots \odot B_k.r_k}{}$ 
%
%Each principal set $p$ in A.r is formed by $p = p_1 \cup \cdots \cup p_k$
%where each $p_i \in B_i.r_i$ for $1 \le i \le k$.
%
%\item $\cred{A.r}{B_1.r_1 \otimes B_2.r_2 \otimes \cdots \otimes B_k.r_k}{}$ 
%
%Each principal set $p$ in A.r is formed by $p = p_1 \cup \cdots \cup p_k$
%where $p_i \cap p_j = \emptyset$ for all $i \ne j$ and $p_i \in B_i.r_i$
%for $1 \le i \le k$.
%
%\end{enumerate}
The features introduced by $\RT^T$ allow threshold policies and separation of
duty policies to be written \cite{Li:DRBTMF}.

$\RT^D$ adds the concepts of role activations and delegations to $RT_0$, via
the delegation credential form $\delcred{A}{B}{\activation{C}{D.r}}$. In this
case $A$ delegates to $B$ the \newterm{role activation} of
$\activation{C}{D.r}$. Empowered with this role activation $B$ can then access
whatever facilities $C$ can access from role $D.r$. This presupposes that $A$
has been delegated the activation $\activation{C}{D.r}$, which holds when $A =
C$ and $A$ is a member of role $D.r$ in the basic case. Hence, delegated
activations don't carry any authority unless there is a chain of delegation
credentials where the credential at the head of the chain was issued by the
entity mentioned in the role activation.

While the original $\RT$ framework does not support revocation in its
policy language, it is proposed to incorporate revocation
\cite{Li:DRBTMF} by leveraging a monotonic approach developed in
\cite{lbi-fc01} based on certificate lifetimes.  While lifetimes and
the requirement for freshness are encoded logically, the proposal
suggests the use of external certificate revocation lists to implement
verification; this is an interesting example of the possible interplay
between the semantics of authorization per se and components external
to them.  In addition, a variant of the $\RT$ framework has been
developed that associates risk values with credentials
\cite{skalka-wang-chapin-jcs06}. These risks are tracked 
through the authorization process so that the role membership is
parameterized by the total membership risk.  The set of risks and
their ordering is left abstract, and can be specialized to a number of
applications, e.g.~risk can be defined as remaining certificate
lifetime, so that role membership is parameterized by the minimal
lifetime of certificates used for authorization.

%The $\RT$ framework does not consider negative credentials or certificate
%revocation. As with SPKI/SDSI certificate validity is not part of the core
%system but must instead by checked externally (again, by component $T_C$ in
%\autoref{figure-tmstruct}) before credentials are submitted to the
%authorization decision logic.
%\cnote{Is revocation stuff stated somewhere in 
%the papers?  I don't think so, killed it.}
%
%Several extensions to the $\RT$ framework have also been described in
%the literature. $RT^R$ extends $\RT_0$ by associating a risk value
%with each credential \cite{Chapin:RADA}. These risks are tracked
%through the authorization process so that the role membership is
%parameterized by the total membership risk.  The set of risks and
%their ordering is left abstract, and can be specialized to a number of
%applications, e.g.~to represent an aggregation of certificate
%legitimacy measures to rate ``trustability'' of an authorization
%decision.  $\RT+_0$ extends $\RT_0$ by adding an integer delegation
%depth control to most credential forms \cite{Hong:DDCTMS}, a
%capability that $RT_0$ lacks. $\RT+_0$ delegation depths limit the
%delegation of authority by tracking the number of namespaces
%(administrative domains) such delegations cross. Delegation depth is
%also allowed to be unlimited, in which case $\RT+_0$ degenerates to
%$\RT_0$.

\subsubsection{Running Example}

To express the medical records example using $\RT$, only the
facilities of $\RT_0$ are necessary. Alice defines a role
\texttt{records} whose members are able to access her medical
records. She creates the policy
\begin{itemize}
\item \texttt{Alice.records} $\leftarrow$ \texttt{Bob}
\item \texttt{Alice.records} $\leftarrow$ \texttt{Bob.alice\_delegates}
\end{itemize}
The first rule grants her doctor, Bob, access to her records. The second
rule allows Bob to further delegate that access by defining the membership
of an \texttt{alice\_delegates} role.

Bob's standing policy is
\begin{itemize}
\item \texttt{Bob.team} $\leftarrow$ \texttt{Bob.team.support}
\item \texttt{Bob.alice\_delegates} $\leftarrow$
  \texttt{Hospital.medical\_staff} $\cap$ \texttt{Bob.team}
\end{itemize}
The first rule defines Bob's team as including all the support personnel
specified by the members of his team. In the second rule, Bob uses an
intersection role to specify that only the medical personnel on his team
should have access to Alice's medical records.

When Bob consults with Carol he adds \texttt{Bob.team} $\leftarrow$
\texttt{Carol} to his policy to add Carol, and indirectly all of Carol's
support people, to his team.

The only part of Carol's policy relevant to this example places Dave in her
\texttt{support} role: \texttt{Carol.support} $\leftarrow$ \texttt{Dave}.
Finally Dave has a credential from the hospital asserting his membership in
the \texttt{medical\_staff} role. $\RT_0$ can use these credentials to
prove that Dave is a member of \texttt{Alice.records} and thus able to
access Alice's medical records.

\subsubsection{Semantics}

The original formal semantics of $\RT$ is based on \datalog\ \cite{Li:DRBTMF}.
Specifically each $\RT$ credential is translated into a \datalog\ rule. The
meaning of a collection of $\RT$ credentials is defined in terms of the
minimum model of the corresponding \datalog\ program. In the case of the
$\RT_1^C$, \datalog\ with constraints is used \cite{Li:DCFTML}.

The translation from $\RT_0$ to \datalog\ requires only a single
predicate \textit{isMember} to assert when a particular entity is a member
of a particular role. The translation rules are shown below where \datalog\
variables are shown prefixed with `\textit{?}' to distinguish them from
constants.
\begin{enumerate}

\item $\cred{A.r}{E}{}$

$\textit{isMember}(E, A, r).$

\item $\cred{A.r}{B.s}{}$

$\textit{isMember}(\textit{?x}, A, r) \leftarrow
 \textit{isMember}(\textit{?x}, B, s).$

\item $\cred{A.r}{B.s.t}{}$

$\textit{isMember}(\textit{?x}, A, r) \leftarrow
 \textit{isMember}(\textit{?y}, B, s),
 \textit{isMember}(\textit{?x}, \textit{?y}, t).$

\item $\cred{A.r}{B_1.s_1 \cap \cdots \cap B_n.s_n}{}$

$\textit{isMember}(\textit{?x}, A, r) \leftarrow
 \textit{isMember}(\textit{?x}, B_1, s_1), \ldots,
 \textit{isMember}(\textit{?x}, B_n, s_n).$

\end{enumerate}
The authorizer associates a permission with a particular role, say $A.g$,
called the \newterm{governing role}. Access is granted to an entity $E$ iff
the \datalog\ query $\textit{isMember}(E, A, g)$ succeeds.

An alternative set-theory semantics has also been defined for $\RT_0$
\cite{Li:DCDTM}. In this semantics each role $A.r$ is represented as a set
of entities $\mbox{rmem}(A.r)$ that are members of that role. For a given
set of credentials $\creds$ these sets are the least sets satisfying the
set of inequalities
\begin{displaymath}
\{ \mbox{rmem}(A.r) \supseteq \mbox{expr}[\mbox{rmem}](e)\,|\,
   A.r \longleftarrow e \in \creds \}
\end{displaymath}
where $\mbox{expr}[\mbox{rmem}](e)$ is the set of entities in a
particular role expression $e$. A \newterm{role expression} includes
both linked roles and intersection roles. In particular:
\begin{eqnarray*}
\mbox{expr}[\mbox{rmem}](B) & = & \{B\} \\
\mbox{expr}[\mbox{rmem}](A.r) & = & \mbox{rmem}(A.r) \\
\mbox{expr}[\mbox{rmem}](A.r_1.r_2) & = &
  \bigcup_{B \in\, \mbox{rmem}(A.r_1)} \mbox{rmem}(B.r_2) \\
\mbox{expr}[\mbox{rmem}](f_1 \cap \cdots \cap f_k) & = &
  \bigcap_{1 \le j \le k} \mbox{expr}[\mbox{rmem}](f_j)
\end{eqnarray*}
The set-theory semantics for $\RT_0$ was developed primarily to provide
theoretical support for a distributed credential chain discovery algorithm
\cite{Li:DCDTM}. The set-theory semantics facilitate proving soundness and
completeness of that algorithm.

Another approach to the semantic specification of $\RT$ is taken by
Polakow and Skalka, who propose the LolliMon linear logic programming
language as a foundation \cite{polakow-skalka-plas06}.  Like the
set-theoretic specification, this approach has the advantage of being
easily extended to the problem of distributed certificate chain
discovery, while enjoying the additional benefit of scalability to the
full $\RT$ framework.  The encoding closely resembles the original
\datalog\  \textit{isMember} predicate defined above, and the logic 
of certificate discovery can be expressed by additional clauses in
LolliMon's rich formula language.

\subsubsection{Implementation}

Li et al.~describe an implementation strategy for $\RT_0$ in terms of a
construct called a credential graph $\mathgraph{\creds}$ \cite{Li:DCDTM}.
Each node in $\mathgraph{\creds}$ represents a role expression with
directed edges corresponding to each credential. In addition,
\newterm{derived edges} are added to represent the indirect relationships
between roles that are introduced by linked roles and intersections. An
entity is a member of a role iff there exists a path from the entity to the
role in $\mathgraph{\creds}$. Li et al.~prove that credential
graphs are sound and complete with respect to the set-theory semantics of
$\RT_0$.

In addition Li et al.~describe a distributed credential chain discovery
algorithm that finds a path in $\mathgraph{\creds}$ given initially incomplete
credentials \cite{Li:DCDTM}. The algorithm assumes that either the issuer or
subject of a credential can be contacted on-line and queried for more
credentials on demand. In this way missing credentials can be found as needed
to complete a proof of authorization. The algorithm can work either backward,
starting at the governing role and following credentials from issuer to
subject, or forward, starting at the entity representing the requester and
following credentials from subject to issuer. In general both approaches are
useful. In some cases a certificate authority will maintain a database of all
credentials issued, making the backward discovery algorithm effective. In
other cases credentials will be held by the subjects, making the forward
discovery algorithm more appropriate. To ensure that searches always succeed
when possible, a type system can be used to assign appropriate types to role
names. These types restrict the way credentials can be formed and specify
where credentials must be stored \cite{Li:DCDTM}.

The complexity of credential chain discovery in $\RT_0$ has been shown to be
log-space $\mathcal{P}$-complete using a reduction from the monotone circuit
value problem \cite{Li:DCDTM}.

%To study the complexity of authorization in a distributed context, Li et
%al.\ address general security analysis questions in the $\RT$ framework
%\cite{Li:BPOCSATM}. Let $\mathcal{P}$ be a collection of policy statements
%(credentials) that are possibly distributed over multiple locations.
%$\mathcal{P}$ includes credentials that are signed and passed from location
%to location as well as credentials that are part of each entities local
%policy. Li et al.~ask whether certain general queries about the state of
%the entire system can be answered with only partial information about the
%overall policy. For example an authorizer might want to check that the set
%of entities who can access a particular resource is included in some other
%set of entities, regardless of what changes are made in the policy defined
%by entities outside the authorizer's control. More specifically, the
%following kinds of queries were studied:
%\begin{enumerate}
%
%\item \textit{Membership:} $A.r \sqsupseteq \{ E_1, \ldots, E_n \}$. This
%means that all the entities $E_1, \ldots, E_n$ are members of $A.r$.
%
%\item \textit{Boundedness:} $\{ E_1, \ldots, E_n \} \sqsupseteq A.r$. This
%means that the members of $A.r$ are among the entities $E_1, \ldots, E_n$.
%
%\item \textit{Inclusion:} $A.r \sqsupseteq B.s$. This means that the
%members of $B.s$ are among the members of $A.r$.
%
%\end{enumerate}
%Li et al.~perform their analysis on the assumption that certain roles
%(provided as a parameter to the analysis) cannot be changed in certain
%ways. The idea is that an authorizer is only concerned about changes in
%roles he or she does not control. Presumably the authorizer would rerun
%the analysis before making any changes to roles he or she does control.
%
%The analysis is performed by defining appropriate \datalog\ programs to
%address the particular query. Li et al.~show that membership and
%boundedness queries can be answered with a complexity that is
%$O(|\mathcal{P}|^3)$, where $|\mathcal{P}|$ is the number of policy
%statements in $\mathcal{P}$ \cite{Li:BPOCSATM}. However, inclusion queries,
%probably the most useful in practice, are more computationally difficult.
%Using a subset of $RT_0$ obtained by removing linked roles and intersection
%roles, such queries can be answered using a suitable \newterm{stratified}
%\datalog\ program\footnote{A stratified \datalog\ program is one where the
%predicates defined in it can be assigned to different strata, such that
%each predicate only depends on the negation of predicates in lower strata
%\cite{Apt:TTDK}.} with a complexity in $\textbf{P}$. However, when
%intersections are added back to the system, the analysis becomes
%$\textbf{coNP}$-complete. When instead linked roles are added back to the
%system, the analysis becomes $\textbf{PSPACE}$-complete. Finally, inclusion
%queries in full $\RT_0$ have complexity that is in $\textbf{coNEXP}$,
%although the exact complexity of such queries is currently an open
%question. Although some of the queries studied are computationally
%intractable they are all decidable. This is significant since in some other
%authorization systems, such as the HRU model, even membership queries are
%in general undecidable
%\cite{Harrison:POS}.

