\section{Features of Trust Management Systems}
\label{section-features}

In this section we describe and discuss features relevant to trust
management.  We do not intend this listing to be exhaustive, rather we
intend to focus on features that are generally considered important
for trust management applications.  Our goal is to more deeply
characterize trust management systems, and to provide a means for
comparison of various systems later in the paper.

We name and discuss features below, commenting on their relevance to trust
management and noting important implementation issues. Particular trust
management systems will be discussed in detail in \autoref{section-review},
but in anticipation of that and in order to provide a thumbnail reference,
Table~\ref{tbl:tmsummary1} and Table~\ref{tbl:tmsummary2} summarize feature
sets for the collection of trust management systems we survey. Recall from
the previous section that we understand trust management systems to
include more than just the core authorization semantics, but also ancillary
components such as translation from certificates to credentials. Thus, some
systems are said to possess features that are realized in ancillary
components instead of the core semantics. Also, while some systems are
not explicitly designed to support certain features, their semantics is
sufficiently expressive to simulate them, and such instances are listed in
the table. The order in which the systems are listed is
intended to follow an approximate chronological order of their development.
The order is approximate because some of the systems were developed over
a considerable span of time and it is difficult to specify precisely
when they reached a mature state.

%
% This file contains a table summarizing various trust management systems.
%

% Width of table cells...
\newlength{\sumw}
\setlength{\sumw}{0.80in}

% Less typing...
\newcommand{\rr}{\raggedright}

% Trick for allowing raggedright in table cells...
\newcommand{\RBS}{\let\\=\tabularnewline}


%%%%%%%%%%%%%%%%%%
%%%%% PART I %%%%%
%%%%%%%%%%%%%%%%%%

\begin{sidewaystable}
\centering

\begin{tabular}{|p{\sumw}||p{\sumw}|p{\sumw}|p{\sumw}|p{\sumw}|p{\sumw}|p{\sumw}|} \hline
               &
   PolicyMaker &
   KeyNote     &
   SPKI/SDSI   &
   REFEREE     &
   QCM         &
   OASIS
   \\ \hline \hline

% Add a row for syntax? What would a typical entry look like?

 \rr Formal Foundation &
   \rr Graph theory                      &  % PolicyMaker
   \rr Graph theory                      &  % KeyNote
   \rr Set theory. First order logic     &  % SPKI/SDSI
   \rr Not formalized                    &  % REFEREE
   \rr Relational algebra                &  % QCM
   \rr\RBS First order logic                % OASIS
   \\ \hline

 \rr Authorization Procedure &
   \rr Graph search                      &  % PolicyMaker
   \rr Graph search                      &  % KeyNote
   \rr Tuple reduction                   &  % SPKI/SDSI
   \rr Arbitrary                         &  % REFEREE
   \rr Distributed database query        &  % QCM
   \rr\RBS Unspecified                      % OASIS
   \\ \hline

 \rr Authorization Complexity &
   \rr Undecidable                       &  % PolicyMaker
   \rr Undecidable                       &  % KeyNote
   \rr Polynomial                        &  % SPKI/SDSI
   \rr Potentially undecidable           &  % REFEREE
   \rr Unpublished                       &  % QCM
   \rr\RBS Undecidable                      % OASIS
   \\ \hline

 \rr Public Key Infrastructure &
   \rr External                          &  % PolicyMaker
   \rr External                          &  % KeyNote
   \rr External                          &  % SPKI/SDSI
   \rr Internal                          &  % REFEREE
   \rr Internal                          &  % QCM
   \rr\RBS External (undefined)             % OASIS
   \\ \hline

 \rr Threshold Policies &
   \rr Yes                               &  % PolicyMaker
   \rr Yes                               &  % KeyNote
   \rr Yes                               &  % SPKI/SDSI
   \rr No                                &  % REFEREE
   \rr No                                &  % QCM
   \rr\RBS No                               % OASIS
   \\ \hline

 \rr Local Name Spaces &
   \rr Can be simulated                  &  % PolicyMaker
   \rr No                                &  % KeyNote
   \rr Yes                               &  % SPKI/SDSI
   \rr No                                &  % REFEREE
   \rr Yes                               &  % QCM
   \rr\RBS Yes                              % OASIS
   \\ \hline

 \rr Role-Based Access Control &
   \rr Can be simulated                  &  % PolicyMaker
   \rr Can be simulated                  &  % KeyNote
   \rr Yes                               &  % SPKI/SDSI
   \rr Can be simulated                  &  % REFEREE
   \rr Yes                               &  % QCM
   \rr\RBS Yes                              % OASIS
   \\ \hline

 \rr Delegation of Rights &
   \rr Yes                               &  % PolicyMaker
   \rr Yes                               &  % KeyNote
   \rr Yes (boolean depth)               &  % SPKI/SDSI
   \rr Yes                               &  % REFEREE
   \rr No                                &  % QCM
   \rr\RBS Yes                              % OASIS
   \\ \hline

 \rr Certificate Validity &
   \rr Depends on assertion language     &  % PolicyMaker
   \rr Internal                          &  % KeyNote
   \rr External                          &  % SPKI/SDSI
   \rr Internal                          &  % REFEREE
   \rr External                          &  % QCM
   \rr\RBS Internal                         % OASIS
   \\ \hline
   
 \rr Credential Negation &
   \rr No                                &  % PolicyMaker
   \rr No                                &  % KeyNote
   \rr No                                &  % SPKI/SDSI
   \rr Yes                               &  % REFEREE
   \rr No                                &  % QCM
   \rr\RBS No                               % OASIS
   \\ \hline
   
 \rr Credential Revocation &
   \rr External                          &  % PolicyMaker
   \rr External                          &  % KeyNote
   \rr On-line revocation check          &  % SPKI/SDSI
   \rr External                          &  % REFEREE
   \rr Internal                          &  % QCM
   \rr\RBS Internal                         % OASIS
   \\ \hline

% QCM: In \cite{Gunter:GCR} the base QCM system is extended to support
% revocation. The formal syntax and semantics of QCM is enhanced and
% monotonicity is proved.

 \rr Distributed Chain Discovery &
   \rr No                                &  % PolicyMaker
   \rr No                                &  % KeyNote
   \rr No                                &  % SPKI/SDSI
   \rr Yes                               &  % REFEREE
   \rr Yes                               &  % QCM
   \rr\RBS No                               % OASIS
   \\ \hline

\end{tabular}

\caption{Summary of Trust Management Systems, Part I}
\label{tbl:tmsummary1}
\end{sidewaystable}

%%%%%%%%%%%%%%%%%%%
%%%%% PART II %%%%%
%%%%%%%%%%%%%%%%%%%

\begin{sidewaystable}

\begin{tabular}{|p{\sumw}||p{\sumw}|p{\sumw}|p{\sumw}|p{\sumw}|p{\sumw}|p{\sumw}|p{\sumw}|p{\sumw}|} \hline
               &
   PCA         &
   TPL         &
   SD3         &
   Binder      &
   RT          &
   Cassandra   &
   \textsc{Protune}
   \\ \hline \hline

 \rr Formal Foundation &
   \rr Higher order logic                &  % PCA
   \rr Prolog                            &  % TPL
   \rr Datalog                           &  % SD3
   \rr Datalog                           &  % Binder
   \rr Set theory. \datalogc             &  % RT
   \rr \datalogc                         &  % Cassandra
   \rr\RBS Stratified logic programs        % Protune
   \\ \hline

 \rr Authorization Procedure &
   \rr None                              &  % PCA
   \rr Not published                     &  % TPL
   \rr Datalog query                     &  % SD3
   \rr Datalog query                     &  % Binder
   \rr Graph search. Datalog query       &  % RT
   \rr Modified SLG resolution           &  % Cassandra
   \rr\RBS Logic program evaluation         % Protune
   \\ \hline

 \rr Authorization Complexity &
   \rr Decidable proof checking          &  % PCA
   \rr Undecidable                       &  % TPL
   \rr Polynomial                        &  % SD3
   \rr Polynomial                        &  % Binder
   \rr Polynomial                        &  % RT
   \rr Depends on constraint domain      &  % Cassandra
   \rr\RBS Potentially undecidable          % Protune
   \\ \hline

 \rr Public Key Infrastructure &
   \rr Internal                          &  % PCA
   \rr External                          &  % TPL
   \rr External                          &  % SD3
   \rr External                          &  % Binder
   \rr External                          &  % RT
   \rr External                          &  % Cassandra
   \rr\RBS Internal                         % Protune
   \\ \hline

 \rr Threshold Policies &
   \rr No                                &  % PCA
   \rr Can be simulated                  &  % TPL
   \rr No                                &  % SD3
   \rr No                                &  % Binder
   \rr Yes ($\RT^T$)                     &  % RT
   \rr Yes                               &  % Cassandra
   \rr\RBS Can be simulated                 % Protune
   \\ \hline

 \rr Local Name Spaces &
   \rr Can be simulated                  &  % PCA
   \rr No                                &  % TPL
   \rr Yes                               &  % SD3
   \rr Yes                               &  % Binder
   \rr Yes                               &  % RT
   \rr Yes                               &  % Cassandra
   \rr\RBS Can be simulated                 % Protune
   \\ \hline

 \rr Role-Based Access Control &
   \rr Can be simulated                  &  % PCA
   \rr Yes                               &  % TPL
   \rr Can be simulated                  &  % SD3
   \rr Can be simulated                  &  % Binder
   \rr Yes                               &  % RT
   \rr Yes                               &  % Cassandra
   \rr\RBS Can be simulated                 % Protune
   \\ \hline

 \rr Delegation of Rights &
   \rr Can be simulated                  &  % PCA
   \rr Yes (integer depth)               &  % TPL
   \rr Can be simulated                  &  % SD3
   \rr Can be simulated                  &  % Binder
   \rr Yes ($\RT^D$)                     &  % RT
   \rr Yes                               &  % Cassandra
   \rr\RBS Can be simulated                 % Protune
   \\ \hline

 \rr Certificate Validity &
   \rr Internal                          &  % PCA
   \rr External                          &  % TPL
   \rr External                          &  % SD3
   \rr External                          &  % Binder
   \rr External                          &  % RT
   \rr Internal                          &  % Cassandra
   \rr\RBS Internal                         % Protune
   \\ \hline
   
 \rr Credential Negation &
   \rr No                                &  % PCA
   \rr Yes                               &  % TPL
   \rr No                                &  % SD3
   \rr No                                &  % Binder
   \rr No                                &  % RT
   \rr\RBS No                            &  % Cassandra
   \rr\RBS Monotonic                        % Protune
   \\ \hline
   
 \rr Credential Revocation &
   \rr Monotonic revocation              &  % PCA
   \rr Internal                          &  % TPL
   \rr External                          &  % SD3
   \rr External                          &  % Binder
   \rr External                          &  % RT
   \rr Internal                          &  % Cassandra
   \rr\RBS External                         % Protune
   \\ \hline

 \rr Distributed Chain Discovery &
   \rr No                                &  % PCA
   \rr Yes (ad hoc)                      &  % TPL
   \rr Yes                               &  % SD3
   \rr No                                &  % Binder
   \rr Yes                               &  % RT
   \rr Yes                               &  % Cassandra
   \rr\RBS Yes                              % Protune
   \\ \hline

\end{tabular}

\caption{Summary of Trust Management Systems, Part II}
\label{tbl:tmsummary2}
\end{sidewaystable}



% The summary table contains:
%
% Formal Foundation
% Semantics
% Authorization Procedure
% Authorization Complexity
% Public Key Infrastructure
% Threshold Policies
% Local Name Spaces
% Role-Based Access Control
% Delegation of Rights
% Certificate Validity
% Distributed Chain Discovery
% Negative Certificates

% Formal Foundation

\subsection{Discussion of Features}

We now briefly describe trust management features at a conceptual
level.  Specific examples of these features in systems are given in
\autoref{section-review}.

\subsubsection{Formal Foundation} 
Since authorization systems are used in security-sensitive contexts,
mathematically precise descriptions of their behavior and formal
assurances of their correctness is essential.  A variety of formalisms
serve as effective foundations for the definition of trust management
authorization semantics.  As we describe later in this survey, these
can be divided into three main categories: logics, database
formalisms, and graph theory.

In the case of trust management systems based on logic, the
authorization problem is expressed in terms of finding a proof of a
particular formula representing successful resource access, with a
collection of suitable axioms representing policy.  Credentials
relevant to a particular decision become additional hypotheses to be
used in the proof.  Trust management systems based on database
formalisms (e.g.~relational algebra) see the authorization decision as
a query against a distributed database. The certificates issued by a
principal contain, in effect, tuples from relations that a principal
controls.  Trust management systems based on graph theory define the
authorization decision in terms of finding a path through a graph. The
request is represented by a particular node in the graph. Principals
are also graph nodes and the certificates they issue denote edges.

%In \autoref{section-review} we focus particularly on the systems
%described by logics, including those systems that use the logic based
%query language \datalog. The systems that use other formalisms are
%also discussed, although we cover them in less detail.

It is not unusual for a particular trust management system to be
described by more than one formalism. In fact, some aspects of trust
management are more naturally expressed using one formalism or
another.  Also, \datalog\ serves as both a database formalism and a
programming logic, and several trust management systems have been
specified in \datalog.

% Authorization Procedure. Authorization Complexity.

\subsubsection{Authorization Procedure. Authorization Complexity} Trust
management systems differ in exactly how the authorization decision is
implemented. In a broad sense this is due to differences in the way
the systems are described; systems using the same style of
formalization tend to use similar authorization procedures. This is
particularly evident among the systems using programming logics such
as \datalog\ as both their formal foundation and implementation.
However, some differences between systems result in significant
differences in how authorization is computed even when the underlying
formalism is the same, if certificate revocation is present in one
system but not another for example.  In some cases no authorization
procedure is given; the details of computing authorization is entirely
left to the implementors.

The computational complexity of the authorization decision is clearly of
practical interest. Authorization should be decidable
and tractable, but there is a trade off between the expressiveness of the
certificate and policy language and the complexity of the authorization
decision. For example, the systems that use Datalog with constraints
(\datalogc) can have various levels of computational complexity depending
on the constraint domain used \cite{Li:DCFTML}. Yet even trust management
systems with undecidable decision procedures can be potentially useful;
realistic policies may be decidable even if the general policy language is
not.
%The danger in such a system is that an
%attacker who submits carefully crafted certificates might be able to
%force the authorization decision to execute infinitely.

% Public Key Infrastructure

\subsubsection{Public Key Infrastructure (PKI)} It is common for trust
management systems to treat keys directly as principals. This creates a
conceptually clean design. In contrast some systems regard the human or
machine participants as the principals and encode a relationship between
principals and the keys that identify them. In the former case key bindings
are not represented in the authorization semantics, where in the latter
case they are. Although PKIs underpin the implementation of trust
management systems, the question here is: to what extent does a particular
trust management system directly concern itself with the details of key
management.

% Threshold Policies

\subsubsection{Threshold and Separation of Duty Policies} Many systems
support threshold policies, where at least $k$ out of a set of $n$ entities
must agree on some point in order to grant access. Threshold policies are
appealing since agreement provides confidence in situations wherein no
single authority is trusted by itself. The concept of separation of duty is
related to threshold policies. In the case of a separation of duty policy
entities from different sets must agree before access is granted.

For example a bank might require that two different cashiers approve a
withdrawal (same set---threshold policy). The bank might also require that a
cashier and a manager, who are not the same person, approve a loan
(different sets---separation of duty policy). In general threshold policies
and separation of duty policies cannot be implemented in terms of each
other, although some trust management systems provide support for both
\cite{Li:DRBTMF}.

% Local Name Spaces

\subsubsection{Local Name Spaces} It is desirable for trust management 
systems to allow each administrative domain to manage its own
name space independently.  Requiring that names be globally unique is
problematic and, in general infeasible. Although there have been
attempts at creating a global name space \cite{X500}, these attempts
have at best only been partially successful. The ability to reference
non-local name spaces is also a keystone of modern trust management,
in that it allows local policy to consider requesters that may not be
directly known to the local system.

% Role-Based Access Control

\subsubsection{Role-Based Access Control} In a large system with many 
principals it is often convenient to use role based access control
(RBAC) \cite{Ferraiolo:RBAC,Sandhu:RBACM}. In such a system
\newterm{roles} are used to associate a group of principals to a set
of permissions.  The use of roles simplifies administration since the
permissions granted to a potentially large group of principals are
defined in a single place.  RBAC is a conceptual foundation of modern
authorization technologies, so many trust management systems provide
features to support RBAC policies.

%Some trust management systems support this concept by casting the access
%control decision as a role membership decision. Access will be granted
%if the requestor is a member of an appropriate role but the precise
%meaning of the roles, in terms of the permissions that are connected to
%them, is defined outside the trust management environment. In contrast
%some trust management systems include a mechanism in their policy
%language to define permissions explicitly. In these systems the access
%control decision is directly rendered for a particular permission. In
%some cases roles are not provided directly but can be simulated by
%assigning an appropriate interpretation to suitable objects within the
%system.

% Delegation of Rights

\subsubsection{Delegation of Rights} All trust management systems allow an
authorizer to delegate authority. In other words, an authorizer can
specify third parties that have the authority to certify particular
attributes. We take this as one of the defining characteristics of a
trust management system.  In many applications a requester will also
want to delegate some or all of his or her rights to an intermediary
who will act on that requester's behalf.

Delegation of rights is important in a distributed environment. For
example a request may be made to an organization's front end system
that accesses internal servers where the request is ultimately
processed. The classic three-tier architecture of web applications
follows this approach.  In many environments the back end servers may
have their own access control requirements, in which case the
requester will need to delegate his or her rights to the front end
system for use when making requests to the internal servers.

Trust management systems differ in their support for rights
delegation. Delegation certificate forms may be formally provided, or
delegation can be simulated via more primitive forms.  Also,
delegation \emph{depth} can be modulated in some systems-- rather than
being purely transitive, delegation of rights may only be allowed to
be transferred between fixed $n$ principals. In some cases rights can be
delegated arbitrarily or not at all. A system that has this latter
feature is said to support boolean delegation depth.
%
%Although delegation of rights seems different than delegation of authority,
%many trust management systems can simulate a kind of delegation of rights
%using delegation of authority. To do this the authorizer writes a policy
%that allows those with access to the resource to grant access to others.
%For example, using the trust management system $\RT_0$, the principal $A$
%can express his willingness to accept delegated rights by writing the
%policy $A.r \leftarrow A.r.s$. Here we assume that a principal must be a
%member of role $A.r$ to gain access. With this policy any principal $E \in
%A.r$ can create a local role $s$ and add another principal $F$ to $E.s$.
%$F$ then becomes a member of $A.r$ and thus has access to the resource.
%Presumably the lifetime of the certificate issued by $E$ placing $F$ in
%$E.s$ would be short so that the delegation would only be valid for the
%duration of a single session.
%
%This approach does require an authorizer to write an appropriate policy
%ahead of time. However, even in systems that support delegation of rights
%as an explicit feature, authorizers often have the option of ignoring
%delegated rights if they choose. Thus it is not unusual for this matter to
%be under the control of the authorizer's policy regardless of the
%mechanism.
%
%[It seems like the trick I mentioned above requires the TM logic to support
%circular policy statements. I'm not sure that's universal, however. Systems
%without such an ability might not be able to simulate delegation of rights
%using delegation of authority.]

% Certificate Validity

\subsubsection{Certificate Validity} Since an authorizer receives certificates
from unknown and potentially untrustworthy entities, the validity of
those certificates must be checked.  Usually, signatures must be
verified and the certificate must not have expired, since in practice
certificates will almost always have a finite lifetime to ensure that
obsolete information cannot circulate indefinitely. In some systems
certificate validity is explicitly treated as part of the structure of
the trust management authorization semantics-- the component $L$
described in \autoref{section-overview-structure}.  In such cases
sufficient expressivity may exist in the policy language to specify
authentication policies \cite{Abadi:CACDS}, or, in a simpler (and
currently more popular) scenario, certificate lifetimes can be
directly represented in credentials and taken into account in policy
\cite{Bauer:GFACSW,lbi-fc01,skalka-wang-chapin-jcs06}. In other
systems, certificate validity is defined externally and checked as
part of the translation of certificates into credentials-- the
component $T_C$-- and not formally reflected in the authorization
semantics \cite{RFC-2693}. We note that it is a topic of lively debate
whether authorizers \cite{Rivest:CWECRL} or certificate authorities
\cite{McDaniel:RTCWECRL} should determine validity intervals for
authorization decisions.

\subsubsection{Credential Negation} Policy languages sometimes 
allow policy makers to specify that a credential \emph{not} be held.
For example, access to a resource may require that requesters not
possess a credential endowing them with a felon role.  In systems
using logic as a foundation for the semantics of authorization, this
is expressed as credential negation.  That is, authorization is
predicated on the negation of a role attribute expressed as a
credential.  Note that this makes the semantics nonmonotonic-- as more
credentials (facts) are added to the system, it is possible that fewer
authorizations succeed.  As noted in \cite{seamons-policy02}, this
makes credential negation a generally undesirable feature, since
nonmonotonic systems are potentially unsound in practice.  For example, if a
certificate is not discovered due to a network failure, access might be
granted that would otherwise have been denied.

\subsubsection{Certificate Revocation} Certificate revocation is 
similar to credential negation, but allows previously granted access
rights to be explicitly eliminated \cite{Rivest:CWECRL}.  Like
certificate validity, this can be implemented in the translation $T_C$
from certificates to credentials. For example, in SPKI/SDSI
\cite{RFC-2693} online revocation lists can be defined that filter out
revoked certificates prior to embedding as credentials for the
authorization decision.  At first glance it may appear that
certificate revocation entails nonmonotonicity, as does credential
negation.  However, it has been demonstrated that certificate
revocation can encoded monotonically in both the Proof Carrying
Authorization framework \cite{Bauer:GFACSW} and a logic-based PKI
infrastructure \cite{lbi-fc01}; we describe how in
\autoref{section-other-PCA}.  The technique points out a relation
between certificate revocation and certificate validity, in that
monotonic revocation can be based on lifetimes and the requirement to
renew certificates.  Various high-level approaches to and nuances of
certificate revocation are discussed in \cite{Rivest:CWECRL}.

% Negative Certificates
%
%\subsubsection{Negative Credentials}  
%In \emph{monotonic} systems each policy statement or credential can
%only add new access rights, never remove existing ones. In a
%non-monotonic system, negative certificates that explicitly deny an
%access right are possible. Many distributed environments avoid
%non-monotonic access control because such systems are often unsafe in
%practice: if a certificate is not received due to a network failure,
%access might be granted that would otherwise have been denied.  Also,
%non-monotonic authorization semantics are more conceptually and
%computationally complex.
%
%Another way of achieving the same effect is via certificate
%\emph{revocation} \cite{Rivest:CWECRL}, and by representing certificate
%revocation lists in the authorization semantics. Like certificate validity,
%this is also an issue that can be handled by the translation $T_C$ from
%certificates to credentials. For example, in SPKI/SDSI online revocation
%lists can be defined that filter out revoked certificates prior to
%embedding as credentials for the authorization decision. In contrast Proof
%Carrying Authorization treats revocation as part of the core logic by using
%an inference rule that only allows a credential to be used if a suitable
%revocation list is available. This allows revocation to be handled without
%introducing any non-monotonicity.

%A related issue is certificate revocation. Practical systems usually
%provide a way for a principal to explicitly revoke a certificate that
%becomes invalid before its expiration time. Since trust management systems
%allow certificates to be made pertaining to arbitrary attributes, this
%issue is more pressing than in other certificate based systems. For
%example, identities do not typically change very often but other attributes
%may be subject to rapid changes that are outside of the control of the
%certificate issuer. Thus it is useful for a trust management application to
%provide some way for an issuer to revoke certificates.
%
%Unfortunately the information that a certificate has been revoked is itself
%a kind of negative certificate. Introducing such a facility into a trust
%management system also introduces the problems of non-monotonicity. For this
%reason we group negative certificates and certificate revocation together.

% Distributed Chain Discovery

\subsubsection{Distributed Certificate Chain Discovery} Where do 
certificates for a particular access request come from?  In the
example in \autoref{section-overview-example}, it was assumed that the
requester presents all relevant certificates upon access request.  It
is also easy to imagine settings in which authorizers maintain local
databases of certificates.  More generally, certificates could be
stored anywhere in the network, as long as the local system has some
way of finding them.  Of course, given the potentially enormous number
of certificates on the network, it is necessary to define some means
of selectively retrieving only certificates that might pertain to a
particular authorization decision.  This problem is sometimes called
\emph{distributed certificate chain discovery} \cite{Li:DCDTM} or
\emph{policy directed certificate retrieval} \cite{Gunter:PDCR}.  In
both of these approaches the process of obtaining certificates is formally
well founded and not left to ad hoc techniques.
