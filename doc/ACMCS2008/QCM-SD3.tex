\subsection{QCM and SD3}

Many trust management systems have focused on authorization decisions
while setting aside issues of certificate storage and retrieval.  In
contrast, Query Certificate Manager (QCM)
\cite{Gunter:DALSI,Gunter:ADDUQCM,Gunter:PDCR} and its successor Secure
Dynamically Distributed Datalog (SD3)
\cite{Jim:STMSCE,Jim:DDQE} address the issue head-on, by treating 
trust management as essentially a distributed database problem.  An
advantage of their approach is that well-studied database techniques
and abstractions can be leveraged.  In particular, the system provides
applications programmers with high-level database query languages for
defining authorization policy over a transparent PKI infrastructure,
where authorization is implemented as a query processed automatically
over a distributed database.  Among the implementation benefits of
their distributed database approach are a variety of optimization
techniques and a natural incorporation of distributed certificate
retrieval.  SD3 also introduces a novel certified evaluation mechanism
that reduces the size of its trusted computing base.
% whereby the results of query evaluation are
% certified by a machine-checkable proof of evaluation correctness.
% Since the evaluator can be a complex program, while the proof checker
% is simple, reliance on the proof checker for correctness significantly
% reduces the size of the trusted computing base.

\subsubsection{Features}
QCM hides from end-users the complexity of distributed query
evaluation and certificate retrieval. Instead it presents a high level
abstraction of a secure, local database.  The original presentation of
QCM proposed a policy language based on relational algebra
\cite{Gunter:DALSI}.  Consider the following example of a web page
content filtering application taken from Gunter and Jim
\cite{Gunter:DALSI}. Here a ratings server $r_1$ maintains a relation
\text{\bf ratings} containing rating information for a large
collection of web pages.  An independent server $r_2$ maintains a
similar relation. A web browser then defines a local relation in terms
of these other two using relational algebra expressions:
\begin{eqnarray*}
\textbf{ratings} & = &
   r_1.\textbf{ratings} \cap r_2.\textbf{ratings} \\
\textbf{ok}      & = &
   \pi_{\text{hash}}(\sigma_{\text{rating} = G}\textbf{ratings})
\end{eqnarray*}
The browser will only accept ratings for which the two rating servers
agree. In addition, the browser's \textbf{ok} table contains only the
page hashes for the pages with a $G$ rating.

Now, suppose that the browser was governed by the policy that it would
only display web pages with a $G$ rating.  Formally, assuming that $h$
is the hash of a web page, it would only be displayed if
$\sigma_{\text{hash}=h}\textbf{ok}$ is not null.  In the simplest
scenario, the browser could enforce this by submitting the query
$\sigma_{\text{hash}=h}\textbf{ok}$ to a local QCM processor, and
the processor would in turn query $r_1$ and $r_2$ remotely.  However,
a more efficient and flexible scheme is allowed in QCM-- tuples in a
database relation can be certified by the relation authority, and
distributed as certificates.  Such certificates, called 
\emph{inclusions}, are denoted via an
ABLP-style \textbf{says} connective.  For example, the web server
that hosts a page with hash $h$ can obtain certificates from the
rating servers for that page. Each certificate is signed by the
corresponding rating server:
\begin{eqnarray*}
r_1\,\,\textbf{says}\,\,r_1.\textbf{ratings} & \supseteq &
   \{(\text{hash}: h, \text{rating}: G)\} \\
r_2\,\,\textbf{says}\,\,r_2.\textbf{ratings} & \supseteq &
   \{(\text{hash}: h, \text{rating}: G)\} 
\end{eqnarray*}
When the page is requested by a browser, the two certificates are sent
as well, and from there to the local QCM processor along with the
query $\sigma_{\text{hash}=h}\textbf{ok}$.  Now the processor does
not need to contact the remote servers because the certificates
contain enough information to answer the query directly, and can even
cache the certificate contents for future use.  This scheme is clearly
more efficient, and has the additional benefit that not all relation
authorities need to be online during authorization.

Also notable is QCM's support for certificate revocation
\cite{Gunter:GCR}. This is done by allowing a set of revoked tuples to
be subtracted from a set of otherwise potentially useful tuples. QCM
provides for explicit \newterm{non-membership certificates} that can
be used to assert that a tuple is not an element of the revoked set.
This later work adopts a set-theoretic model of the QCM database to
accommodate the notion of non-membership, and a new language of set
comprehension is defined for QCM programming.

As an example of this set comprehension language, consider the earlier
example of web page content filtering. Rating server $r_1$ maintains a
relation that binds page hashes to rating values
$r_1.\textbf{ratings}(\text{hash}, \text{rating})$, and similarly for rating
server $r_2$. The browser's policy then defines a set \textbf{ok} of
acceptable page hashes with the statement
\begin{eqnarray*}
\text{\bf ok} = \{h\,|\,
  \langle \text{hash} = h, \text{rating} = G \rangle & \leftarrow & 
    r_1.\textbf{ratings}, \\
  \langle \text{hash} = h, \text{rating} = G \rangle & \leftarrow &
    r_2.\textbf{ratings}\,\}
\end{eqnarray*}
This defines a set of hash values for pages that both ratings servers agree
are $G$ rated. Queries also have the form of set expressions. When the
browser retrieves a page it asks its local QCM processor to evaluate the
query $\{p\,|\,p \leftarrow \textbf{ok}, p = h\}$ where $h$ is the hash of
the retrieved page. If this set evaluates to the singleton $\{h\}$ then the
browser can display the page as $G$ rated.

Secure Dynamically Distributed Datalog (SD3) is the successor of
QCM. It adds to QCM an extended version of \datalog\ as its policy and
credential language, allowing recursive policies to be defined. In SD3
predicates are scoped by public keys; rules can refer to predicates in
other name spaces by prepending the key to the predicate name. For
example, suppose that the predicate $E$ under control of key $K$
defines the edge relation of a particular graph. The following SD3
program computes the transitive closure of that graph.
\begin{eqnarray*}
T(X, Y) &\leftarrow& K\$E(X, Y). \\
T(X, Y) &\leftarrow& T(X, Z), T(Z, Y).
\end{eqnarray*}
SD3 also adds other notable implementation features, including
intentional responses and certified evaluation, discussed below.

\subsubsection{Running Example}

Here we demonstrate how SD3 would express the medical records example. We
need to first express the information to be processed as a collection of
relations. Alice maintains a one-place relation
\texttt{records} with tuples storing the keys of those principals who can
access her medical records. We can then represent Alice's policy as the
following two SD3-style \datalog\ rules.
\begin{itemize}
\item \texttt{$K_a$\$records($K_b$)}
\item \texttt{$K_a$\$records(X) $\leftarrow$ $K_b$\$alice\_delegates(X)}
\end{itemize}
\datalog\ has no problems expressing either recursion or intersections
(conjunctions). Bob's policy becomes
\begin{itemize}
\item \texttt{$K_b$\$team(X) $\leftarrow$ $K_b$\$team(Y), Y\$support(X)}
\item \texttt{$K_b$\$alice\_delegates(X) $\leftarrow$ $K_h$\$medical\_staff(X), $K_b$\$team(X)}
\item \texttt{$K_b$\$team($K_c$)}
\end{itemize}
The remaining assertions made by the hospital and by Carol are
\begin{itemize}
\item \texttt{$K_h$\$medical\_staff($K_d$)}
\item \texttt{$K_c$\$support($K_d$)}
\end{itemize}
This example only hints at the expressivity of the general SD3 policy
language.

SD3 distinguishes between global names that are key-qualified and local
names that are not. In this way SD3 supports multiple, independent name
spaces. If the hospital database evaluates Dave's request to update Alice's
medical records in the context of Alice's name space, then the $K_a$ prefix
on Alice's policy is superfluous. The tuples from relations in other name
spaces would be signed by the corresponding key and obtained from some
source external to Alice's name space.

\subsubsection{Semantics} The authors of QCM and SD3 have used 
a variety of formal semantics for different aspects and versions of
the system.  In the original presentation of QCM \cite{Gunter:DALSI},
the core authorization semantics are the semantics of the relational
algebra.  Additionally, an I/O automata model of network communication
is developed to verify that certain checks during updates guarantee
data consistency \cite{Gunter:ADDUQCM}.  In this model each QCM node
maintains a set of pending queries, a set of inclusions that it
accepts (initialized to the definitions known directly by the node),
and a set of requests that the node has made. The automaton specifies
how this state changes for each possible input or output action. A
corresponding automaton models the network itself. This allows
different network models, including potentially hostile models, to be
studied in a straight forward way.

In later work that addresses certificate revocation, a set-theoretic
model of the more recent QCM language of set comprehension is
developed as a denotational semantics \cite{Gunter:GCR}.  QCM
expressions are interpreted as set operations in a universe of QCM
values, which include numbers, strings, keys, and finite and cofinite
sets of values.  An operational semantics describing the behavior of
QCM evaluation is defined, and is shown to be sound with respect to
the denotational semantics.  It is important for the operational
semantics that QCM objects are only modeled as single values or finite
or cofinite sets, since this means that they can be finitely
represented.

The semantics of SD3 are based on the semantics of \datalog\ in a
fairly straightforward manner; the only complication is the
interpretation of key-qualifiers on predicate names.  To describe the
semantics of a distributed SD3 program, first a \newterm{global
\datalog\ program} is constructed from a given SD3 program by
replacing each $n$-ary predicate $R$ with an $(n+1)$-ary predicate
$R^g$, and each atom of the form $s\$R(t_1, \ldots, t_n)$ with an atom
of the form $R^g(s, t_1, \ldots, t_n)$. The semantics of an SD3
program is the minimum model of the resulting \datalog\ program
\cite{Jim:DDQE}.

While the formal meaning of QCM and SD3 programs has evolved
throughout development of these systems, the authors argue that their
interpretation has been essentially consistent, since relational
algebra, set comprehension, and \datalog\ are ``roughly equivalent by
variations of Codd's Theorem'' \cite{Jim:STMSCE}.


\subsubsection{Implementation} Algorithms for query processing in 
the QCM and SD3 systems has been defined and proven correct
\cite{Gunter:GCR,Jim:STMSCE} with respect to a number of safety and
security requirements, e.g.~soundness of the algorithms with respect
to the denotational meaning of programs.  The distributed database
approach also allows a number of standard optimization techniques to
be applied, notably magic set rewriting \cite{Jim:DDQE}.  Beyond this,
QCM and SD3 also offer several novel implementation features.

When a QCM node is queried the result is a collection of signed
tuples, possibly obtained indirectly from other nodes, forming an
\newterm{extensional} response to the query. SD3 extends this by
allowing a node to instead return an \newterm{intensional} response
consisting of one or more rules, perhaps in terms of relations held by
other nodes, that define the result of the query. In such a case the query
originator could contact the other nodes if necessary to obtain the
information needed to fully evaluate the query.

An example from Jim and Suciu \cite{Jim:DDQE} illustrates this distinction.
Suppose that an SD3 server has the rule $R(x,y)\,\texttt{:-}\,E(x, w, z),
w\$R(z, y)$ and it receives from the client the query $R(1, y)$. Suppose
also that the server has the tuples $E(1, s_2, 2)$ and $E(1, s_3, 3)$ in
its local table. The server could return the intensional response of
\begin{displaymath}
\begin{array}{c}
R(1, y)\,\texttt{:-}\,s_2\$R(2, y) \\
R(1, y)\,\texttt{:-}\,s_3\$R(3, y)
\end{array}
\end{displaymath}
The client could then contact sites $s_2$ and $s_3$ to complete the
query based on these rules.  

Jim describes a prototype SD3 system that implements the DNSSEC protocol
\cite{Jim:STMSCE}. In order to obtain the performance needed in DNS
applications, Jim's implementation uses a number of elaborate optimization
techniques. These optimizations add complexity to the implementation and
increase the size of the trusted computing base. To deal with this Jim's
implementation uses \newterm{certified evaluation}. The output of the SD3
query evaluator is checked by a relatively simple proof checker. If the
check fails, the results of the query are considered erroneous. Since proof
checking is easier than proof construction, the proof checker can be small
and simple, thus reducing the size of the trusted computing base.

QCM supports distributed credential chain discovery, which the QCM authors
refer to as \newterm{policy-directed certificate retrieval}
\cite{Gunter:PDCR}. Such distributed queries are satisfied extensionally.
For example if a QCM node $a$ defines a relation $\text{\bf ratings} =
b.\text{\bf ratings}$ that node would answer queries about the membership
of $\text{\bf ratings}$ by querying $b$. If $b$ defined its $\text{\bf
ratings}$ relation in some complex way it might query other nodes as
appropriate. However, node $b$ would return signed tuples from its
$\text{\bf ratings}$ relation rather than a signed policy rule.

QCM's system of distributed credential chain discovery should be contrasted
with that described earlier for $\RT_0$. In the $\RT_0$ case distributed
queries are satisfied intentionally: policy rules are passed back to the
authorizing node where the entire credential chain is computed. This allows
$\RT_0$ to make direct use of credentials provided with the request without
having to transmit those credentials to other nodes. 

% ===========================
% Here is the original draft.
% 
% Query Certificate Manager (QCM)
% \cite{Gunter:DALSI,Gunter:ADDUQCM,Gunter:PDCR} and its successor Secure
% Dynamically Distributed Datalog (SD3)
% \cite{Jim:STMSCE,Jim:DDQE} use a distributed database approach to the trust
% management problem. This approach is based on the observation that an
% authority on an attribute is, in effect, the maintainer of a relation about
% that attribute. For example the student records office at a university
% might maintain a one-place relation listing the public keys of all honors
% students or, more generally, a two-place relation binding the public keys
% of students to the students' grade point averages. Even if the relations
% are not made explicit, it is conceptually useful to regard the information
% maintained by an authority to be organized in this way. Doing so allows
% trust management systems to take advantage of the considerable body of
% knowledge about database systems.
% 
% When a QCM node is queried the result is a collection of signed tuples,
% possibly obtained indirectly from other nodes, forming an
% \newterm{extensional} response to the query. SD3 extends this by allowing a
% node to instead return an \newterm{intensional} response consisting of a
% rule, perhaps in terms of relations held by other nodes, that defines the
% result of the query. In such a case the query originator could contact the
% other nodes if necessary to obtain the information needed to fully evaluate
% the query. \cnote{This needs more description-- why isn't the original 
% query a sifficient formula?}
% 
% QCM and SD3 express the distributed authorization problem as a distributed
% database query problem, making the considerable research in that field
% \cite{Kossmann:SADQP} relevant to trust management applications. Thus
% QCM/SD3 provides support for distributed credential chain discovery as a
% consequence of their handling of distributed queries. In addition because
% QCM/SD3 provides a mechanism to securely query distributed databases, they
% can also be used in other contexts besides authorization applications. For
% example, it is straightforward to define a public key infrastructure using
% QCM/SD3 where distributed queries are used to obtain certified keys.
% 
% Note also that during a query it is not necessary for all attribute
% authorities to be on-line at that time. A query can be accompanied by
% relevant tuples or rules that have been previously signed by appropriate
% authorities. This is similar to the way a requester in a SPKI/SDSI or RT
% system can provide certificates containing signed policy statements along
% with the request.
% 
% QCM uses sets of tuples with relational algebra operations as its formal
% foundation. In this context monotonicity is obtained by disallowing the set
% difference operation, thus rendering the system relationally incomplete but
% offering robust behavior in the face of network failures. SD3, using
% \datalog, extends the relational algebra by adding support for recursive
% queries. Thus \datalog\ provides a bridge between the logical approach and
% the database approach for handling trust management systems.
% 
% \cnote{Maybe too much detail in intro section, should be 
% inlined throughout other appropriate sections.}
% 
% \subsubsection{Features}
% 
% QCM hides from end-users the complexity of distributed query
% evaluation and certificate retrival. Instead it presents a high level
% abstraction of a secure, local database, via a policy language based
% on relational algebra.  Consider the following example of a web page
% content filtering application taken from Gunter and Jim
% \cite{Gunter:DALSI}. Here a ratings server
% $r_1$ maintains a relation \text{\bf ratings} containing rating
% information for a large collection of web pages.  An independent
% server $r_2$ maintains a similar relation. A web browser then defines
% a local relation in terms of these other two using relational algebra
% expressions.
% \begin{eqnarray*}
% \text{\bf ratings} & = & r_1.\text{\bf ratings} \cap r_2.\text{\bf ratings} \\
% \text{\bf ok} & = & \pi_{\text{hash}}(\sigma_{\text{rating} =
%   G}\text{\bf ratings})
% \end{eqnarray*}
% The browser will only accept ratings for which the two rating servers
% agree. In addition, the browser's {\bf ok} table contains only the
% page hashes for the pages with a $G$ rating.
% 
% A web server that hosts a particular page with hash $h$ can obtain
% certificates from the rating servers for that page. Each certificate is
% signed by the corresponding rating server, indicated in ABLP style using a
% \text{\bf says} connective in the formulae below. The certificates assert
% that a particular tuple is a member of a particular relation. For example
% \begin{eqnarray*}
% r_1\,\,\text{\bf says}\,\,r_1.\text{\bf ratings} & \supseteq &
% \{(\text{\bf hash}: h, \text{\bf rating}: G)\} \\
% r_2\,\,\text{\bf says}\,\,r_2.\text{\bf ratings} & \supseteq &
% \{(\text{\bf hash}: h, \text{\bf rating}: G)\} 
% \end{eqnarray*}
% These expressions are called \newterm{inclusions} since they only assert a
% lower bound on the query result. When the page is requested by a browser,
% the two certificates are sent as well. The browser submits the certificates
% to a local QCM processor along with the query
% $\sigma_{\text{hash}=h}\text{\bf ok}$. In this case QCM does not need to
% contact the remote servers because the certificates contain enough
% information for it to answer the query directly.
% 
% Of course the web browser would need the public keys associated with the
% two remote servers $r_1$ and $r_2$. Presumably those keys could be obtained
% from a key server which is itself a kind of remote database. Thus the
% complete QCM program at the browser would contain statements about the
% composition of a public key relation as well as the rating relation.
% Similarly a QCM principal can define a group or role by defining a relation
% that contains the group's members.
% 
% QCM supports distributed credential chain discovery, which the QCM authors
% call \newterm{policy-directed certificate retrieval} \cite{Gunter:PDCR}.
% Such distributed queries are satisfied extensionally. For example if a QCM
% node $a$ defines a relation $\text{\bf ratings} = b.\text{\bf ratings}$
% that node would answer queries about the membership of $\text{\bf ratings}$
% by querying $b$. If $b$ defined its $\text{\bf ratings}$ relation in some
% complex way it might query other nodes as appropriate. However, node $b$
% would return signed tuples from its $\text{\bf ratings}$ relation rather
% than a signed policy rule.
% 
% QCM's system of distributed credential chain discovery should be contrasted
% with that described earlier for $\RT_0$. In the $\RT_0$ case distributed
% queries are satisfied intentionally: policy rules are passed back to the
% authorizing node where the entire credential chain is computed. This allows
% $\RT_0$ to make direct use of credentials provided with the request without
% having to transmit those credentials to other nodes. 
% \cnote{The discovery stuff is in the implementation subsection of the 
% $\RT$ section, why not here?}
% 
% Although QCM does not support general negation, it does provide a limited
% form of negation that allows certificate revocation to be handled
% \cite{Gunter:GCR}. This is done by allowing a set of revoked tuples to be
% subtracted from a set of otherwise potentially useful tuples. QCM provides
% for explicit \newterm{non-membership certificates} that can be used to
% assert that a tuple is not an element of the revoked set.
% 
% Secure Dynamically Distributed Datalog (SD3) is a successor of QCM. It uses
% an extended version of \datalog\ as its policy and credential language. In
% SD3 predicates are scoped by public keys; rules can refer to predicates in
% other name spaces by prepending the key to the predicate name. For example
% suppose that the predicate $E$ under control of key $K$ defines the edge
% relation of a particular graph. The following SD3 program computes the
% transitive closure of that graph.
% \begin{verbatim}
% T(X, Y) :- K$E(X, Y).
% T(X, Y) :- T(X, Z), T(Z, Y).
% \end{verbatim}
% \cnote{You should be consistent with fonts in datalog(-like) syntax.}
% 
% As with QCM the SD3 query evaluator can fetch relations, such as $E$, from
% remote systems. However, unlike QCM, SD3 provides intentional query
% responses. In some cases signed \datalog\ rules are returned from a query
% \cite{Jim:DDQE}.
% 
% \subsubsection{Running Example}
% 
% Here we demonstrate how SD3 would express the medical records example. We
% need to first express the information to be processed as a collection of
% relations. Alice maintains a one-place relation
% \texttt{records} with tuples storing the keys of those principals who can
% access her medical records. We can then represent Alice's policy as the
% following two SD3-style \datalog\ rules.
% 
% \begin{itemize}
% \item \texttt{$K_a$\$records($K_b$).}
% \item \texttt{$K_a$\$records(X) :- $K_b$\$alice\_delegates(X).}
% \end{itemize}
% 
% \datalog\ has no problems expressing either recursion or intersections
% (conjunctions). Bob's policy becomes
% 
% \begin{itemize}
% \item \texttt{$K_b$\$team(X) :- $K_b$\$team(Y), Y\$support(X).}
% \item \texttt{$K_b$\$alice\_delegates(X) :- $K_h$\$medical\_staff(X), $K_b$\$team(X).}
% \item \texttt{$K_b$\$team($K_c$).}
% \end{itemize}
% 
% The remaining assertions made by the hospital and by Carol are
% 
% \begin{itemize}
% \item \texttt{$K_h$\$medical\_staff($K_d$).}
% \item \texttt{$K_c$\$support($K_d$).}
% \end{itemize}
% 
% This example does not illustrate much of SD3's expressiveness. All the
% relations involved are one-place relations containing keys. These
% correspond to the roles of $\RT$ and the names of SDSI. However, SD3 is
% more general because the relations involved can be arbitrary.
% 
% SD3 distinguishes between global names that are key-qualified and local
% names that are not. If the hospital database evaluates Dave's request to
% update Alice's medical records in Alice's context, then the $K_a$ prefix on
% Alice's policy is superflous. The tuples from the other relations would be
% signed by the corresponding key and obtained ``externally'' from Alice's
% context.
% 
% \subsubsection{Semantics}
% 
% QCM is fundamentally a language about sets. In a database context these
% sets are the relations in a distributed database. QCM has a type system
% consisting of application specific basic datatypes, a type for principals,
% product types and set types. The type system ensures that certain nonsense
% expressions, such as selecting a field from a relation that doesn't have
% such a field, are rejected.
% 
% The semantics of QCM are described in two parts: one part deals with the
% network communication and the other part deals with the computations done
% on a single node \cite{Gunter:ADDUQCM}. I/O automata are used to describe
% how nodes exchange messages. Each QCM node maintains a set of pending
% queries, a set of inclusions that it accepts (initialized to the
% definitions known directly by the node), and a set of requests that the
% node has made. The automaton specifies how this state changes for each
% possible input or output action. A corresponding automaton models the
% network itself. This allows different network models, including potentially
% hostile models, to be studied in a straight forward way. \pnote{Should we
% include a figure with some specifics on the I/O automata used?}
% 
% The computational aspects of QCM are described with both a denotational
% semantics and an operational semantics \cite{Gunter:GCR}. These semantics
% are given for an extension of QCM supporting certificate revocation. The
% idea is to use a polarity discipline to define both positive and negative
% names. Negative names describe co-finite sets\footnote{A co-finite set is a
% set that is the complement of a finite set in an infinite universe.}. In
% this context a revocation defines a co-finite set of non-membership
% credentials. \pnote{The formal semantics of QCM does not appear to be
% spelled out anywhere. In the ``main'' QCM paper, which is unpublished, the
% authors allude to another paper where the sematics are provided, but they
% don't give a reference and I can't find any such paper. On the other hand,
% the semantics in ``Generalized Certificate Revocation'' is complicated by
% the whole polarity issue.}
% 
% QCM does not necessary accept every certificate submitted to it. Doing so
% invites the possibility of accepting an inconsistent collection of
% certificates. In such a case there is no model for the relations given and
% any inclusion can follow. Thus a QCM node first verifies that the contents
% of a certificate, $(p\,\,\textbf{says}\,\,M \supseteq N)$, is acceptable.
% This means that every relation symbol in $M$ must be qualified by $p$, $N$
% is a value, and $M \supseteq N$ is consistent. During evaluation, as more
% inclusions are computed, no inconsistent acceptance set will be produced
% \cite{Gunter:ADDUQCM}.
% 
% The semantics of SD3 are based on the semantics of the underlying \datalog\
% \cite{Jim:DDQE}. In the following discussion we identify the keys used to
% qualify predicates as \newterm{sites}. This is in keeping with the idea
% that each key corresponds to a principal and each principal manages a
% local (site specific) name space based on that key.
% 
% To describe the semantics of a distributed SD3 program first construct a
% \newterm{global \datalog\ program} from the SD3 program by replacing each
% $n$-ary predicate $R$ with an $(n+1)$-ary predicate $R^g$, and each atom of
% the form $s\$R(t_1, \ldots, t_n)$ with an atom of the form $R^g(s, t_1,
% \ldots, t_n)$ The semantics of an SD3 program is the minimum model of the
% corresponding \datalog\ program.
% 
% A \datalog\ rule is considered safe if all the variables in the head also
% appear in the body. This is important because it gaureentees that the
% program's minimum model is finite and that query evaluation will terminate.
% Because of the distributed nature of SD3 an additional safety rule is
% necessary. First define a \newterm{site determined} atom as one in which
% the site specified is not a variable. Let rules have the form $a\,
% \text{:-}\, a_1, \ldots, a_n$. A rule is said to be \newterm{site safe} if
% it is both safe and if for every atom $a_i = x\$R_i(\ldots)$ that is not
% site determined the variable $x$ occurs in some atom $a_j$ where $j < i$.
% Thus the first atom in the body of a rule must be site determined. Site
% safety is a stronger condition than safety. A rule such as $s\$Q(x)\,
% \text{:-}\, x\$R(x)$ is not site safe even though the corresponding
% \datalog\ rule is safe.
% 
% Site safety is important because a site in a distributed SD3 program is not
% necessarily aware of all the sites in the entire program. The
% \newterm{active domain} of a \datalog\ program $P$, denoted
% $\textit{adom}(P)$ is the set of all constants in $P$. A safe \datalog\
% program has the same minimum model over its active domain as it does over
% an infinite domain. An SD3 site $s$, however, does not know the entire
% active domain but instead can only compute a portion of it,
% $\textit{adom}_s$, defined recursively as
% $$
% \textit{adom}_s = \forall s_1 \in \textit{adom}_s \,\,.\,\,
%      (\textit{adom}(P_s) \cup \textit{adom}_{s1})
% $$
% where $\textit{adom}(P_s)$ is the active domain of the program at site $s$
% only. It can be shown that if an SD3 program is site safe, the answer to a
% query at a site $s$ over the accessible part of the active domain at $s$,
% $\textit{adom}_s$ is the same as over an infinite domain \cite{Jim:DDQE}.
% Notice that site safety imposes restrictions on the order in which atoms
% can appear in the body of a rule.
% 
% \subsection{Implementation}
% 
% Because QCM and SD3 are essentially distributed database applications,
% their approach to many implementation issues tends to be quite generic. For
% example, the purpose of the signatures on the certificates used by most
% trust management systems is to protect the integrity of the credentials
% those certificates contain. However there are many ways to obtain this
% protection besides using digital signatures. A node $X$ wishing to query
% another node $Y$ can connect to node $Y$ and engage in a network protocol
% that allows the authenticated transmission of data. Such a protocol might
% use efficient symmetric encryption or message authentication codes instead
% of more costly digital signature algorithms. This would allow a large query
% result to pass between the nodes with a minimum of resource consumption. In
% this way QCM and SD3 can potentially disconnect credentials from the
% certificates that contain them, even in the case where remote nodes are
% being queried.
% 
% The prototype implementations of QCM and SD3 do not attempt to apply access
% control to the policy data itself. In other words, there is no support for
% automated trust negotiation. Anonymous users can query any node and obtain
% information about the relations maintained by that node. In the case of
% QCM, where queries are answered extensionally, the actual policy rules held
% by a node are never revealed, although the results of their evaluation are.
% 
% Another consequence of QCM answering queries extensionally is that
% credentials submitted with the request $C_R$ are only used if they
% immediately pertain to the query on the authorizing node. Otherwise the QCM
% processor will not recognize the usefulness of such credentials. When a
% distributed query is evaluated a remote node might want to use $C_R$ but
% the orginating node does not pass $C_R$ forward since the general utility
% of doing so would be unknown. Thus the remote node must look up the
% components of $C_R$ that it needs on its own. A result of this is that
% cycles are not detected since doing so would require some sort of
% distributed cycle detection algorithm. Thus if policy statements refer to
% each other they will cause a query to loop infinitely.
% 
% In contrast SD3 may answer queries intensionally so that a single node
% obtains a complete set of all relevant rules. This allows all the submitted
% credentials to be potentially used and it allows cycles in the policy to be
% handled gracefully.
% 
% Jim describes a prototype SD3 system that implements the DNSSEC protocol
% \cite{Jim:STMSCE}. In order to obtain the performance needed in DNS
% applications, Jim's implementation uses a number of elaborate optimization
% techniques. These optimizations add complexity to the implementation and
% increase the size of the trusted computing base. To deal with this Jim's
% implementation uses \newterm{certified evaluation}. The output of the SD3
% query evaluator is checked by a relatively simple proof checker. If the
% check fails, the results of the query are considered erronous. Since proof
% checking is easier than proof construction, the proof checker can be small
% and simple, thus reducing the size of the trusted computing base.
% 