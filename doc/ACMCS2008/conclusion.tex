\section{Conclusion}
\label{section-conclusion}

Trust management technology responds to the security demands of modern
distributed systems-- or in the words of previous authors
\cite{Blaze:RTMDSS}, ``the trust management approach to distributed
system security was developed as an answer to the inadequacy of
traditional authorization mechanisms''.  Compared to other simpler
systems such as identity or role-based access control, trust
management systems provide modern distributed applications programmers
a more effective and scalable means of specifying and enforcing
authorization policies.  At the heart of any trust management system
is a language for expressing policy and access rights, comprising a
mixture of features that address the character and requirements of
modern distributed security.  We have reviewed a number of language
features, and summarized which systems possess which features (see
especially Tables \ref{tbl:tmsummary1} and \ref{tbl:tmsummary2}).
Overall, we conclude that a subset of them are fundamental to any
trust management system:
\begin{description}
\item[Linked local namespaces] No single, monolithic namespace
exists on the Internet.  Instead, distinct security domains 
define their own namespaces.  Any trust management system language must provide some means
to refer to non-local namespaces, within the local namespace.
\item[Roles] Role membership is a fundamental abstraction
in trust management systems.  The specification of authorization via roles allows
policies to be defined independently of identities, so that 
addressing the needs of unknown future users does not require
policy to be rewritten.
\item[Delegation of authority] No single, monolithic policy authority
exists on the Internet.  Rather, distinct security domains define
their own local policies, and security domains commonly delegate
authority over local policy to trusted non-local authorities; any trust management system
language must provide some way to express this.
\item[Delegation of rights] In many cases, the users of a system 
desire to delegate their access rights to other entities, to 
act on their behalf.  trust management system languages should address this need,
including important nuances such as whether certain rights
should be specified as undelegatable by local authorities.
\item[Certificate revocation] Access rights should never be
permanently granted, rather authorities should be able to revoke them
or set finite lifetimes for their use.  While certificate revocation
appears to be an essentially nonmonotonic feature, this has been
disproven, as the authors of both PCA \cite{Bauer:ACWPCA} and RT
\cite{lbi-fc01} have developed monotonic inference rules for
incorporating revocation in authorization.  Some systems such as
SPKI/SDSI offer a simpler implementation-based solution
\cite{RFC-2693}, where revocation is not featured in the authorization
language but processed during the parsing of certificates into 
credentials.
\end{description}

At a higher level, we argue that rigorous formal foundations are a
necessary design feature of trust management systems, since they allow
rigorous guarantees of security.  We have shown that graph theories,
logics, and database formalisms are the most common formalisms for
trust management system design, though logic stands out as the most
popular.  Indeed, previous authors have argued that monotonic
programming logics are uniquely well-suited as trust management
languages \cite{Li:DCFTML}, and in general the rich semantic domains
available in logic provide a great deal of flexibility and scalability
in trust management system applications \cite{polakow-skalka-plas06}.
A variety of trust management system foundations use domain-specific
logical constructs originally developed for authentication settings
\cite{Burrows:LA}, witnessing the evolution of authorization systems
from earlier access control systems based mainly on authentication.

Many of the systems surveyed in this paper exist primarily in theory,
and have not yet been deployed.  As trust management systems become
more commonplace, and theory develops into practice, we believe a
major challenge will be whole-system assurances.  As we have
discussed, trust management systems comprise more than just a
semantics for their core authorization language, including a
collection of features for storing, collecting, and processing
certificates.  Significantly, some trust management system features
are sometimes implemented in these other components, such as when
certificate revocation or expiration is realized during certificate
parsing.  And of course, correctness of an entire system depends on
correctness of all of its components, as well as their interaction.
Thus, formalisms for assurances of correctness of \emph{systems} must
transcend the core authorization semantics, and address the myriad
components of trust management systems.  In many ways QCM and SD3 set
the standard in this regard \cite{Gunter:DALSI,Jim:STMSCE}, by using
database theory and technology as a uniform setting for implementing
certificate storage and retrieval, defining the semantics of
authorization, and formally modeling systems.  As trust management
systems mature into vital components of Internet communications and
commerce, a holistic formal view that takes into account the variety
of trust management system components will be essential to coherence
and reliability of these systems.
 

% Trust management systems (trust management systems) provide modern distributed
% applications a means to specify and enforce authorization policies.
% Through a combination of features, they allow distinct security
% domains to establish trust relationships for the purposes of access
% control and resource protection in volatile distributed environments.
% At the core of any trust management system is the subsystem that governs authorization
% upon access request-- the authorization decision.  The authorization
% decision takes as input a policy specification and access rights (aka
% credentials) of the requester, and determines whether access should be
% granted on the basis of an authorization semantics.
% 
% In this paper we have surveyed current state-of-the-art in trust
% management, with an emphasis on systems that have a formally
% well-founded authorization semantics.  Such systems have the advantage
% of supporting the formal verification of correctness that security
% concerns demand.  Our survey has focused on features for expressing
% policies and credentials, and how they are reflected in the
% authorization semantics.  We have also observed that some features of
% trust management system authorization are realized in components of authorization
% decisions that are external to the formal semantics.  For example,
% credentials usually have a certificate wire format, which is parsed
% into credential form.  Expired or revoked certificates may be rejected
% during parsing, whereas the authorization semantics may be strictly
% monotonic.  Consequently, we have shown how authorization decisions
% comprise more than just an implementation of a core semantics, and that
% other decision components contribute to system functionality.
% 
% Our survey has covered foundations of trust management, especially
% formal calculi that underly many authorization semantics.  We have
% also reviewed a number of current systems, with a focus on three
% representatives-- SPKI/SDSI \cite{RFC-2693}, RT \cite{Li:DRBTMF}, and
% QCM/SD3 \cite{Gunter:DALSI,Jim:Strust management systemCE}-- that embody some of the most
% interesting and advanced features of and approaches to trust
% management.  We have characterized these systems according to a
% generic model of authorization decisions, as well as their underlying
% semantics.  Overall, our survey serves as a resource for understanding
% trust management systems, what features they possess, and how the behavior of such systems
% depends on the integration of implementation components including, but
% not restricted to, a formal core authorization semantics.
% 
% %Numerous trust management systems have been described in the literature.
% %They all solve the distributed authorization problem by allowing an
% %authorizer to define a policy describing the attributes of allowed
% %requesters. The authorizer then delegates authority over these attributes
% %to specific third parties. In addition to this defining characteristic, the
% %various trust management systems that have been described support a wide
% %array of additional features.
% %
% %To ensure that a system has understandable security properties, it is
% %desirable for the system to have a precise formal description. Many modern
% %trust management systems use \datalog\ or \datalogc\ as their semantic
% %foundation, although modal logics, first order logic, higher order logic,
% %and graph theory have also been used. In addition, distributed
% %authorization has been identified as a specific application of distributed
% %databases, making important results from that field applicable to trust
% %management.
% %
% %Many features of a practical system, including important features such as
% %certificate revocation and expiration, are often not included in the
% %logical formalism. Instead these features are often handled outside the
% %logical core. Consequently such external features can typically be shared
% %among several cores; the sharing being an implementation detail. For
% %example, SPKI/SDSI style on-line revocation could be supported just as
% %easily in an RT based system. Also certificate formats are independent of
% %the core logic.
% %
% %Most trust management systems are just different organizations of the same
% %basic concepts. A generic implementation might use, for example, a
% %\datalog\ or \datalogc\ engine as its core logic, $L$ in
% %\autoref{figure-tmstruct}, with different ``plug-ins'' implementing various
% %$T_C$, $T_P$, and $T_Q$ components according to the needs of the systems we
% %have reviewed. Such an implementation could support any of the \datalog\
% %based systems such as SDSI, RT, SD3, or Binder. This formulation helps to
% %clarify the significance of issues like extensional vs intensional queries,
% %credential storage, and policy confidentiality. These matters are largely
% %independent of any particular trust management system. Regardless of the
% %name attached to the system, the underlying issues, and their potential
% %solutions, are essentially the same.
% %
% %As the field of distributed authorization matures, it is likely that no
% %single system that we have reviewed here will prevail above the others.
% %Instead the concepts and features these systems provide will be mixed and
% %matched according to the needs of applications. The abundance of trust
% %management systems described in the literature over the last ten years is a
% %prelude to a more unified and principled approach to distributed
% %authorization that we expect to see in the future.
% 