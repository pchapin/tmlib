%+++++++++++++++++++++++++++++++++
% Preamble and global declarations
%+++++++++++++++++++++++++++++++++
\documentclass{article}
%\documentclass[12pt]{article}

% Common packages. Uncomment as needed.
%\usepackage{amsmath}
%\usepackage{amssymb}
%\usepackage{amstext}
%\usepackage{amsthm}
%\usepackage{doublespace}
%\usepackage{fullpage}
%\usepackage[dvips]{graphics}
%\usepackage{listings}
%\usepackage{mathpartir}
%\usepackage{url}
%\usepackage{hyperref}

% The following are settings for the listings package.
%\lstset{language=Java,
%        basicstyle=\small,
%        stringstyle=\ttfamily,
%        commentstyle=\ttfamily,
%        xleftmargin=0.25in,
%        showstringspaces=false}

\setlength{\parindent}{0em}
\setlength{\parskip}{1.75ex plus0.5ex minus0.5ex}

\usepackage{amsmath}
\usepackage{amssymb}
\usepackage{amstext}
\usepackage[T1]{fontenc}
\usepackage{fp-frame}
\usepackage{mathpartir}
\def\TirName#1{\text{\sc #1}}
\newtheorem{condition}{Condition}
\newtheorem{definition}{Definition}
\newtheorem{lemma}{Lemma}

\newcommand{\RT}{\text{RT}}
\newcommand{\RTR}{\RT^R}
\newcommand{\WSMDB}{\text{WS}_{MDB}}
\newcommand{\WSM}{\text{WS}_{M}}
\newcommand{\WS}{\text{WS}}
\newcommand{\says}{\textit{ says }}
\newcommand{\as}{\textit{ as }}
\newcommand{\controls}{\textit{ controls }}
\newcommand{\serves}{\textit{ serves }}
\newcommand{\for}{\textit{ for }}
\newcommand{\carrier}[3]{#1 \textit{ carries } #2 \textit{ for } #3}
\newcommand{\speaksfor}{\Rightarrow}
\renewcommand{\implies}{\supset}
\newcommand{\network}{\mathcal{N}}
\newcommand{\cod}[1]{\llbracket#1\rrbracket}
\newcommand{\subcod}[2]{\llbracket#2\rrbracket_{#1}}
\newcommand{\audit}{\text{audit}}
\newcommand{\aud}{\text{aud}}
\newcommand{\assert}{\text{assert}}
\newcommand{\lookup}{\text{lookup}}
\newcommand{\ok}[1]{\mathit{OK}(#1)}
%\newcommand{\priv}{\ok{T}}
\newcommand{\priv}{\mathbf{priv}}
%\newcommand{\privmdb}{\ok{T_{MDB}}}
\newcommand{\privmdb}{\priv_{MDB}}
\newcommand{\aclmdb}{\acl_{MDB}}
\newcommand{\rawreq}{\mathbf{req}}
\newcommand{\acl}{\mathcal{A}}
\newcommand{\ccl}{\mathcal{C}}
\newcommand{\cclmdb}{\ccl_{MDB}}
\newcommand{\rolecerts}{\mathcal{R}}
\newcommand{\delcerts}{\mathcal{D}}
\newcommand{\keycontrol}{\mathcal{K}}
\newcommand{\true}{\mathbf{true}}
\newcommand{\metaonreq}[1]{\hat{#1}}
\newcommand{\onloc}[2]{#1 \cdot #2}
\newcommand{\onreq}[2]{#1 \cdot #2}
\newcommand{\cloc}{\iota}
\newcommand{\cedge}[1]{\stackrel{#1}{\longleftarrow}}
\newcommand{\cred}[3]{#1 \cedge{#3} #2}
\newcommand{\mathitcred}[3]{\mathit{#1} \cedge{\mathit{#3}} \mathit{#2}}
\newcommand{\expr}{\mathrm{expr}}
\newcommand{\rmem}{\mathrm{rmem}}
\newcommand{\creds}{\mathcal{C}}
\newcommand{\risk}{\kappa}
\newcommand{\risks}{K}
\newcommand{\po}{\preccurlyeq}
\newcommand{\entityrisk}{\mathit{EntityRisk}}
\renewcommand{\implies}{\Rightarrow}
\newcommand{\ents}{\mathrm{ents}}
\newcommand{\credsmean}[1]{\mathcal{S}_{#1}}
\newcommand{\inexprs}{\mathrm{inexprs}}
\newcommand{\bounds}{\mathrm{bounds}}
%\def\twoheadrightarrowfill@{\arrowfill@\relbar\relbar\twoheadrightarrow} 
%\newcommand{\xtwoheadrightarrow}[2][]{\ext@arrow 0359\xtwoheadrightarrowfill@{#1}{#2}} 
\newcommand{\wtedge}[3]{#1 \sarrow{->}{#3} #2}
\newcommand{\xtwoheadrightarrow}[1]{\sarrow{->>}{#1}}
\newcommand{\wtpath}[3]{#1 \xtwoheadrightarrow{#3} #2}
\newcommand{\nodes}[1]{\mathcal{N}_{#1}}
\newcommand{\wtedges}[1]{\mathcal{E}_{#1}}
\newcommand{\graph}[1]{\mathcal{G}_{#1}}
\newcommand{\role}[2]{$\mathit{#1}.\mathit{#2}$}
\newcommand{\CCD}{\textit{GetRisks}}
\newcommand{\checkmem}{\mathrm{checkmem}}
\newcommand{\cesnote}[1]{\textit{(#1 -- c.e.s.)}}

\newcommand{\thmnumbering}{section}
\newtheorem{theorem}{Theorem}[\thmnumbering]
%\newtheorem{definition}{Definition}[\thmnumbering]
\newtheorem{corollary}{Corollary}[\thmnumbering]
\newtheorem{example}{Example}[\thmnumbering]
%\newtheorem{lemma}{Lemma}[\thmnumbering]
%\newtheorem{proposition}{Proposition}[\thmnumbering]
\newenvironment{proof}{\noindent\textit{Proof. }}{\qed\medskip}
%\newcommand{\qed}{$\Box$}
\def\squareforqed{\hbox{\rlap{$\sqcap$}$\sqcup$}}
\def\qed{\ifmmode\squareforqed\else{\unskip\nobreak\hfil
\penalty50\hskip1em\null\nobreak\hfil\squareforqed
\parfillskip=0pt\finalhyphendemerits=0\endgraf}\fi}

\newcommand{\colfigurebox}[5]
{
        \begin{figure}
        \begin{center}
        \framebox[#1]{
                \begin{minipage}{#2}
                \footnotesize
                \vspace{2mm}
                #5
                \vspace{1mm}
                \end{minipage}
                \normalsize
        }
        \caption{#4} 
        \label{#3}
        \end{center}
        \end{figure}
}

\newcommand{\figurebox}[5]
{
        \begin{figure*}
        \begin{center}
        \framebox[#1]{
                \begin{minipage}{#2}
                \footnotesize
                \vspace{2mm}
                #5
                \vspace{1mm}
                \end{minipage}
                \normalsize
        }
        \caption{#4} 
        \label{#3}
        \end{center}
        \end{figure*}
}

%%%%% dduct %%%%%%
\newcommand{\dduct}[2]{
\ensuremath{
\renewcommand{\arraystretch}{1.3}
	\begin{array}[b]{c}
		#1\\\hline
		#2
	\end{array}
\renewcommand{\arraystretch}{1}
}
}
%%%%%%%%%%%%%%%%%%%

\newcommand{\algtab}[1]
{
\vspace*{-3mm}
\begin{tabbing}
\hspace*{12mm}\=\hspace{9mm}\=\hspace{9mm}\=\hspace{6mm}\=\hspace{6mm}\=
\hspace{6mm}\=\\\newcommand{\cmnts}{\noindent\textbf{Comments\hspace{2mm}}}
#1
\end{tabbing}
}
\newcommand{\lt}{\left\{}
\newcommand{\rt}{\right\}}
\newcommand{\Lt}{\left\{\!\!\right.}
\newcommand{\Rt}{\left.\!\!\right\}}
\newcommand{\const}{\mathbf{c}}
\newcommand{\heading}[1]{\noindent\textbf{#1\ \  }}

\newcommand{\schemecomp}[3]{#1\vdash^i#2\preccurlyeq#3}

\newcommand{\Estep}{\rightarrow}
\newcommand{\instance}{\preccurlyeq}
\newcommand{\geninstance}{\trianglelefteq}
\newcommand{\gen}{\mathit{gen}}
\newcommand{\normalize}{\mathit{normalize}}

\newcommand{\iruleone}{\textsc{Sub$^\instance$}}
\newcommand{\iruletwo}{$(\preccurlyeq\!\forall)$}
\newcommand{\irulethree}{$(\forall\!\preccurlyeq)$}

%\newcommand{\seclang}{\lambda_{\text{\rm sec}}}
%\newcommand{\setlang}{\lambda_{\text{\rm set}}}
\newcommand{\pinit}{\ensuremath{p_0}}
%\newcommand{\reduce}{\leadsto}
\newcommand{\reduce}{\rightarrow}
%\newcommand{\eqdefn}{=_{\scriptscriptstyle{\mathit{def}}}}
\newcommand{\subn}{\varphi}
\newcommand{\subndefn}[1]{[#1]}
\newcommand{\restrict}[2]{{#1}\!\mid_{#2}}
\newcommand{\rrow}[2]{\lt \drow{r}{#1}{#2} \rt}
%\newcommand{\refcell}{\mathrm{ref}}
%\newcommand{\tref}{\mathrm{ref}}
%\newcommand{\hmxconfig}[2]{#1/#2}
%\newcommand{\store}{\varsigma}
\newcommand{\rangevec}[3]{#1^{\hspace{4pt}#2 \in \setdefn{1,\ldots,#3}}}
%\newcommand{\initenv}{\Gamma_\iota}
\newcommand{\initenv}{\Delta}
%\newcommand{\extend}[3]{#1\oplus #2 \mapsto #3}
\newcommand{\gdesc}[1]{\text{\textit{#1}}}
\newcommand{\loc}{l}
\newcommand{\setdefn}[1]{\lt#1\rt}
\newcommand{\deref}{\,!}
\newcommand{\assignfn}[1]{{:=}_{#1}}
%\renewcommand{\eabs}[3]{\lambda\langle#1\rangle #2 . #3 }
\newcommand{\credentials}{\mathcal{A}}
\newcommand{\constrain}[2]{#1\!\mid_{#2}}
\newcommand{\remove}[2]{#1\backslash{#2}}
\newcommand{\subrule}{\TirName{Sub}}
\newcommand{\varrule}{\TirName{Var}}
\newcommand{\absrule}{\TirName{Abs}}
\newcommand{\apprule}{\TirName{App}}
\newcommand{\letrule}{\TirName{Let}}
\newcommand{\assignrule}{\TirName{Assign}}
\newcommand{\refrule}{\TirName{Ref}}
\newcommand{\derefrule}{\TirName{Deref}}
\newcommand{\construle}{\TirName{Const}}
\newcommand{\forallelim}{\TirName{$\forall$ Elim}}
\newcommand{\forallintro}{\TirName{$\forall$ Intro}}
\newcommand{\configrule}{\TirName{Config}}
\newcommand{\labinferrule}[3]{\inferrule*[right=\scriptsize#3]{#1}{#2}}
\newcommand{\Labinferrule}[3]{\inferrule*[Right=\scriptsize#3]{#1}{#2}}
\newcommand{\vvec}{\bar{\regv}}
\newcommand{\tvec}{\bar{\tau}}
\newcommand{\existsC}[2]{\exists #1 . #2}
\newcommand{\rename}{\varrho}
\newcommand{\blankheading}[2]
{	
	\vspace{17pt}
	\noindent\textbf{\Large #2}
	\addcontentsline{toc}{#1}{#2}
	\vspace{10pt}
}
\newcommand{\blankchapter}[1]
{	
	\chapter*{#1}
	\addcontentsline{toc}{chapter}{#1}
}
\newcommand{\blanksection}[1]
{	
	\section*{#1}
	\addcontentsline{toc}{section}{#1}
}

%
% ICFP secty definitions
%
\newcommand{\chse}{\ensuremath{\ |\ }}
\newcommand{\chkpriv}[2]{\ensuremath{\mathrm{checkpriv\ }#1\ \mathrm{for\ }#2}}
\newcommand{\inspect}{\ensuremath{\mathrm{inspect}}}
\newcommand{\ite}[3]{\ensuremath{\mathrm{if\ }#1\ \mathrm{then\ }#2\ \mathrm{else\ } #3}}
\newcommand{\nil}{\mathit{nil}}
\newcommand{\fail}{\ensuremath{\mathbf{secfail}}}
\newcommand{\needs}{\ensuremath{\mathrm{\ needs\ }}}
%\newcommand{\fnty}[3]{\ensuremath{#1\rightarrow#2\needs#3}}
\newcommand{\fnty}[3]{\ensuremath{#1\xrightarrow{#3}#2}}
\newcommand{\privs}{\ensuremath{\mathrm{privs}}}
\newcommand{\rmpriv}{\ensuremath{\mathrm{rmpriv}}}
\newcommand{\satisfy}{\ensuremath{\mathit{satisfy}}}
\newcommand{\hsleq}{\ensuremath{\mathit{\preceq_{hs'}}}}
\newcommand{\minsafe}{\ensuremath{\mathit{minsafe}}}
\newcommand{\leastsub}{\ensuremath{\mathit{leastsub}}}
\newcommand{\safesub}{\ensuremath{\mathit{safesub}}}
\newcommand{\vars}{\ensuremath{\mathit{vars}}}
%\newcommand{\pinit}{\ensuremath{p_{\iota}}}
\newcommand{\pset}{\ensuremath{\mathbf{s}}}
\newcommand{\psetinit}{\ensuremath{\pset_{\iota}}}
\newcommand{\cons}[2]{\ensuremath{#1\!::\!#2}}
\newcommand{\constr}{\ensuremath{\ /\ }}
\newcommand{\stack}{\ensuremath{\mathtt{S}}}
%\newcommand{\stack}{\ensuremath{\mathbf{s}}}
\newcommand{\gpset}{\ensuremath{R}}
%\newcommand{\subn}{\ensuremath{S}}
\newcommand{\vsubn}{\ensuremath{\hat{S}}}
\newcommand{\isecty}{\ensuremath{\mathit{isecty}}}
\newcommand{\mt}{\ensuremath{\mathit{mt_{\scriptscriptstyle{\tpc}}}}}
%\newcommand{\mt}{\ensuremath{\mathit{mt}}}
\newcommand{\MT}{\ensuremath{\mathit{MT}}}
\newcommand{\MTfn}{\ensuremath{f}}
\newcommand{\dom}{\ensuremath{\mathit{Dom}}}
\newcommand{\var}{\ensuremath{\mathit{Vars}}}
\newcommand{\sectyfail}{\ensuremath{\mathbf{sectyfail}}}
\newcommand{\infsecty}{\ensuremath{\mathit{infer\_secty}}}
\newcommand{\sconstrs}{\ensuremath{\mathit{scs}}}
\newcommand{\tpc}{\ensuremath{\mathit{TPC}}}
\newcommand{\pc}{\ensuremath{\mathit{PC}}}
\newcommand{\tc}{\ensuremath{\mathit{TC}}}
\newcommand{\close}{\ensuremath{\mathit{close}}}
\newcommand{\verify}{\ensuremath{\mathit{verify}}}
\newcommand{\glb}{\ensuremath{\mathit{glb}}}
\newcommand{\istack}{\ensuremath{\stack_{\iota}}}
\newcommand{\stackinit}{\ensuremath{\stack_{\iota}}}
%\newcommand{\mapping}[2]{#1 \triangleright #2}
\newcommand{\mapping}[2]{[#2/#1]}
\newcommand{\stackframe}[3]{\langle #1, #2, #3 \rangle}
\newcommand{\constack}[4]{\ensuremath{\stackframe{#1}{#2}{#3}\hspace{-2pt} :: \! #4}}
\newcommand{\revstack}[4]{\ensuremath{#4 @ \stackframe{#1}{#2}{#3}}}
%\newcommand{\signedfn}[3]{\ensuremath{{}^{#3}\!\lambda #1.#2}}
\newcommand{\signedfn}[3]{\ensuremath{\lambda #1.#2.#3}}
%\newcommand{\signedfn}[3]{\ensuremath{\lambda\, #1.#2 \rhd #3}}
\newcommand{\access}{\ensuremath{\credentials}}
%\newcommand{\enabled}{\ensuremath{\mathcal{F}}}
\newcommand{\enabled}{\pset}
%\newcommand{\secderv}{\ensuremath{\vdash_{\!\scriptscriptstyle{\access}}}}
\newcommand{\secderv}{\ensuremath{\vdash}}
\newcommand{\secvar}{\ensuremath{\alpha}}
\newcommand{\secty}{\ensuremath{\tau}}
\newcommand{\lp}{\ensuremath{\mathit{lp}}}
%\newcommand{\seclang}{\ensuremath{\lambda_{\text{sec}}}}
\newcommand{\unit}{\ensuremath{\text{unit}}}
\newcommand{\Let}{\ensuremath{\mathrm{\ let\ }}}
\newcommand{\In}{\ensuremath{\mathrm{\ in\ }}}
\newcommand{\translt}{\TirName{trans$\le$}}
\newcommand{\reflt}{\TirName{ref$\le$}}
\newcommand{\sub}{\TirName{sub}}
\newcommand{\subtr}{\TirName{sub}}
\newcommand{\vartr}{\TirName{var}}
\newcommand{\fntr}{\TirName{abs}}
\newcommand{\appltr}{\TirName{app}}
\newcommand{\lptr}{\RuleLetPrivYes}
\newcommand{\lpftr}{\RuleLetPrivNo}
\newcommand{\cptr}{\RuleCheckPriv}
\newcommand{\fnlt}{$\rightarrow\le$}
\newcommand{\inlt}{\TirName{coerce$\le$}}
\newcommand{\psvar}{\ensuremath{\rho}}
\newcommand{\tysubn}{\ensuremath{\Theta}}
\newcommand{\soln}[1]{\ensuremath{\lfloor#1\rfloor}}
\newcommand{\exf}[1]{\ensuremath{\mathit{#1}}}
\newcommand{\fullvers}[1]{}
\newcommand{\shortvers}[1]{#1}
\newcommand{\slredex}[2]{#1,#2}
\newcommand{\false}{\text{\bf false}}
\newcommand{\jdirect}[6]{#1, #2, #3, #4 \vdash #5 : #6}

\def\snote#1{\ \ [[\ \ \textbf{#1}\ \ ]]\ \ }

%
% ECOOP capty definitions
%
%\newcommand{\gdesc}[1]{\text{\textit{#1}}}
\newcommand{\idworld}{\mathit{ID}}
\newcommand{\idset}{}
\newcommand{\lab}{d}
\newcommand{\labset}{D}
\newcommand{\Alabworld}{\mathcal{L}_a}
\newcommand{\Blabworld}{\mathcal{L}_b}
\newcommand{\labworld}{\mathcal{D}}
\newcommand{\intface}{\iota}
\newcommand{\uintface}{\varphi}
\newcommand{\coreobj}[1]{[#1]}
\newcommand{\objdefn}[3]{\coreobj{#1}\cdot #2 \cdot #3}
\newcommand{\varobjdefn}[3]{#1\cdot #2 \cdot #3}
\newcommand{\recdefn}[1]{\{ #1 \}}
\newcommand{\recgen}[1]{{\pi}_{#1}}
%\newcommand{\setdefn}[1]{\recdefn{#1}}
%\newcommand{\capab}[2]{\langle #1,#2 \rangle}
%\newcommand{\capab}[3]{\objdefn{#1}{#2}{#3}}
%\newcommand{\itranslate}[1]{\mathit{trans}\ #1}
\newcommand{\itranslate}[1]{\hat{#1}}
\newcommand{\itranslatewide}[1]{\widehat{#1}}
%\newcommand{\stranslate}[1]{\hat{#1}}
\newcommand{\stranslate}[1]{#1}
\newcommand{\itrans}[1]{\itranslate{#1}}
\newcommand{\strans}[1]{\stranslate{#1}}
\newcommand{\itranswide}[1]{\widehat{#1}}
\newcommand{\stranswide}[1]{#1}
%\newcommand{\project}[2]{#1/#2}
%\newcommand{\modify}[3]{#1[#2 = #3]}
%\newcommand{\elevate}[1]{[#1]}
%\newcommand{\rowabbrv}[1]{[#1]}
\newcommand{\project}[2]{#1.#2}
\newcommand{\modify}[3]{#1\{#2 = #3\}}
\newcommand{\elevate}[1]{\{#1\}}
\newcommand{\rowabbrv}[1]{\{#1\}}
\newcommand{\memcheck}[2]{#2\ni#1}
\newcommand{\notmemcheck}[2]{#2\not\ni#1}
\newcommand{\memcheckprim}[1]{\ni_{#1}}
\newcommand{\projprim}[1]{/_{#1}}
%\newcommand{\methlist}{\mathit{ml}}
\newcommand{\methlist}{\varrho}
%\newcommand{\loc}{l}
\newcommand{\send}[3]{#1.#2(#3)}
\newcommand{\select}[2]{#1.#2}
%\newcommand{\store}{\sigma}
\newcommand{\config}[3]{#1,#2,#3}
\newcommand{\isoopconfig}[2]{#1/#2}
\newcommand{\hmxconfig}[2]{#1,#2}
%\newcommand{\ocap}{\ensuremath{o_{\mathrm{cap}}}}
\newcommand{\ocap}{\ensuremath{\mathrm{pop}}}
%\newcommand{\isoop}{\ensuremath{\mathrm{isoop}_{\mathrm{set}}}}
\newcommand{\isoop}{\ensuremath{\mathrm{isoop}}}
%\newcommand{\projml}{\ensuremath{P\!\Lambda_{\mathrm{ref}}}}
\newcommand{\projml}{\ensuremath{\mathrm{pml}_{\scriptstyle{B}}}}
%\newcommand{\reduce}{\rightarrow}
%\newcommand{\extend}[3]{#1\!\parallel\! #2 \mapsto #3}
%\newcommand{\extend}[3]{#1\!\oplus\! #2 \mapsto #3}
\newcommand{\extend}[3]{#1[#2 \mapsto #3]}
\newcommand{\meths}[1]{\text{\rm meths}(#1)}
\newcommand{\ivars}[1]{\text{\rm ivars}(#1)}
\newcommand{\ifacevars}[1]{\text{\rm vars}_{\intface}(#1)}
%\newcommand{\hole}{\text{\vbox{\hrule height 0.015cm width 0.15cm}}}
%\newcommand{\fv}[1]{\text{\rm fv}(#1)}
%\newcommand{\domain}[1]{\text{\rm dom}(#1)}
\newcommand{\snks}[1]{\text{\rm usr}(#1)}
\newcommand{\obj}[2]{\text{\rm obj}_{#1}(#2)}
%\newcommand{\cast}[2]{(#1)#2}
%\newcommand{\cast}[2]{\llcorner\!#2\!\lrcorner_{#1}}
%\newcommand{\cast}[3]{#1\!\shortmid\!_{\!\scriptscriptstyle{#2}}{#3}}
\newcommand{\cast}[3]{#1\!\shortmid\!(#2,#3)}
\newcommand{\multicast}[2]{#1\!\shortmid\!#2}
%\newcommand{\elet}[3]{\mathrm{let}\,#1=#2\,\mathrm{in}\,#3}
%\newcommand{\etestprim}[1]{\mathord{?}_{#1}}
%\newcommand{\eaccess}[2]{#1.#2}
%\newcommand{\eaccessprim}[1]{._{#1}}
\newcommand{\refcell}{\text{\rm ref}}
\newcommand{\successor}{\text{\rm succ}}
\newcommand{\pred}{\text{\rm pred}}
\newcommand{\iszero}{\text{\rm zero}}
\newcommand{\olab}{\text{\rm obj}}
\newcommand{\weaklab}{\text{\rm strong}}
\newcommand{\snkslab}{\text{\rm usr}}
\newcommand{\srclab}{\text{\rm src}}
\newcommand{\intfacelab}{\text{\rm ifc}}
\newcommand{\setlab}{\text{\rm set}}
\newcommand{\getlab}{\text{\rm get}}
\newcommand{\deletelab}{\text{\rm delete}}
\newcommand{\readlab}{\text{\rm read}}
\newcommand{\writelab}{\text{\rm write}}
\newcommand{\subsetfn}{\text{\rm subseteq}}
\newcommand{\castfn}{\text{\rm cast}}
\newcommand{\upcastfn}{\text{\rm upcast}}
\newcommand{\caprec}[2]{\recdefn{\olab = #1, \intfacelab = #2}}
%\newcommand{\credentials}{\mathcal{A}}
\newcommand{\vpre}{\tpre}
\newcommand{\vabs}{\tabs}
%\newcommand{\defeq}{=_{\scriptscriptstyle{\mathrm{def}}}}
%\newcommand{\defeq}{\stackrel{\mathrm{def}}{=}}
\newcommand{\defeq}{\triangleq}
\newcommand{\bisimilar}{\simeq}
\newcommand{\redexsimilar}{\approxeq}
\newcommand{\contextsimilar}{\approxeq}
%\renewcommand{\tset}[1]{[#1]}
%\renewcommand{\ourX}{\ensuremath{\mathrm{ROWS}}}
\newcommand{\tref}{\refcell}
\newcommand{\ycomb}{Y}
%\newcommand{\self}{\varsigma}
\newcommand{\ivar}{i}
%\newcommand{\initenv}{\Gamma_1}
%\newcommand{\objt}[2]{#1\ \mathbf{withface}\ #2}
\newcommand{\prettyrow}[1]{[#1]}
%\newcommand{\objt}[2]{#1\ \cdot\ #2}
\newcommand{\objt}[3]{{#1}^{#2}_{#3}}
%\newcommand{\objt}[3]{{#1}\cdot{#2}-{#3}}
\newcommand{\intt}{\mathrm{int}}
\newcommand{\objintt}{\mathbf{int}}
\newcommand{\unitt}{\mathrm{unit}}
\newcommand{\labinit}{\lab_1}
%\newcommand{\schemecomp}[3]{#1\vdash^i#2\preccurlyeq#3}
%\newcommand{\jocap}[5]{#1,#2,#3\vdash#4 : #5}
\newcommand{\jocap}[5]{#1,#3\vdash#4 : #5}
\newcommand{\assume}{\Gamma}
%\newcommand{\vecdefn}[3]{#1^{\hspace{6pt}#2 \in \setdefn{1,\ldots,#3}}}
\newcommand{\vecdefn}[3]{{#1}^{\hspace{4pt}0<{#2}\le{#3}}}
%\newcommand{\methods}[2]{#1^{\hspace{6pt}#2}}
\newcommand{\codt}[1]{\llparenthesis\,#1\,\rrparenthesis}
\newcommand{\codtin}[1]{\llparenthesis\,#1\,\rrparenthesis_{\tin}}
\newcommand{\codtout}[1]{\llparenthesis\,#1\,\rrparenthesis_{\tout}}
\newcommand{\invcodt}[1]{\rrparenthesis#1\llparenthesis}
\newcommand{\fndestruct}[3]{#1;\ #2 \mapsto #3}
\newcommand{\ksset}[1]{{\text{\emph{Set}}}_{#1}}
\newcommand{\kmeth}[1]{{\text{\emph{Meth}}}_{#1}}
\newcommand{\kifc}[1]{{\text{\emph{Ifc}}}_{#1}}
%\newcommand{\weaken}[3]{\mathbf{weak}(#1,#2,#3)}
\newcommand{\weaken}[2]{\mathbf{weak}_{#2}(#1)}
%\newcommand{\eabs}[2]{\lambda#1.#2}
\newcommand{\cellobj}[2]{#1 \cdot #2}
\newcommand{\newcell}[2]{\refcell_{#1}#2}
\newcommand{\unitv}{()}
\newcommand{\cget}[1]{\send{#1}{\getlab}{}}
\newcommand{\cset}[2]{\send{#1}{\setlab}{#2}}
\newcommand{\intersect}[2]{#1\wedge#2}
\newcommand{\union}[2]{#1\vee#2}
\newcommand{\unionprim}[1]{\vee_{#1}}
\newcommand{\difference}[2]{#1\ominus#2}
\newcommand{\diffprim}[1]{\ominus_{\!#1}}
\newcommand{\differenceprim}[1]{\ominus_{#1}}
\newcommand{\reduceisect}{\reduce_\wedge}
%\newcommand{\existsC}[2]{\exists#1.#2}
\newcommand{\noneset}{\varnothing}
\newcommand{\allset}{\coset{\noneset}}
\newcommand{\addmem}[2]{#1{\scriptstyle\oplus}#2}
\newcommand{\removemem}[2]{#1{\scriptstyle\ominus}#2}
\newcommand{\setmeaning}[1]{\langle#1\rangle}
%\newcommand{\tin}{\cdot}
%\newcommand{\tout}{-}
\newcommand{\tin}{+}
\newcommand{\tout}{-}
\newcommand{\ttop}{\top}
\newcommand{\tbot}{\bot}
\newcommand{\setv}{\rowv}
\newcommand{\allsett}{\omega}
\newcommand{\nonesett}{\varnothing}
%\newcommand{\fielduk}[2]{\stackrel{\;#2}{#1}}
\newcommand{\fielduk}[2]{#1#2}
\newcommand{\fieldint}[1]{\fielduk{#1}{\tin}}
\newcommand{\fieldoutt}[1]{\fielduk{#1}{\tout}}
%\newcommand{\fieldint}[1]{\dot{#1}}
%\newcommand{\fieldoutt}[1]{\bar{#1}}
\newcommand{\kcon}{\text{\emph{Con}}}
\newcommand{\compoundC}{\mathbf{C}}
\newcommand{\srow}[3]{\fielduk{#1}{#2},#3}
\newcommand{\snext}{,}
\newcommand{\sys}{\ensuremath{\system{1}{=}}}
\newcommand{\inlong}[1]{#1}
\newcommand{\labs}[2]{\lambda #1 . #2}
\newcommand{\tyinf}{\ensuremath{\mathit{infer}}}
\newcommand{\unify}{\ensuremath{\mathit{unify}}}
\newcommand{\kprow}[1]{{\text{\emph{Prow}}}_{#1}}
\newcommand{\rurow}[2]{\tset{\unionrow{\fielduk{r}{#1}}{#2}}}
\newcommand{\simplify}{\ensuremath{\mathit{simplify}}}
\newcommand{\confusion}{\ensuremath{\mathit{confusion}}}
%\newcommand{\close}{\ensuremath{\mathit{close}}}
\newcommand{\ctrans}{\ensuremath{\mathit{Trans}}}
\newcommand{\ctransp}{\ensuremath{\mathit{Trans_{row}}}}
\newcommand{\cfn}{\ensuremath{\mathit{Fn}}}
\newcommand{\crow}{\ensuremath{\mathit{Row}}}
%\newcommand{\tyinf}{\ensuremath{\mathit{infer}}}
\newcommand{\absrow}{\partial\tabs}
\newcommand{\groundrow}{\varrho}
%\newcommand{\fail}{\mathbf{fail}}
\newcommand{\fieldel}{\mathit{f}}
\newcommand{\rowsplit}{\mathit{split}}
\newcommand{\multivar}{\ensuremath{\gamma}}
\newcommand{\extnsn}{\varepsilon}
\newcommand{\compsubn}{\vartheta}
\newcommand{\privset}{\mathcal{R}}
\newcommand{\mvar}{\mu\nu}
\newcommand{\abrvrow}{ar}
\newcommand{\arow}[3]{\langle #1, #2, #3 \rangle}
\newcommand{\smid}{\!\shortmid\!}
%\newcommand{\range}{\mathrm{rng}}
\newcommand{\fuse}{\mathit{fuse}}
\newcommand{\prow}{\pi}
\newcommand{\emptyprow}{\epsilon}
\newcommand{\urset}{\mathcal{R}}
\newcommand{\unionrow}[2]{#1\ltimes#2}
\newcommand{\cprow}[2]{#1\smid#2}
\newcommand{\prowt}{\pi}
\newcommand{\rawform}[4]{(\unionrow{#1}{\cprow{#2}{#3}}\snext #4)}
\newcommand{\tground}{\hat{\tau}}
\newcommand{\nat}{\mathbb{N}}
\newcommand{\pair}[2]{#1 * #2}
\newcommand{\coset}[1]{\bar{#1}}
%\newcommand{\stacked}[1]{\lfloor#1\rfloor}
\newcommand{\stacked}[1]{\cdot#1\cdot}
\newcommand{\venv}{\gamma}
\newcommand{\closure}[2]{\mathtt{C}({#1},{#2})}
\newcommand{\onflag}[1]{\mathbf{on}(#1)}
\newcommand{\offlag}{\mathbf{off}}
\newcommand{\shadow}[2]{#1 ; #2}
\newcommand{\flagsym}{\mathbf{f}}
\newcommand{\gleanenv}{\mathrm{env}}
\newcommand{\valueworld}{\mathbb{V}}
\newcommand{\OK}{\mathit{ok}}
\newcommand{\tOK}{\tau_{\OK}}
\newcommand{\tdead}{\tau_{dp}}
\newcommand{\upddef}{\triangleright}
\newcommand{\defupd}[2]{#1 \upddef #2}
\newcommand{\tsetp}[1]{\tset{\cdot#1\cdot}}
\newcommand{\inject}[1]{{}^\backprime#1}
\newcommand{\match}[3]{\text{\rm match}\ #1\ \text{\rm{with}}\ #2 \rightarrow #3}
\newcommand{\defuser}{\partial}
\newcommand{\defmatch}[1]{\hole \rightarrow #1}
\newcommand{\scod}[1]{\codp{#1}{}}
\newcommand{\succeedmacro}{\text{\bf succeed}}
\newcommand{\failmacro}{\text{\bf fail}}
\newcommand{\freshlab}[1]{\text{\bf{fresh}}_{#1}()}
\newcommand{\sframe}{\mathit{f}}
\newcommand{\oldest}{\text{\rm oldest}}
\newcommand{\sectrans}{$\seclang$-to-$\projml$}
\newcommand{\vb}[1]{\texttt{#1}}
\newcommand{\public}{\text{\texttt{public}}}
\newcommand{\private}{\text{\texttt{private}}}
%\renewcommand{\protected}{\text{\texttt{protected}}}
\newcommand{\metaj}{\mathcal{J}}
\newcommand{\locrule}{\TirName{Loc}}
\newcommand{\eassign}[2]{\mathord{:=}\,#1\,#2}
\newcommand{\eass}[1]{\mathord{:=}\,#1}
\newcommand{\vcod}[1]{\langle#1\rangle}
\newcommand{\rcodp}[2]{\langle#1\rangle_{#2}}
\newcommand{\weakface}[2]{#1\backslash_{#2}}
\newcommand{\labstack}{\delta}
\newcommand{\labcons}[2]{#1::#2}
\newcommand{\rev}{\mathrm{rev}}
\newcommand{\codconfig}[3]{\codp{\hmxconfig{#1}{#2}}{#3}}
\newcommand{\codsim}{\lhd}
\newcommand{\oreduce}{\hookrightarrow}
\newcommand{\Ehole}{[\,]}
\newcommand{\selfobj}[1]{[#1]}
\newcommand{\selft}[1]{\prettyrow{#1}}
\newcommand{\eseclang}{\seclang^{\stack}}
\newcommand{\codE}[3]{\cod{(#2, #3, #1)}}
\newcommand{\slreduce}{\reduce}
\newcommand{\frun}{f_{\mathrm{run}}}
\newcommand{\silenturl}[1]{}
\newcommand{\eg}{e.g.~}
\newcommand{\linfer}[3]{\inferrule*[right=\textsc{\scriptsize#3}]{#1}{#2}}

%
% DANGEROUS REDEFINES!
%
%\renewcommand{\drow}[3]{\fielduk{#1}{#2},#3}
%\renewcommand{\tpre}{\tin}
%\renewcommand{\tabs}{\tout}






%++++++++++++++++++++
% The document itself
%++++++++++++++++++++
\begin{document}

%-----------------------
% Title page information
%-----------------------
\title{Applications of Trust Management Systems}
\author{Peter C. Chapin}
\date{June 2006}
\maketitle

This informal document is intended to catalog possible applications of
trust management (TM) systems. The purpose of this catalog is to find
compelling examples as well as, perhaps, guide future implementation
efforts.

\begin{enumerate}

\item \textit{Calendar Application}. Li and Mitchell \cite{Li:RRBTMF}
  implemented a distributed calendar application called `August' that uses
  $\textrm{RT}_0$ for access control. They say, ``In August each user has a
  calendar and can specify policy that determines who is allowed to view
  each part of the user's calendar and who is allowed to add an activity of
  certain kinds at a certain time.''

\item \textit{IPsec/Firewalls}. Blaze and Ioannidis \cite{Blaze:TMIPS}
  describe an extension to the IPsec architecture that uses KeyNote to
  check if packet filters proposed by a remote host comply with a local
  policy for the creation of such filters. This is a specific case of a
  more general concept of using a TM system for firewall management. For
  example, one could imagine a proxy server using a TM system to control
  who it will proxy for and under what conditions.

\item \textit{Email}. People would like to be able to allow JavaScript and
  other executable content in their email because doing so would enrich the
  email experience. However, nobody does this currently because of the
  dangers of malicious software. Accepting executable content from known
  individuals who sign their messages is inflexible. Using TM techniques,
  an email application could accept executable content from a larger array
  of users, according to local policy.

\item \textit{Web Page Content Ratings}. Several TM systems described in
  the literature use a web page content rating example
  \cite{Gunter:ADDUQCM,Chu:RTMWA}. The idea is that a ratings authority
  issues certificates that associate [the hash of] a web page to a rating.
  The web server delivers this certificate to the browser where local
  policy is consulted to determine if the page should be displayed. There
  was some effort in the W3C to standardize how this might be done
  \cite{Resnick:PIACWC}, although that work appears to have stagnated.

\item \textit{File and Database Sharing}. In environments where
  collaborators are spread across administrative domains, access to common
  files and data can be granted using TM techniques. That way the owner of
  the file or data does not have to know the identity of users on other
  domains. Many potential applications of this exist: web servers,
  conventional file servers, revision control systems, database access,
  etc.

\item \textit{Streaming Media}. Suppose a television station partners with
  MSN to offer streaming video to selected MSN subscribers. The television
  station could use TM techniques to control access to the video feed by
  delegating the authority to specify legitimate users to MSN. This is a
  special case of multi-domain file sharing.

\item \textit{Medical Records}. Several TM systems described in the
  literature refer to medical record applications \cite{Bacon:MORBACSAS,
  Becker:CFTMAEHR}. This is a special case of multi-domain database
  sharing.

\end{enumerate}

\section{Email Application}

% Alice.orgs <- Bob.orgs
% Alice.programmers <- Alice.orgs.programmers
% Alice.executable_content <- Alice.programmers & Bob.dbms_project
%
% Bob.orgs <- ACME
% Bob.dbms_project <- ACME.dbms_project
%
% ACME.programmers <- Carol
% ACME.dbms_project <- Carol

Alice is a programmer at a company working on a new database management
system. She would like to receive complex email messages containing
executable content from other programmers working on the same project.
Those working on her project include locally known individuals defined by
her manager, Bob, as well as certain individuals who are working at an
affiliated organization, ACME Software. However, not all of the project
members are programmers. We assume that each organization maintains a list
of its programmers and that Bob maintains a list of all organizations that
might be relevant to Alice. Note that to protect herself from malicious
content, Alice does not want to process executable content in arbitrary
messages.

In the traditional approach Alice might insist that messages with
executable content be digitally signed and contain an attached certificate
from a certificate authority she knows and trusts that binds the signing
key to a name Alice understands. Alice then creates an access control list
in her mail client specifying which names may send her executable content.

The problem with this approach is that it is relatively inflexible. Alice
must know the names of all programmers in her project, including
programmers at a different organization. Alice is not likely to be made
aware when new programmers are added or removed from her project by ACME
Software and so her access control list will be often be out of date.
\pnote{Add some comments about how RBAC might handle this problem}.

In the trust management approach, Alice loads the following policy into her
email program.

\begin{enumerate}
\item Alice will regard as an organization any entity that Bob regards as
  an organization.

\item Alice will regard as a programmer any entity that one of her listed
  organizations regards as a programmer.

\item Alice will accept executable content from anyone on the DBMS project,
  according to Bob, and who is also a programmer.
\end{enumerate}

Bob's policy specifies ACME Software as an affiliated organization. Also,
in addition to specifically adding certain local people to the DBMS
project, Bob's policy merges ACME Software's group of DBMS people into his
locally defined group. Notice that the security policy is distributed. Not
only is Alice unaware of the precise set of individuals from which she will
accept executable content, she is unaware of precisely how that set is
constructed. However, she has delegated trust to her manager, Bob, and
indirectly to ACME Software to specify these sets correctly. If some
programmers from an additional organization join the DBMS project, Bob
modifies his definition of the project's membership appropriately. Alice
does not need to do anything to begin accepting email with executable
content from these new programmers.

Suppose Alice receives some email from Carol, a programmer at ACME Software
working on the DBMS project. The message is signed with Carols's private
key and is accompanied by several certificates.

\begin{enumerate}
\item A certificate signed by ACME Software asserting that Carol is working
  on the DBMS project.

\item A certificate signed by ACME Software asserting that Carol is a
  programmer.

\item A certificate signed by Bob asserting that ACME Software is an
  affiliated organization. This certificate contains ACME Software's public
  key which could be used to check the signature on the previous two
  certificates.

\item A certificate signed by Bob asserting that ACME Software's definition
  of its DBMS project is part of his definition of the DBMS project.
\end{enumerate}

Based on this information Alice's email client computes that the received
message complies with Alice's policy for executable content.

Notice that two of the certificates above were signed by Bob. In practice
it is likely that Carol will not know to include those certificates in her
message. Furthermore Alice may be initially unaware of them because she
does not know Bob's policy regarding who is in her current project or which
organizations are on his list of affiliated organizations. Thus it may be
necessary, in general, for Alice's email client to automatically attempt to
fetch these additional certificates from Bob's server. Trust management
systems vary in if, and how, they support this sort of credential
discovery.

\section{Medical Records Application}

% Alice.records <- Bob
% Alice.records <- Bob.alice_delegates
%
% Bob.team <- Carol
% Bob.team <- Bob.team.support
% Bob.alice_delegates <- Hospital.medical_staff & Bob.team
%
% Hosptial.medical_staff <- Dave (a lab technician)
%
% Carol.support <- Dave

Suppose Alice is a cancer patient at a hospital being treated by Bob, a
doctor. Alice grants Bob access to her medical records and also allows Bob
to delegate such access to others as he sees fit.

Bob defines his team as a particular collection of individuals together
with the people supporting them. Notice that this definition is recursive;
a person supporting one of Bob's team members becomes a team member. Here
we assume that Bob's team includes both medical and non-medical personnel
(for example other doctors as well as, perhaps, receptionists). Bob then
delegates his access to Alice's medical records to only the medical staff
on his team.

Suppose further that Bob consults with another doctor, Carol, on Alice's
condition. Bob modifies his policy to add Carol temporally to his team.
Carol orders some blood tests that are then analyzed by Dave, a lab
technician and one of Carol's support people. The policy described allows
Dave to access Alice's medical records, for example to input the test
results.

Dave signs the test results when he uploads them to the hospital database.
He also includes the following certificates.

\begin{enumerate}
\item A certificate signed by Carol asserting that Dave is one of her support
  people.

\item A certificate signed by the hospital asserting that Dave is a member of
  the medical staff.

\item A certificate signed by Bob asserting that he delegates his access to
  Alice's medical records to the medical staff on his team.

\item A certificate signed by Bob asserting that his team contains all
  the people supporting those on his team.

\item A certificate signed by Bob asserting that Carol is on his team.
\end{enumerate}

In this scenario the hospital database is assumed to already know Alice's
policy. As with the earlier example, Dave may not realize that he needs to
submit the various certificates that have been signed by Bob. Since Carol
requested the test Dave may not realize that Bob is involved at all.
Instead the hospital database fetches Bob's policy from Bob's server in
order to complete the authorization decision.

Complex access control scenarios such as this one are difficult to express
using traditional methods. Neither Alice nor Bob realize that Dave needs to
be granted access to Alice's medical records. Although Dave's role as one
of Carol's support people might be enough to grant him access to the
records of Carol's patients, the fact that Bob is consulting with Carol on
Alice's case is not something that would be initially assumed or expected.
This connection is established by Bob's policy, which he writes on demand,
and not by any previous decision made by the hospital.

% Extend this example to include the case where Bob consults with a doctor
% at a different hospital?

\subsection{RT}

To express the medical records example using $\RT$, only the facilities of
$\RT_0$ are necessary. Alice defines a role \texttt{records} whose members
are able to access her medical records. She creates the policy

\begin{itemize}
\item \texttt{Alice.records} $\leftarrow$ \texttt{Bob}
\item \texttt{Alice.records} $\leftarrow$ \texttt{Bob.alice\_delegates}
\end{itemize}

The first rule grants her doctor, Bob, access to her records. The second
rule allows Bob to further delegate that access by defining the membership
of an \texttt{alice\_delegates} role.

Bob's standing policy is

\begin{itemize}
\item \texttt{Bob.team} $\leftarrow$ \texttt{Bob.team.support}
\item \texttt{Bob.alice\_delegates} $\leftarrow$
  \texttt{Hospital.medical\_staff} $\cap$ \texttt{Bob.team}
\end{itemize}

The first rule defines Bob's team as including all the support personnel
specified by the members of his team. In the second rule, Bob uses an
intersection role to specify that only the medical personnel on his team
should have access to Alice's medical records.

When Bob consults with Carol he adds \texttt{Bob.team} $\leftarrow$
\texttt{Carol} to his policy to add Carol, and indirectly all of Carol's
support people, to his team.

The only part of Carol's policy relevant to this example places Dave in her
\texttt{support} role: \texttt{Carol.support} $\leftarrow$ \texttt{Dave}.
Finally Dave has a credential from the hospital asserting his membership in
the \texttt{medical\_staff} role. $\RT_0$ can use these credentials to
prove that Dave is a member of \texttt{Alice.records} and thus able to
access Alice's medical records.

\subsection{SDSI}

SDSI is very similar to $\RT_0$ except that SDSI lacks intersections.
\pnote{This isn't strictly true, although much of the literature acts as if
it is.} This presents a problem in the example as originally stated since
Bob would then be forced to grant his entire team, including non-medical
personnel, access to Alice's medical records. To work around this Bob could
distinguish between his overall team and his medical team and assume that
the members of \texttt{support} are only medical personnel. Assuming that
$K_a$ is Alice's key and so forth, the credentials become

\begin{itemize}

\item $K_a\, \text{\tt records} \rightarrow K_b$
\item $K_a\, \text{\tt records} \rightarrow K_b\, \text{\tt alice\_delegates}$
\item $K_b\, \text{\tt medical\_team} \rightarrow K_b\, \text{\tt
  medical\_team}\,\, \text{\tt support}$
\item $K_b\, \text{\tt alice\_delegates} \rightarrow K_b\, \text{\tt
  medical\_team}$
\item $K_b\, \text{\tt medical\_team} \rightarrow K_c$
\item $K_c\, \text{\tt support} \rightarrow K_d$

\end{itemize}

\subsection{SPKI}

In the SDSI example above, a request signed by a key $K$ is granted access
to Alice's records if the name $K_a\, \text{\tt records}$ can be resolved
to $K$. SPKI version 2.0 contains SDSI and so this approach would apply in
a SPKI setting as well. However, SPKI also provides authorization
certificates.

So far the example has treated Alice's medical records as a single entity.
If access is granted to any part of Alice's records, access is granted to
all of Alice's records. If Alice's medical records contain many components
one could assert access control over each component individually using
separate names (or roles) to represent the different components. However,
for indefinitely large structured resources such an approach is infeasible.
SPKI authorization tags allow an indefinite subset of a structured resource
to be specified and thus offers a degree of expressiveness that is not
possible using SDSI (or $\RT_0$) alone.

For example, Alice might issue a SPKI authorization certificate that grants
Bob access to her medical records (or some portion thereof) and grant Bob
the power to delegate that access to others. Such a certificate might look
like

\vspace{1.5ex}
\centerline{$(K_a, K_b, \text{\tt true}, \text{\tt records:*})$}

Here the string \texttt{records:*} is assumed to convey all access to all
records. The precise format of this string is application dependent. As
long as the hospital database, acting on behalf of Alice, understands its
meaning authorizations will only be carried out according to Alice's
wishes.

Bob could delegate the authorization he received from Alice to his medical
team by issuing another authorization certificate such as

\vspace{1.5ex}
\centerline{$(K_b, K_b\,\,\text{\tt medical\_team}, \text{\tt false},
  \text{\tt records:*})$}

Here Bob prevents further delegations of the authorization. In this
example, Bob passes his entire set of permissions to his medical team.
Assuming Bob understands the format of the authorization tag, he could
optionally pass a subset of his permissions to his team. When a request is
made the hospital database would intersect the authorization tags to find
the overall set of permissions allowed in the request.

In this example, no further authorization certificates are necessary. When
Dave submits his test results to the hospital database, he must sign his
request with his key and provide SDSI name certificates to prove his key's
association with $K_b\,\,\text{\tt medical\_team}$. He must also provide
the two authorization certificates showing that $K_b\,\,\text{\tt
medical\_team}$ is authorized to access Alice's medical records.

\subsection{SD3}

SD3 treats the authorization problem as a distributed database query.
Accordingly we need to first express the information to be processed as a
collection of relations. Alice maintains a one-place relation
\texttt{records} with tuples storing the keys of those principals who can
access her medical records. We can then represent Alice's policy as the
following two SD3-style \datalog\ rules.

\begin{itemize}
\item \texttt{$K_a$\$records($K_b$).}
\item \texttt{$K_a$\$records(X) :- $K_b$\$alice\_delegates(X).}
\end{itemize}

\datalog\ has no problems expressing either recursion or intersections
(conjunctions). Bob's policy becomes

\begin{itemize}
\item \texttt{$K_b$\$team(X) :- $K_b$\$team(Y), Y\$support(X).}
\item \texttt{$K_b$\$alice\_delegates(X) :- $K_h$\$medical\_staff(X), $K_b$\$team(X).}
\item \texttt{$K_b$\$team($K_c$).}
\end{itemize}

The remaining assertions made by the hospital and by Carol are

\begin{itemize}
\item \texttt{$K_h$\$medical\_staff($K_d$).}
\item \texttt{$K_c$\$support($K_d$).}
\end{itemize}

This example does not illustrate much of SD3's expressiveness. All the
relations involved are one-place relations containing keys. These
correspond to the roles of $\RT$ and the names of SDSI. However, SD3 is
much more general because the relations involved can be arbitrary.

SD3 distinguishes between global names that are key-qualified and local
names that are not. If the hospital database evaluates Dave's request to
update Alice's medical records in Alice's context, then the $K_a$ prefix on
Alice's policy is superfluous. The tuples from the other relations would be
signed by the corresponding key and obtained ``externally'' from Alice's
context.

\bibliographystyle{plain}
\bibliography{survey}

\end{document}
