%----> cut <----
%
%\inferrule[signature-check]
%  { P \mkeyword{believes} (\stackrel{K}{\longmapsto} Q) \\
%    P \mkeyword{sees} \{X\}_{K^{-1}} }
%  { P \mkeyword{sees} X }
%
%The \textsc{signature-check} rule formalizes a partial notion of message
%validity. It allows one to unwrap a signed message. If $P$ believes key
%$K$ is $Q$'s public key and $K$ successfully checks the signature on
%$\{X\}_{K^{-1}}$, then $P$ sees the message content $X$. A similar rule
%exists for extracting the message content of a symmetrically encrypted
%message $\{X\}_K$. None of the inference rules use $P \mkeyword{sees} X$
%as a premise, however, but instead work directly with encrypted or signed
%messages. Thus the \textsc{signature-check} rule is used only to unwrap an
%outer later of cryptographic protection to access a nested signed or
%encrypted message. In any case, AAN logic incorporates a notion of
%signature checking into its formal structure. This operation is also
%essential in trust management systems but, as we will see, is often
%performed outside the main logic in such systems.
%
%----> cut <----

\section{Foundations of Authorization}
\label{section-foundations}

Although trust management systems comprise a number of components as
discussed in \autoref{section-overview-structure}, the heart of any
system is its authorization semantics, denoted $L$ in
\autoref{figure-tmstruct}, where the authorization decision is
realized. The semantic foundations of authorization are well studied,
having evolved from logical formalisms originally developed for
verifying distributed authentication protocols
\cite{Burrows:LA,Abadi:CACDS}, and early work on access control as a
distinct concern in distributed systems \cite{woo93authorizations}. While
formalisms other than logic have been used to specify authorization
semantics, notably graph theory and relational algebra as discussed at
various points in this paper, logic is probably the most popular, and
logical approaches have had a broad impact on the foundations and
practice of authorization and security \cite{bonatti-logicsforauth}.
Thus, a review of logical foundations provides historical perspective
and insight into standard features and themes of authorization. To
this end we now review BAN \cite{Burrows:LA} logic, the so-called
logic of authentication \cite{Abadi:CACDS} which is commonly
abbreviated ABLP, and aspects of programming logics such as Prolog and
Datalog that are relevant to the issue.

\subsection{BAN Authentication Logic}

The first thorough study of a logic for specifying and verifying
security protocols was presented in \cite{Burrows:LA} where a logic,
commonly called BAN logic, was introduced.  In that paper the authors
analyze several authentication protocols, including Kerberos, Andrew
Secure RPC, Needham-Schroeder Public-Key, and X.509.  Although BAN was
not intended as a foundation for authorization semantics, it is
instructive to observe how it became so.  From an authorization
perspective, BAN is historically and technically significant because
it introduces the ideas of representing beliefs, statements, and
capabilities of participants in a distributed protocol using a formal
logical framework.

BAN logic is a many-sorted modal logic that distinguishes between
atomic principals $P$ and encryption keys $K$. Formulae are created from
propositional conjunction along with several additional
constructs. These constructs include the following forms.
\begin{itemize}
\item $\,P \mkeyword{believes} X$: principal $P$ might act as
  if statement $X$ were true.
\item $\,P \mkeyword{sees} X$: principal $P$ has received a
  message containing statement $X$.
\item $\,P \mkeyword{said} X$: at some point in the past (not
  necessarily during the current authentication session) principal $P$
  sent a message containing statement $X$.
\item $\,P \mkeyword{controls} X$: principal $P$ is an
  authority over $X$ and should be trusted on it.
\item $\,\keyword{fresh}(X)$: statement $X$ is fresh. It has not
  been asserted during any previous authentication session.
\item $\,P \stackrel{K}{\longleftrightarrow} Q$: principals
  $P$ and $Q$ can communicate using the shared key $K$.
\item $\,\stackrel{K}{\longmapsto}P$: principal $P$ has public
  key $K$. The private key corresponding to $K$ is called $K^{-1}$.
\item $\,\{X\}_K$: statement $X$ is encrypted under key $K$. A
  statement encrypted under a private key is a signed statement.
\end{itemize}

BAN logic allows representation of statements a given principal says
and believes as well as statements over which a principal has
authority; these same ideas are used in authorization logics as
well. In addition BAN logic allows one to talk about encryption keys,
incorporating key security into the logic.

Inference rules formally specify the proof theory of BAN language
constructs. A sampling of these inference rules is given in
\autoref{figure-BANrules}\footnote{The rule names in \autoref{figure-BANrules} 
and \autoref{figure-ABLPinferencerules}
have been made up by us to ease discussion.}. For example
\textsc{message-meaning-1} says that if $P$ shares a key $K$ with $Q$, and
$P$ receives a message encrypted with $K$, then $P$ can conclude $Q$
is the source of the message. The \textsc{message-meaning-2} rule allows a
similar inference for signed messages. These rules form the connection
between the principals and the keys they use.  Later authorization
logics that consider interactions of principals and keys use similar
rules to characterize these interactions.

\BANrulesfig

The \textsc{jurisdiction} rule formalizes a notion of delegation of
authority, an essential ingredient in all trust management systems. The
rule says that if $P$ regards $Q$ as an authority over $X$ and $P$ believes
$Q$ is asserting $X$, then $P$ will accept $Q$'s authority and believe $X$
as well.

The \textsc{signature-check} rule encodes signature authentication and
message encryption, allowing one to unwrap a signed message. If $P$
believes key $K$ is $Q$'s public key and $K$ successfully checks the
signature on $\{X\}_{K^{-1}}$, then $P$ sees the message content $X$.
A similar rule exists for extracting the message content of a
symmetrically encrypted message $\{X\}_K$.  Inference rules that
accept a message that $P$ sees as a premise require that message to
be signed or encrypted.  Since the details of key
signatures and encryption are generally hidden from the authorization
component of most trust management systems as discussed in
\autoref{section-introduction}, these features of BAN make it more
appropriate as a logic of authentication.

As an example of representations in BAN logic, the statement:
\begin{displaymath}
A \mkeyword{believes} (A \stackrel{K_{as}}{\longleftrightarrow} S)
\end{displaymath}
expresses $A$'s belief that she shares a key $K_{as}$ with some
authentication server $S$. The statement:
\begin{displaymath}
A \mkeyword{sees} \{A \stackrel{K_{ab}}{\longleftrightarrow} B\}_{K_{as}}
\end{displaymath}
represents a message sent to $A$ containing a key intended to be
shared between $A$ and $B$ and encrypted using a key shared between
$A$ and some authentication server $S$. The goal of the analysis might
be to prove a formula such as
\begin{displaymath}
(A \mkeyword{believes} (A \stackrel{K}{\longleftrightarrow} B)) \wedge
(B \mkeyword{believes} (A \stackrel{K}{\longleftrightarrow} B))
\end{displaymath}
for some key $K$. Such a formula asserts that both $A$ and $B$ believe they
have a key they share with each other.

\subsection{ABLP Distributed Authorization Logic}

\ABLPinferencerulesfig

The logical basis of many trust management systems is due to a general
calculus for distributed access control called the logic of
authentication and commonly abbreviated ABLP logic
\cite{Abadi:CACDS}. (ABLP is an acronym for the author list of the
seminal paper that develops the logic \cite{Abadi:CACDS}).  ABLP
develops many ideas introduced with BAN logic, but is intended less as
a low-level specification language for authentication protocols, and
more as a logic for reasoning about access control in general.  As the
authors discuss, this includes authorization issues-- groups, roles,
delegation, etc.  Hence, ABLP logic formalizes a rich authorization
semantics, and has inspired much subsequent development in trust
management.

Full ABLP logic is sufficiently expressive to be undecidable. However,
\cite{Abadi:CACDS} describes a number of restrictions to ABLP logic
that allow for decidable access control decisions while still
retaining enough expressivity to be useful.  In practice, it has
been used to support access control in the Taos operating system
\cite{Wobber:ATOS}. 

In ABLP logic principals $P$ can be users, roles, machines, I/O channels,
encryption keys or any other convenient abstraction. In addition to atomic
principals, denoted $A,B,C$, etc., \emph{compound} principals can
constructed with the use of connectives:
\begin{itemize}
\item $A \vee B$: this principal represents the group containing $A$ and $B$.
\item  $A \wedge B$: this principal issues statements signed jointly by 
$A$ and $B$.
\item $A|B$: pronounced ``$A$ quoting $B$'', this principal issues statements
said by $A$ to originate from $B$.
\item $A \for B$: this principal represents $A$ speaking on behalf of 
$B$, which is a stronger notion than $A|B$.
\item $A \as R$: this principal represents $A$ assuming the role 
$R$.
\end{itemize}
ABLP formulae $s$ are then built from principals, standard logical
connectives, and special connectives for representing authorization
concepts:
\begin{itemize}
\item $p$, $\neg s$, $s \wedge s, s \implies s$: 
propositional atoms, negation, conjunction, and implication.
\item $P_1 \speaksfor P_2$: pronounced ``$P_1$ speaks for $P_2$'', this denotes that
$P_1$ speaks with all the authority of $P_2$.
\item $P \says s$: this denotes that $P$ has uttered the statement $s$.
\end{itemize}

In addition to the usual inference rules of propositional logic,
inference rules and axioms are provided for the authorization-specific
connectives, including those defined in
\autoref{figure-ABLPinferencerules}.  Notably, rule
\TirName{Speaksfor} says that if $A$ speaks for $B$ and $A$ has
asserted some $s$, then implicitly $B$ has also asserted $s$.  Rule
$\TirName{As}$ establishes that the principal connective $\mathit{as}$ can be defined as
a derived form of ($|$). These rules together comprise a proof theory,
where consequences can be deduced from assumptions.

In particular, resources can be represented as atomic propositions
$\priv$, and access to the resource can be granted if the associated
proposition can be proved given assumptions about policy and
credentials.  For example, with $A
\controls s$ defined as syntactic sugar for $(A \says s) \implies s$,
access control lists may be modeled as conjunctions of assertions $A
\controls \priv$.  An access request for $\priv$ by $A$ is represented
as the assumption $\vdash A \says \priv$.  Observe that assumptions
$\vdash A \says s$ and $\vdash A \controls s$ together imply $\vdash
\priv$ by modus ponens, allowing access to the denoted resource.

The ABLP formula language can express a rich collection of authorization
features.  In addition to access control list encodings as described
above, role membership can be modeled.  To specify that $A$ is a
member of a role $R$, policy can include the assumption $\vdash A
\speaksfor R$.  Then, when $A$ assumes the role $R$ to make an
assertion $s$, represented as assumption $\vdash A \as R \says s$, rules
\TirName{As}, \TirName{Speaksfor}, and \TirName{Quoting} allow deduction of 
$\vdash A \says R \says s$, from which can be derived $\vdash R \says
R \says s$ by \TirName{Speaksfor}, hence $\vdash R|R \says s$ by
\TirName{Quoting}, and finally $R \says s$ follows by assuming idempotence 
of ($|$) for roles
\cite{Abadi:CACDS}.  Note that ACL representations in this model need
only take into account roles, e.g.~$R \controls \priv$ allows any
specified member of $R$ to gain access to $\priv$.

A variety of delegation idioms can also be modeled
\cite{Abadi:CACDS}.  An assertion of the form $A \for B \says
s$ means that $A$ has asserted $s$ on behalf of $B$, and denotes that
$B$ has delegated the assertion of $s$ to $A$.  In contrast, $A \says
B \says s$ merely represents $A$'s claim that $B$ asserts $s$, and
requires no verification of the statement.  The meaning of 
$A \for B$ can be altered according to desired delegation policies,
for example $A \for B \says s$ can be taken as syntactic sugar for
the formula $A \wedge D \says s$, where $D$ represents a 
delegation server.

\subsection{Programming Logics}

Programming logics such as \prolog\ and \datalog\ have played an
important role in the development of trust management systems.  As
discussed above, logics provide useful abstractions for authorization
semantics, furthermore specifications in executable programming logics
provide prototype implementations for free.  Programming logics have
served as target languages for the compilation of higher-level
authorization languages \cite{Li:DCFTML,woo93authorizations}, have
served as the foundation for enriched authorization languages
\cite{Li:USSUFOL,Jim:STMSCE,DeTreville:BLBSL,Li:DRBTMF,Li:DLLBADA},
and have been used for the formalization and study of trust management
systems \cite{Li:USSUFOL,polakow-skalka-plas06}.

Both \prolog\ and \datalog\ are \emph{Horn-Clause} logics, in which
all formulae are restricted to the form $\mbox{\it head} \leftarrow
\mbox{\it body}$, where $\leftarrow$ is a right-to-left implication
symbol, $\mbox{\it head}$ is a proposition, and $\mbox{\it body}$ is a
conjunction of propositions.  If variables $X$ appear in a rule, the
rule is implicitly universally quantified over those variables. The
head of each rule is the consequent of the body.  If $\mbox{\it body}$
is empty then the rule is a \emph{fact}.

As a simple example of how logics can apply in a trust management
framework, imagine that delegation should be transitive.  Suppose that
$\mathit{delegation}(X, Y)$ is defined to mean that the rights of $X$ have been
delegated to $Y$. Suppose also that $\mathit{cert}(X, Y)$ represents a
delegation certificate passing rights directly from $X$ to $Y$. The
following Horn clauses obtain transitivity of delegation:
\begin{mathpar}
\mathit{delegation}(X, Y) \leftarrow \mathit{cert}(X, Y)

\mathit{delegation}(X, Y) \leftarrow \mathit{cert}(X, Z), \, \mathit{delegation}(Z, Y)
\end{mathpar}
Letting $a,b,c,...$ denote constants, the following represents a
collection of delegation certificates:
\begin{mathpar}
\mathit{cert}(a, b)

\mathit{cert}(b, c)

\mathit{cert}(b, d)

\mathit{cert}(c, e)
\end{mathpar}
From these facts and the definition of $\mathit{delegation}$, the
query $\mathit{delegation}(a, e)$ will succeed while
$\mathit{delegation}(d, e)$ fails.

Datalog was developed as a query language for databases. It is not a full
programming language. In contrast Prolog is Turing complete and thus more
expressive than Datalog. This extra expressivity is useful in certain
contexts. For example, a full-featured authorization logic called
Delegation Logic has been defined as a strict extension of Datalog at a
high level, that is ultimately compiled to Prolog for practical
implementation \cite{Li:DLLBADA}. However, Datalog has certain advantages
in the authorization setting: the combination of monotonicity, a bottom-up
proof strategy, and Datalog's \newterm{safety condition} (any variable
appearing in the head of a rule must also appear in the body) guarantee
program termination in polynomial time. In contrast, Prolog's top-down
proof search can cause non-termination in the presence of cyclic
dependencies. For example, if we added the certificate $\mathit{cert}(e,b)$
to the above fact set, some queries would not terminate. This problem is
resolved by \emph{tabling} as in XSB \cite{xsb-page}, but it has been
argued that this solution adds too much size and complexity to the
implementation for authorization decisions \cite{Li:DRBTMF}. And while
Datalog is not capable of expressing structured data, Datalog with
constraints (\datalogc), a restricted form of constraint logic programming
\cite{jaffar-maher-jlp94}, has been shown sufficiently expressive for a
wide range of trust management idioms \cite{Li:DCFTML}.

Prolog is able to express negation-as-failure, and so-called
Disjunctive Datalog is likewise able to express a restricted form of
negation \cite{eiter-etal-tdbs97}.  
Therefore nonmonotonic
authorization features such as credential negation can be provided in
systems where programming logics are intended to serve as a basis for
semantic interpretation or implementation
\cite{woo93authorizations,bonatti-logicsforauth}.  However, as
discussed in \autoref{section-features}, nonmonotonicity in
authorization semantics is generally considered undesirable, since it
introduces the possibility of unsoundness in practice
\cite{seamons-policy02}.  Also, while certificate revocation seems at
first blush to entail nonmonotonicity, it has been shown to be
definable monotonically with appropriately constructed logical
inference rules \cite{lbi-fc01,Bauer:GFACSW}.  Or, as discussed in
\autoref{section-features}, revocation can be handled by components
external to the authorization semantics (via component $T_C$ in
\autoref{figure-tmstruct}), for example by filtering certificates
through certificate revocation lists prior to authorization decisions
as in SPKI/SDSI.  For these reasons previous authors have argued that
monotonic (subsets of) programming logics are adequate foundations for
trust management applications, such as safe Datalog with constraint
domains \cite{Li:DCFTML}.

Recently, more expressive programming logics have been proposed to
address restrictions in the Horn-clause formula languages of Datalog
and Prolog.  Relevant work has proposed use of the higher-order linear
logic programming language LolliMon as a foundation for trust
management systems \cite{polakow-skalka-plas06}.  LolliMon is not
restricted to a Horn-clause form, and the availability of hypothetical
(vs. strictly literal) subgoals and linear assumptions in particular
allow the formal modeling of distributed certificate chain discovery
(component $D$ in \autoref{figure-tmstruct}), as interleaved with the
authorization semantics of a trust management system.
