
\subsection{SPKI/SDSI}

The Simple Distributed Security Infrastructure (SDSI)
\cite{Rivest:SDSI-10,Rivest:SDSI-11} is a system for managing
distributed name spaces.  In addition to global names, a primary
contribution of SDSI is linked local name space management, where
name spaces are defined and structured locally, but can be referenced
non-locally.  An authorizer associates access rights with a particular
local name, and any principals \emph{bound} to that name by SDSI
\emph{name certificates} are authorized for access.  By signing access
requests with names, an authorization logic is obtained based on
name-to-key bindings, and linking relations between names.

The Simple Public Key Infrastructure (SPKI) was developed concurrently
with SDSI to provide more complex authorization policies in
distributed systems without the need for managing identities. These
technologies merged into SPKI/SDSI version 2.0
\cite{RFC-2693}.  SPKI adds to SDSI the ability to
directly bind a capability, called an
\newterm{authorization certificate}, to a name or key. Such bindings are only
meaningful if the issuing principal has a superset of the capabilities
being bound. Thus SPKI/SDSI allows a principal to explicitly delegate
a subset of his or her rights to another principal or to a name
representing a collection of principals, resulting in a rich
authorization language.  

\subsubsection{Features of SDSI}

In SDSI public keys have the status of principals, and there is no
attempt to associate public keys with individual people, machines, or
other entities, this being regarded as an external consideration.  When
discussing SPKI/SDSI we use the terms ``principal'' and
``public key'' interchangeably.  The system implicitly assumes that
private keys are secure and that statements signed by those keys
reflect the intentions of the corresponding principal. Thus the SDSI
authorization semantics does not concern itself with certificate
signature checking, rather such checking is handled entirely by the
processing of certificates prior to authorization (that is, by
component $T_C$ in \autoref{figure-tmstruct}).

In SDSI each principal defines a structured name space local to that
principal by issuing \newterm{name certificates}, binding names to
principals in a manner that confers the rights of the name to the
principal. A SDSI name is of the form $K\,A$, where $K$ is a key
identifying a name space and $A$ fully qualifies the name. Intuitively, we
may read $K\,A$ as ``$K$'s $A$'', i.e., a local name $A$ in $K$'s name
space. In addition, SDSI provides \emph{extended names} of the form
$K\,A_1\cdots A_n$ that allows linkage to non-local name spaces as discussed
below. A name certificate is abstracted as a 4-tuple of the form $(K, A, S,
V)$ \cite{RFC-2693} where:
\begin{itemize}
\item $K$ is the principal issuing the certificate.
\item $A$ is the name in $K$'s name space being defined.
\item $S$ is the subject of the certificate (the name being bound to $A$).
\item $V$ is certificate validity information.
\end{itemize}
Certificate subjects $S$ can be other principals, names, or linked
names.  The certificate validity field $V$ contains expiration times
or information about where to obtain revocation or revalidation
information on-line. An authorizer can use this information to check
if a certificate has been revoked in real-time. A SDSI system
disregards any certificates that are expired or have been revoked so validity
information does not play a direct role in authorization decisions.

Since validity information $V$ is not relevant to the SDSI
authorization semantics, the following useful syntactic sugar can be
defined for the consideration of SDSI authorization logic:
$$
K A \rightarrow S\ \defeq\ (K,A,S,V) \qquad \text{$V$ is valid}
$$
Informally this means that the name $A$ in $K$'s name space is being
defined as a local name for the subject $S$. For example, the
certificate $K_a\,\,\text{\tt robert}\,\,\rightarrow K_b$ indicates
that $K_b$ is bound to $K_a$'s \texttt{robert}.

The meaning of a SDSI name is the set of principals bound to that name
by valid name certificates.  Using only the above certificate, the
meaning of $K_a$'s $\text{\tt robert}$ is the set $\{K_b\}$.  SDSI
specifically allows multiple name certificates to define the same
name, so if $K_a$ issued a second name certificate asserting
$K_a\,\,\text{\tt robert}\,\,\rightarrow K_{B'}$ then the meaning of
$K_a$'s $\text{\tt robert}$ would be the set $\{K_b, K_{B'}\}$.  Thus
SDSI names are essentially group names.  When an authorizer associates
access rights to a local name, that name behaves similarly to a role
as defined by the role-based access control (RBAC) community
\cite{Ferraiolo:RBAC}. In this way SDSI provides support for RBAC.

An important aspect of SDSI is that it allows certificate subjects to
refer to non-local names. This can be done via a certificate of the
form $K_1\,A_1 \rightarrow K_2\,A_2$ denoting that the meaning of
$K_1\,A_1$ subsumes that of $K_2\,A_2$, i.e.~all names bound to
$K_2\,A_2$ are also bound to $K_1\,A_1$.  Certificate subjects can
also be an extended name $K\,A_1\,\cdots\,A_n$.  For $n=2$, the
extended name $K\,A_1\,A_2$ has a meaning that is based on the meaning
of $K\,A_1$: it is the set of all names bound to $K_x\,A_2$ such that
$K_x$ is bound to $K\,A_1$.  Iterating this idea obtains meaning for
extended names with higher values of $n$.

%% Cut this example, doesn't seem to add much.
%
%Each SDSI principal is the root of a name hierarchy, delegating authority
%over the subordinate names to the principals that defined them. For example
%the name certificate
%$$
%K_1\,\,\text{post-people} \rightarrow K_{\text{usps}}\,\,\text{\tt Florida}\,\,{\tt Miami}\,\,\text{\tt postal-workers}
%$$
%issued by $K_1$ defines the local name \texttt{post-people} to refer to the
%principals in the \texttt{postal-workers} group of the Miami, Florida post
%office. Here we assume that the principal representing the US Postal
%Service, $K_{\text{usps}}$, has issued a name certificate for its
%\texttt{Florida} local name that binds that name to the key used by the
%Florida postal operations center. The Florida postal operations center has,
%in turn, issued a name certificate that binds the Miami post office key to
%the name \texttt{Miami}, and so forth. This decentralizes control over the
%membership of $K_1\,\,\text{post-people}$ since the bindings involved in
%the name certificate above can be changed or extended by various
%principals.

\subsubsection{Running Example (SDSI)}

Here we show how to encode the policies described in the medical records
example in \autoref{section-overview-example} using SDSI. Assuming that
$K_a$, $K_b$, $K_c$, and $K_d$ are Alice, Bob, Carol, and Dave's keys
respectively, Alice's policy is expressed as:
\begin{itemize}
\item $K_a\, \text{\tt records} \rightarrow K_b$
\item $K_a\, \text{\tt records} \rightarrow K_b\, \text{\tt alice\_delegates}$
\end{itemize}
Although the original SDSI definitions provided a way to define groups of
principals using set intersections, SDSI version 2.0 lacks this facility.
This presents a problem in the example as originally stated since Bob would
then be forced to grant his entire team, including non-medical personnel,
access to Alice's medical records. To work around this Bob can distinguish
between his overall team and his medical team, so his policy is:
\begin{itemize}
\item $K_b\, \text{\tt medical\_team} \rightarrow K_b\, \text{\tt
  medical\_team}\,\, \text{\tt support}$
\item $K_b\, \text{\tt alice\_delegates} \rightarrow K_b\, \text{\tt
  medical\_team}$
\item $K_b\, \text{\tt medical\_team} \rightarrow K_c$
\end{itemize}
Carol's policy includes $K_c\, \text{\tt support} \rightarrow K_d$ defining
Dave as a member of her support staff. Again because SDSI 2.0 lacks
intersections the hospital's assertion that Dave is a medical staff member
is not used; we must presume that Carol will only add medical staff members
to her support staff. In a more realistic situation Carol may want to
distinguish between her medical support staff and her non-medical support
staff by using two distinct names.

\subsubsection{Features of SPKI}

SPKI extends the SDSI framework with authorization certificates,
allowing authorization rights to be delegated from principal to
principal. Such certificates have the form of a 5-tuple, $(K, S, D, T,
V)$ where:
\begin{itemize}
\item $K$ is the principal issuing the certificate.
\item $S$ is the subject of the certificate.
\item $D$ is a boolean delegation flag.
\item $T$ is the authorization tag.
\item $V$ is certificate validity information.
\end{itemize}
The $K$, $S$, and $V$ fields are the same as for SDSI name
certificates.  The $T$ field of the certificate is the
\newterm{authorization tag}. It is formatted as an s-expression with
specific rules regarding its structure.  The meaning of the
tag is left undefined by SPKI and is application specific.  For
example a tag such as
\texttt{(http (port 8080) (read (url /downloads)))} might represent the
capability of being able to read from the \texttt{downloads} directory
on the HTTP server at port 8080. In this way, SPKI authorization tags
provide a way to make statements about structured resources.

The $D$ field of the certificate is the delegation flag.  If set the
subject is allowed to further delegate the authorization to others by
issuing new authorization certificates as appropriate. SPKI provides
only boolean delegation control where authorizations can be delegated
arbitrarily or not at all. SPKI's design does not allow a principal to
specify an integer delegation depth because of the inherent difficulty
in specifying an appropriate depth.  The argument for this is that in
general principals can't easily know how many levels of delegation an
authorization might reasonably need. Also since controlling the depth
of delegation does not restrict the width of the delegation tree, a
limited depth does not necessarily prevent rampant delegation
\cite{RFC-2693}.

\subsubsection{Running Example (SPKI)}

In the SDSI example above, a request signed by a key $K$ is granted access
to Alice's records if the name $K_a\, \text{\tt records}$ can be resolved
to $K$. SPKI/SDSI version 2.0 contains SDSI and so this approach would
apply in a SPKI/SDSI setting as well. However, SPKI also provides
authorization certificates.

So far the example has treated Alice's medical records as a single
entity.  If access is granted to any part of Alice's records, access
is granted to all of Alice's records. If Alice's medical records
contain many components one could assert access control over each
component individually using separate names to represent the different
components. However, for indefinitely large structured resources such
an approach is infeasible.  SPKI authorization tags allow an
indefinite subset of a structured resource to be specified and thus
offers a granularity of control that is not possible using SDSI alone.

For example, Alice might issue a SPKI authorization certificate that
grants Bob access to her medical records (or some portion thereof) and
the power to delegate that access to others. Such a certificate might
look like:

\vspace{1.5ex}
\centerline{$(K_a, K_b, \text{\tt true}, \text{\tt (records Alice (rw *))})$}
\vspace{1.5ex}
\noindent Here the authorization tag \texttt{(records Alice (rw *))} is assumed
to convey full (read and write) access to all of Alice's
records. Although the precise format of this string is application
dependent, as long as the hospital database acting on behalf of Alice
understands its meaning, authorizations will only be carried out
according to Alice's wishes.

Bob could delegate the authorization he received from Alice to his
medical team by issuing another authorization certificate:

\vspace{1.5ex}
\centerline{$(K_b, K_b\,\,\text{\tt medical\_team}, \text{\tt false},
  \text{\tt (records Alice (rw *))})$}
\vspace{1.5ex}
\noindent Here Bob prevents further delegations of the authorization. In this
example, Bob passes his entire set of permissions to his medical team.
Assuming Bob understands the format of the authorization tag, he could
optionally pass a subset of his permissions to his team. When a
request is made the hospital database would intersect the
authorization tags to find the overall set of permissions allowed in
the request.

In this example, no further authorization certificates are necessary. When
Dave submits his test results to the hospital database, he must sign his
request with his key and provide SDSI name certificates to prove his key's
association with $K_b\,\,\text{\tt medical\_team}$. He must also provide
the two authorization certificates showing that $K_b\,\,\text{\tt
medical\_team}$ is authorized to access Alice's medical records. Notice
that one of these authorization certificates is signed by Alice and thus
authority over her records ultimately comes from her.


\subsubsection{Semantics}

The original presentations of SPKI and SDSI 
\cite{Rivest:SDSI-10,Rivest:SDSI-11,RFC-2693} provide a thorough 
informal specification of its semantics and also sketch an
operational meaning of certificates via rewrite rules as discussed
below in \autoref{section-SDSI-SPKI-implementation}, but a rigorous
formal specification has been a distinct project carried out after the
initial development of the system.  The problem is surprisingly
subtle, with a number of authors proposing alternate solutions.

The semantics proposed by Clarke et al.~\cite{Clarke:CCDSS} is
constructed as the least solution of a system of containment
constraints imposed by a given set of certificates.  Let $\mathcal{K}$
be the set of principals mentioned in a given collection of name
certificates $\mathcal{C}$. Let $\mathcal{N}_L$ be the set of local
names of the form $K\,A$ where $K \in
\mathcal{K}$, and $A$ is one of the name identifiers mentioned in
$\mathcal{C}$. Let $\mathcal{N}_E$ be the set of extended names of the form
$K\,A_1\,A_2\,\ldots\,A_n,\,n \ge 2$ where $K \in \mathcal{K}$ and the
$A_1, A_2, \ldots, A_n$ are all name identifiers mentioned in
$\mathcal{C}$. Finally let $\mathcal{T} = \mathcal{K} \cup \mathcal{N}_L
\cup \mathcal{N}_E$ be the set of all \newterm{terms} that can be formed
using the principals and name identifiers in $\mathcal{C}$. Then the
semantics of name spaces is defined via the
\emph{valuation function} $\mathcal{V} : \mathcal{T} \rightarrow
\mathcal{P}(\mathcal{K})$  satisfying the equations:
\begin{eqnarray*}
\mathcal{V}(K) &=& \{K\} \qquad \text{for all } K \in \mathcal{K}\\
\mathcal{V}(K A_1 A_2 \ldots A_n) &=&
  \bigcup_{K' \in \mathcal{V}(K A_1)} \mathcal{V}(K' A_2 A_3 \ldots A_n)
\end{eqnarray*}
Furthermore, $\mathcal{V}$ is defined to be the least function 
satisfying the above equalities and the following system of inequalities:
$$
\mathcal{V}(K\,A) \supseteq \mathcal{V}(S)
\qquad (K, A, S, V) \in \mathcal{C}
$$
This is a succinct and intuitive description, well-suited to modeling
the meaning of name certificates.

In contrast to this approach, Abadi developed a logic of SDSI's linked
local names \cite{Abadi:OSLLNS}.  The model theory of the logic plays
a role similar to that of Clarke's semantics, but the proof theory is
technically closer to the certificate rewrite rules proposed in
RFC-2693 \cite{RFC-2693}, allowing characteristics of names such
as associativity to be clarified, and relations between the rewrite rules
and semantic model to be drawn more easily.  In particular, Abadi
shows the rewrite rules are sound with respect to the logic. However,
Abadi's logic does allow conclusions about names to be drawn that
would not be possible in Rivest and Lampson's Scheme.

Abadi uses the notation $n \mapsto v$ to indicate a binding (aka 
\emph{mapping}) of name
$n$ to the value $v$ in the name space of some \newterm{current
principal}.  Intuitively $n \mapsto v$ means ``$v$ speaks-for $n$'':
any statement asserted by $v$ is implicitly a statement
asserted by $n$, a relation similar to $\speaksfor$ in ABLP logic
(\autoref{section-foundations}).  Here values are terms consisting of
local names, public keys aka \emph{global names}, and extended names.
Some bindings are published by the assumed current principal as signed
certificates. For example a binding such as $\mbox{\tt alice} \mapsto
(\mbox{\tt bob}\,\,\mbox{\tt mother})$ maps the current principal's
local name {\tt alice} to the extended name $(\mbox{\tt
bob}\,\,\mbox{\tt mother})$.  Abadi also extends the notion of mapping
so that arbitrary principal expressions can map to other arbitrary
principal expressions. For example $(n_1\,n_2)
\mapsto K$ conveys that the extended name
$n_1\,n_2$ is bound to the key $K$.

Abadi's logic for SDSI names is a modal logic with an obvious debt to
ABLP logic. Formulae of the form $p\,\keyword{says}\,s$ formalize
certificates asserting proposition $s$ that are signed by $p$. The
standard axioms and rules of inference from propositional logic and
modal logic are then extended to include the axioms in
\autoref{figure-AbadiSDSIaxioms}.  Sets of certificates are represented
as logical assumptions, and name-to-key bindings $n \mapsto v$ are
considered valid iff they can be deduced from these assumptions given
the rules of inference.  Abadi shows that the rules of deduction
required to simulate the name resolution algorithm given by Rivest and
Lampson are sound in this setting.

\AbadiSDSIaxiomsfig

However, Abadi's general logic is more powerful than the name resolution
rules allow. Consider the following example given by Abadi where $f_1$,
$f_2$, and $h$ are global names (keys). The assumptions are
\begin{mathpar}
m \mapsto f_1

m \mapsto f_2

f_1\,\,\keyword{says}\,\,(n_1 \mapsto n_2)

f_2\,\,\keyword{says}\,\,(n_2 \mapsto h)
\end{mathpar}
The logic allows one to deduce $(m\,n_1) \mapsto h$ whereas this result
can not be obtained by the name resolution rules. Abadi suggests that such
results may not be harmful. 
%% Killed this, it was confusing - chris
%Yet they are counter the natural intuition for
%SDSI names. Let $K_p$ be the current principal. Then, using notation
%introduced earlier, $\mathcal{V}(K_p\,m) \supseteq
%\{f_1, f_2\}$. Thus $\mathcal{V}(K_p\,m\,n_1) \supseteq
%\mathcal{V}(f_1\,n_1) \cup \mathcal{V}(f_2\,n_1) \supseteq
%\mathcal{V}(f_1\,n_2)$. Using $\mathcal{V}(f_2\,n_2)$ in this
%context would seem to be a violation of the local scope of $f_1\,n_2$.

Halpern and van der Meyden present an alternative to Abadi's logic
\cite{Halpern:LSSLLNS} that attempts to avoid some of the surprising
conclusions in Abadi's logic while maintaining the correspondence
between name resolution and proof.  In the logic of Halpern and van der
Meyden formulae of the form $p \mapsto q$ intuitively express the idea
that all keys bound to $q$ are also bound to $p$. This intuition is
very similar to that expressed by the valuation function $\mathcal{V}$
described in \cite{Clarke:CCDSS}. Halpern and van der Meyden avoid
using ``$q$ speaks for $p$'' as an intuitive explanation for $p
\mapsto q$, regarding such a meaning as one about delegation and thus
outside the scope of their study.  Halpern and van der Meyden
distinguish between general principal expressions and keys,
restricting some of Abadi's axioms to operate only over keys rather
than general principal expressions.  These authors later accounted for
additional features in SPKI/SDSI \cite{Halpern:LRS}, including
authorization certificates and certificate lifetime and revocation
issues. Howell and Kotz also provide a logical accounting of SPKI/SDSI
\cite{Howell:FSS,Howell:NSRAAB} building on Abadi's concepts
for SDSI names but with a restricted speaks-for relation.


\LiSDSIlogicfig

Li provides yet another formulation of the logic of SDSI local names
\cite{Li:LNSS}, but based on general purpose programming logics rather
than special purpose authorization logics. Li regards the handling of
$\mapsto$ in Abadi as too general and observes that even under the
more restricted axioms of Halpern and van der Meyden there are some
undesirable consequences. Instead Li presents the Prolog logic program in
\autoref{figure-LiSDSIlogic} that performs SDSI name resolution. In
this program an extended SDSI name $(a m_1 m_2
\ldots m_k)$ is represented by a list $[a, m_1, m_2, \ldots, m_k]$
\footnote{Note that the Prolog syntax $[X,Y|T]$ represents a 
list whose first and second elements are $X$ and $Y$ respectively,
and $T$ is the rest (the tail) of the list.}.
Name certificates are translated into facts using the
$\mathit{includes}$ predicate, such that $K\,A \rightarrow S$ becomes
$\mathit{includes}([K,A],[S])$.  In related work Li and Mitchell show 
the equivalence of a logic programming semantics for SPKI/SDSI and 
a set-theory semantics in the style of Clarke \cite{Li:USSUFOL}.

Jha and Reps describe a connection between SPKI/SDSI and pushdown systems
\cite{Jha:ASSCUMC}, and show how to use model checking techniques to
compute a proof of authorization. Existing model checking algorithms allow
a variety of other questions to be answered as well, for example given
a resource one might ask what names are able to access that resource.

\subsubsection{Implementation}
\label{section-SDSI-SPKI-implementation}

RFC-2693 \cite{RFC-2693} defines a 4-tuple reduction rule that can be
used to combine two related name certificates into a third
certificate.  This rule involves replacing a local name in one
certificate with a key binding established by another.
So, letting $\circ$ be an infix denotation of the rewrite
operation, we have that:
$$
K_1\,A \rightarrow K_2\,B_1\,B_2\,B_3 \circ K_2\,B_1 \rightarrow K_3
$$
results in:
$$
K_1\,A \rightarrow K_3\,B_2\,B_3
$$
A similar reduction is defined for authorization certificates,
describing how an authorization is explicitly delegated from one
subject to the next. For example:
$$
(K_1, S_1, \text{\tt true}, T_1, V_1) \circ (S_1, S_2, D_2, T_2, V_2)
$$
results in:
$$
(K_1, S_2, D_2, A_I(T_1, T_2), V_I(V_1, V_2))
$$
where $A_I$ computes the intersection of the two authorizations and
$V_I$ computes the intersection of the two validity conditions. Rules
for computing these intersections are given in RFC-2693.  Finally a
similar reduction rule describes how the subject of an authorization
can be rewritten according to bindings specified in a name
certificate. 

Note that RFC-2693 does not give a specific algorithm for finding
which reductions should be used, given a set of certificates for a
particular access request. Instead the requester is required to send
the appropriate certificates in the correct order. This puts the
burden of constructing the proof of authorization on the requester;
the authorizer merely checks this proof. This general concept is
extended in Proof Carrying Authorization which we discuss in more
detail in \autoref{section-other-PCA}.

To relieve the access requester of the burden of proof, Clarke et
al.~describe a credential chain discovery algorithm, that will
automatically check for authorization given a set of certificates and
a particular request \cite{Clarke:CCDSS}. The algorithm uses a graph
construction to search for a particular sequence of reductions to
a certificate delegating the requested permission from the local
name space to the requester. This algorithm runs in $O(n^3
L)$ time (worst case) where $n$ is the number of input certificates
and $L$ is the length of the longest extended name in any of the
certificates.

In both Clarke et al.'s scheme and that proposed by RFC-2693, it is
assumed that all certificates are on hand when the authorization
decision is made.  To our knowledge no method for retrieving
non-local SPKI/SDSI certificates dynamically has been described in the
literature, in other words there are no distributed certificate chain
discovery techniques developed for SPKI/SDSI in the sense 
described in \autoref{section-overview-structure}.

