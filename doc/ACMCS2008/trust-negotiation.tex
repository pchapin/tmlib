\section{Other Components: Trust Negotation}
\label{section-trust-negotiation}

In this survey we have focused specifically on authorization in trust
management.  However, modern trust management systems include other
major components, to address problems other than the semantics of
authorization.  While space considerations prevent a complete review
of issues and approaches, in this section we provide a brief overview
of \newterm{trust negotiation}, which is a topic of considerable interest
in modern trust management research.  Along with providing a more
complete view of trust management, the purpose of this section is to
provide a better practical context for the topics covered in this
survey, and a better picture of current research directions.

When considering the semantics of authorization, it is simplest to
assume that requesters' credentials are publicly available, so that
authorizers have full access to them as well as their own policy.  We
call this the \newterm{basic model}, and it is typically assumed in most
of the systems we survey here for the initial development of an
authorization semantics.  In the basic model authorizers do not
directly disclose their policy to any requester while requesters are
assumed to disclose their credentials freely. Requesters either send
their credentials with each request or, in some cases, make their
credentials available on public servers where authorizers can locate
them using some form of credential chain discovery.

Furthermore, the result of the authorization decision in the basic
model is a simple boolean value specifying if access is allowed or
not. If access is allowed the requester does not know which
credentials were actually necessary to gain that access. If access is
denied the requester does not know why the denial occurred and has no
way of knowing how to obtain missing credentials.

%Treating requester credentials as public information is not necessarily a
%problem. In some cases the information in the credentials might be a matter of
%public record. In other cases a requester might be able to apply heuristics
%external to the trust management system to decide which credentials should be
%sent in support of any particular request. For example Alice, a cancer
%patient, might have a credential signed by her physician certifying her
%medical status. Using this credential Alice can access a cancer support web
%site where she can converse with other patients. However, when Alice accesses
%her employer's web site on work related business she can simply avoid sending
%the credential that proves she's a cancer patient. She knows from the social
%context of her request that this credential will be irrelevant.

However, in practice the basic model is not always sufficient.  For
example, rather than expending overhead on distributed chain
discovery, an authorizer may associate additional information with
credentials that associates them with particular resources, so
requesters can supply a subset of their credentials in an informed
manner.  But a deeper issue is that a requester may regard some of her
credentials as sensitive and have a complex access policy for
them. The requester may require an authorizer comply with that policy
before she is willing to disclose those credentials.  In addition the
authorizer may wish to control access to his resource access policy,
allowing some or all of that policy to be disclosed to suitable
requesters. In this situation, the requester and authorizer can engage
in a process called trust negotiation during which
credentials and policy statements are shared incrementally between the
authorizer and requester as the parties gain increasing trust in each
other.

An example in \cite{seamons-policy02} illustrates the concepts. Alice, a
university student, orders her textbooks from an online bookstore. She
requests a student discount not knowing what credentials the bookstore will
require. In response the bookstore asks to see her digital student ID and her
digital credit card. Alice is willing to disclose her ID but will only
disclose her credit card to web sites that have been certified by the Better
Business Bureau; accordingly she requests this credential from the bookstore
site. Once the bookstore discloses its Better Business Bureau credential,
Alice's access policy on her credit card is satisfied and she discloses her
credentials as well.

Many trust negotiation systems have been described in the literature
\cite{Winslett:NTW,gavriloaie-eswp04,protune-05}. Some, such as
PeerTrust \cite{gavriloaie-eswp04}, are extensions of other trust
management systems, including systems we have surveyed here.  For
example, PeerTrust extends SD3 with trust negotiation, and a trust
negotiation framework has been developed for the RT system
\cite{Winsborough:TPATN}. In other cases, such as with Protune
\cite{protune-05}, support for trust negotiation and trust management
were designed together from scratch. In either case trust negotiation
systems build on trust management concepts and thus have many
overlapping concerns.

In a survey of trust negotiation systems a list of requirements on
trust negotiation policy languages is given in
\cite{seamons-policy02}. Because trust negotiation systems include
trust management functionality that we survey here, many of the these
policy language requirements overlap with or are embedded in our list
of trust management features. For example, Seamons et al.\ note the
importance of using languages with well defined semantics, and discuss
features such as monotonicity, credential chains, delegation depth,
and local name spaces.  However, Seamons et al.\ also require trust
negotiation policy languages to have the power to express access
control information on the policies themselves. This is the essence of
trust negotiation.

In addition \cite{seamons-policy02} gives requirements on compliance checkers.
In a trust negotiation context compliance checkers can no longer return a
simple yes/no result. If the request for access fails, the checker must
provide information about what additional credentials are needed to gain that
access. The authorizer can use this information to request those credentials
from the requester. Furthermore the requester uses a compliance checker to
control access to her credentials, and in some cases to locally process
policy information provided by the authorizer to determine which of her
credentials might be relevant to a particular request.

Some of the earliest work on trust negotiation focused on how the requester
could select precisely the credentials necessary for the desired access and
thus avoid sending sensitive credentials unnecessarily. In
\cite{seamons97internet} \newterm{credential acceptance policies} written in a
restricted form of Prolog are downloaded from the authorizer by the requester
and then executed with the requester's database of credentials as part of the
collection of facts available to the Prolog program. The result is a list of
required credentials that the requester must send to the authorizer to gain
access to the protected resource. The authorizer executes essentially the same
program to check the access request.

Later work generalizes this process by allowing the requester to assign
\newterm{credential access} or \newterm{credential disclosure} policies to
control the conditions under which those credentials can be revealed
\cite{Winsborough:ATN,Yu:PECSATNI}. In addition authorizers might want to
control the disclosure of their access policies as well
\cite{Seamons:LDACPATN}. When a negotiating party sends all the credentials or
policy statements for which the access policy has been met, that party is said
to follow an \newterm{eager} strategy. On the other hand if the negotiating
party only sends a more narrowly focused collection of credentials in response
to specific requests from the negotiating partner, that party is said to
follow a \newterm{parsimonious} strategy. In this later case, however, a
negotiating party can inadvertantly reveal sensitive information about her
credentials indirectly by way of her reactions to specific requests
\cite{Winsborough:ATN,Winsborough:TPATN,Winsborough:SATN}. Fundamentally this
problem arises because credential access policies are only associated with
credentials a negotiating party actually has. To address this issue
\newterm{ack policies} can be created that cause a negotiating party to enter
into a negotiation about attributes she considers sensitive even if she lacks
any specific credentials about those attributes
\cite{Winsborough:TPATN,Winsborough:SATN}.

The PeerTrust system is an example of a fully developed trust
negotation policy system.  PeerTrust extends SD3 to provide trust
negotiation \cite{gavriloaie-eswp04}. The Horn clauses used in SD3 are
enhanced to allow guards. Essentially the body of a clause is broken
into sections where the successful evaluation of one section is
required before the evaluation of the next section is allowed to
begin. An example in \cite{gavriloaie-eswp04} shows a rule Alice might
use to specify that she will only reveal her signed credential that
she is a California state police officer to web sites that are members
of the Better Business Bureau:
\begin{displaymath}
\begin{array}{l}
\textit{policeOfficer}(\textrm{alice})\,\, @\,\, \textrm{CSP}\,\, \leftarrow \\
\hspace{0.5cm} \textit{member}(\textrm{Requester})\,\, @\,\, \textrm{BBB}\,\, @\,\, \textrm{Requester}\,\, |\,\, \textit{signedBy}[\textrm{CSP}].
\end{array}
\end{displaymath}
In this rule the ``@'' is used to indicate a non-local predicate. The
ability to evaluate non-local predicates nests so that, for example,
``\textit{member}(Requester) @ BBB @ Requester'' means that the requester
must show that he is a member of the Better Business Bureau. In this
context the requester is a web server asking Alice if she is a police
officer. This predicate must be validated before the evaluation of the rule
can proceed beyond the ``|'' and Alice can send the necessary signed
certificate to the web server.

For another example, \textsc{Protune} is a particularly rich trust
negotiation system providing support for authentication, policy rules with
actions and side effects (\newterm{provisional} rules), as well as
hierarchical services and credentials
\cite{bonatti-ccs00,Bonatti:UFRAIRW,Jajodia:FSMACP,protune-05}.
Metapolicies are used to arbitrate trust negotiation; as credentials are
gathered a negotiating party's metapolicies activate new policy rules as
appropriate. \textsc{Protune} has also been extended to provide support for
advanced queries allowing users to ask high level questions about
authorization decisions such as, for example, \textit{why?}, \textit{what
if?}, and \textit{how to?} queries \cite{bonatti-ecai06}.

While \textsc{Protune} provides extensive support for trust negotiation, at
its core it also contains support for many of the authorization features we
discuss in this work. \textsc{Protune} libraries can be created that
simulate the semantics of other trust management systems. For example,
\textsc{Protune} can encode the four credential forms of $\RT_0$
\cite{Bonatti:PLS}, thus providing all the capabilities of that system such
as local name spaces, role-based access control, and delegation of rights.
In addition \textsc{Protune}'s features can be used to encode a public key
infrastructure \cite{bonatti-ccs00,Bonatti:UFRAIRW} and distributed chain
discovery \cite{Bonatti:PLS}. The last column in Table~\ref{tbl:tmsummary2}
compares \textsc{Protune} with the other systems we review in terms of its
authorization semantics.

%In their earlier survey on trust negotiation policy language, Seamons
%et al.\ consider some other lower level features that are outside the
%scope of this survey, in particular \newterm{inter-credential
%constraints}, \newterm{external functions}, and authentication
%issues. Inter-credential constraints allow the policy author to
%specify relationships between credentials that are not directly
%expressed by the credentials themselves.  External functions can be
%used, for example, to get the current time, compare dates, or to
%otherwise express complex constraints on the values of attributes
%stored in a credential. Authentication issues are relevant to 
%the issue of certificate validity, which is another relevant 
%component of complete trust management systems.
%
% Here is a short section on authentication and validity issues.
%
%\subsection{Certificate Validity}
%
%Previous work discussed the importance of supporting authentication policies
%in trust negotiation policy languages \cite{seamons-policy02}. It is not
%enough, in general, to prove that a particular key has access to a resource.
%The requester must also show that she controls that key by authenticating to
%the authorizer in some way. The trust management systems we review typically
%assume that the requester signs her requests with her key to prove she is the
%key owner. Alternative authentication mechanisms are usually not considered.
%
%Verifying the authenticity of a certificate is just one aspect of a more
%general issue of certificate validity. To be valid a certificate must not be
%revoked, for example due to key compromise, and the cryptographic signature
%algorithm must not have been broken. In addition, even if a certificate is
%still valid in a cryptographic sense, the information contained in that
%certificate may be stale. For this reason certificate authorities usually
%specify a \newterm{validity interval} for each certificate during which the
%authority vouches for the binding in that certificate. Although some
%attributes, such as membership in an organization, have clearly defined
%validity intervals other attributes, such as a principal's income status, can
%change unexpectedly. In this later case it is possible the binding in a
%certificate will be incorrect during some part of the certificate's validity
%interval.
%
%Often the authorizer stands the most to loose by accepting a certificate
%containing stale data. It thus makes sense in such a case for the authorizer
%(instead of the certificate authority) to define the validity interval based
%on when the certificate was issued and on the authorizer's tolerance for
%accidently using an inaccurate information \cite{Rivest:CWECRL}. In other
%cases it is the certificate authority who stands the most to loose, for
%example in terms of his reputation as an authority, if stale certificates
%signed by him remain in wide circulation \cite{McDaniel:RTCWECRL}. In practice
%it is likely that authorizer will apply his own validity interval based on,
%and intersecting with, the interval specified by the certificate authority.
%Alternatively the authorizer can choose to encode the degree of certificate
%staleness as a risk value attached to the certificate. The authorization
%decision can then be made taking into account these risks
%\cite{skalka-wang-chapin-jcs06}.

