\section{$\RTR$ Credential Chain Discovery}
\label{section-chain-discovery}
\label{section-rtr-discovery}

In this section we discuss an algorithm for authorization with risk in
a distributed environment.  Following $\RT$ credential chain discovery
\cite{Li:2003-02}, our technique is to characterize credential sets
graph-theoretically, except that our credential graphs are
risk-weighted multigraphs, to accommodate risk assessments.
Credential graphs are shown to be a full abstraction of solutions as
in \autoref{def-solution}, and the $\RTR$ discovery algorithm is shown
to correctly reconstruct credential graphs.

In addition to theoretical correctness, our chain discovery algorithm
has two important practical features:
\begin{enumerate}
  \item The algorithm need not verify a role membership in a
  risk-optimal fashion, but rather is parameterized by 
  a risk threshold, that is a maximum tolerable risk for role 
  membership verification.
 \item The discovery procedure is \emph{directed}, in the sense that
it is aborted along search paths whose risk overruns the maximum
threshold.
\end{enumerate}
The first feature allows end-users to modulate tolerable levels of
risk in authorization.  The second feature reaps any efficiency
benefits intended by associating risks with credentials, as high risk
may be associated with high expense, e.g.~if risks are wait times.


\subsection{Credential Graphs}
\label{section-credential-graphs}

We begin by defining an interpretation of credential sets $\creds$ as
a credential graph.  More precisely, sets of credentials are
interpreted as a weighted multigraph, where nodes are role
expressions, edges are credentials, and weights are risks.
Authorization is implemented by determining reachability, via risk
weighted paths, where the aggregation of edge risk along the path is
the risk of authorization.  Reachability is predicated on simple
paths, since traversing cycles can only increase risk, and any path
with a cycle would otherwise generate an infinite number of risk
weighted paths.  Allowing the latter would preclude a constructive
definition of credential graphs, since chains are distinguished by
risk and cycle traversal increases risk monotonically.
\begin{definition}[Risk weighted credential chains]
Let $\graph{} = (\nodes{}, \wtedges{})$ be a weighted multigraph with
nodes $f \in \nodes{}$ and edges $\wtedge{f_1}{f_2}{\risk} \in
\wtedges{}$ weighted by elements $\risk$ of a given risk ordering.
The pair:
$$((f_1,\ldots,f_n), \risk_1 \oplus \cdots \oplus \risk_{n-1})$$ is a
\emph{risk weighted path in $\graph{}$} iff for all $i \in [1..n-1]$,
there exists $\wtedge{f_i}{f_{i+1}}{\risk_i} \in \wtedges{}$.  A
weighted path $((f_1,\ldots,f_n), \risk)$ is \emph{simple} iff no node
is repeated in $(f_1,\ldots,f_n)$.  We write $\wtpath{f}{f'}{\risk}$,
pronounced ``there exists a credential chain from from $f$ to $f'$
with risk $\risk$'', iff $((f,\ldots,f'), \risk)$ is a simple risk
weighted path.  We write $\wtpath{f}{f'}{\risk} \in \graph{}$ iff
$\wtpath{f}{f'}{\risk}$ holds given $\graph{}$.
\end{definition}

The definition of credential graphs is founded on the definition of
risk weighted chains, since edges derived from linked and intersection
credentials are supported by them.  
\begin{definition}[Credential graph]
\label{def-credentialgraph}
Given $\creds$, its \emph{credential graph} is a weighted multigraph
$\graph{\creds} = (\nodes{\creds}, \wtedges{\creds})$, where:
$$
\nodes{\creds} = \bigcup\limits_{\cred{A.r}{e}{\risk}} \setdefn{A.r,e}
$$
And $\wtedges{\creds}$ is the least set of risk-weighted edges satisfying 
the following closure properties:
\begin{enumerate}
\item If $\cred{A.r}{e}{\risk} \in \creds$ then 
$\wtedge{e}{A.r}{\risk} \in \wtedges{\creds}$.
\item If $B.r_2, A.r_1.r_2 \in \nodes{\creds}$ and
$\wtpath{B}{A.r_1}{\risk}$, then
$\wtedge{B.r_2}{A.r_1.r_2}{\risk} \in \wtedges{\creds}$.
\item If $D,f_1\cap \cdots \cap f_n \in \nodes{\creds}$ and for each
$i \in [1..n]$ there exists 
$\wtpath{D}{f_i}{\risk_i}$, then $\wtedge{D}{f_1 \cap \cdots \cap
f_n}{\risk} \in \wtedges{\creds}$, where $\risk = \risk_1 \oplus
\cdots \oplus \risk_n$.
\end{enumerate}
\end{definition}

The definition of credential graphs can be made constructive by
iterating closure over an initial initial edge set
$\wtedges{\creds}^0$:
$$
\wtedges{\creds}^0 = \setdefn{\wtedge{A.r}{e}{\risk} \mid 
\cred{A.r}{e}{\risk} \in \creds}
$$ 
In rules (2) and (3), the paths predicating membership in
$\wtedges{\creds}$ are called \emph{support paths}, and the edges are
called \emph{derived}.  On each iteration, add a new weighted edge
according to closure rule (2) or (3).  Since $\creds$ is finite, and
support paths must be simple, the process will reach a fixpoint in
a finite number of iterations; this fixpoint is $\wtedges{\creds}$.

We observe that the characterization of credential sets $\creds$
is sound and complete with respect to the set theoretic semantics
given in the previous section.  These results will form a 
bridge with the semantics of $\RTR$ for establishing correctness
of credential chain discovery.  The statement of soundness reflects
the fact that while risk assessments of credential sets express
minimum risk bounds of role membership, the credential graph 
does not preclude reachability via paths of higher risk.
\begin{theorem}[Soundness]
For all $B,A.r$, if $\wtpath{B}{A.r}{\risk} \in \graph{\creds}$, then
$(B,\risk') \in \credsmean{\creds}(A.r)$ with $\risk' \po \risk$.
\end{theorem}
The statement of completeness reflects that any assessed risk is the
weight of some related path in the graph:
\begin{theorem}[Completeness]
For all $A.r$, if $(B,\risk) \in \credsmean{\creds}(A.r)$, then 
$\wtpath{B}{A.r}{\risk} \in \graph{\creds}$.
\end{theorem}

\subsection{Backward Chain Discovery Algorithm}

As discussed in \cite{Li:2003-02}, any role $A.r$ is defined by its
credentials.  In centralized chain discovery, all credentials are
maintained locally by assumption.  In distributed chain discovery,
some credentials may be stored remotely.  \emph{Backwards} chain
discovery assumes that the credentials defining a role $A.r$ are
obtained through the entity $A$, so that chains need to be
reconstructed ``backwards'', beginning with the governing role of an
authorization decision.  We now define a backwards credential chain
discovery algorithm $\checkmem$ for $\RTR$, possessing features
described at the beginning of \autoref{section-rtr-discovery}.  We
abstract the details of credential retrieval and risk assignment,
other than its ``backwards'' nature, assuming that remote
risk-weighted credentials can always be retrieved on demand (and
cached, presumably).  While forwards and mixed discovery techniques
for $\RT$ are also discussed in \cite{Li:2003-02}, analogous
techniques for $\RTR$ are beyond the scope of this paper.  For brevity
and clarity in the presentation, we textually describe the algorithm
$\checkmem$.

\subsubsection{Definition of $\checkmem$}

\newcommand{\threshold}{\risk_{\mathrm{max}}}

The algorithm $\checkmem(A,f,\threshold)$ reconstructs a proof graph,
to check membership of $A$ in role $f$ within a given threshold
$\threshold$.  The algorithm maintains the following mutable
datastructures: a solution of type $\mathit{RoleExpressions}
\rightarrow \mathit{RiskAssessment}$, an association of \emph{solution
monitors} with graph nodes, discussed below, and an association of
sets of \emph{search risks} with graph nodes.  Search risks are the
accumulated risks along any discovery path to the given node; it is
important to note that search risk associations are different than
risk assessments.  When a node is first encountered during search, it
is added to the queue for future search.  No node is added to the
queue more than once.

Initially, the solution is a default mapping to the empty risk
assessment, and every node is associated with an empty set of solution
monitors and search risks.  To begin the search, the node $f$ is added
to the queue, and associated with the search risk $\bot$.

Nodes are taken from the queue individually for searching, but not
indiscriminately; rather, only nodes that have a search risk below the
threshold $\risk$ are searched.  In this way, the algorithm
short-circuits discovery along paths that are too risky. 
Over-threshold nodes are not removed from the queue, since
future discovery might find a less risky path to that node.  Hence,
nodes wait in the queue until they are the next below-threshold node
to be searched.  The algorithm runs until there are no below-threshold
nodes left in the queue, or until a solution for $A$ in $f$ 
below threshold $\threshold$ is found.
  
Solution monitors propagate solution elements $(A,\risk)$ forward
along discovered edges, aggregating edge risks as they go; their
control flow structure mimics the discovered graph structure.
Whenever a monitor notifies a node $f$ to add a solution element
$(A,\risk)$, if there does not exist $\risk' \po \risk$ such that
$(A,\risk')$ is already in $f$'s solution (in which case we
say it is \emph{canonically new}), the solution is added to $f$, 
and all of $f$'s solution
monitors are applied to the new solution.  There are three classes of
solution monitors:
\begin{enumerate}
  \item A \emph{role monitor} for a given role $A.r$ and 
edge risk $\risk$ is a function abstracted on solution 
elements $(B,\risk')$, that notifies $A.r$ to add
$(B,\risk' \oplus \risk)$ to its solutions.
  \item A \emph{linking monitor} for a given linked role $A.r_1.r_2$
is a function abstracted on solution elements $(B,\risk)$, that
creates a role monitor for $A.r_1.r_2$ and $\risk$, applies
it to each known element of $B.r_2$'s solution, and adds it to
$B.r_2$'s solution monitors to propagate solutions yet to be
discovered.  Finally, given all search risks $\risk'$ of $A.r_1.r_2$,
$\risk' \oplus \risk$ is added to $B.r_2$'s search risks, and $B.r_2$
is added to the queue if it hasn't already been.
  \item An \emph{intersection monitor} for a given intersection role
$f_1 \cap \cdots \cap f_n$ is a function abstracted on solution
elements $(B,\risk)$, that creates a role monitor for $f_1 \cap \cdots
\cap f_n$ and $\bot$, and applies it to each element 
$(B,\risk')$ in the canonical form of the assessment $R_1 \oplus
\cdots \oplus R_n$, where each $R_i$ is the assessment of $f_i$ in the
current solution.
\end{enumerate}

Whenever nodes are taken from the queue according to the above
described discipline, they are processed depending on their
form:
\begin{enumerate}
  \item To process an entity $A$, the node $A$ is notified to 
add $(A,\bot)$ as a solution to itself.
  \item To process a role $A.r$, the credentials defining $A.r$ are
retrieved.  For each such credential $\cred{A.r}{f}{\risk}$, a role
monitor for $A.r$ and $\risk$ is created, is applied to all of $f$'s
known solutions, and is added to $f$'s solution monitors for
propagating solutions still to be discovered.  Finally, given all 
search risks $\risk'$ of $A.r$, $\risk' \oplus \risk$ is added to $f$'s
search risks, and $f$ is added to the queue if it hasn't already been.
  \item To process a linked role $A.r_1.r_2$, a linking monitor for
$A.r_1.r_2$ is created, is applied to all of $A.r_1$'s known
solutions, and is added to $A.r_1$'s solution monitors.  Every
search risk $\risk$ of $A.r_1.r_2$ is added to
$A.r_1$'s search risks, and $A.r_1$ is added to the queue if it hasn't
already been.
  \item To process an intersection role $f_1 \cap \cdots \cap f_n$, an
intersection monitor for $f_1 \cap \cdots \cap f_n$ is created, and
added to each $f_i$.  Every search risk $\risk$ of $f_1 \cap \cdots
\cap f_n$ is added to each $f_i$'s search risks, and
each $f_i$ is added to the queue if it hasn't already been.
\end{enumerate}
When node processing for an invocation $\checkmem(A,f,\threshold)$
halts, the algorithm returns true iff there exists $(A,\risk)$ in the
solution of $f$ such that $\risk \po \threshold$.

\subsubsection{Properties}

Assuming that defining credentials can always be obtained for any
role, we assert that $\checkmem$ satisfies the following properties,
demonstrating that it correctly reconstructs credential graphs.  Since
credential graphs are full abstractions of the $\RTR$ semantics as
discussed in \autoref{section-credential-graphs}, these results
demonstrate that $\checkmem$ is a correct implementation of $\RTR$.
\begin{theorem}[Soundness]
$\checkmem(A,f,\threshold)$ implies 
$\wtpath{A}{f}{\risk}$ such that $\risk \po \threshold$.
\end{theorem}

\begin{theorem}[Completeness]
$\wtpath{A}{f}{\risk}$ implies that $\checkmem(A,f,\risk)$ holds.
\end{theorem}

We also observe that the algorithm terminates, regardless of the given
risk threshold.  This is because nodes are never visited more than
once, and solution monitors will not traverse any graph cycle, and
hence are guaranteed to terminate.  Solution monitors only propagate
canonically new members, but traversal of a cycle necessarily causes a
monotonic increase in a solution's risk assessment, hence canonical
containment in an existing solution.
\begin{theorem}[Termination]
For all $A$, $f$, and $k$, $\checkmem(A,f,\risk)$ terminates.
\end{theorem}

\subsubsection{Discussion} There are two particular instances where 
the definition of $\checkmem$ could be enhanced, for more eager
short-circuiting of chain discovery in case risk thresholds are
exceeded along discovery paths. First, observe that credentials are
retrieved before being checked to see if their risks will force the
discovery threshold to be exceeded.  However, risks such as expected
wait time suggest that it is more practical for credentials to be
retrieved after ensuring they won't overrun the threshold. A number
of minor variations on $\checkmem$ can be imagined that will 
address this.

A more interesting enhancement is relevant to the propagation of
search risks along discovery paths leading from intersection nodes.
Observe that from any intersection role $f_1 \cap \cdots \cap f_n$,
the search risks of $f_1 \cap \cdots \cap f_n$ are propagated to each
$f_i$.  However, this could be a under-approximation of search risks
for any given $f_i$.  For example, suppose that $A$ is being checked
for authorization and $(A,\risk)$ is known to be the only possible 
assessment of $A$ in $f_1$'s solution.  When checking $f_n$, the
search risks of $f_n$ inherited from $f_1 \cap \cdots \cap f_n$
could be aggregated with $\risk$, since $\risk$ is certain to 
be a component risk of any authorization supported by
discovery from $f_n$.  A generalization of this idea is beyond the 
scope of this paper, but is an interesting topic for future work.
