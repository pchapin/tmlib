\section{The System $\RTR$}
\label{section-rtr}

The system $\RTR$ is $\RT_0$ extended with a formal definition of risk
assessments.  In this section we define the syntax and semantics of
$\RTR$, and give some examples of risk-assessed authorization
decisions in $\RTR$.  As for $\RT$ in \cite{Li:2003-02}, we define a
set theoretic semantics for $\RTR$, since this allows an easy
correspondence with the graph theoretic characterization of $\RTR$ for
distributed chain discovery given in the next section.  While a
constraint datalog semantics for $\RTR$-- similar to the datalog semantics of
$\RT$ in \cite{Li:2003-01}-- is an interesting possibility, it is
beyond the scope of this paper.

\rmemfig

\subsection{Syntax and Semantics}

The system $\RTR$ is defined as a framework, parameterized by a
\emph{risk ordering}, which is required to be a complete lattice
$(\mathcal{K},\po)$.  We let $\risk$ and $\risks$ range over elements
and subsets of $\mathcal{K}$ respectively, and let $\top$ and $\bot$
denote top and bottom.  Any instantiation of $\RTR$ is expected to
supply a risk ordering with $\po$ decidable, and must also supply an
associative, commutative, monotonic \emph{risk aggregation} operation
$\oplus$.
%, with $\bot$ a 0-element of $\oplus$-- i.e. 
%$\bot \oplus \risk = \risk$.  
The relation $\po$ allows risks to be compared, and ``greater'' and
``lesser'' risks assessed, while $\oplus$ allows risks to be
combined when authorization decisions involve multiple risks.
Examples of particular risk orderings are given in
\autoref{section-rtrexamples} and \autoref{section-application},
below.

The basis of risk assessment is the association of risk with
individual credentials, since credentials are the fundamental assertions
used in authorization decisions.  Thus, credentials in $\RTR$ are of the
following form:
%\footnote{In this definition and elsewhere we assume the syntactic
%definitions given in \cite{Li:2003-02}}:
$$
\cred{A.r}{f}{\risk}
$$ where $\risk$ is the risk associated with the credential.  We leave
unspecified the precise mechanism of risk association, though in many
cases it is likely that the authorizing agent will automatically
assign risk to credentials.  In essence, the aggregation of risks
associated with credentials used in some authorization decision
constitutes the risk of that decision.

Formally, the semantics of $\RTR$ associates risk $\risk$ with the
membership of entities $B$ in roles $A.r$.  Thus, the meaning of roles
$A.r$ are finite sets of pairs of the form $(B,\risk)$, called
$\mathit{RiskAssessment}$s; we let $R$ range over such sets.  For any
$\mathcal{A}\subseteq \mathit{Entities}$,
$\mathit{RiskAssessment}(\mathcal{A})$ denotes the set of risk
assessments $R$ such that $(A,\risk) \in R$ implies $A \in
\mathcal{A}$.  Note that any $R$ may associate more than one risk with
any entity, i.e.~there may exist $(A,\risk_1),(A,\risk_2) \in R$ such
that $\risk_1 \ne \risk_2$.  This reflects the possibility of more
that one path to role membership, each associated with incomparable
risk.  Taking the glb of incomparable risks in risk assessments is
unsound, since the glb will assess a lesser risk of membership than is
in fact possible to obtain through any path.

However, if a risk assessment associates two distinct but comparable
risks with a given role membership, the lesser of the two can be taken
as representative; in general, risk assessments can be taken as a set
of lower bound constraints on risk in authorization.  Thus, we define
equivalence on risk assessments as follows:
$$
R \cup \setdefn{(A,\risk_1),(A,\risk_2)} = R \cup \setdefn{(A,\risk_1)} 
\qquad \text{where } \risk_1 \po \risk_2 
$$ 
We call \emph{canonical} those risk assessments $R$ such that there
exist no $(A,\risk_1), (A,\risk_2) \in R$ where $\risk_1 \po \risk_2$,
and observe that any equivalence class of risk assessments has 
a unique canonical form.  Furthermore, the canonical representation 
of any assessment $R$, denoted $\hat{R}$, is decidable since
assessments are finite and $\po$ is decidable.   We extend the 
ordering $\po$ to risk assessments as follows:
\begin{eqnarray*}
R_1 \po R_2 &\iff& \forall (A,\risk_1) \in \hat{R_1} . \exists \risk_2. 
(A,\risk_2) \in \hat{R_2} \wedge \risk_1 \po \risk_2
\end{eqnarray*}
The relation is clearly decidable.  We also observe that it is
is a partial order:
\begin{corollary}
The relation $\po$ on risk assessments is a partial order.
\end{corollary} 
Hereafter we restrict our consideration to canonical risk assessments
without loss of generality.  We immediately observe the following,
which will be useful in the development of a formal semantics for 
$\RTR$:
\begin{lemma}
\label{lemma-assessmentlattice}
For all finite $\mathcal{A} \subset \mathit{Entities}$, the poset:
$$(\mathit{RiskAssessment}(\mathcal{A}),\po)$$
is a complete lattice. 
\end{lemma}
\begin{proof}
Given $\mathcal{R} \subseteq \mathit{RiskAssessment}(\mathcal{A})$.
For each $A \in \mathcal{A}$, let $\risks_A =
\setdefn{\risk\ |\ \exists R \in \mathcal{R} . (A,\risk) \in R}$, 
and let $\risk_A$ be the lub of $\risks_A$, which must exist since 
we require risk orderings to be complete lattices.  Let
$R_{\mathcal{R}} = \setdefn{(A,\risk_A) \mid A \in \mathcal{A}}$.
Clearly $R_{\mathcal{R}}$  is an element of 
$\mathit{RiskAssessment}(\mathcal{A})$,
and is a lub of $\mathcal{R}$.  The existence of a glb for 
$\mathcal{R}$ follows dually.
\end{proof}
A notion of aggregation of risk assessments 
is useful to define:
\begin{eqnarray*}
R \oplus \risk &\ \defeq\ & \setdefn{(A,\risk' \oplus \risk) \ \mid\ (A,\risk') \in R}\\
R_1 \oplus R_2 &\ \defeq\ &
\setdefn{(A,\risk_1 \oplus \risk_2) \ \mid\ 
(A,\risk_1), (A,\risk_2) \in R_1 \times R_2} 
\end{eqnarray*}
We assert monotonicity of this operation:
\begin{corollary}
\label{cor-oplusmonotonic}
The operation $\oplus$ on risk assessments is monotonic.
\end{corollary}

As we will see below, solutions to sets of credentials are 
functions of type:
$$
\mathit{Role} \rightarrow \mathit{RiskAssessment}
$$ 
Letting $f$ and $g$ be functions of this type, we define:
$$
f \po g \quad\iff\quad f(A.r) \po g(A.r) \text{ for all roles } A.r
$$
%\begin{corollary}
%\label{cor-assessmentlattice}
%The poset $(\mathit{Role} \rightarrow \mathit{RiskAssessment}, \po)$ is
%a complete lattice.
%\end{corollary}
Now, we can define the semantics of $\RTR$, by extending the
semantics of $\RT_0$ in \cite{Li:2003-02} to assess risk:
\begin{definition}[Semantics of $\RTR$] 
\label{def-solution}
Given a set $\creds$ of $\RTR$ credentials, the semantics
$\credsmean{\creds}$ of $\creds$ is a function mapping role expressions to risk
assessments.  In particular, $\credsmean{\creds}$ is the least function
$\rmem : \mathit{Role} \rightarrow \mathit{RiskAssessment}$ 
(ordered by $\po$ as above) such that $\bounds[\rmem] \po \rmem$,
where $\bounds[\rmem]$ and the auxiliary function $\expr[\rmem]$, mapping
role expressions to risk assessments, are defined as in \autoref{figure-rmem}.
\end{definition}

Given any $\creds$, we can construct $\credsmean{\creds}$ by a least
fixpoint argument, showing that any set of credentials has a solution.
The technique follows \cite{Li:2003-02}.  The solution is constructed
as the limit of the sequence $\setdefn{\rmem_i}_{i \in \mathbb{N}}$,
where $\rmem_i : \mathit{Roles}(\creds) \rightarrow 
\mathit{RiskAssessment}(\mathit{Entities}(\creds))$
for every $i$.  The sequence is defined inductively by taking 
$\rmem_0(A.r) = \varnothing$ for every role $A.r$, and letting:
$$
\rmem_{i+1}(A.r) = \bounds[\rmem_i](A.r)
$$ 
for every $A.r$.  The function relating the values in
$\setdefn{\rmem_i}_{i \in \mathbb{N}}$ is monotonic, since $\cup$ and
$\oplus$ are monotonic, the latter by definition and
\autoref{cor-oplusmonotonic}.  Further, the pointwise ordering of
functions $\rmem_i$ under $\po$ forms a complete lattice, by
\autoref{lemma-assessmentlattice}.  Therefore, a least fixpoint of the
sequence $\setdefn{\rmem_i}_{i \in \mathbb{N}}$ exists.  Let 
$\rmem_\omega$ be this fixpoint, and define:
$$
\begin{array}{rclr}
\credsmean{\creds}(A.r) &=& \rmem_\omega(A.r) \qquad &  A.r \in \mathit{Roles}(\creds) \\
\credsmean{\creds}(A.r) &=& \varnothing & A.r \not\in \mathit{Roles}(\creds)
\end{array}
$$
It is easily shown that $\credsmean{\creds}(A.r)$ so defined is a
least solution to $\creds$ as specified in \autoref{def-solution}.

\subsection{Examples}
\label{section-rtrexamples}

We now give some examples of risk assessments for authorizations in
two different risk models, illustrating applications of the system.
Other more complex examples are discussed in
\autoref{section-application}.

\subsubsection{Bound-of-Risks}  

In \cite{Den:latt}, an information flow security model is presented
where all static data is assigned to a security class.  Security
classifications of variables are then assigned based on the
combination of security classes of data flowing into those variables,
as determined by an abstract program interpretation.  Security classes
are identified by elements in a complete lattice, where
``class-combination'' is defined as the lub of combined classes.

We propose that an adaptation of this model is useful in the context
of authorization risk assessment.  We do not propose an abstract
interpretation of authorization, incorporating some form of
``may-analysis'', but rather a purely dynamic authorization and risk
assessment model, so in this sense we differ from the model proposed
in \cite{Den:latt}.  Nevertheless, we may adopt the use of least upper
bounds as a ``class-combination'' mechanism-- in our terminology,
``risk aggregation''-- that assesses the risk of any authorization
decision as the least upper bound of risks associated with all
credentials used in the decision.

Consider a risk ordering where three classifications $\mathcal{K} =
\setdefn{\mathit{low,medium,high}}$ are defined, and the following
relations are imposed:
$$
\mathit{low} \po \mathit{medium} \po \mathit{high}
$$ 
and $\oplus$ is taken to be the lub operator.  Imagine also that an
online vendor called $\mathit{Store}$ maintains a purchasing policy
whereby representatives of the $\mathit{Acme}$ corporation have
$\mathit{buyer}$ power only if they are both employees and official
purchasers.  Since this policy is maintained locally, it is associated
with a $\mathit{low}$ risk of usage, hence $\mathit{Store}$ could
specify:
\begin{mathpar}
\mathitcred{Store.buyer}{Acme.purchaser \cap Acme.employee}{low}
\end{mathpar}
Imagine further that $\mathit{Ed}$ attempts to make a purchase from
$\mathit{Store}$, providing certificates claiming $\mathit{employee}$
and $\mathit{purchaser}$ status.  However, if we assume that these
certificates can possibly be faked, or that role membership within the
$\mathit{Acme}$ corporation has a volatile status, higher risk can be
assigned to these certificates:
\begin{mathpar}
\mathitcred{Acme.employee}{Ed}{medium}

\mathitcred{Acme.purchaser}{Ed}{high}
\end{mathpar}
We also assume that a less risky path of establishing
$\mathit{Ed}$'s membership in the $\mathit{Acme.purchaser}$ role is
through a $\mathit{manager}$ certificate obtained directly from
$\mathit{Personnel}$, and via $\mathit{Acme}$'s own policy specifying
$\mathit{purchaser}$ power for all $\mathit{manager}$s:
\begin{mathpar}
\mathitcred{Acme.purchaser}{Personnel.manager}{low}

\mathitcred{Personnel.manager}{Ed}{low}
\end{mathpar}
Although using $\mathit{Ed}$'s certificate asserting his membership in
the $\mathit{Acme.purchaser}$ role will incur a $\mathit{high}$ risk,
because of the less risky path to this relation, the risk assessment of
this set of credentials will find that establishing $\mathit{Ed}$'s
membership in the $\mathit{Store.buyer}$ role requires a lower bound of
$\mathit{medium}$ risk.  The least solution for all given roles
is as follows:
\begin{eqnarray*}
\mathit{Store.buyer}&:& \setdefn{(\mathit{Ed, medium})}\\
\mathit{Acme.employee}&:& \setdefn{(\mathit{Ed, medium})}\\
\mathit{Acme.purchaser}&:& \setdefn{(\mathit{Ed, low})}\\
\mathit{Personnel.manager}&:& \setdefn{(\mathit{Ed, low})}
\end{eqnarray*}
Of course, in certain cases it may be preferable to use the
certificate $\mathit{Ed}$ provides, instead of going through
$\mathit{Personnel}$-- if wait times for distributed communication
with that node are prohibitively long, for example.  However, in this
case it should be specified that a $\mathit{high}$ level of risk will
be tolerated in the credential chain.  In
\autoref{section-chain-discovery} and \autoref{section-application},
we define a technique for credential chain discovery that implements
this idea.

Returning to the example, for the purposes of illustration we imagine
that the risk ordering is extended with an element $\mathit{moderate}$,
that is incomparable with $\mathit{medium}$, inducing the lattice:
\begin{center}
\risklattice
\end{center}
We also imagine that $\mathit{Store}$ has cached an old certificate,
establishing $\mathit{Ed}$'s membership in the $\mathit{Acme.employee}$
role with $\mathit{moderate}$ risk:
\begin{mathpar}
\mathitcred{Acme.employee}{Ed}{moderate}
\end{mathpar}
In this case, since $\mathit{moderate}$ and $\mathit{medium}$ are 
incomparable, the risk assessment will reflect that $\mathit{Ed}$'s
membership in the $\mathit{Store.buyer}$ and $\mathit{Acme.employee}$
roles can be established via two paths with incomparable risk:
\begin{eqnarray*}
\mathit{Store.buyer}&:& \setdefn{(\mathit{Ed, medium}),(\mathit{Ed, moderate})}\\
\mathit{Acme.employee}&:& \setdefn{(\mathit{Ed, medium}),(\mathit{Ed, moderate})}
\end{eqnarray*}
Precision and safety in the assessment of minimal risk is not lost by
taking the glb of incomparable risk assessments.

\subsubsection{Sum-of-Risks}
\label{sec:sum-of-risks}

An alternative to the bound-of-risks model is a sum-of-risks model,
where credentials are assigned numeric risk values and the total risk
for any authorization decision is the sum of all risks associated with
the credentials used in the decision.  Thus, we take the risk ordering
in this model to be the lattice of natural numbers up to $\omega$
induced by $\leq$, and we take $\oplus$ to be addition.  This model is
useful in case risk is considered additive, or in case the number of
credentials used in an authorization decision is an element of risk,
the more the riskier.

Imagining a similar situation as above, the following risks could 
be assigned, where 1 is considered ``not risky'' and 4 is considered
``risky'':
\begin{mathpar}
\mathitcred{Store.buyer}{Acme.purchaser \cap Acme.employee}{1}

\mathitcred{Acme.employee}{Ed}{3}

\mathitcred{Acme.purchaser}{Ed}{4}

\mathitcred{Acme.purchaser}{Personnel.manager}{2}

\mathitcred{Personnel.manager}{Ed}{3}
\end{mathpar}
Note that $\mathit{Ed}$'s certificate claiming membership in the role
$\mathit{Acme.purchaser}$ is still assigned higher risk than both
the certificate establishing his $\mathit{manager}$ status and the
certificate establishing $\mathit{purchaser}$ rights for
$\mathit{manager}$s.  However, the sum-of-risks model will still
ascertain that the use of $\mathit{Ed}$'s certificate will be the least
risky way to establish his membership in the $\mathit{Store.buyer}$
role.  The solution of the given credentials will comprise the following
risk assessments:
\begin{eqnarray*}
\mathit{Store.buyer}&:& \setdefn{(\mathit{Ed, 8})}\\
\mathit{Acme.employee}&:& \setdefn{(\mathit{Ed, 3})}\\
\mathit{Acme.purchaser}&:& \setdefn{(\mathit{Ed, 4})}\\
\mathit{Personnel.manager}&:& \setdefn{(\mathit{Ed, 3})}
\end{eqnarray*}
If a pure count of credentials used in authorization is the 
basis of risk assessment, this model can be formally obtained 
in the sum-of-risks model by associating risk 1 with every 
credential.
