\section{Applications}
\label{section-application} 

In this section we discuss interesting applications of $\RTR$.
Details of these applications are avenues for future work; here we
describe how $\RTR$ could be used to support trust management systems
that incorporate notions of risk.

\subsection{Trust but Verify}

The Trust but Verify (TbV) framework \cite{skalka-wang-sws04} provides
a setting for distributed trust management that takes into account a
notion of \emph{trust} for online authorization decisions, backed up
by offline \emph{verification}.  Many realistic authorization
decisions require ``softening'' of security in the online phase; this
amounts to trusting the validity of certain assertions in this phase,
that would otherwise be too expensive to verify.  However, online
trust should be specified so that sound offline verification is
well-defined, providing formal certainty that offline verification
supports online trust.

Any authorization decision in the TbV framework is abstractly
specified as derivability of a target privilege $\priv$ given a
security context $s$, written $s\vdash\priv$.  Any instance of the TbV
framework comprises a \emph{trust transformation}, that formalizes the
definition of trust in terms of a function, mapping initial security
contexts $s$ to contexts $\cod{s}$, that contain assertions that are
trusted solely for efficient online verification.  Furthermore, the
trust transformation should be reversible, via an $\audit$ technique
that is required to reconstruct a security context that is at least as
strong as the pre-image $s$ of any trust-transformed security context
$\cod{s}$.  The $\audit$ technique is the implementation of offline
verification.  In \cite{skalka-wang-sws04}, the TbV framework is
developed using ABLP logic \cite{ABLP93}.  However, the $\RT$
framework is a more modern trust management system, with a variety of
implementation techniques and variations \cite{Li:2003-02}.  The
$\RTR$ variation offers a unique dimension of support for TbV, since
trust can be encoded using definitions of risk in $\RTR$.

The TbV framework is characterized by three conditions, that we
recount here.  We show how $\RTR$ can be used to instantiate the
framework in a system that satisfies these conditions.  The first
condition requires that authorization decisions are decidable:
\begin{condition}
\label{cond-decidable}
Let $s$ be an authorization context; then validity of $s \vdash \priv$
is decidable.
\end{condition}
In $\RTR$, authorization decisions are implemented as role membership
decisions with an assessed risk, and security contexts are sets of
credentials $\creds$.  That is, if the role $A.r$ represents a target
privilege and $B$ is a privilege requester, then authorization amounts
to discovery of $\wtpath{B}{A.r}{\risk} \in \graph{\creds}$,
where $\risk$ must be within a specified threshold.

The second condition specifies that auditing reverses trust
transformation, though since trust transformations can be many-to-one,
the context returned by auditing need not be the exact preimage
of trust transformation:
\begin{condition}
\label{cond-extrapolate}
Let $s$ be a trusted context; then success of $\audit(s)$ entails
$\cod{\audit(s)} = s$.
\end{condition}
The last condition sufficiently strengthens the requirements of
auditing to formally establish that any auditing is a sound
verification of trust injected by the trust transformation:
\begin{condition}
\label{cond-extrapolate}
Let $s$ be a trusted context; then if $\audit(s)$ succeeds, for all 
$\priv$ it is the case that $\audit(\cod{s}) \vdash \priv$ implies
$s \vdash \priv$. 
\end{condition}
The condition requires that auditing of a trust-transformed context
must reconstruct a context that is at least as strong as the initial
context, pre-trust-transformation.  In $\RTR$, since authorization
contexts are credentials $\creds$, and the authorization decision
includes a risk threshold, trust transformation may be implemented as
the extension of a credential base $\creds$ with additional
``riskier'' credentials, along with an increase in the tolerable risk
threshold in chain discovery.  Returning to the example in
\autoref{section-rtrexamples}, the initial authorization decision
could be to determine
$\wtpath{\mathit{Ed}}{\mathit{Store.buyer}}{\risk}$, where $\risk \po
\mathit{medium}$.  An online trust transformation could add
$\mathit{Ed}$'s credential $\mathitcred{Acme.purchaser}{Ed}{high}$ to
the credential base, and tolerate $\risk \po \mathit{high}$.  Auditing
in this case would entail removing $\mathit{Ed}$'s certificate from
the credential base, and restoring the risk threshold $\risk \po
\mathit{medium}$.  In fact, just raising the risk threshold would be
sufficient, since raising the risk threshold would eliminate the
possibility of using $\mathit{Ed}$'s certificate in the proof of his
membership in $\mathit{Store.buyer}$, and in general trust
transformation and auditing could be implemented in $\RTR$ purely by
modulation of risk thresholds in chain discovery.

\subsection{Cost/Benefit Analysis}

Risk in $\RTR$ is defined in an abstract manner.  Although the
examples in this paper have used atomic risk values, it is possible to
define a risk ordering on compound risk values.  For example, suppose
both levels of ``trustability'' and expected wait times for retrieval
of specific credentials are considered components of risk.  The set
$\mathcal{K}$ could then contain elements of the form $(\risk,t)$,
where $\risk \in \setdefn{\mathit{low,medium,high}}$ as in
\autoref{section-rtrexamples} and $t$ is a wait time represented as an
integer, and:
$$
(\risk_1,t_1) \po (\risk_2,t_2) \iff \risk_1 \po \risk_2 \wedge
t_1 \le t_2
$$ 
reflecting that lower wait times, as well as higher confidence in
validity, define lower risk.  Maximum risk in chain discovery would
then specify both a tolerable level of trust, and a tolerable wait
time for any particular credential.

This suggests an \emph{interactive} procedure for chain discovery,
where the costs of raising the level of one component of risk could be
balanced against benefits in another risk dimension.  In the above
scenario, if chain discovery in some instance fails given a threshold
$(\risk, t)$, chain discovery could be re-run with a higher threshold,
but notice there is a choice of which element(s) of risk to raise.
The cost of raising $\risk$ can then be balanced against the benefits
in the time dimension, by re-running chain discovery with the
threshold $(\risk',t)$, where $\risk\po\risk'$.  The opposite is also
clearly the case.  This cost/benefit analysis would be further
enhanced by optimizing chain discovery.  The backward chain discovery
algorithm presented in this paper ensures that risks are kept below a
certain threshold, but does not attempt to optimize risk.  By
extending chain discovery with optimization techniques, in the
presence of compound risk, benefit dimensions could be optimized
within a fixed cost dimension.  For example, optimal wait times could
be sought given a $\mathit{high}$ level of trust risk.  Development of
optimizing algorithms is a topic for future work.
