\section{Introduction}

Trust management systems provide a formal means to specify and enforce
distributed authorization policies.  From its origins in BAN
\cite{BAN} and ABLP logic \cite{ABLP93}, research progress in this
field now comprises systems such as SDSI/SPKI
\cite{rivest-lampson-96,ellison-etal-rfc99} and $\RT$
\cite{Li:2003-04}.  The expressiveness and rigor of these systems has
become increasingly important to security in modern distributed
computing infrastructures, as web-based interactions continue to 
evolve in popularity and complexity.

Authorization in trust management usually takes into account several
facts and assertions, including certificates provided by non-local,
untrusted actors.  Although e.g.~cryptographic techniques provide
certain measures of confidence in this setting, not all components of
authorization can realistically be used with the same level of
confidence; the Pretty Good Privacy (PGP) framework acknowledges this,
by including a notion of trustworthiness of certificates.
Furthermore, efficient online authorization decisions often require a
weakening of ideal security, since the latter may be prohibitively
expensive.  This weakening may involve the acceptance of assertions
that would otherwise be verified, in case lowered confidence levels are
more tolerable than the danger of intractability.  Thus, many
practical distributed authorization decisions include elements of
\emph{risk} associated with authorization components, where risk could
be associated with trust, or computational cost, or any other
practical consideration making some facts more or less risky than
others.

A rigorous assessment of authorization should accurately assess risk,
but risk in trust management is usually an informal consideration.  In
this paper, we introduce a trust management logic, called $\RTR$, that
formally incorporates formal risk assessment.  The system is a variant
of $\RT$ \cite{Li:2003-04}, and includes an abstract definition of
risk, a means to associate risk with individual assertions, and a
semantics that assesses risk of authorization by combining the risk of
assertions used in authorization decisions.  This formalization
promotes development of a distributed authorization algorithm allowing
tolerable levels of risk to be precisely specified and rigorously
enforced.

\subsection{Paper Outline}

The remainder of the paper is organized as follows. In
\autoref{section-rt}, an overview of the $\RT_0$ system is given for
background.  In \autoref{section-rtr}, we define the syntax and
set-theoretic semantics of $\RTR$, an authorization logic with risk
assessment.  In \autoref{section-rtr-discovery}, we give a
graph-theoretic interpretation of $\RTR$ that is equivalent to the
set-theoretic semantics, and show that so-called credential graphs can
be automatically reconstructed by a distributed chain discovery
algorithm, as an implementation of distributed authorization.  In
\autoref{section-application}, we discuss some interesting
applications motivating the development of $\RTR$, and we conclude
with a summary of the paper and remarks on related work in
\autoref{section-conclusion}.
