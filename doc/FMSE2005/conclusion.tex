\section{Conclusion}
\label{section-conclusion}

We now conclude with comments on related work and a short summary of
the paper.

\subsection{Related Work}

Many trust management systems have been developed by previous
authors. In such a system resource owners write policy statements
using a suitable policy language that describes the attributes of
authorized users. When a request is made, the requesting entity
provides signed credentials that prove the requester complies with the
policy.  Proofs are constructed automatically, and implement a formal
semantics.  Previous systems include BAN \cite{BAN} and ABLP logic
\cite{ABLP93}, PolicyMaker \cite{Blaze96a}, KeyNote \cite{RFC-2704},
SDSI/SPKI \cite{rivest-lampson-96}, \cite{ellison-etal-rfc99}, and RT
\cite{Li:2002-05}, \cite{Li:2003-02}, \cite{Li:2003-04}, to name a
few.  However, our focus is not on trust management, but 
trust management extended with risk assessment.

Proof carrying carrying authorization (PCA)
\cite{bauer-phd,Appel99ccs} is a framework for specifying and
enforcing webpage access policies.  It is based on ABLP logic, but
includes primitives for detecting timestamp expiration.  While this
capability reflects some sense of risk assessment, it is not as
general as the notion of risk expressed in our system.

In \cite{anderson-ms-thesis}, semantics for a number of $\RT$ variants
are obtained via embedding in constraint datalog.  An implementation
of ``confidence levels'', similar to our notion of risk assessment, is
suggested via the use of constraints, though not developed in detail.
While it is possible that many interesting risk assessment schemes can
be defined using $\RT_1$ or $\RT_2$, we believe that defining a new
$\RT$ variant to explicitly capture the notion of risk assessments is
appealing in various respects.  In particular, we are able to define
risk in a general manner, and isolate issues related to online
authorization with components of risk.

Dealing with trustworthiness in distributed systems has been an active
research area (see, e.g., \cite{grandison-ieeecst00}). In
\cite{josang-ndss99}, an algebra is provided for reasoning about trust
in certificate chains. Our notion of risk is related to the notion of
trust, and some relevant operators of \cite{josang-ndss99} may be
directly incorporated into our framework.  Comparative expressiveness
of risk and trust operators is an interesting research topic, but
is beyond the scope of this paper.
 
\subsection{Summary}

In this paper we have defined $\RTR$, a role-based trust management
framework with formal risk assessment.  This system is a variation on
$\RT$ \cite{Li:2003-04}, and includes the capability to associate
credentials with risk, and to assess risk levels of authorization as
the aggregated risks of authorization components.  Risks are defined
in an abstract manner, under the requirement that the set of risks be
a complete lattice, with a monotonic aggregation operator.  A formal
semantics has been given, that associates role membership with risk
levels.  An algorithm has also been defined for implementation of this
semantics, providing an automatic risk assessed authorization
procedure.  The algorithm is specialized for functionality in a
distributed environment, and can be parameterized by risk thresholds,
specifying a maximum tolerable risk for authorization.  The algorithm
is directed, to avoid proof paths whose aggregate risks exceed the
given threshold, hence to risk as little as possible during the course
of authorization.
