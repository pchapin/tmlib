\conferenceinfo{FMSE'05,} {November 11, 2005, Fairfax, Virginia, USA.}

\CopyrightYear{2005}

\crdata{1-59593-231-3/05/0011} 

\title{Risk Assessment in Distributed Authorization}

\numberofauthors{3}

\author{
\alignauthor Peter Chapin   \\
  \affaddr{Department of Computer Science}\\
  \affaddr{University of Vermont}\\
  \email{pchapin@cs.uvm.edu}
\alignauthor Christian Skalka   \\
  \affaddr{Department of Computer Science}\\
  \affaddr{University of Vermont}\\
  \email{skalka@cs.uvm.edu}
\alignauthor X. Sean Wang \\
  \affaddr{Department of Computer Science}\\
  \affaddr{University of Vermont}\\
  \email{xywang@cs.uvm.edu}
}

\maketitle

\begin{abstract} 
Distributed authorization takes into account several elements,
including certificates that may be provided by non-local actors.
While most trust management systems treat all assertions as equally
valid up to certificate authentication, realistic considerations may
associate risk with some of these elements; some actors may be less
trusted than others, some elements may be more computationally
expensive to obtain, and so forth.  Furthermore, practical online
authorization may require certain levels of risk to be tolerated. In
this paper, we introduce a trust management logic that incorporates
formal risk assessment.  This formalization allows risk levels to be
associated with authorization elements, and promotes development of a
distributed authorization algorithm allowing tolerable levels of risk
to be precisely specified and rigorously enforced.
\end{abstract}

\category{C.2.0}{Computer-Communication Networks}{\\General}[Security and protection]

\terms{Security, Languages, Theory}

\keywords{Distributed Authorization, Trust Management Logic}
